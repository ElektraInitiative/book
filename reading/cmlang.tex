\documentclass[final,a4paper,11pt]{memoir} % Remove option 'final' to obtain debug information.

\newif\ifdebug
\debugfalse

\usepackage[type={CC},modifier={by-sa},version={4.0}]{doclicense}

\chapterstyle{veelo}%
\usepackage[scaled]{helvet}%
\usepackage{lmodern}%
\usepackage{courier}%
\usepackage[T1]{fontenc}%
\usepackage[english,naustrian]{babel}%
\usepackage[nodayofweek]{datetime}%
\usepackage{geometry}%
\usepackage{calc}%
\usepackage{etoolbox}%
\usepackage{graphicx}%

% Load packages to allow in- and output of non-ASCII characters.
\usepackage{lmodern}        % Use an extension of the original Computer Modern font to minimize the use of bitmapped letters.
\usepackage[T1]{fontenc}    % Determines font encoding of the output. Font packages have to be included before this line.
\usepackage[utf8]{inputenc} % Determines encoding of the input. All input files have to use UTF-8 encoding.
\usepackage{amsmath}    % Extended typesetting of mathematical expression.
\usepackage{amssymb}    % Provides a multitude of mathematical symbols.
\usepackage{mathtools}  % Further extensions of mathematical typesetting.

% Make font spacing more pretty (must be after font)
\usepackage[activate={true,nocompatibility},final,tracking=true,kerning=true,spacing=true,factor=1100,stretch=10,shrink=10]{microtype}
\SetTracking{encoding={*}, shape=sc}{0}  % http://www.khirevich.com/latex/microtype/

\usepackage[inline]{enumitem} % User control over the layout of lists (itemize, enumerate, description).
\usepackage{syntax}  % if I want a listings of listings http://tex.stackexchange.com/questions/96765/treat-grammar-as-listing
\usepackage{multicol} % For columns next to each other (e.g. for concurrency)
\usepackage{bibentry} % For adding bibentry within text
\usepackage{graphicx} % needed for scale box
\usepackage{mathtools}
\usepackage{currvita}
\usepackage{multido}




\usepackage{letltxmacro}
\usepackage{xfrac}
\usepackage{multirow}   % Allows table elements to span several rows.
\usepackage{booktabs}   % Improves the typesettings of tables.
\usepackage[usenames,dvipsnames,table]{xcolor} % Allows the definition and use of colors. This package has to be included before tikz.
\usepackage{tabularx}   % for tables (needs xcolor)
\usepackage{nag}       % Issues warnings when best practices in writing LaTeX documents are violated.

\ifdebug
\overfullrule=5mm
\usepackage[backgroundcolor=white]{todonotes}  % Provides tooltip-like todo notes.
\usepackage{etoolbox} % for debugging of ntheorems
\else
\usepackage[disable]{todonotes}  % Provides tooltip-like todo notes. [DISABLED]
\fi

\usepackage{pdfpages} % must be after xcolor (option clash)

\usepackage[sort&compress,numbers]{natbib}  % to be compatible with \citet, \citeauthor, \citeyear
\nobibliography*  % see http://tex.stackexchange.com/questions/49048/how-to-cite-one-bibentry-in-full-length-in-the-body-text

\usepackage[
	bookmarks=true, % PDF bookmarks allowed. NB! The level depth of bookmarks is the same as in the TOC.
	unicode=true, % PDF bookmarks in Unicode.
	bookmarksnumbered=true, % Section numbers in PDF bookmarks.
	bookmarksopenlevel=1, % The open level in PDF bookmarks.
	hyperindex=true, % Hyperlinked index.
	%linkcolor = blue, % The colour for in-document links (e.g. in the table of contents).
	%citecolor = cyan, % The colour for bibliographic citations.
	%urlcolor = red% The colour for hyperlinks to the Net.
]
{hyperref} % pdf features like links
\usepackage[all]{hypcap}

\usepackage[thmmarks,amsmath,amsthm]{ntheorem}
\usepackage{thm-restate}

\usepackage{wrapfig} % KPS figures
\usepackage{xifthen}
\usepackage{siunitx}
\usepackage{listings}
\usepackage[nodayofweek]{datetime} % for nicely formatted dates

\usepackage[acronym,toc]{glossaries} % Enables the generation of glossaries and lists fo acronyms. This package has to be included last.

\setcounter{chapter}{-1} % chapters start with 0
\setcounter{topnumber}{5}
\setcounter{bottomnumber}{5}
\setcounter{totalnumber}{5}

\setparaheadstyle{\itshape}
\setsubparaheadstyle{\itshape}


\OnehalfSpacing

\newcommand*{\SavedLstInline}{}
\LetLtxMacro\SavedLstInline\lstinline
\DeclareRobustCommand*{\lstinline}{%
  \ifmmode
    \let\SavedBGroup\bgroup
    \def\bgroup{%
      \let\bgroup\SavedBGroup
      \hbox\bgroup
    }%
  \fi
  \SavedLstInline
}

\newcommand{\authorname}{Markus Raab} % The author name without titles.
\newcommand{\booktitle}{Context-aware Configuration} % The title of the book.


\hypersetup
{
	pdfpagelayout = TwoPageRight,           % How the document is shown in PDF viewers (optional).
	pdftitle={Elektra}, %% title of the work
	pdfauthor={\authorname}, %% my name
	pdfsubject={\booktitle}, %% the title of the book
	pdfcreator={LaTeX2e and pdfLaTeX with hyperref-package.},
	pdfproducer={LaTeX2e, pdfLaTeX, hyperref-package, vim on GNU/Linux}, %%
	pdfkeywords={elektra, backends, modules, components,
		configuration, settings, preferences,
		plugin, plugins, type, checker, database,
		Komponenten, Datenbank, Kontrakte, Konfiguration}
	% colorlinks=true
}


\nonzeroparskip             % Create space between paragraphs
\setlength{\parindent}{0pt} % Remove paragraph indentation

\setlength\columnsep{1cm}   % for multicol

\setpnumwidth{2.5em}        % Avoid overfull hboxes in the table of contents (see memoir manual).
\setrmarg{4em}
\setlength{\cftpartnumwidth}{3em}

\setsecnumdepth{subsection} % Enumerate subsections.
\maxtocdepth{subsection}

\makeindex      % Use an optional index.

\csappto{theindex}{\label{index}} %to index the index

\rowcolors{1}{white}{lightgray}

\newcolumntype{R}{>{\raggedleft\arraybackslash}X}%

\renewcommand{\floatpagefraction}{.8}%

\clubpenalty = 10000
\widowpenalty = 10000
\displaywidowpenalty = 10000




\lstdefinelanguage{dump}
{
	morekeywords={kdbOpen,ksNew,keyNew,keyMeta,keyCopyMeta,keyEnd,ksEnd,kdbClose},
	sensitive=false,
	morecomment=[l]{//},
	morecomment=[s]{/*}{*/},
	morestring=[b]",
}


\lstdefinelanguage{SpecElektra}{
	%
	comment=[l]{;},
	commentstyle=\color{purple}\ttfamily,
	%
	morestring=[b]',
	morestring=[b]`,
	morestring=[b]",
	stringstyle=\color{purple}\ttfamily,
	%
	sensitive=f,% keywords are not case sensitive
	%
	% Colors see https://en.wikibooks.org/wiki/LaTeX/Colors
	%
	keywordstyle=\color{BlueViolet}\bfseries,
	keywordstyle=[2]\color{Green},
	keywordstyle=[3]\color{Aquamarine}\bfseries\textit,
	keywordstyle=[4]\color{NavyBlue}\bfseries,
	keywordstyle=[5]\color{Mahogany},
	%
	keywords={layer, require, validation, check, range, description, rationale, requirement, visibility, accessibility, enable, condition, message, default, opt, readonly, type, context, property1, property2, description, file, content, mountpoint, metadata, infos, plugins},
	keywords=[2]{},
	keywords=[3]{order, interface, network, emphasized},
	%keywords=[4]{[, ]},  %Not needed
	keywords=[4]{},
	keywords=[5]{},
	%
	literate={:=}{{{\color{red}\textbf:=}}}2
		 {\%}{{{\color{NavyBlue}\textbf\%}}}1
		 {[}{{{\color{Sepia}\textbf[}}}1
		 {]}{{{\color{Sepia}\textbf]}}}1,
}

\lstdefinelanguage{Cpp}{%
	language     = C++,
	literate=
}


\lstdefinelanguage{CfgElektra}{
	comment=[l]{;},
	commentstyle=\color{purple}\ttfamily,
	%
	morestring=[b]',
	morestring=[b]`,
	morestring=[b]",
	stringstyle=\color{purple}\ttfamily,
	%
	%
	sensitive=f,% keywords are not case sensitive
	%
	% Colors see https://en.wikibooks.org/wiki/LaTeX/Colors
	%
	keywordstyle=\color{Bittersweet}\bfseries,
	keywordstyle=[2]\color{DarkOrchid}\bfseries,
	keywordstyle=[3]\color{ForestGreen}\bfseries\textit,
	keywordstyle=[4]\color{Goldenrod}\bfseries,
	keywordstyle=[5]\color{CarnationPink},
	%
	keywords={},
	keywords=[2]{},
	keywords=[3]{},
	keywords=[4]{},
	keywords=[5]{},
	%
	literate={=}{{{\color{ForestGreen}\textbf=}}}1
		 %{<-}{{{\color{ForestGreen}\textbf<-}}}2
		 %{*}{{{\color{Bittersweet}\textbf*}}}1
		 {\%}{{{\color{NavyBlue}\textbf\%}}}1,
}




\lstset{language=SpecElektra, % Use SpecElektra as default programming language
	%boxpos=t, % make boxes a bit more unbreakable
	%frame=lines, % top+bottom line
	basicstyle=\ttfamily, % Use normal-size true type font
	showspaces,%
	showstringspaces=false, % Don't put marks in string spaces
	showlines=true, % make sure empty lines at end are shown (needed for concurrency
	tabsize=4, % spaces per tab
	xleftmargin=\parindent, % should be 18pt or 1.5em as defined by memoir
	%Does not really work well (needs to be deactivated for shortlistings):
	breaklines=false,
	%postbreak=\mbox{\textcolor{red}{$\hookrightarrow$}\space},
	%breakautoindent=true,
	%prebreak={\mbox{\ensuremath{\curvearrowright}}} % Zeichen am Zeilenende (Umbruch)
	%breaklines=true,
	%breakautoindent=true,
	%prebreak=\small\symbol{'134}, % backslash
	%prebreak={\mbox{\ensuremath{\curvearrowright}}} % lange kure
	%prebreak={\mbox{\ensuremath{\hookleftarrow}}} % lange kure
	%xleftmargin=3.0ex, %for some formats
	%xrightmargin=1.0ex, %for some formats
	%
	% Files do not work in utf8 see also:
	% http://stackoverflow.com/questions/1116266/listings-in-latex-with-utf-8-or-at-least-german-umlauts
	% http://tex.stackexchange.com/questions/24528/having-problems-with-listings-and-utf-8-can-it-be-fixed
	% Should work but doesn't? (Maybe add to literate broken?)
	%add to literate={ö}{{\"o}}1
	%	{ä}{{\"a}}1
	%	{ü}{{\"u}}1
	%	{Ö}{{\"O}}1
	%	{Ä}{{\"A}}1
	%	{Ü}{{\"U}}1
	%	{ß}{{\ss}}1,
	%
	% listingsutf8 did not work, made umlauts in comments very strange
	%extendedchars=true,
	%inputencoding=utf8,
	%
	%morecomment=[l][\color{blue}]{...}, % Line continuation (...) e.g. blue comment
	morekeywords={for_each},
	numbers=left, % Line numbers on left
	firstnumber=1, % Line numbers start with line 1
	numberstyle=\small\color{blue}, % Line numbers are blue and small
	numbersep=5pt,
	%stepnumber=5 % Line numbers go in steps of 5
}



\graphicspath{{../pic/}{../figures/}{../graphics/}{../ipe/}{../ggplot/}}

%amsmath stuff
\DeclareMathSymbol{\mlq}{\mathord}{operators}{``}  % math left quote
\DeclareMathSymbol{\mrq}{\mathord}{operators}{`'}  % math right quote


%operators
\DeclareMathOperator*{\concat}{\mathnormal{+\!\!+}}
\DeclareMathOperator*{\concatkey}{\mathnormal{\colon\!\!\!/}}

%symbols
%make Linebreak work without math
\newcommand{\NullValue}{\epsilon}
\newcommand{\CleanString}{\overrightarrow{C'}}
\newcommand{\RelativeKeyNames}{\mathbb{R}}
\newcommand{\KeyNames}{\mathbb{N}}
\newcommand{\MetaKeyValues}{\bigcup\limits_{g \in G} X_g}
\newcommand{\EmptyNameSpace}{\epsilon}
\newcommand{\Keys}{\mathbb{K}}
\newcommand{\KeySets}{\mathcal{K}}
\newcommand{\NullKey}{\varnothing}
\newcommand{\EmptyKeySet}{\kappa}
\newcommand{\NameSpaces}{\mathnormal{N}}
\newcommand{\LineBreak}{\ensuremath{\hookleftarrow}}
\newcommand{\WhiteSpace}{\textvisiblespace}
\newcommand{\SetOfValues}{\ensuremath{\mathbb{C}_\NullValue}}
\newcommand{\Strings}{\ensuremath{\mathbb{C}}}
\newcommand{\NonEmptyStrings}{\ensuremath{\mathbb{C}_{\geq1}}}
\newcommand{\CleanStrings}{\ensuremath{\mathbb{C}'}}
\newcommand{\NonEmptyCleanStrings}{\ensuremath{\mathbb{C}_{\geq1}'}}


%% For Elektra names
\newcommand{\elektra}[2][]
{%
	\ifthenelse{\isempty{#2}}%
		{\textsc{Elektra}}% if #2 is empty
		{%
		\ifthenelse{\isempty{#1}}%
			{\textsc{#2\-Elektra}}% if #1 is empty, do not index it!
			{\index{#1Elektra|boldindex}\textsc{#2\-Elektra}}% if #1 is not empty, index it
		}%
}

%% Highlight an important word + index it
\newcommand{\empha}[2][]
{%
	\ifthenelse{\isempty{#1}}%
		{\index{#2}\emph{#2}}% if #1 is empty
		{\index{#1}\emph{#2}}% if #1 is not empty
}


%% https://tex.stackexchange.com/questions/391448/list-all-theorems-that-were-not-restated/391484#391484
\makeatletter
\long\def\g@addto@toks#1#2{\global#1\expandafter{\the#1#2}}
\newtoks\@toks@restatable
\newtoks\listofrestatable
\newtoks\listofrestatated
\let\tre@restatable\restatable
\let\tre@endrestatable\endrestatable
\renewenvironment{restatable}[3][]{%
  \g@addto@toks\@toks@restatable{#3,}%
  \g@addto@toks\listofrestatable{\process@comma#3 (#2)}%
  \global\@namedef{add@restated@#3}{%
    \g@addto@toks\listofrestatated{#3; }%
    \global\cslet{add@restated@#3}{\relax}%
  }
  \tre@restatable[#1]{#2}{#3}%
  \csgpreto{#3}{\@nameuse{add@restated@#3}}%
}{\tre@endrestatable}
\def\process@comma{\def\process@comma{,\space}}
\def\notrestated{}
\def\checknotrestated{%
  \@for\@rthm:=\the\@toks@restatable\do{%
    \ifx\@rthm\@empty\else
      \expandafter\ifx\csname add@restated@\@rthm\endcsname\relax\else
        \xdef\notrestated{\notrestated\@rthm; }\fi
    \fi
  }
  \ifx\notrestated\empty\else
    \par\vspace{1pc}\par\noindent
    Warning! -- There were theorems that have not been restated: \notrestated
    \@latex@warning{There were theorems that have not been restated: \notrestated}%
\fi
}
\makeatletter



% https://en.wikibooks.org/wiki/LaTeX/Fonts
\DeclareTextFontCommand\textintro{\normalfont\bfseries\itshape} % nice!
%\DeclareTextFontCommand\textintro{\normalfont\slshape\scshape} % also nice! needs slantsc
%\DeclareTextFontCommand{\question}{\em}

%amsmath stuff
\DeclareMathSymbol{\mlq}{\mathord}{operators}{``}  % math left quote
\DeclareMathSymbol{\mrq}{\mathord}{operators}{`'}  % math right quote

%% Introduces a new word in the book + index it
\newcommand{\intro}[2][]
{%
	\ifthenelse{\isempty{#1}}%
		{\index{#2|boldindex}\textintro{#2}}% if #1 is empty
		{\index{#1|boldindex}\textintro{#2}}% if #1 is not empty
}

% https://tex.stackexchange.com/questions/390144/why-does-lstlisting-removes-a-space-after-morestring/390156#390156
\makeatletter
%\def\lst@visiblespace{\lst@ttfamily{\ }{\ }}
\def\lst@visiblespace{\lst@ttfamily{\phantom{\char32}}{\phantom{\textvisiblespace}}}
\makeatother

%https://stackoverflow.com/questions/2767389/referencing-a-theorem-like-environment-by-its-name
\makeatletter
\def\namedlabel#1#2{\begingroup
   \def\@currentlabel{#2}%
   \label{#1}\endgroup
}
\makeatother


% From here:
% https://tex.stackexchange.com/questions/24101/theorem-decorations-that-stay-with-theorem-environment
\newcommand{\theoremhang}{% top theorem decoration
  \begingroup%
    \setlength{\unitlength}{.005\linewidth}% \linewidth/200
    \begin{picture}(0,0)(1.5,0)%
      \linethickness{0.45pt} \color{Black}%
      \put(-3,2){\line(1,0){206}}% Top line
      \multido{\iA=2+-1,\iB=50+-10}{5}{% Top hangs
        \color{black!\iB}%
        \put(-3,\iA){\line(0,-1){1}}% Top left hang
        \put(203,\iA){\line(0,-1){1}}% Top right hang
      }%
    \end{picture}%
  \endgroup%
}%

\newcommand{\theoremhung}{% bottom theorem decoration
  \nobreak%
  \vspace*{-1.8ex}%
  \begingroup%
    \setlength{\unitlength}{.005\linewidth}% \linewidth/200
    \begin{picture}(0,0)(1.5,0)%
      \linethickness{0.45pt} \color{Black}%
      \put(-3,0){\line(1,0){206}}% Bottom line
      \multido{\iA=0+1,\iB=50+-10}{5}{% Bottom hangs
        \color{black!\iB}%
        \put(-3,\iA){\line(0,1){1}}% Bottom left hang
        \put(203,\iA){\line(0,1){1}}% Bottom right hang
      }%
    \end{picture}%
  \endgroup%
}%

\newcommand{\theoremspace}{\needspace{2\baselineskip}}
\newcommand{\theoremspuce}{\nopagebreak\noindent}

\newcommand{\chapterhung}{\vspace{-1.4em}\theoremhung\vspace{2em}}

\newcounter{defexample}
\numberwithin{defexample}{chapter}


\theoremstyle{plain}
\newtheorem{definition}[defexample]{Definition}

\theoremstyle{definition}
\theoremsymbol{\ensuremath{\blacktriangle}}%
\newtheorem{example}[defexample]{Example}

\theoremstyle{definition}
\theoremsymbol{}%
\theoremprework{\theoremspace\theoremhang\vspace*{-1.3ex}}%
\theorempostwork{\vspace{-1em}\theoremspuce\theoremhung\vspace{1em}}%
\theoremheaderfont{\upshape\bfseries}
\theoremseparator{.}
\theorembodyfont{\upshape}
\newtheorem{lemma}{Lemma}

\theoremstyle{plain}
\newtheorem{contribution}{Contribution}

\theoremstyle{plain}
\newcounter{reqcounter}
\newtheorem{requirement}[reqcounter]{Requirement}

\theoremstyle{nonumberplain}
\newtheorem{goal}{Goal}

\theoremstyle{definition}
\newtheorem{RQ}{RQ}

\theoremstyle{definition}
\newtheorem{RQsub}{RQ}[RQ]

\theoremstyle{definition}
\newtheorem{RQsubsub}{RQ}[RQsub]


\theoremstyle{nonumberplain}
\theoremheaderfont{\sffamily\bfseries\upshape}%
\theorembodyfont{\itshape}%
\theoremsymbol{}%
\theoremseparator{\ }%
\theoremprework{\theoremspace\theoremhang\vspace*{-1.3ex}}%
\theorempostwork{\vspace{-1em}\theoremspuce\theoremhung\vspace{1em}}%
\setlength{\theorempreskipamount}{-1ex}
\setlength{\theorempostskipamount}{-1ex}
\newtheorem{hypothesis}{Hypothesis}

\theoremstyle{nonumberplain}
\theoremheaderfont{\sffamily\bfseries\upshape}%
\theorembodyfont{\itshape}%
\theoremsymbol{}%
\theoremseparator{\ }%
\theoremprework{\theoremspace\theoremhang\vspace*{-1.3ex}}%
\theorempostwork{\vspace{-1em}\theoremspuce\theoremhung\vspace{1em}}%
\setlength{\theorempreskipamount}{-1ex}
\setlength{\theorempostskipamount}{-1ex}
\newtheorem{finding}{Finding}

\theoremstyle{nonumberplain}
\newtheorem{implication}{Implication}


% https://tex.stackexchange.com/questions/229383/alignment-of-modified-veelo-chapter-style-memoir
\makeatletter
\newlength{\numberheight}
\setlength{\numberheight}{\beforechapskip}

\makechapterstyle{meelo}{
	\setlength{\afterchapskip}{40pt}
	\renewcommand*{\chapterheadstart}{\vspace*{40pt}}
	\renewcommand*{\afterchapternum}{\par\nobreak\vskip 25pt}
	\renewcommand*{\chapnamefont}{\normalfont\LARGE\flushright}
	\renewcommand*{\chapnumfont}{\normalfont\HUGE}
	\renewcommand*{\chaptitlefont}{\normalfont\HUGE\bfseries\flushright}
	\renewcommand*{\printchaptername}{%
		\chapnamefont\MakeUppercase{\@chapapp}}
	\renewcommand*{\chapternamenum}{}
	\setlength{\beforechapskip}{18mm}
	\setlength{\midchapskip}{\paperwidth}
	\addtolength{\midchapskip}{-\textwidth}
	\addtolength{\midchapskip}{-\spinemargin}
	\addtolength{\midchapskip}{-11.5em}
	\renewcommand*{\printchapternum}{%
		\enspace\resizebox{!}{\numberheight}{\chapnumfont\thechapter}%
			\rlap{\hspace{2cm}\rule{\midchapskip}{\beforechapskip}}%
	}%
	\makeoddfoot{plain}{}{}{\thepage}%
}

\chapterstyle{meelo}
\makeatother


%% Make grammar(s) pretty:
%\setlength{\grammarparsep}{20pt plus 1pt minus 1pt} % increase separation between rules
\setlength{\grammarindent}{12em} % increase separation between LHS/RHS

%% to be used to mark the main page of an index
\newcommand{\boldindex}[1]{\textbf{\hyperpage{#1}}}

%\newunicodechar{⏎}{$\hookleftarrow$} % \hookleftarrow ≅ ↩︎


%% Refer to a Goal
\newcommand{\goalref}[1]
{Goal~\ref{goal:#1}}

%% Refer to a Section
\newcommand{\secref}[1]
{Section~\ref{sec:#1}}
%{Section~\ref{sec:#1} on page~\pageref{sec:#1}}

%% Refer to a Table
\newcommand{\tabref}[1]
{Table~\ref{tab:#1}}

%% Refer to a Figure
\newcommand{\figref}[1]
{Figure~\ref{fig:#1}}

%% Refer to a Example
\newcommand{\exref}[1]
{Example~\ref{ex:#1}}

%% Refer to a Lemma
\newcommand{\lemmaref}[1]
{Lemma~\ref{lemma:#1}}

%% Refer to a Definition
\newcommand{\defref}[1]
{Definition~\ref{def:#1} on page~\pageref{def:#1}}

\let\oldeqref\eqref
%% Refer to an equation
\renewcommand{\eqref}[1]
{Equation~\oldeqref{eq:#1}}

%% Refer to RQ for sections
\newcommand{\rqsubsection}[2]{\subsection{RQ \ref*{rq:#1}: #2}}


\newcommand{\fixtheorem}[0]{\leavevmode\vspace{-1em}}

%% Refer to a chapter
\newcommand{\chapref}[1]
{Chapter~\ref{chapter:#1}}
%{Chapter~\ref{chapter:#1} on page~\pageref{chapter:#1}}

\newcommand{\question}[1]
{\emph{O: ``#1''}}

\newcommand{\methodQuestion}{\emph{Q:}}

\newcommand{\methodSource}{\emph{S:}}



%define namespace
%textsc also used in backend.tex:344
\makeatletter
\newcommand{\namespace}[1]{\@namespace{#1}}
\newcommand{\@@namespace}[1]{\textsc{\let\@namespace\@@@namespace#1}}
\newcommand{\@@@namespace}[1]{\textnormal{\let\@namespace\@@namespace#1}}
\let\@namespace\@@namespace
\makeatother


\makeatletter
\newcommand{\notsupported}[1]{\@notsupported{Not supported#1}}
\newcommand{\@@notsupported}[1]{\textsc{\let\@notsupported\@@@notsupported#1}}
\newcommand{\@@@notsupported}[1]{\textnormal{\let\@notsupported\@@notsupported#1}}
\let\@notsupported\@@notsupported
\makeatother

% code environment (avoids listings to have page breaks within them)
% \begin{code}[caption={a},language=Cpp]
% \end{code}
\lstnewenvironment{code}[1][]%
{%
\microtypesetup{activate=false}%
\noindent%
\minipage{\linewidth}%
%\vspace{0.5\baselineskip}\medskip%
\theoremspace%
\theoremhang%
\lstset{#1}}%
{\theoremhung\endminipage%
\microtypesetup{activate=true}%
}
%{}

\newcommand{\StashGrammar}{\let\syntleft\relax \let\syntright\relax}

%two lines within a cell
\newcommand{\blap}[1]{{\begin{tabular}[t]{@{}c@{}}#1\end{tabular}}}

%operators
\DeclareMathOperator*{\concat}{\mathnormal{+\!\!+}}
\DeclareMathOperator*{\concatkey}{\mathnormal{\colon\!\!\!/}}

%symbols
%make Linebreak work without math
\newcommand{\NullValue}{\epsilon}
\newcommand{\CleanString}{\overrightarrow{C'}}
\newcommand{\RelativeKeyNames}{\mathbb{R}}
\newcommand{\KeyNames}{\mathbb{N}}
\newcommand{\MetaKeyValues}{\bigcup\limits_{g \in G} X_g}
\newcommand{\EmptyNameSpace}{\epsilon}
\newcommand{\Keys}{\mathbb{K}}
\newcommand{\KeySets}{\mathcal{K}}
\newcommand{\NullKey}{\varnothing}
\newcommand{\EmptyKeySet}{\kappa}
\newcommand{\NameSpaces}{\mathnormal{N}}
\newcommand{\LineBreak}{\ensuremath{\hookleftarrow}}
\newcommand{\WhiteSpace}{\textvisiblespace}
\newcommand{\SetOfValues}{\ensuremath{\mathbb{C}_\NullValue}}
\newcommand{\Strings}{\ensuremath{\mathbb{C}}}
\newcommand{\NonEmptyStrings}{\ensuremath{\mathbb{C}_{\geq1}}}
\newcommand{\CleanStrings}{\ensuremath{\mathbb{C}'}}
\newcommand{\NonEmptyCleanStrings}{\ensuremath{\mathbb{C}_{\geq1}'}}

%% Define a key
\newcommand{\key}[1]{\lstinline[language=,literate={/}{}{0\discretionary{/}{}{/}},postbreak=,breaklines=false]{#1}}

% write a string
\newcommand{\str}[1]
{\syntax{\lq\texttt{#1}'}}

% write a plugin (hardly used)
\newcommand{\plugin}[1]
{\texttt{#1}}

%% Define a property (also used for clauses)
\newcommand{\property}[1]{\lstinline[language=,literate={/}{}{0\discretionary{/}{}{/}},postbreak=,breaklines=false]{#1}}


%: WARNING: needs to be kept in sync with background.tex:833 (XXX)
%\lstMakeShortInline[postbreak=,keywordstyle={}]^
%\lstMakeShortInline[breaklines=true]^



%define strong
\makeatletter
\newcommand{\strong}[1]{\@strong{#1}}
\newcommand{\@@strong}[1]{\noindent \textbf{\let\@strong\@@@strong#1}}
\newcommand{\@@@strong}[1]{\textnormal{\let\@strong\@@strong#1}}
\let\@strong\@@strong
\makeatother

%define rqref
\makeatletter
\newcommand{\rqref}[1]{\@rqref{RQ~\ref{rq:#1}}}
\newcommand{\@@rqref}[1]{\noindent \textbf{\let\@rqref\@@@rqref#1}}
\newcommand{\@@@rqref}[1]{\textnormal{\let\@rqref\@@rqref#1}}
\let\@rqref\@@rqref
\makeatother


%define reqref
\makeatletter
\newcommand{\reqref}[1]{\@reqref{\ref{req:#1}}}
\newcommand{\@@reqref}[1]{\noindent \textbf{\let\@reqref\@@@reqref#1}}
\newcommand{\@@@reqref}[1]{\textnormal{\let\@reqref\@@reqref#1}}
\let\@reqref\@@reqref
\makeatother

%define p
\makeatletter
\newcommand{\p}[1]{\SI{#1}{\percent}}
\let\@p\@@p
\makeatother

\hyphenation{plug-in plug-ins meta-data meta-key meta-value inte-grated back-end front-end aware-ness con-text-ual gn-ome libre-office meta-data list-en-er eval-uate prop-er-ty prop-er-ties con-tex-tual name-space name-spaces ap-pli-ca-tions}


\setcounter{chapter}{1} % chapters start with 0


\title{Configuration Management Languages}
\author{Markus Raab}


\begin{document}

\begin{titlingpage}
\maketitle
\doclicenseThis
\end{titlingpage}

\paragraph{Configuration management language}
is a relatively vaguely defined term---it is a language where some kind of configuration is specified.
In this reading text for the lecture Configuration Management, we will investigate different kinds of configuration management languages.

We investigated who already created configuration specification languages to improve configuration access for configuration management tools, i.e. configuration management language, answering the following research question:
\begin{restatable}{RQ}{rqBackgroundSpecificationLanguages}
 \label{rq:background-specification-languages}
 Which configuration specification languages are suitable to improve configuration access for configuration management tools?
\end{restatable}
%\rqBackgroundSpecificationLanguages*

\begin{hypothesis}[RQ~\ref{rq:background-specification-languages}]
We expect to find a large variety of configuration specification languages that already improve configuration access for configuration management tools.
\end{hypothesis}

\section{Method}

We did a survey of all configuration specification languages as revealed by Google Scholar with the search term:
\begin{verbatim}
language
"configuration specification" OR
"configuration description" OR
"configuration definition" OR
"configuration declaration"
\end{verbatim}

This search yielded several thousand articles.
We grouped them by dates because of download limits:
\begin{verbatim}
1950-1998     946 articles
1999-2004     919 articles
2005-2007     786 articles
2008-2010     872 articles
2011-2012     723 articles
2013-2016     810+ articles
\end{verbatim}

The ^+^ sign means that we subscribed to the search term to keep track of new incoming articles.
We scanned through the titles of all papers---or if this was not enough, we read the abstract---to filter off-topic papers.
In particular, we removed all articles that describe general purpose languages, behavioral descriptions, or that are domain-specific.
After this process, we grouped papers that described the same configuration specification language.
As result, we found 92 configuration specification languages.
Due to lack of time, we only further processed the ones that are at least remotely related to \elektra{Spec}, the specification language of \elektra{}, and are of interest for this reading text.
In this step, we excluded about \sfrac{3}{4} of the configuration specification languages.

In the rest of the section, we will describe four selected properties, i.\,e.\ expressiveness, reasoning, modularity, and reusability, for some configuration specification languages.
Others are mentioned in ``Others''.



\section{CFEngine}

CFEngine~\cite{burgess1995cfengine,pandey2012investigating} is a language-based system administration tool that pioneered idempotent behavior.
It uses declarative class-based decision structures.
\citet{burgess2003theory}~introduces theory behind it.

\paragraph*{Expressiveness:}
CFEngine allows us to declare dependences and facilitates some high-level configuration specification constructs.
In its initial variants it neither had validation specifications, cardinalities, nor higher-level relationships.
\paragraph*{Reasoning:}
\notsupported{}
\paragraph*{Modularity:}
\notsupported{}

\paragraph*{Reusability:}
Existing system administrator scripts can be profitably run from CFEngine.


\section{NIX}

The NIX language~\cite{dolstra2007purely} claims to be purely functional as a novel feature.
The main concept is the referential transparency both for the configuration specification language and for the system itself.
A large-scale deployment shows that the approach is feasible and practical.

Read \url{https://nixos.org/} for more information.

\paragraph*{Expressiveness:}
NIX expressions, for example functions, describe how to build software packages.
The unit of variability is a package.
Additionally, a hierarchical set of properties describes the configuration specification.
Otherwise, the expressiveness is low, NIX describes neither cardinalities nor relationships.

\paragraph*{Reasoning:}
Because of the referential transparency of the system itself, every solution derived from the NIX expressions should be valid, so no reasoning or conflict handling is necessary.
Some operations, however, might lead to a completely new system.

\paragraph*{Modularity:}
The NIX expressions are modular because they ensure absence of side effects and thus can be easily composed.

\paragraph*{Reusability:}
Derivations that describe atomic build actions are reused in other derivations.
Import and inherit features are used to create packages, improving reusability.




\section{ConfValley (CPL)}

\citet{huang2015confvalley} introduce systematic validation for cloud services.
ConfValley uses a unified configuration settings representation for tens of different configuration file formats.
Its configuration specification language, called CPL, does not aim to be a type-safe configuration specification language.
It enables, however, system administrators of cloud services to write declarative specifications of properties with correctness constraints.

\paragraph*{Expressiveness:}
CPL introduces many concepts and has non-trivial language features.
Its most expressive elements are first-order quantifiers.
CPL is not able to specify dynamic and complex requirements.

\paragraph*{Reasoning:}
Constraints can be inferred by running an inference engine on configuration settings that are considered good (black-box approach).
Within the validation engine, however, no constraint solver is available.

\paragraph*{Modularity:}
CPL aims at easy grouping of constraints.
Its extensibility has limitations, for example, adding language primitives need modifications in the compiler.
The authors claim, however, that these changes can be done in a straightforward way---at least for predicates.%
{\parfillskip=0pt plus .7\textwidth \emergencystretch=.5\textwidth \par}

\paragraph*{Reusability:}
Using transformations and compositions, predicates can be reused in different contexts.
Also with language constructs like ^let^, specifications can be reused.




\section{Quattor (Pan)}

\citet{cons2002pan} invented and used PAN for many machines within CERN.
Furthermore, the language is still used by Quattor.
The configuration database in Pan comprises high-level and low-level descriptions.
The low-level descriptions are in XML syntax.
Here we focus on the declarative, high-level description.

\paragraph*{Expressiveness:}
The Pan language allows users to specify data types, validation with code snippets and constraints.
It only supports lists but no configurable cardinality nor is-a/part-of relationships.
The compiler uses a 5 step process: compilation, execution, insertions-of-defaults, validation, and serialization.

\paragraph*{Reasoning:}
Pan focuses on validating configurations, it is not able to generate new configurations.
Pan provides type enforcement with embedded validation code.

\paragraph*{Modularity:}
The language has user-defined data types (called templates) but otherwise has only minimal support for modularity.
In particular, side effects and assignments hinder modularity of validation code.

\paragraph*{Reusability:}
Reusability and collaboration is only possible via simple include statements and a simple inheritance mechanism of templates.


\section{UML}

\citet{felfernig1999knowledge,felfernig2000uml,felfernig2002joint} describe an approach where the unified modeling language (UML) is used as notation to simplify the construction of a logic-based description.
The papers formally describe the semantics. Tools are available and experimental results show feasibility.

\paragraph*{Expressiveness:}
All UML features, including cardinality, domain-specific stereotypes and OCL-constraints are available.
The basic structure of the system is specified using classes, generalization and aggregation.
Resources impose additional constraints on the possible system structure.
Finally, the require-relation and incompatible-relation allow us to limit valid configurations.

\paragraph*{Reasoning:}
Customers provide additional input data and requirements for the actual variant of the product.
The logical sentences are range-restricted first-order-logic with a set extension and interpreted function symbols.
For decidability, the term-depth is limited to a fixed number.
It is possible to show that the configuration is consistent or that no solution exists.

\paragraph*{Modularity:}
Generalization is present without multiple inheritance with disjunctive semantics, i.\,e., only one of the given subtypes will be instantiated.

\paragraph*{Reusability:}
For shared aggregation additional ports are defined for a part.


\section{Others}

\textsc{Proteus}~\cite{tryggeseth1995modelling} shows the tight relation between software configuration management, like Git or Svn, and configuration specification languages.
\textsc{Proteus} combines both worlds in a powerful build system.
For example:

\begin{code}[basicstyle=\tiny,morekeywords={family,attributes,end,physical,default,classifications},gobble=4,language=]
family CalcProg
	attributes
		HOME : string default "/home/ask/proteus/test";
		workspace := HOME ++ "/calc/src/"; // string concatenation
		repository := "calc/";
		end
	physical
		main => "main.C";
		defs => "defs.h";
		exe => "calc.x" attributes workspace := HOME ++ "/calc/bin"; end
		classifications status := standard.derived; end;
	end
end
\end{code}

\citet{lock2005strider} invented Strider that supports modeling and analysis of complex systems.

ConfSolve~\cite{hewson2011modelling,hewson2012declarative} is a configuration specification language that is translated to a standard constraint programming language called MiniZinc.
Their focus is in finding configurations for machines and not to compute configuration settings.
ConfSolve generates Puppet code for deployment.

Many other configuration specification languages have been found during the survey~\cite{roll2003towards,pandey2012investigating,hill2011modeling,anderson2002lcfg,deliverable1996tina,lujak2015orcas,sommerville1992configuration,giese2012industrial,huang2007system%
,novak2005automatic%Puppet, Quattor,..
,gunther2012software,berger2013survey,magableh2010primitive,friedrich1999consistency}, but they do not provide configuration access specifications for FLOSS applications.

\section{Result}

The result of the survey was that we could not find a configuration specification language to be used as basis for \elektra{Spec}.
Instead all configuration specification languages we investigated had a different focus, which leads us to our answer of:
\rqBackgroundSpecificationLanguages*

\begin{finding}
We have to reject our hypothesis for \rqref{background-specification-languages}:
We did not find any configuration specification language that supports our goal of improving configuration access for configuration management tools.
Instead earlier work had at least one of the following two assumptions:
\begin{itemize}
\item Configuration files need to be generated instead of directly accessing configuration settings.
\item Applications need to be reimplemented using new development methods.
Architecture description languages, software product lines, and similar approaches have this assumption.
\end{itemize}
\end{finding}
\par
Both assumptions hinder progress in improvements of configuration access for configuration management tools.

\bibliographystyle{plainnat}

\begingroup
\sloppy
\makeatletter
\g@addto@macro{\UrlBreaks}{\UrlOrds}
\makeatother
\bibliography{../shared/elektra.bib}
\endgroup



\end{document}
