%\ifdefined\handout
%\documentclass[handout,aspectratio=1610,xcolor={usenames,dvipsnames,table}]{beamer}
%\else
\documentclass[aspectratio=1610,xcolor={usenames,dvipsnames,table}]{beamer}
%\fi

\newcommand{\mylecture}{Configuration Management}

%\ifdefined\handout
%\documentclass[handout,aspectratio=1610,xcolor={usenames,dvipsnames,table}]{beamer}
%\else
\documentclass[aspectratio=1610,xcolor={usenames,dvipsnames,table}]{beamer}
%\fi

\newcommand{\mylecture}{Configuration Management}

%\ifdefined\handout
%\documentclass[handout,aspectratio=1610,xcolor={usenames,dvipsnames,table}]{beamer}
%\else
\documentclass[aspectratio=1610,xcolor={usenames,dvipsnames,table}]{beamer}
%\fi

\newcommand{\mylecture}{Configuration Management}

\input{../setup}
\input{../shared/setup}

\lstdefinelanguage{dump}
{
	morekeywords={kdbOpen,ksNew,keyNew,keyMeta,keyCopyMeta,keyEnd,ksEnd,kdbClose},
	sensitive=false,
	morecomment=[l]{//},
	morecomment=[s]{/*}{*/},
	morestring=[b]",
}


\lstdefinelanguage{SpecElektra}{
	%
	comment=[l]{;},
	commentstyle=\color{purple}\ttfamily,
	%
	morestring=[b]',
	morestring=[b]`,
	morestring=[b]",
	stringstyle=\color{purple}\ttfamily,
	%
	sensitive=f,% keywords are not case sensitive
	%
	% Colors see https://en.wikibooks.org/wiki/LaTeX/Colors
	%
	keywordstyle=\color{BlueViolet}\bfseries,
	keywordstyle=[2]\color{Green},
	keywordstyle=[3]\color{Aquamarine}\bfseries\textit,
	keywordstyle=[4]\color{NavyBlue}\bfseries,
	keywordstyle=[5]\color{Mahogany},
	%
	keywords={layer, require, validation, check, range, description, rationale, requirement, accessibility, enable, condition, message, default, opt, readonly, type, context, property1, property2, description, file, content, mountpoint, metadata, infos, plugins},
	keywords=[2]{},
	keywords=[3]{order, interface, network, emphasized},
	%keywords=[4]{[, ]},  %Not needed
	keywords=[4]{},
	keywords=[5]{},
	%
	literate={:=}{{{\color{red}\textbf:=}}}2
		 {\%}{{{\color{NavyBlue}\textbf\%}}}1
		 {[}{{{\color{Sepia}\textbf[}}}1
		 {]}{{{\color{Sepia}\textbf]}}}1,
}

\lstdefinelanguage{Cpp}{%
	language     = C++,
	literate=
}


\lstdefinelanguage{CfgElektra}{
	comment=[l]{;},
	commentstyle=\color{purple}\ttfamily,
	%
	morestring=[b]',
	morestring=[b]`,
	morestring=[b]",
	stringstyle=\color{purple}\ttfamily,
	%
	%
	sensitive=f,% keywords are not case sensitive
	%
	% Colors see https://en.wikibooks.org/wiki/LaTeX/Colors
	%
	keywordstyle=\color{Bittersweet}\bfseries,
	keywordstyle=[2]\color{DarkOrchid}\bfseries,
	keywordstyle=[3]\color{ForestGreen}\bfseries\textit,
	keywordstyle=[4]\color{Goldenrod}\bfseries,
	keywordstyle=[5]\color{CarnationPink},
	%
	keywords={},
	keywords=[2]{},
	keywords=[3]{},
	keywords=[4]{},
	keywords=[5]{},
	%
	literate={=}{{{\color{ForestGreen}\textbf=}}}1
		 %{<-}{{{\color{ForestGreen}\textbf<-}}}2
		 %{*}{{{\color{Bittersweet}\textbf*}}}1
		 {\%}{{{\color{NavyBlue}\textbf\%}}}1,
}




\lstset{language=SpecElektra, % Use SpecElektra as default programming language
	%boxpos=t, % make boxes a bit more unbreakable
	%frame=lines, % top+bottom line
	basicstyle=\ttfamily, % Use normal-size true type font
	showspaces,%
	showstringspaces=false, % Don't put marks in string spaces
	showlines=true, % make sure empty lines at end are shown (needed for concurrency
	tabsize=4, % spaces per tab
	xleftmargin=\parindent, % should be 18pt or 1.5em as defined by memoir
	%Does not really work well (needs to be deactivated for shortlistings):
	breaklines=false,
	%postbreak=\mbox{\textcolor{red}{$\hookrightarrow$}\space},
	%breakautoindent=true,
	%prebreak={\mbox{\ensuremath{\curvearrowright}}} % Zeichen am Zeilenende (Umbruch)
	%breaklines=true,
	%breakautoindent=true,
	%prebreak=\small\symbol{'134}, % backslash
	%prebreak={\mbox{\ensuremath{\curvearrowright}}} % lange kure
	%prebreak={\mbox{\ensuremath{\hookleftarrow}}} % lange kure
	%xleftmargin=3.0ex, %for some formats
	%xrightmargin=1.0ex, %for some formats
	%
	% Files do not work in utf8 see also:
	% http://stackoverflow.com/questions/1116266/listings-in-latex-with-utf-8-or-at-least-german-umlauts
	% http://tex.stackexchange.com/questions/24528/having-problems-with-listings-and-utf-8-can-it-be-fixed
	% Should work but doesn't? (Maybe add to literate broken?)
	%add to literate={ö}{{\"o}}1
	%	{ä}{{\"a}}1
	%	{ü}{{\"u}}1
	%	{Ö}{{\"O}}1
	%	{Ä}{{\"A}}1
	%	{Ü}{{\"U}}1
	%	{ß}{{\ss}}1,
	%
	% listingsutf8 did not work, made umlauts in comments very strange
	%extendedchars=true,
	%inputencoding=utf8,
	%
	%morecomment=[l][\color{blue}]{...}, % Line continuation (...) e.g. blue comment
	morekeywords={for_each},
	numbers=left, % Line numbers on left
	firstnumber=1, % Line numbers start with line 1
	numberstyle=\small\color{blue}, % Line numbers are blue and small
	numbersep=5pt,
	%stepnumber=5 % Line numbers go in steps of 5
}



\lstMakeShortInline[postbreak=,keywordstyle={}]^

\graphicspath{{../pic/}{../figures/}{../graphics/}{../ipe/}{../ggplot/}}



\lstdefinelanguage{dump}
{
	morekeywords={kdbOpen,ksNew,keyNew,keyMeta,keyCopyMeta,keyEnd,ksEnd,kdbClose},
	sensitive=false,
	morecomment=[l]{//},
	morecomment=[s]{/*}{*/},
	morestring=[b]",
}


\lstdefinelanguage{SpecElektra}{
	%
	comment=[l]{;},
	commentstyle=\color{purple}\ttfamily,
	%
	morestring=[b]',
	morestring=[b]`,
	morestring=[b]",
	stringstyle=\color{purple}\ttfamily,
	%
	sensitive=f,% keywords are not case sensitive
	%
	% Colors see https://en.wikibooks.org/wiki/LaTeX/Colors
	%
	keywordstyle=\color{BlueViolet}\bfseries,
	keywordstyle=[2]\color{Green},
	keywordstyle=[3]\color{Aquamarine}\bfseries\textit,
	keywordstyle=[4]\color{NavyBlue}\bfseries,
	keywordstyle=[5]\color{Mahogany},
	%
	keywords={layer, require, validation, check, range, description, rationale, requirement, accessibility, enable, condition, message, default, opt, readonly, type, context, property1, property2, description, file, content, mountpoint, metadata, infos, plugins},
	keywords=[2]{},
	keywords=[3]{order, interface, network, emphasized},
	%keywords=[4]{[, ]},  %Not needed
	keywords=[4]{},
	keywords=[5]{},
	%
	literate={:=}{{{\color{red}\textbf:=}}}2
		 {\%}{{{\color{NavyBlue}\textbf\%}}}1
		 {[}{{{\color{Sepia}\textbf[}}}1
		 {]}{{{\color{Sepia}\textbf]}}}1,
}

\lstdefinelanguage{Cpp}{%
	language     = C++,
	literate=
}


\lstdefinelanguage{CfgElektra}{
	comment=[l]{;},
	commentstyle=\color{purple}\ttfamily,
	%
	morestring=[b]',
	morestring=[b]`,
	morestring=[b]",
	stringstyle=\color{purple}\ttfamily,
	%
	%
	sensitive=f,% keywords are not case sensitive
	%
	% Colors see https://en.wikibooks.org/wiki/LaTeX/Colors
	%
	keywordstyle=\color{Bittersweet}\bfseries,
	keywordstyle=[2]\color{DarkOrchid}\bfseries,
	keywordstyle=[3]\color{ForestGreen}\bfseries\textit,
	keywordstyle=[4]\color{Goldenrod}\bfseries,
	keywordstyle=[5]\color{CarnationPink},
	%
	keywords={},
	keywords=[2]{},
	keywords=[3]{},
	keywords=[4]{},
	keywords=[5]{},
	%
	literate={=}{{{\color{ForestGreen}\textbf=}}}1
		 %{<-}{{{\color{ForestGreen}\textbf<-}}}2
		 %{*}{{{\color{Bittersweet}\textbf*}}}1
		 {\%}{{{\color{NavyBlue}\textbf\%}}}1,
}




\lstset{language=SpecElektra, % Use SpecElektra as default programming language
	%boxpos=t, % make boxes a bit more unbreakable
	%frame=lines, % top+bottom line
	basicstyle=\ttfamily, % Use normal-size true type font
	showspaces,%
	showstringspaces=false, % Don't put marks in string spaces
	showlines=true, % make sure empty lines at end are shown (needed for concurrency
	tabsize=4, % spaces per tab
	xleftmargin=\parindent, % should be 18pt or 1.5em as defined by memoir
	%Does not really work well (needs to be deactivated for shortlistings):
	breaklines=false,
	%postbreak=\mbox{\textcolor{red}{$\hookrightarrow$}\space},
	%breakautoindent=true,
	%prebreak={\mbox{\ensuremath{\curvearrowright}}} % Zeichen am Zeilenende (Umbruch)
	%breaklines=true,
	%breakautoindent=true,
	%prebreak=\small\symbol{'134}, % backslash
	%prebreak={\mbox{\ensuremath{\curvearrowright}}} % lange kure
	%prebreak={\mbox{\ensuremath{\hookleftarrow}}} % lange kure
	%xleftmargin=3.0ex, %for some formats
	%xrightmargin=1.0ex, %for some formats
	%
	% Files do not work in utf8 see also:
	% http://stackoverflow.com/questions/1116266/listings-in-latex-with-utf-8-or-at-least-german-umlauts
	% http://tex.stackexchange.com/questions/24528/having-problems-with-listings-and-utf-8-can-it-be-fixed
	% Should work but doesn't? (Maybe add to literate broken?)
	%add to literate={ö}{{\"o}}1
	%	{ä}{{\"a}}1
	%	{ü}{{\"u}}1
	%	{Ö}{{\"O}}1
	%	{Ä}{{\"A}}1
	%	{Ü}{{\"U}}1
	%	{ß}{{\ss}}1,
	%
	% listingsutf8 did not work, made umlauts in comments very strange
	%extendedchars=true,
	%inputencoding=utf8,
	%
	%morecomment=[l][\color{blue}]{...}, % Line continuation (...) e.g. blue comment
	morekeywords={for_each},
	numbers=left, % Line numbers on left
	firstnumber=1, % Line numbers start with line 1
	numberstyle=\small\color{blue}, % Line numbers are blue and small
	numbersep=5pt,
	%stepnumber=5 % Line numbers go in steps of 5
}



\lstMakeShortInline[postbreak=,keywordstyle={}]^

\graphicspath{{../pic/}{../figures/}{../graphics/}{../ipe/}{../ggplot/}}




\title{L04 Configuration Sources}
\date{14.04.2021}

\begin{document}

%%%%%%%%%%%%%%%%%%%%%%%%%%%%%%%%%%%%%%%%%% 
\section{Configuration Files}

\begin{frame}
	\frametitle{Learning Outcomes}
	Students will be able to
	\begin{itemize}
	\item differentiate between configuration sources
	\item unify configuration sources via specifications
	\item (calculate complexity of configuration settings)
	\end{itemize}
\end{frame}

\begin{frame}[fragile]
	\frametitle{Definition}
	A \intro{configuration file} is a file containing configuration settings.

	\pause
	A Web server configuration file:

	\begin{lstlisting}[gobble=4]
	port=80 ; comment
	address=127.0.0.1\end{lstlisting}

	\only<2-2>{
	\begin{quest}
	What are keys? What are configuration values? What is metadata?
	\end{quest}
	}
	\pause

	The configuration values are ^80^ and ^127.0.0.1^, respectively.
	Other information in the configuration file is metadata for the configuration settings (such as the comment).
\end{frame}

\begin{frame}
	\frametitle{Configuration File Formats}
	\begin{itemize}
	\item CSV (comma-separated values)
	\item semi-structured
	\item programming language
	\item literate
	\end{itemize}
\end{frame}

\begin{frame}
	\frametitle{CSV formats}
	\begin{itemize}
	\item passwd: \formatdate{3}{11}{1971}
	\item passwd and group use : as separator
	\item are difficult to extend (e.g., GECOS)
	\item today mostly used for legacy reasons
	\item are replaced one-by-one (e.g., inetd, crontab)
	\end{itemize}
\end{frame}

\begin{frame}
	\frametitle{Programming Language}
	\begin{description}
	\item[$+$] trivial for developers (source the file)
	\item[$+$] above-overage quality of error message
	\item[$-$] makes automatic change of individual values harder
	\item[$-$] very hard to use for people who do not know the programming language
	\item[$-$] does not separate code and data
	\end{description}
\end{frame}

\begin{frame}
	\frametitle{Trends}
	\begin{itemize}
	\item away from CSV
	\item towards general-purpose serialization formats (INI, JSON)
	\item human-read/writable (YAML, TOML)
	\item programming language as configuration file
	\end{itemize}
\end{frame}

\begin{frame}
	\frametitle{Method}

	What do FLOSS developers say?

	\begin{description}
	\item[\methodQuestion{}] survey with 672 persons visiting, 162 persons completing the survey~\cite{raab2017challenges}
	\item[\methodSource{}] source code analysis of 16 applications, comprising 50 million lines of code~\cite{raab2017challenges}
	\end{description}
\end{frame}

\begin{frame}
	\frametitle{Why are so many formats present?}
	\methodQuestion{} \question{In which way have you used or contributed to the configuration system/library/API in your previously mentioned FLOSS project(s)?}~\cite{raab2017challenges}
	\begin{itemize}
	\item \p{19} persons ($n=251$) have introduced a configuration file format.
	\item \p{29} implemented a configuration file parser.
	\item \p{15} introduced a configuration system/library/API.
	\item \p{34} used external configuration access APIs.
	\end{itemize}
\end{frame}

\begin{frame}
	\frametitle{Multitude of Formats}
	\begin{itemize}
	\item on every system a multitude of (legacy) configuration file formats exist
	\item the number grows fast
	\item thus applications usually have to deal with some legacy formats
	\end{itemize}
	

	\begin{restatable}{requirement}{reqLegacy}
	A configuration library must be able to integrate (legacy) systems and must fully support (legacy) configuration files.%
	\label{req:legacy}
	\end{restatable}
\end{frame}





%%%%%%%%%%%%%%%%%%%%%%%%%%%%%%%%%%%%%%%%%% 
\section{Command-line Arguments}

\begin{frame}
	\frametitle{Is there something else?}
	\begin{itemize}
	\item configuration files are the most researched of all configuration sources~\cite{jin2014configurations}
	\item but it is neither the most used nor most popular~\cite{raab2017challenges}
	\end{itemize}
\end{frame}

\begin{frame}
	\methodQuestion{} \question{Which configuration systems/libraries/APIs have you already used or would like to use in one of your FLOSS project(s)?}
	\begin{itemize}
	\item command-line arguments (\p{92}, $n=222$)
	\item environment variables (\p{79}, $n=218$)
	\item \methodSource{} API \texttt{getenv} is used omnipresently with 2,683 occurrences
	\item configuration files (\p{74}, $n=218$))
	\end{itemize}
\end{frame}


\begin{frame}
	\methodQuestion{} \question{What is your experience with the following configuration systems/libraries/APIs?}
	\begin{itemize}
	\item \texttt{getenv} (\p{10}, $n=198$)
	\item configuration files (\p{6}, $n=190$)
	\item command-line options (\p{4}, $n=210$)
	\item X/Q/GSettings (\p{41}, \p{14}, \p{35})
	\item KConfig (\p{21})
	\item dconf (\p{42})
	\item plist (\p{32})
	\item Windows Registry (\p{69})
	\end{itemize}
\end{frame}

\begin{frame}
	\frametitle{Semantics}
	\begin{itemize}
	\item passed by main for a new process via \\ (\texttt{int argc, char ** argv})
	\item visible from other processes (e.g., via \texttt{ps aux})
	\item could be passed along to subprocesses but hardly done
	\item need to be parsed by process
	\item portability: differences in parsing
	\item cannot be changed from outside (requires restart, no IPC)
	\end{itemize}
\end{frame}



%%%%%%%%%%%%%%%%%%%%%%%%%%%%%%%%%%%%%%%%%% 
\section{Environment Variables}

\begin{frame}
	\frametitle{Usage}
	\begin{enumerate}
	\item bypassing other configuration accesses (\methodQuestion{} \p{45})
	\item locating configuration files
	\item debugging and testing (\methodQuestion{} \p{55}, \methodSource{} 1,152, i.\,e. \p{43})
	\item sharing configuration settings across applications (\methodQuestion{} \p{53}, \methodSource{} 716, i.\,e. \p{47})
	\item for configuration settings unlikely to be changed by a user (\methodQuestion{} \p{20})
	\item \question{even when it is used inside a loop} (\methodQuestion{} \p{2})
	\end{enumerate}
\end{frame}

\begin{frame}
	\frametitle{Semantics}
	\begin{itemize}
	\item are also per-process (\texttt{/proc/self/environ})
	\item are not visible from other processes
	\item are automatically inherited by subprocesses
	\item need to be parsed by process (\texttt{[extern] char **environ}) but API is provided (\texttt{getenv})
	\item cannot be changed from outside (requires restart or an additional IPC mechanism)
	\end{itemize}
\end{frame}

\begin{frame}
	\frametitle{getenv}
	\begin{itemize}
	\item is widely standardized, including SVr4, POSIX.1-2001, 4.3BSD, C89, C99~\cite{man2017getenv},
	\item is supported by many programming languages, and
	\item enforces \texttt{key=value} convention.
	\end{itemize}
\end{frame}

\begin{frame}
	\frametitle{Portability}
	\begin{itemize}
	\item no separators for values defined
	\item case sensitivity problems
	\item often many environment variables for the same purpose: TMP, TEMP, or TMPDIR
	\item sometimes one environment variable for different purposes: PATH
	\end{itemize}
\end{frame}





%%%%%%%%%%%%%%%%%%%%%%%%%%%%%%%%%%%%%%%%%% 
\section{Abstractions}

\begin{frame}
	\frametitle{User View}
	\begin{itemize}
	\item command-line for trying out configuration settings
	\item environment variables for configuration settings within a shell
	\item configuration files for persistent configuration settings
	\end{itemize}
\end{frame}

\begin{frame}
	\frametitle{Abstraction}
	\reqLegacy*

	\vspace{1cm}

	How can we deal with the many formats?
\end{frame}

\begin{frame}
	\frametitle{Key-Value}
	A key-value pair is the simplest generic data structure~\cite{strang2004context}.
	While all these formats above have many differences, all of them represent configuration settings as \intro[key-value pair]{key-value pairs}~\cite{jin2014configurations,rabkin2011static,xu2013blame,lathia2013open}.
\\[1cm]

For configuration as program you need to execute them first.
\end{frame}

\begin{frame}
	\frametitle{KeySet (Recapitulation)}

	The common data structure between plugins:
	\vspace{1cm}

	\includegraphics{keyset}
\end{frame}

\begin{frame}
	\frametitle{Mounting}
	\intro[mounting]{Mounting} integrates a backend into the key database~\cite{raab2008thesis}.
	Hence, \elektra{} allows several backends to deal with configuration files at the same time.
	Each backend is responsible for its own subtree of the key database.

	\includegraphics{mounting}
\end{frame}

\begin{frame}[fragile]
	\frametitle{Elektra}

	\begin{code}[gobble=4]
	[kdb/printversion]
	description = "print version information"
	opt = v
	opt/long = version
	opt/arg = none
	\end{code}

	\begin{itemize}
	\item ^gopts^ puts ^Key^s in the ^proc^ namespace
	\item \url{https://www.libelektra.org/tutorials/command-line-options}
	\end{itemize}

	\verb^kdb -v^ \hspace{0.5cm} \verb^kdb --version^ \hspace{0.5cm} \verb^VERSION=1 kdb^
\end{frame}

\begin{frame}
	How can we deal with the many sources?

	\vspace{1cm}

	\begin{restatable}{requirement}{reqEnvironment}
	A configuration library must support all three popular ways for configuration access:
	configuration files, command-line options, and environment variables.
	\end{restatable}
\end{frame}

\begin{frame}
	\frametitle{Plugins}

	Different backends can use different plugins:
	\begin{description}[labelsep=10cm,align=right]
	\item[\texttt{/sw}] in the INI file config.ini
	\item[\texttt{/sw/libreoffice}] in the XML file libreoffice.xml
	\end{description}

	\includegraphics{mounting}
\end{frame}

\begin{frame}
	\frametitle{Cascading}
	\includegraphics{cascading}
\end{frame}


\begin{frame}
	\frametitle{Conclusion}
	\begin{itemize}
	\item three different configuration sources widely used
	\item all three used for different reasons but often for the same configuration settings
	\item many different configuration file formats
	\item abstractions: key-value, mounting, and cascading
	\end{itemize}
\end{frame}





%%%%%%%%%%%%%%%%%%%%%%%%%%%%%%%%%%%%%%%%%% 
\section{Complexity}

\subsection{Trend}

\begin{frame}
	\frametitle{Trend Firefox}
	\includegraphics[scale=0.7]{firefox}
\end{frame}

\begin{frame}
	\frametitle{Trend Chromium}
	\includegraphics[scale=0.7]{chromium}
\end{frame}

\begin{frame}
	\frametitle{Trend Configuration Files}
	\includegraphics[scale=0.5]{pics/trend.png}
	\citet{xu2015hey}
\end{frame}

\subsection{Calculation}

\begin{frame}
	\frametitle{Types of Complexity}
	\begin{itemize}
	\item complexity in access:
		\begin{itemize}
		\item many different formats
		\item non-uniformity
		\item transformations
		\end{itemize}
	\item configuration settings
		\begin{itemize}
		\item number of settings $s$
		\item number of values $n$
		\item dependences between settings
		\end{itemize}
	\end{itemize}
\end{frame}

\lstDeleteShortInline^
\begin{frame}
	\frametitle{Calculation of Complexity}

	Using enumerative combinatorics:
	\begin{itemize}
	\item number of configurations: $n^s$
	\item for $N$ groups of different $n$ and $s$ (i.e., $n_1 \dots n_N$ with $s_1 \dots s_N$ occurrences):  $$\prod_{i=1}^{N} n_i^{s_i}$$
	\item more difficult to calculate (or unbounded) for dependences, module instantiations, arrays, \dots
	\end{itemize}
\end{frame}

\begin{frame}
	\frametitle{Calculation of Complexity}

	Examples:
	\begin{itemize}
	\item 600 boolean settings in Apache httpd (let us assume $n=2$):
	\pause
	$2^{600} \approx 10^{180}$

	\item 19 integer settings:
	\pause
	${2^{32}}^{19} = 2^{32 \cdot 19} = 2^{609} \approx 10^{183}$

	\item for 20 boolean and 20 enums with 5 possibilities:
	\pause
	$$2^{20}*5^{20} = 10^{20}$$

	\end{itemize}
\end{frame}

\begin{frame}
	\frametitle{Calculation of Complexity (cont.)}

	Examples:
	\begin{itemize}
	\item an array with $1-20$ boolean settings:
	\pause
	$2^{20}$

	\item MySQL has 461 settings, of which 216 are non-simple types~\cite{xu2015hey} \\ (let us assume $n=\{3,20\}$):
	\pause
	$3^{245} * 20^{216} \approx 10^{397}$ \\
	(settings are explained in 5560 pages\footnote{\url{https://downloads.mysql.com/docs/refman-5.7-en.pdf}})
	\end{itemize}
\end{frame}

\begin{frame}
	\frametitle{Calculation of Complexity (cont.)~\cite{jin2014configurations}}

	Examples:
	\begin{itemize}
	\item in Firefox resulting in 846 boolean options and 1,111 options of either integer or string, each with three values

	\pause
	$$2^{846}*3^{1111} \approx 6.46 * 10^{259}$$

	\item LibreOffice
	\pause
	$$2^{4433} * 3^{31889}$$
	\end{itemize}
\end{frame}
\lstMakeShortInline[postbreak=,keywordstyle={},showspaces=no]^
%XXX




%%%%%%%%%%%%%%%%%%%%%%%%%%%%%%%%%%%%%%%%%%
\section{Meeting}

\subsection{Recapitulation}

\begin{frame}
	\begin{task}
	Do you have any questions?
	\end{task}
\end{frame}

\begin{frame}
	\begin{alertblock}{Question}
	How can we share configuration settings?
	\end{alertblock}

	\pause
	\begin{exampleblock}{Answer}
	\begin{itemize}
	\item Implement support directly in application.
	\item Override/fallback links in specification.
	\item Calculate/transform values in specification.
	\end{itemize}
	\end{exampleblock}
\end{frame}

\begin{frame}[fragile]
	\frametitle{Definition Configuration File}

	\pause

	A \intro{configuration file} is a file containing configuration settings.

	A Web server configuration file:

	\begin{lstlisting}[gobble=4]
	port=80 ; comment
	address=127.0.0.1\end{lstlisting}

	\only<2-2>{
	\begin{quest}
	What are keys? What are configuration values? What is metadata?
	\end{quest}
	}
	\pause

	The configuration values are ^80^ and ^127.0.0.1^, respectively.
	Other information in the configuration file is metadata for the configuration settings (such as the comment).
\end{frame}

\begin{frame}
	\frametitle{Configuration File Formats}

	\begin{task}
	What are the trends?
	\only<1-1>{
	How can we deal with the many formats?
	}
	\end{task}

	\pause
	\begin{itemize}
	\item away from CSV
	\item towards general-purpose serialization formats (INI, JSON)
	\item human-read/writable (YAML, TOML)
	\item programming language as configuration file
	\end{itemize}

	\only<2-2>{
	\begin{task}
	How can we deal with the many formats?
	\end{task}
	}

	\begin{itemize}
	\item Key-value
	\item Mounting
	\item Plugins
	\end{itemize}
\end{frame}

\begin{assignment}
	\frametitle{Discussion}
	\begin{task}
	What is your favourite configuration file format?
	\end{task}

	\begin{task}
	Did you implement a configuration file parser and/or invented a new configuration file format?
	\end{task}
\end{assignment}

\begin{assignment}
	\begin{task}
	Break.
	\end{task}
\end{assignment}

\begin{frame}
	\frametitle{Semantics of Command-line Arguments}

	\pause
	\begin{itemize}
	\item passed by main for a new process via \\ (\texttt{int argc, char ** argv})
	\item visible for anyone: \texttt{/proc/self/cmdline}, e.g., via \texttt{ps aux}
	\item could be passed along to subprocesses but hardly done
	\item need to be parsed by process
	\item portability: differences in parsing
	\item cannot be changed from outside (requires restart, no IPC)
	\end{itemize}
\end{frame}

\begin{frame}
	\frametitle{Semantics of Environment Variables}

	\pause
	\begin{itemize}
	\item are per-process
	\item only static view by owner in \texttt{/proc/self/environ}
	\item are not visible from other processes
	\item subprocesses inherit by default
	\item need to be parsed by process (\texttt{[extern] char **environ}) but API is provided (\texttt{getenv})
	\item cannot be changed from outside (requires restart or an additional IPC mechanism)
	\end{itemize}
\end{frame}

\begin{assignment}
	\begin{task}
	What are the differences between mounting and cascading?
	\end{task}
\end{assignment}

\subsection{Assignments}

\begin{frame}
	\frametitle{0.9.5 Release}

	\begin{itemize}[<+-| alert@+>]
	\item upgrade your installation to 0.9.5
	\item read release notes \url{https://www.libelektra.org/news/0.9.5-release}
	\item rebase PRs
	\end{itemize}
\end{frame}

\begin{assignment}
	\frametitle{T1 Extra}

	\begin{task}
	Calculate complexity of your teamwork and add to PR.
	\end{task}

	See scripts/complexity.rb
\end{assignment}


\subsection{Preview}

\begin{frame}
	\frametitle{Feedback}
	\hfill \includegraphics[width=2cm]{pics/feedback.png}
	\vspace{-1cm}
	\begin{itemize}
		\item Slides added
		\item Did TUWEL grading improve?
	\end{itemize}
\end{frame}


\begin{frame}
	\frametitle{Outlook}

	Will be online within this week:
	\begin{itemize}[<+-| alert@+>]
	\item History of Configuration Management
	\item CM Languages
	\item CM Tools
	\end{itemize}
\end{frame}



\appendix

\begin{frame}[allowframebreaks]
	\bibliographystyle{plainnat}
	\bibliography{../shared/elektra.bib}
\end{frame}


\end{document}
