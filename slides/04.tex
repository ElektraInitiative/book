%TODO: add cliffhanger with preview for next time

%variability
%code-generation
%testing

%modularity? + dependences

%TODO: argumentation/motivation for code generation:
% - misconfiguration caused by transformation/defaults/...

% Make nice A4 pages for print:
%\usepackage{pgfpages}
%\pgfpagesuselayout{resize to}[a4paper,border shrink=5mm,landscape]

\beamertemplatenavigationsymbolsempty

\setbeamertemplate{bibliography item}[text]

\usepackage[type={CC},modifier={by-sa},version={4.0}]{doclicense}

\usepackage[utf8]{inputenc}
\usepackage{hyperref}
\usepackage{breakurl}
\usepackage{graphicx}
\usepackage{pgfplots}
\usepackage{pgf}
\usepackage{tikz}
\usetikzlibrary{positioning}
\usetikzlibrary{arrows}
\usetikzlibrary{decorations.markings}
\usetikzlibrary{calc}
\usetikzlibrary{matrix}
\usetikzlibrary{shapes}
\usetikzlibrary{decorations.pathmorphing}
\usetikzlibrary{fit}
\usetikzlibrary{backgrounds}
\usetikzlibrary{plotmarks}
\usepackage{stmaryrd}
\usepackage{listings}
\usepackage{pdflscape}
\usepackage{perpage}
\usepackage{appendixnumberbeamer}

%\usepackage[thmmarks,amsmath,amsthm]{ntheorem} % already included in beamer
\usepackage{thm-restate}

\usepackage[sort&compress,numbers]{natbib}  % to be have \citet, \citeauthor, \citeyear

\MakePerPage{footnote}

\tikzstyle{o}=[r,ppBlue]
\tikzstyle{r}=[thick,rectangle,align=center]
\tikzstyle{t}=[r,ppTrans] %,font=\bfseries]
\tikzstyle{dd}=[densely dashed]
\tikzstyle{n}=[r,ppBlue]
\tikzstyle{p}=[r,ppRed]
\tikzstyle{ppRed}  =[draw=red,  fill=  red!20]
\tikzstyle{ppBlue} =[draw=blue, fill= blue!20]
\tikzstyle{ppGreen}=[draw=green,fill=green!20]
\tikzstyle{ppTrans}=[draw=none, fill=none]

\usetheme{Warsaw}

\useoutertheme[subsection=true]{smoothbars}
%\useoutertheme[subsection=false]{miniframes}

\definecolor{bblue}{HTML}{D7DF01}	% yellow-ish actually, for better black/white printing
\definecolor{rred}{HTML}{C0504D}
\definecolor{ggreen}{HTML}{9BBB59}
\definecolor{ppurple}{HTML}{9F4C7C}
\definecolor{lightgray}{rgb}{0.3,0.3,0.3}
\definecolor{lightergray}{rgb}{0.9,0.9,0.9}
\definecolor{UniBlue}{RGB}{83,121,170}

\DeclareTextFontCommand\textintro{\normalfont\bfseries\itshape} % nice!
\newcommand{\intro}[2][]
{%
	\textintro{#2}%
}
\newcommand{\empha}[2][]
{%
	\emph{#2}%
}

%\theoremstyle{plain}
\newcounter{reqcounter}
\newtheorem{requirement}[reqcounter]{Requirement}

%setbeamercolor{structure}{fg=violet}

\makeatletter
\def\th@task{%
    \normalfont % body font
    \setbeamercolor{block title example}{bg=orange,fg=white}
    \setbeamercolor{block body example}{bg=orange!20,fg=black}
    \def\inserttheoremblockenv{exampleblock}
  }
\makeatother

\theoremstyle{task}
\newtheorem{task}{Task}

\newenvironment{assignment}%
{%\setbeamercolor{background canvas}{bg=violet}%
%\setbeamercolor{structure}{fg=cyan!90!black}%
 \setbeamercolor{frametitle}{bg=orange,fg=white}
\begin{frame}}%
{\end{frame}}%

\AtBeginSection[]{
  \begin{frame}
  \vfill
  \centering
  \begin{beamercolorbox}[sep=8pt,center,shadow=true,rounded=true]{title}
    \usebeamerfont{title}\insertsectionhead\par%
  \end{beamercolorbox}
  \tableofcontents
  \vfill
  \end{frame}
}




\pgfplotsset{compat=1.14}
\author{Markus Raab}


\date{13.4.2018}

\begin{document}

\renewcommand{\enquote}[1]{\emph{``#1''}} % Cannot be done earlier

%%%%%%%%%%%%%%%%%%%%%%%%%%%%%%%
\begin{frame}
	\titlepage
	\doclicenseThis
\end{frame}

\begin{frame}
	\frametitle{Organization}
	Next dates:
	\begin{description}
		\item[13.4.2018:] \textbf{homework submitted, topics of team exercise}
		\item[27.4.2018:] lecture
		\item[4.5.2018:] lecture
		\item[18.5.2018:] guest lecture
		\item[25.5.2018:] team exercise submitted
		\item[1.6.2018:] lecture
		\item[8.6.2018:] lecture
		\item[15.6.2018:] last corrections of team exercise
		\item[22.6.2018:] test
	\end{description}
\end{frame}


\begin{frame}
	\frametitle{Popular Topics}
	\vspace{-0.5cm}
	\begin{multicols}{2}
	\begin{description}
	\item[4] validation
	\item[4] user interface
	\item[3] tools (benefits?)
	\color{red}
	\item[3] testability
	\color{gray}
	\item[3] complexity reduction (when conf. needed?)
	\item[3] architectural decisions
	\color{black}
	\item[2] Puppet
	\item[2] modularity
	\color{gray}
	\item[2] environment variables
	\color{black}
	\item[2] documentation
	\color{red}
	\item[2] configuration specification
	\color{gray}
	\item[2] command-line args\color{black}
	\color{red}
	\item[2] code generation
	\item[1] variability
	\color{black}
	\item[1] self-description
	\item[1] round-tripping
	\item[1] introspection
	\item[1] early
	\item[1] dependences
	\item[1] context-awareness
	\item[1] auto-detection
	\item[1] administrators
	\end{description}
	\end{multicols}
\end{frame}

\begin{frame}
	\hspace*{-1cm}\includegraphics[width=\paperwidth]{dot/topics}
\end{frame}

\begin{frame}
	\frametitle{Configuration Access (Recapitulation)}
	\pause

	\ExecuteMetaData[../book/background.tex]{definition-configuration-access}

	\ExecuteMetaData[../book/background.tex]{definition-configuration-access-points}
\end{frame}

\begin{frame}[fragile]
	\frametitle{Trend (Recapitulation)}

	\begin{itemize}[<+-| alert@+>]
	\item alarming trend in number and complexity of configuration settings
	\item sharing, visibility and default value calculation often helps
	\item needs abstraction: configuration specification
	\item but also more courageous decisions and periodical reevaluation
	\item different ways to reduce configuration space
	\end{itemize}
\end{frame}

\begin{frame}
	\frametitle{SpecElektra (Recapitulation)}
	\pause

	\fontsize{18}{0}\selectfont
	\ExecuteMetaData[../book/approach.tex]{definition-spec}
\end{frame}


\begin{frame}
	\frametitle{Modularity (Recapitulation)}
	\pause
	\Large
	\ExecuteMetaData[../book/backend.tex]{definition-modularity}
\end{frame}

\begin{frame}
	\frametitle{Plugins (Recapitulation)}
	\pause
	\Large
	\ExecuteMetaData[../book/approach.tex]{definition-plugins}
\end{frame}

\begin{frame}
	\frametitle{Introspection (Recapitulation)}
	\pause
	\begin{itemize}
	\item unified get/set access to (meta*)-key/values
	\item access via applications, CLI, GUI, web-UI, ...
	\item access via any programming language (similar to file systems)
	\item GUI, web-UI can semantically interpret metadata
	\end{itemize}
\end{frame}


%%%%%%%%%%%%%%%%%%%%%%%%%%%%%%%%%%%%%%%%%% 
\section{Code Generation}

\subsection{Why?}

\begin{assignment}
	\begin{task}
	How to ensure that configuration access points match with present configuration settings?
	\end{task}
\end{assignment}

\begin{frame}
	\frametitle{Rationale (Partly Recapitulation)}
	Configuration Specification:
	\begin{itemize}
	\item without specification you and others do not even know which settings are available
	\item needed for any further techniques we will discuss:
		\begin{itemize}
		\color{red}
		\item code generation guarantees that configuration access points match with specification
		\item validation guarantees that configuration settings match with specification
		\end{itemize}
	\item essential for \intro[no-futz computing]{no-futz computing}~\citet{holland2001nofutz}
	\item the foundation for any advanced tooling like configuration management tools
	\item needed as communication of producers and consumers of configuration
	\end{itemize}
\end{frame}

\begin{assignment}
	\begin{task}
	Brainstorming: Which artefacts can we produce with code generation?
	\end{task}
\end{assignment}

\begin{frame}
	\frametitle{Implementations}
	For example:
	\begin{itemize}
	\item generate examples/documentation
	\item auto-completion/syntax highlighting/IDE support
	\item tooling (GUI, Web UI)
	\item validation code
	\end{itemize}
	\pause
	Are all of them useful?
\end{frame}

\begin{frame}[fragile]
	\frametitle{Current Challenges}
	Configuration Access Code usually has:
	\begin{itemize}
	\item code duplications
	\item hard-coded default values
	\item unexpected transformations
	\item no introspection facilities
	\end{itemize}
	\begin{example}
	\begin{code}[gobble=4,language=C++]
	if (!strcasecmp(token, "on")) {
		*var = 1;
	} else {
		*var = 0;
	} /* src/cache_cf.cc from Squid */
	\end{code}\end{example}
\end{frame}

\begin{frame}
	\frametitle{Goal}

	\begin{goal}
	Configuration settings should adhere the specification from source to destination.
	\end{goal}

	\begin{restatable}{requirement}{reqGeneration}
	The specification must enable code generation and inconsistencies must be ruled out during compilation.
	\end{restatable}
\end{frame}


\subsection{How?}

\begin{frame}
	\frametitle{Code Generation}

	\ExecuteMetaData[../book/approach.tex]{code-generation}
\end{frame}

\begin{frame}
	\frametitle{Possible Properties}
	For example, SpecElektra has following properties:
	\begin{description}
	\item[type] represents the type to be used in the emitted source code.
	\item[opt] is used for short command-line options to be copied to the namespace \namespace{proc}.
	\item[opt/long] is used for long command-line options, which differ from short command-line options by supporting strings and not only characters.
	\item[readonly] yields compilation errors when developers assign a value to a contextual value within the program.
	\item[default] enables us to start the application even if the backend does not work.
	\end{description}
\end{frame}



%%%%%%%%%%%%%%%%%%%%%%%%%%%%%%%%%%%%%%%%%% 
\section{Introspection vs. Generation}

\begin{frame}
	\begin{alertblock}{Question}
	Variability vs. Configurability
	\end{alertblock}
\end{frame}

\begin{frame}
	\begin{alertblock}{Question}
	Introspection vs. Generation
	\end{alertblock}
\end{frame}

\begin{frame}
	Example for Artifacts:
	\begin{itemize}
	\item configuration management code
	\end{itemize}
\end{frame}

\begin{frame}
	Limitations of Introspection:
	\begin{itemize}
	\item no whole-program optimizations (API barriers)
	\item no static checks
	\end{itemize}
\end{frame}


%%%%%%%%%%%%%%%%%%%%%%%%%%%%%%%%%%%%%%%%%% 
\section{Testability}


\begin{frame}
	\frametitle{Conclusion}
	\begin{itemize}
	\item definitions:
	\item code generation vs. introspection
	\item testability: reduction and automatic testing
	\end{itemize}
\end{frame}

\begin{frame}
	\frametitle{Preview}
	\begin{itemize}
	\item modularity
	\item dependences
	\item auto-detection
	\end{itemize}
\end{frame}



%%%%%%%%%%%%%%%%%%%%%%%%%%%%%%%%%%%%%%%%%% 
\nocite{raab2017introducing}

\appendix

\begin{frame}[allowframebreaks]
	\bibliographystyle{plainnat}
	\bibliography{../shared/elektra.bib}
\end{frame}

\end{document}


