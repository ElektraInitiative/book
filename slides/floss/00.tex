%\ifdefined\handout
%\documentclass[handout,aspectratio=1610,xcolor={usenames,dvipsnames,table}]{beamer}
%\else
\documentclass[aspectratio=1610,xcolor={usenames,dvipsnames,table}]{beamer}
%\fi

\newcommand{\mylecture}{Configuration Management}

%\ifdefined\handout
%\documentclass[handout,aspectratio=1610,xcolor={usenames,dvipsnames,table}]{beamer}
%\else
\documentclass[aspectratio=1610,xcolor={usenames,dvipsnames,table}]{beamer}
%\fi

\newcommand{\mylecture}{Configuration Management}

%\ifdefined\handout
%\documentclass[handout,aspectratio=1610,xcolor={usenames,dvipsnames,table}]{beamer}
%\else
\documentclass[aspectratio=1610,xcolor={usenames,dvipsnames,table}]{beamer}
%\fi

\newcommand{\mylecture}{Configuration Management}

\input{../setup}
\input{../shared/setup}

\lstdefinelanguage{dump}
{
	morekeywords={kdbOpen,ksNew,keyNew,keyMeta,keyCopyMeta,keyEnd,ksEnd,kdbClose},
	sensitive=false,
	morecomment=[l]{//},
	morecomment=[s]{/*}{*/},
	morestring=[b]",
}


\lstdefinelanguage{SpecElektra}{
	%
	comment=[l]{;},
	commentstyle=\color{purple}\ttfamily,
	%
	morestring=[b]',
	morestring=[b]`,
	morestring=[b]",
	stringstyle=\color{purple}\ttfamily,
	%
	sensitive=f,% keywords are not case sensitive
	%
	% Colors see https://en.wikibooks.org/wiki/LaTeX/Colors
	%
	keywordstyle=\color{BlueViolet}\bfseries,
	keywordstyle=[2]\color{Green},
	keywordstyle=[3]\color{Aquamarine}\bfseries\textit,
	keywordstyle=[4]\color{NavyBlue}\bfseries,
	keywordstyle=[5]\color{Mahogany},
	%
	keywords={layer, require, validation, check, range, description, rationale, requirement, accessibility, enable, condition, message, default, opt, readonly, type, context, property1, property2, description, file, content, mountpoint, metadata, infos, plugins},
	keywords=[2]{},
	keywords=[3]{order, interface, network, emphasized},
	%keywords=[4]{[, ]},  %Not needed
	keywords=[4]{},
	keywords=[5]{},
	%
	literate={:=}{{{\color{red}\textbf:=}}}2
		 {\%}{{{\color{NavyBlue}\textbf\%}}}1
		 {[}{{{\color{Sepia}\textbf[}}}1
		 {]}{{{\color{Sepia}\textbf]}}}1,
}

\lstdefinelanguage{Cpp}{%
	language     = C++,
	literate=
}


\lstdefinelanguage{CfgElektra}{
	comment=[l]{;},
	commentstyle=\color{purple}\ttfamily,
	%
	morestring=[b]',
	morestring=[b]`,
	morestring=[b]",
	stringstyle=\color{purple}\ttfamily,
	%
	%
	sensitive=f,% keywords are not case sensitive
	%
	% Colors see https://en.wikibooks.org/wiki/LaTeX/Colors
	%
	keywordstyle=\color{Bittersweet}\bfseries,
	keywordstyle=[2]\color{DarkOrchid}\bfseries,
	keywordstyle=[3]\color{ForestGreen}\bfseries\textit,
	keywordstyle=[4]\color{Goldenrod}\bfseries,
	keywordstyle=[5]\color{CarnationPink},
	%
	keywords={},
	keywords=[2]{},
	keywords=[3]{},
	keywords=[4]{},
	keywords=[5]{},
	%
	literate={=}{{{\color{ForestGreen}\textbf=}}}1
		 %{<-}{{{\color{ForestGreen}\textbf<-}}}2
		 %{*}{{{\color{Bittersweet}\textbf*}}}1
		 {\%}{{{\color{NavyBlue}\textbf\%}}}1,
}




\lstset{language=SpecElektra, % Use SpecElektra as default programming language
	%boxpos=t, % make boxes a bit more unbreakable
	%frame=lines, % top+bottom line
	basicstyle=\ttfamily, % Use normal-size true type font
	showspaces,%
	showstringspaces=false, % Don't put marks in string spaces
	showlines=true, % make sure empty lines at end are shown (needed for concurrency
	tabsize=4, % spaces per tab
	xleftmargin=\parindent, % should be 18pt or 1.5em as defined by memoir
	%Does not really work well (needs to be deactivated for shortlistings):
	breaklines=false,
	%postbreak=\mbox{\textcolor{red}{$\hookrightarrow$}\space},
	%breakautoindent=true,
	%prebreak={\mbox{\ensuremath{\curvearrowright}}} % Zeichen am Zeilenende (Umbruch)
	%breaklines=true,
	%breakautoindent=true,
	%prebreak=\small\symbol{'134}, % backslash
	%prebreak={\mbox{\ensuremath{\curvearrowright}}} % lange kure
	%prebreak={\mbox{\ensuremath{\hookleftarrow}}} % lange kure
	%xleftmargin=3.0ex, %for some formats
	%xrightmargin=1.0ex, %for some formats
	%
	% Files do not work in utf8 see also:
	% http://stackoverflow.com/questions/1116266/listings-in-latex-with-utf-8-or-at-least-german-umlauts
	% http://tex.stackexchange.com/questions/24528/having-problems-with-listings-and-utf-8-can-it-be-fixed
	% Should work but doesn't? (Maybe add to literate broken?)
	%add to literate={ö}{{\"o}}1
	%	{ä}{{\"a}}1
	%	{ü}{{\"u}}1
	%	{Ö}{{\"O}}1
	%	{Ä}{{\"A}}1
	%	{Ü}{{\"U}}1
	%	{ß}{{\ss}}1,
	%
	% listingsutf8 did not work, made umlauts in comments very strange
	%extendedchars=true,
	%inputencoding=utf8,
	%
	%morecomment=[l][\color{blue}]{...}, % Line continuation (...) e.g. blue comment
	morekeywords={for_each},
	numbers=left, % Line numbers on left
	firstnumber=1, % Line numbers start with line 1
	numberstyle=\small\color{blue}, % Line numbers are blue and small
	numbersep=5pt,
	%stepnumber=5 % Line numbers go in steps of 5
}



\lstMakeShortInline[postbreak=,keywordstyle={}]^

\graphicspath{{../pic/}{../figures/}{../graphics/}{../ipe/}{../ggplot/}}



\lstdefinelanguage{dump}
{
	morekeywords={kdbOpen,ksNew,keyNew,keyMeta,keyCopyMeta,keyEnd,ksEnd,kdbClose},
	sensitive=false,
	morecomment=[l]{//},
	morecomment=[s]{/*}{*/},
	morestring=[b]",
}


\lstdefinelanguage{SpecElektra}{
	%
	comment=[l]{;},
	commentstyle=\color{purple}\ttfamily,
	%
	morestring=[b]',
	morestring=[b]`,
	morestring=[b]",
	stringstyle=\color{purple}\ttfamily,
	%
	sensitive=f,% keywords are not case sensitive
	%
	% Colors see https://en.wikibooks.org/wiki/LaTeX/Colors
	%
	keywordstyle=\color{BlueViolet}\bfseries,
	keywordstyle=[2]\color{Green},
	keywordstyle=[3]\color{Aquamarine}\bfseries\textit,
	keywordstyle=[4]\color{NavyBlue}\bfseries,
	keywordstyle=[5]\color{Mahogany},
	%
	keywords={layer, require, validation, check, range, description, rationale, requirement, accessibility, enable, condition, message, default, opt, readonly, type, context, property1, property2, description, file, content, mountpoint, metadata, infos, plugins},
	keywords=[2]{},
	keywords=[3]{order, interface, network, emphasized},
	%keywords=[4]{[, ]},  %Not needed
	keywords=[4]{},
	keywords=[5]{},
	%
	literate={:=}{{{\color{red}\textbf:=}}}2
		 {\%}{{{\color{NavyBlue}\textbf\%}}}1
		 {[}{{{\color{Sepia}\textbf[}}}1
		 {]}{{{\color{Sepia}\textbf]}}}1,
}

\lstdefinelanguage{Cpp}{%
	language     = C++,
	literate=
}


\lstdefinelanguage{CfgElektra}{
	comment=[l]{;},
	commentstyle=\color{purple}\ttfamily,
	%
	morestring=[b]',
	morestring=[b]`,
	morestring=[b]",
	stringstyle=\color{purple}\ttfamily,
	%
	%
	sensitive=f,% keywords are not case sensitive
	%
	% Colors see https://en.wikibooks.org/wiki/LaTeX/Colors
	%
	keywordstyle=\color{Bittersweet}\bfseries,
	keywordstyle=[2]\color{DarkOrchid}\bfseries,
	keywordstyle=[3]\color{ForestGreen}\bfseries\textit,
	keywordstyle=[4]\color{Goldenrod}\bfseries,
	keywordstyle=[5]\color{CarnationPink},
	%
	keywords={},
	keywords=[2]{},
	keywords=[3]{},
	keywords=[4]{},
	keywords=[5]{},
	%
	literate={=}{{{\color{ForestGreen}\textbf=}}}1
		 %{<-}{{{\color{ForestGreen}\textbf<-}}}2
		 %{*}{{{\color{Bittersweet}\textbf*}}}1
		 {\%}{{{\color{NavyBlue}\textbf\%}}}1,
}




\lstset{language=SpecElektra, % Use SpecElektra as default programming language
	%boxpos=t, % make boxes a bit more unbreakable
	%frame=lines, % top+bottom line
	basicstyle=\ttfamily, % Use normal-size true type font
	showspaces,%
	showstringspaces=false, % Don't put marks in string spaces
	showlines=true, % make sure empty lines at end are shown (needed for concurrency
	tabsize=4, % spaces per tab
	xleftmargin=\parindent, % should be 18pt or 1.5em as defined by memoir
	%Does not really work well (needs to be deactivated for shortlistings):
	breaklines=false,
	%postbreak=\mbox{\textcolor{red}{$\hookrightarrow$}\space},
	%breakautoindent=true,
	%prebreak={\mbox{\ensuremath{\curvearrowright}}} % Zeichen am Zeilenende (Umbruch)
	%breaklines=true,
	%breakautoindent=true,
	%prebreak=\small\symbol{'134}, % backslash
	%prebreak={\mbox{\ensuremath{\curvearrowright}}} % lange kure
	%prebreak={\mbox{\ensuremath{\hookleftarrow}}} % lange kure
	%xleftmargin=3.0ex, %for some formats
	%xrightmargin=1.0ex, %for some formats
	%
	% Files do not work in utf8 see also:
	% http://stackoverflow.com/questions/1116266/listings-in-latex-with-utf-8-or-at-least-german-umlauts
	% http://tex.stackexchange.com/questions/24528/having-problems-with-listings-and-utf-8-can-it-be-fixed
	% Should work but doesn't? (Maybe add to literate broken?)
	%add to literate={ö}{{\"o}}1
	%	{ä}{{\"a}}1
	%	{ü}{{\"u}}1
	%	{Ö}{{\"O}}1
	%	{Ä}{{\"A}}1
	%	{Ü}{{\"U}}1
	%	{ß}{{\ss}}1,
	%
	% listingsutf8 did not work, made umlauts in comments very strange
	%extendedchars=true,
	%inputencoding=utf8,
	%
	%morecomment=[l][\color{blue}]{...}, % Line continuation (...) e.g. blue comment
	morekeywords={for_each},
	numbers=left, % Line numbers on left
	firstnumber=1, % Line numbers start with line 1
	numberstyle=\small\color{blue}, % Line numbers are blue and small
	numbersep=5pt,
	%stepnumber=5 % Line numbers go in steps of 5
}



\lstMakeShortInline[postbreak=,keywordstyle={}]^

\graphicspath{{../pic/}{../figures/}{../graphics/}{../ipe/}{../ggplot/}}




\title{\mylecture}

\begin{document}

%%%%%%%%%%%%%%%%%%%%%%%%%%%%%%%%%%%%%%%%%% 
\section{Preliminaries}

\begin{frame}
	\frametitle{BigBlueButton}
	\begin{itemize}[<+-| alert@+>]
		\item used for weekly hybrid meetings
		\item is FLOSS
		\item raise the hand immediately on any issues
		\item use ``[on] @GitHubName Real Name'' as your name \\
			([on] for on-site alone, [tw] for on-site with neighbor, [vi] for virtual)
		\item if audio does not work, write in the chat
		\item on technical problems, try another browser, e.g., recent Firefox or Chromium
	\end{itemize}
\end{frame}

\begin{frame}
	\frametitle{Language}
	Materials are in English:
	\begin{itemize}
		\item Slides are in English
		\item Reading texts are in English
		\item Videos are in English
	\end{itemize}
\end{frame}

\begin{assignment}
	\frametitle{Language during the meetings?}
	\begin{task}
	\begin{description}
	\item[A] English
	\item[B] Slightly Prefer English
	\item[C] Both are fine
	\item[D] Slightly Prefer German
	\item[E] German
	\end{description}
	\end{task}
\end{assignment}

\begin{assignment}
	\frametitle{Video}
	I am trying to keep meetings short and with breaks. \\
	You are allowed to:
	\begin{itemize}
		\item stretch
		\item move
		\item eat
		\item look somewhere else
		\item leave your place
	\end{itemize}

	\begin{task}
	If not on-site: please turn video on.
	\end{task}
\end{assignment}

\begin{frame}
	\frametitle{Inverted Classroom}
	Meetings are most Wednesdays 14:00 c.t. - 16:00 (max.)

	\begin{itemize}[<+-| alert@+>]
		\item always read/watch the material in advance
		\item within meetings we will do recapitulations, discussions, etc.
		\item for today it was enough to read TISS
		\item the more you participate, the more you learn
		\item guest meeting on L08 Collaboration
	\end{itemize}
\end{frame}

\begin{frame}
	\frametitle{Programming Languages}
	Elektra supports following programming languages:
	\begin{itemize}
		\item C\footnote{support on problems with the programming language by the lecturer}
		\item C++\footnotemark[1]
		\item Java\footnotemark[1]
		\item Python\footnotemark[1]
		\item Rust
		\item Go
		\item Lua
		\item Ruby
		\item Kotlin
	\end{itemize}

	\begin{quest}[1]
	Which language(s) do you need to know?
	\end{quest}

	\begin{quest}[2]
	Which language can you use?
	\end{quest}
\end{frame}

\begin{assignment}
	\begin{task}
	Break.
	\end{task}
\end{assignment}

\begin{frame}
	\frametitle{Overview Assignments}
	\begin{description}[<+-| alert@+>]
	\item[30\,\%:] homework
	\item[30\,\%:] teamwork
	\item[40\,\%:] project
	\item[0\,\%:] presentation
	\end{description}

	\pause[\thebeamerpauses]

	\begin{quest}
	How to get a positive grade?
	\end{quest}

	\pause

	\begin{itemize}[<+-| alert@+>]
	\item To get a positive grade all parts must be positive.
	\item Extrapoints can be earned in the lecture.
	\item After you did H0, you get a grade.
	\end{itemize}
\end{frame}

\begin{frame}
	\frametitle{Deadlines}

	\begin{itemize}[<+-| alert@+>]
	\item if you make submissions earlier, you get feedback earlier
	\item dates are in ``assignments.pdf'', ``schedule.pdf'', ICS file and calender of TUWEL
	\end{itemize}

	\pause[\thebeamerpauses]
	\vspace{1em}

	There are up to three deadlines for each homework, teamwork or project:

	\begin{itemize}[<+-| alert@+>]
	\item deadline for submission of the work
	\item deadline for review (review the submission of others)
	\item deadline for corrections (based on the feedback of submission)
	\end{itemize}
\end{frame}

%%%%%%%%%%%%%%%%%%%%%%%%%%%%%%%%%%%%%%%%%% 
\section{Motivation}
\subsection{}
\begin{frame}
	\frametitle{FLOSS}
	\floss{} allows you to:
	\vspace{1em}
	\begin{enumerate}[<+-| alert@+>]
		\setcounter{enumi}{-1}
		\item Use
		\item Share
		\item Study
		\item Improve
	\end{enumerate}
	\vspace{1em}
	\pause[\thebeamerpauses]
	the software (binary and source) for any purpose without restrictions.
\end{frame}

\begin{frame}
	\frametitle{Implications}

	There are countless implications\footnote{many of which we will discuss in the course}:
	\begin{enumerate}[<+-| alert@+>]
		\item technology knowledge doesn't becomes irrelevant after changing employee
		\item people give you money so that you improve FLOSS for them
		\item you can do research on FLOSS without any restriction
		\item you can modify FLOSS as you see fit for yourself or your employee
	\end{enumerate}
\end{frame}

\begin{frame}[fragile]
	\frametitle{Sustainable FLOSS}

	\begin{itemize}[<+-| alert@+>]
		\item at university, development during theses
		\item taking FLOSS from job to job (``GitHub as CV'')
		\item improve FLOSS on customer requests
		\item selling of hardware
		\item providing a service
	\end{itemize}
\end{frame}

\begin{assignment}
	\frametitle{First Assignment}
	\begin{itemize}[<+-| alert@+>]
		\item Have you already used FLOSS?
		\item Did you already participate in FLOSS?
		\item Which (other) implications are relevant for you?
	\end{itemize}
	\pause[\thebeamerpauses]

	\begin{task}
	Discuss in breakout room and tell your partner's story.
	\end{task}
\end{assignment}

\begin{assignment}
	\begin{task}
	Break.
	\end{task}
\end{assignment}

%%%%%%%%%%%%%%%%%%%%%%%%%%%%%%%%%%%%%%%%%% 
\section{Elektra}

\begin{frame}
	\frametitle{Elektra}

	\url{https://libelektra.org}

	\hfill \includegraphics[width=2cm]{../figures/logo}

	\vspace{-1cm}
	\begin{itemize}[<+-| alert@+>]
		\item very active: new release was today
		\item object of study in FLOSS
		\item Elektra is mainly developed at TU Wien
	\end{itemize}
\end{frame}

\begin{assignment}
	\begin{task}
	Break.
	\end{task}
\end{assignment}

\begin{frame}
	\frametitle{Use Cases of Elektra}
	\begin{itemize}
	\item Embedded systems
	\uncover<1->{
	\begin{itemize}
	\item Olimex
	\item OpenWRT (distribution)
	\item Broadcom (blue-ray devices)
	\item Kapsch (cameras)
	\item Toshiba (TVs)
	\end{itemize}
	}\uncover<2->{
	\item Server
	\begin{itemize}
	\item ansible-libelektra
	\item Allianz (insurance)
	\item TU Wien
	\item Other Universities
	\end{itemize}
	}\uncover<3->{
	\item Desktop
	\begin{itemize}
	\item KDE, GNOME and XFCE
	\item Oyranos
	\item Redshift
	\item LCDproc
	\end{itemize}
	}
	\end{itemize}
\end{frame}

\begin{assignment}
	\frametitle{Use Cases}
	\begin{task}
	Can you name some other FLOSS for each use case?
	\end{task}
\end{assignment}

%%%%%%%%%%%%%%%%%%%%%%%%%%%%%%%%%%%%%%%%%% 
\section{Content Overview}

\begin{frame}
	\textit{learning outcomes:}
	\begin{itemize}
		\item remember learning outcomes
		\item remember the topics
	\end{itemize}
\end{frame}

\begin{frame}
	\textit{learning outcomes:}

	\begin{quest}
	What are the main learning outcomes as written in TISS?
	\end{quest}

	\pause

	\begin{itemize}
		\item participate in FLOSS initiatives,
		\item found new FLOSS initiatives,
		\item use FLOSS methods in your business context.
	\end{itemize}
\end{frame}


\begin{frame}
	\includegraphics[width=12cm]{Overview}
\end{frame}

\begin{assignment}
	In which FLOSS topics are you interested? \\
	(Can be other topics not mentioned.)

	\begin{task}[1]
	Discuss topics with your partner.
	\end{task}

	\begin{task}[2]
	Write down the most interesting topics in the shared notes.
	\end{task}
\end{assignment}


\section{Outlook}

\begin{frame}
	\frametitle{M01 Issue Tracking}
	TUWEL already contains materials for M01
	\begin{task}
	\begin{itemize}
		\item read assignments.pdf for H0
		\item reading text, videos
		\item register for the course by doing H0
	\end{itemize}
	\end{task}

	\pause[\thebeamerpauses]

	\begin{task}
	Any questions?
	\end{task}
\end{frame}




%%%%%%%%%%%%%%%%%%%%%%%%%%%%%%%%%%%%%%%%%% 
\nocite{raab2017introducing}

\appendix

\begin{frame}[allowframebreaks]
	\bibliographystyle{plainnat}
	\bibliography{../shared/elektra.bib}
\end{frame}

\end{document}


