%\ifdefined\handout
%\documentclass[handout,aspectratio=1610,xcolor={usenames,dvipsnames,table}]{beamer}
%\else
\documentclass[aspectratio=1610,xcolor={usenames,dvipsnames,table}]{beamer}
%\fi

\newcommand{\mylecture}{Configuration Management}

%\ifdefined\handout
%\documentclass[handout,aspectratio=1610,xcolor={usenames,dvipsnames,table}]{beamer}
%\else
\documentclass[aspectratio=1610,xcolor={usenames,dvipsnames,table}]{beamer}
%\fi

\newcommand{\mylecture}{Configuration Management}

%\ifdefined\handout
%\documentclass[handout,aspectratio=1610,xcolor={usenames,dvipsnames,table}]{beamer}
%\else
\documentclass[aspectratio=1610,xcolor={usenames,dvipsnames,table}]{beamer}
%\fi

\newcommand{\mylecture}{Configuration Management}

\input{../setup}
\input{../shared/setup}

\lstdefinelanguage{dump}
{
	morekeywords={kdbOpen,ksNew,keyNew,keyMeta,keyCopyMeta,keyEnd,ksEnd,kdbClose},
	sensitive=false,
	morecomment=[l]{//},
	morecomment=[s]{/*}{*/},
	morestring=[b]",
}


\lstdefinelanguage{SpecElektra}{
	%
	comment=[l]{;},
	commentstyle=\color{purple}\ttfamily,
	%
	morestring=[b]',
	morestring=[b]`,
	morestring=[b]",
	stringstyle=\color{purple}\ttfamily,
	%
	sensitive=f,% keywords are not case sensitive
	%
	% Colors see https://en.wikibooks.org/wiki/LaTeX/Colors
	%
	keywordstyle=\color{BlueViolet}\bfseries,
	keywordstyle=[2]\color{Green},
	keywordstyle=[3]\color{Aquamarine}\bfseries\textit,
	keywordstyle=[4]\color{NavyBlue}\bfseries,
	keywordstyle=[5]\color{Mahogany},
	%
	keywords={layer, require, validation, check, range, description, rationale, requirement, accessibility, enable, condition, message, default, opt, readonly, type, context, property1, property2, description, file, content, mountpoint, metadata, infos, plugins},
	keywords=[2]{},
	keywords=[3]{order, interface, network, emphasized},
	%keywords=[4]{[, ]},  %Not needed
	keywords=[4]{},
	keywords=[5]{},
	%
	literate={:=}{{{\color{red}\textbf:=}}}2
		 {\%}{{{\color{NavyBlue}\textbf\%}}}1
		 {[}{{{\color{Sepia}\textbf[}}}1
		 {]}{{{\color{Sepia}\textbf]}}}1,
}

\lstdefinelanguage{Cpp}{%
	language     = C++,
	literate=
}


\lstdefinelanguage{CfgElektra}{
	comment=[l]{;},
	commentstyle=\color{purple}\ttfamily,
	%
	morestring=[b]',
	morestring=[b]`,
	morestring=[b]",
	stringstyle=\color{purple}\ttfamily,
	%
	%
	sensitive=f,% keywords are not case sensitive
	%
	% Colors see https://en.wikibooks.org/wiki/LaTeX/Colors
	%
	keywordstyle=\color{Bittersweet}\bfseries,
	keywordstyle=[2]\color{DarkOrchid}\bfseries,
	keywordstyle=[3]\color{ForestGreen}\bfseries\textit,
	keywordstyle=[4]\color{Goldenrod}\bfseries,
	keywordstyle=[5]\color{CarnationPink},
	%
	keywords={},
	keywords=[2]{},
	keywords=[3]{},
	keywords=[4]{},
	keywords=[5]{},
	%
	literate={=}{{{\color{ForestGreen}\textbf=}}}1
		 %{<-}{{{\color{ForestGreen}\textbf<-}}}2
		 %{*}{{{\color{Bittersweet}\textbf*}}}1
		 {\%}{{{\color{NavyBlue}\textbf\%}}}1,
}




\lstset{language=SpecElektra, % Use SpecElektra as default programming language
	%boxpos=t, % make boxes a bit more unbreakable
	%frame=lines, % top+bottom line
	basicstyle=\ttfamily, % Use normal-size true type font
	showspaces,%
	showstringspaces=false, % Don't put marks in string spaces
	showlines=true, % make sure empty lines at end are shown (needed for concurrency
	tabsize=4, % spaces per tab
	xleftmargin=\parindent, % should be 18pt or 1.5em as defined by memoir
	%Does not really work well (needs to be deactivated for shortlistings):
	breaklines=false,
	%postbreak=\mbox{\textcolor{red}{$\hookrightarrow$}\space},
	%breakautoindent=true,
	%prebreak={\mbox{\ensuremath{\curvearrowright}}} % Zeichen am Zeilenende (Umbruch)
	%breaklines=true,
	%breakautoindent=true,
	%prebreak=\small\symbol{'134}, % backslash
	%prebreak={\mbox{\ensuremath{\curvearrowright}}} % lange kure
	%prebreak={\mbox{\ensuremath{\hookleftarrow}}} % lange kure
	%xleftmargin=3.0ex, %for some formats
	%xrightmargin=1.0ex, %for some formats
	%
	% Files do not work in utf8 see also:
	% http://stackoverflow.com/questions/1116266/listings-in-latex-with-utf-8-or-at-least-german-umlauts
	% http://tex.stackexchange.com/questions/24528/having-problems-with-listings-and-utf-8-can-it-be-fixed
	% Should work but doesn't? (Maybe add to literate broken?)
	%add to literate={ö}{{\"o}}1
	%	{ä}{{\"a}}1
	%	{ü}{{\"u}}1
	%	{Ö}{{\"O}}1
	%	{Ä}{{\"A}}1
	%	{Ü}{{\"U}}1
	%	{ß}{{\ss}}1,
	%
	% listingsutf8 did not work, made umlauts in comments very strange
	%extendedchars=true,
	%inputencoding=utf8,
	%
	%morecomment=[l][\color{blue}]{...}, % Line continuation (...) e.g. blue comment
	morekeywords={for_each},
	numbers=left, % Line numbers on left
	firstnumber=1, % Line numbers start with line 1
	numberstyle=\small\color{blue}, % Line numbers are blue and small
	numbersep=5pt,
	%stepnumber=5 % Line numbers go in steps of 5
}



\lstMakeShortInline[postbreak=,keywordstyle={}]^

\graphicspath{{../pic/}{../figures/}{../graphics/}{../ipe/}{../ggplot/}}



\lstdefinelanguage{dump}
{
	morekeywords={kdbOpen,ksNew,keyNew,keyMeta,keyCopyMeta,keyEnd,ksEnd,kdbClose},
	sensitive=false,
	morecomment=[l]{//},
	morecomment=[s]{/*}{*/},
	morestring=[b]",
}


\lstdefinelanguage{SpecElektra}{
	%
	comment=[l]{;},
	commentstyle=\color{purple}\ttfamily,
	%
	morestring=[b]',
	morestring=[b]`,
	morestring=[b]",
	stringstyle=\color{purple}\ttfamily,
	%
	sensitive=f,% keywords are not case sensitive
	%
	% Colors see https://en.wikibooks.org/wiki/LaTeX/Colors
	%
	keywordstyle=\color{BlueViolet}\bfseries,
	keywordstyle=[2]\color{Green},
	keywordstyle=[3]\color{Aquamarine}\bfseries\textit,
	keywordstyle=[4]\color{NavyBlue}\bfseries,
	keywordstyle=[5]\color{Mahogany},
	%
	keywords={layer, require, validation, check, range, description, rationale, requirement, accessibility, enable, condition, message, default, opt, readonly, type, context, property1, property2, description, file, content, mountpoint, metadata, infos, plugins},
	keywords=[2]{},
	keywords=[3]{order, interface, network, emphasized},
	%keywords=[4]{[, ]},  %Not needed
	keywords=[4]{},
	keywords=[5]{},
	%
	literate={:=}{{{\color{red}\textbf:=}}}2
		 {\%}{{{\color{NavyBlue}\textbf\%}}}1
		 {[}{{{\color{Sepia}\textbf[}}}1
		 {]}{{{\color{Sepia}\textbf]}}}1,
}

\lstdefinelanguage{Cpp}{%
	language     = C++,
	literate=
}


\lstdefinelanguage{CfgElektra}{
	comment=[l]{;},
	commentstyle=\color{purple}\ttfamily,
	%
	morestring=[b]',
	morestring=[b]`,
	morestring=[b]",
	stringstyle=\color{purple}\ttfamily,
	%
	%
	sensitive=f,% keywords are not case sensitive
	%
	% Colors see https://en.wikibooks.org/wiki/LaTeX/Colors
	%
	keywordstyle=\color{Bittersweet}\bfseries,
	keywordstyle=[2]\color{DarkOrchid}\bfseries,
	keywordstyle=[3]\color{ForestGreen}\bfseries\textit,
	keywordstyle=[4]\color{Goldenrod}\bfseries,
	keywordstyle=[5]\color{CarnationPink},
	%
	keywords={},
	keywords=[2]{},
	keywords=[3]{},
	keywords=[4]{},
	keywords=[5]{},
	%
	literate={=}{{{\color{ForestGreen}\textbf=}}}1
		 %{<-}{{{\color{ForestGreen}\textbf<-}}}2
		 %{*}{{{\color{Bittersweet}\textbf*}}}1
		 {\%}{{{\color{NavyBlue}\textbf\%}}}1,
}




\lstset{language=SpecElektra, % Use SpecElektra as default programming language
	%boxpos=t, % make boxes a bit more unbreakable
	%frame=lines, % top+bottom line
	basicstyle=\ttfamily, % Use normal-size true type font
	showspaces,%
	showstringspaces=false, % Don't put marks in string spaces
	showlines=true, % make sure empty lines at end are shown (needed for concurrency
	tabsize=4, % spaces per tab
	xleftmargin=\parindent, % should be 18pt or 1.5em as defined by memoir
	%Does not really work well (needs to be deactivated for shortlistings):
	breaklines=false,
	%postbreak=\mbox{\textcolor{red}{$\hookrightarrow$}\space},
	%breakautoindent=true,
	%prebreak={\mbox{\ensuremath{\curvearrowright}}} % Zeichen am Zeilenende (Umbruch)
	%breaklines=true,
	%breakautoindent=true,
	%prebreak=\small\symbol{'134}, % backslash
	%prebreak={\mbox{\ensuremath{\curvearrowright}}} % lange kure
	%prebreak={\mbox{\ensuremath{\hookleftarrow}}} % lange kure
	%xleftmargin=3.0ex, %for some formats
	%xrightmargin=1.0ex, %for some formats
	%
	% Files do not work in utf8 see also:
	% http://stackoverflow.com/questions/1116266/listings-in-latex-with-utf-8-or-at-least-german-umlauts
	% http://tex.stackexchange.com/questions/24528/having-problems-with-listings-and-utf-8-can-it-be-fixed
	% Should work but doesn't? (Maybe add to literate broken?)
	%add to literate={ö}{{\"o}}1
	%	{ä}{{\"a}}1
	%	{ü}{{\"u}}1
	%	{Ö}{{\"O}}1
	%	{Ä}{{\"A}}1
	%	{Ü}{{\"U}}1
	%	{ß}{{\ss}}1,
	%
	% listingsutf8 did not work, made umlauts in comments very strange
	%extendedchars=true,
	%inputencoding=utf8,
	%
	%morecomment=[l][\color{blue}]{...}, % Line continuation (...) e.g. blue comment
	morekeywords={for_each},
	numbers=left, % Line numbers on left
	firstnumber=1, % Line numbers start with line 1
	numberstyle=\small\color{blue}, % Line numbers are blue and small
	numbersep=5pt,
	%stepnumber=5 % Line numbers go in steps of 5
}



\lstMakeShortInline[postbreak=,keywordstyle={}]^

\graphicspath{{../pic/}{../figures/}{../graphics/}{../ipe/}{../ggplot/}}




\title{L00 Preliminary Talk}
\date{6.10.2021}

\begin{document}

%%%%%%%%%%%%%%%%%%%%%%%%%%%%%%%%%%%%%%%%%% 
\section{Preliminaries}

\begin{frame}
	\frametitle{BigBlueButton}
	\begin{itemize}
		\item used for weekly virtual meetings
		\item is FLOSS
		\item set status (e.g. raise the hand) immediately on any issues
		\item use ``Real Name @GitHubName'' as your name
		\item on technical problems use the chat
		\item you can connect several times (e.g. phone+laptop)
		\item on issues, try another browser (recent Firefox+Chromium)
	\end{itemize}
\end{frame}

\begin{frame}
	\frametitle{Language}
	Materials are in English:
	\begin{itemize}
		\item Slides are in English
		\item Papers are in English
		\item Videos are in English
	\end{itemize}
\end{frame}

\begin{assignment}
	\frametitle{Language of the Talk?}
	\begin{task}
	\begin{description}
	\item[A] English
	\item[B] German
	\item[C] Don't care
	\end{description}
	\end{task}
\end{assignment}

\begin{assignment}
	\frametitle{Video}
	I am trying to keep meetings short with many breaks. \\
	You are allowed to:
	\begin{itemize}
		\item stretch
		\item move
		\item eat
		\item look somewhere else
		\item leave your place
	\end{itemize}
	\begin{task}
	But please turn video on.
	\end{task}
\end{assignment}

%%%%%%%%%%%%%%%%%%%%%%%%%%%%%%%%%%%%%%%%%% 
\section{Motivation}
\subsection{}
\begin{frame}
	\frametitle{FLOSS}
	\floss{} allows you to:
	\begin{enumerate}
		\setcounter{enumi}{-1}
		\item Use
		\item Share
		\item Study.
		\item Improve
	\end{enumerate}
	the software (binary and source) for any purpose without restrictions.

	See \url{https://fsfe.org/freesoftware/freesoftware.en.html}
\end{frame}

\begin{frame}
	\frametitle{Implications}

	There are countless implications\footnote{many of which we will discuss in the course}:
	\begin{enumerate}
		\item none of your knowledge becomes irrelevant after changing employee
		\item people can give you money so that you improve FLOSS for them
		\item you can do research on FLOSS without any restriction
		\item you can modify FLOSS as you see fit
	\end{enumerate}
\end{frame}

\begin{assignment}
	\frametitle{First Assignment}
	\begin{itemize}
		\item Have you already used FLOSS?
		\item Did you already participate in FLOSS?
	\end{itemize}
	\begin{task}
	Discuss in breakout room and tell your partner's story.
	\end{task}
\end{assignment}

\begin{assignment}
	\begin{task}
	Break.
	\end{task}
\end{assignment}

%%%%%%%%%%%%%%%%%%%%%%%%%%%%%%%%%%%%%%%%%% 
\section{Content Overview}

\begin{frame}
	\textit{learning outcome:}
	\begin{itemize}
		\item remember learning outcomes
		\item remember the topics
	\end{itemize}
\end{frame}

\begin{frame}
	\textit{learning outcome:}
	\begin{itemize}
		\item participate in FLOSS initiatives,
		\item found new FLOSS initiatives,
		\item use FLOSS methods in your business context.
	\end{itemize}
\end{frame}




\begin{frame}
	\frametitle{Elektra}
	\hfill \includegraphics[width=2cm]{../figures/logo}
	\vspace{-1cm}
	\begin{itemize}
		\item Elektra is one FLOSS initiative of what we \\ discuss in this lecture.
		\item Elektra is developed at TU Wien (\url{https://libelektra.org}).
	\end{itemize}
\end{frame}

\begin{assignment}
	\frametitle{In which topics are you interested?}
	\begin{task}[1]
	Discuss topics with your partner.
	Can be new topics not mentioned before.
	\end{task}

	\begin{task}[2]
	Write down the most interesting in the shared notes.
	\end{task}
\end{assignment}





%%%%%%%%%%%%%%%%%%%%%%%%%%%%%%%%%%%%%%%%%% 
\section{Organisation}

\subsection{Preliminaries}
\begin{frame}
	\frametitle{Communication}
	\begin{itemize}
		\item TUWEL \url{https://tuwel.tuwien.ac.at/course/view.php?idnumber=194114-2021W}
		\item TISS \url{https://tiss.tuwien.ac.at/course/courseDetails.xhtml?courseNr=194114&semester=2021W}
		\item GitHub \url{https://git.libelektra.org}
		\item EMail \url{markus.raab@complang.tuwien.ac.at}
		\item before/after/during meetings
	\end{itemize}
\end{frame}

\begin{frame}
	\frametitle{Inverted Classroom}
	Meetings are most Wednesday 09:00 c.t. - 11:00 (max.)

	\begin{itemize}
		\item always read/watch the material in advance
		\item TUWEL already contains materials for L01
		\item within meetings we will do recapitulations, discussions, etc.
	\end{itemize}
\end{frame}

\begin{frame}
	\frametitle{Previous Knowledge}
	\begin{itemize}
		\item Obviously \textit{no} prior knowledge about FLOSS necessary.
		\item If you already have experience, you can use it in your talk and assignments.
		\item You should have some understanding of software engineering and software requirements.
		\item Programming skills is a must.
	\end{itemize}
\end{frame}

\begin{frame}
	\frametitle{Programming Languages}
	Elektra supports following programming languages:
	\begin{itemize}
		\item C
		\item C++
		\item Java
		\item Python
		\item Rust\footnote{no support on problems with the programming language by the lecturer}
		\item Go\footnotemark[1]
		\item Lua\footnotemark[1]
		\item Ruby\footnotemark[1]
	\end{itemize}
	You can use either of these languages.
\end{frame}

\subsection{Grades}

\begin{frame}
	You will get a grade once you did the registration H0.
	\vspace{1cm}

	To get a positive grade:
	\begin{itemize}
		\item All parts must be done.
		\item All parts must be positive.
	\end{itemize}
\end{frame}

\begin{frame}
	Grade is calculated from:
	\begin{description}
	\item[30\,\%:] homework
	\item[30\,\%:] teamwork
	\item[30\,\%:] project
	\item[10\,\%:] presentation
	\item[+:] extrapoints
	\end{description}
\end{frame}

\subsection{Assignments}
\begin{frame}
	\frametitle{Talk}
	\begin{itemize}
		\item About anything related to FLOSS.
		\item The duration must be 10--20 minutes.
		\item It must be about your experience.
		\item I.e., not only about study of literature.
		\item E.g., about:
		\begin{itemize}
			\item homework you did (e.g. H1 or T1)
			\item some FLOSS tool or application you developed
		\end{itemize}
	\end{itemize}
\end{frame}

\begin{frame}
	\frametitle{Deadlines}

	\begin{itemize}
	\item if you make submissions earlier, you get feedback earlier
	\item dates are both in ``semester schedule.pdf'' and calender of TUWEL
	\end{itemize}

	There are up to three deadlines for each homework, teamwork or project:

	\begin{itemize}
	\item deadline for submission of the work
	\item deadline for review (review the submission of others)
	\item deadline for corrections (based on the feedback of submission)
	\end{itemize}
\end{frame}

\begin{assignment}
	\begin{task}
	Talk with someone about a potential collaboration in the team exercise.
	\end{task}
\end{assignment}

\begin{assignment}
	\frametitle{Questions?}
	\begin{task}
	Please register for the course by doing H0.
	\end{task}

	\begin{task}
	Any questions?
	\end{task}
\end{assignment}



%%%%%%%%%%%%%%%%%%%%%%%%%%%%%%%%%%%%%%%%%% 
\nocite{raab2017introducing}

\appendix

\begin{frame}[allowframebreaks]
	\bibliographystyle{plainnat}
	\bibliography{../shared/elektra.bib}
\end{frame}

\end{document}


