% Make nice A4 pages for print:
%\usepackage{pgfpages}
%\pgfpagesuselayout{resize to}[a4paper,border shrink=5mm,landscape]

\beamertemplatenavigationsymbolsempty

\setbeamertemplate{bibliography item}[text]

\usepackage[type={CC},modifier={by-sa},version={4.0}]{doclicense}

\usepackage[utf8]{inputenc}
\usepackage{hyperref}
\usepackage{breakurl}
\usepackage{graphicx}
\usepackage{pgfplots}
\usepackage{pgf}
\usepackage{tikz}
\usetikzlibrary{positioning}
\usetikzlibrary{arrows}
\usetikzlibrary{decorations.markings}
\usetikzlibrary{calc}
\usetikzlibrary{matrix}
\usetikzlibrary{shapes}
\usetikzlibrary{decorations.pathmorphing}
\usetikzlibrary{fit}
\usetikzlibrary{backgrounds}
\usetikzlibrary{plotmarks}
\usepackage{stmaryrd}
\usepackage{listings}
\usepackage{pdflscape}
\usepackage{perpage}
\usepackage{appendixnumberbeamer}

%\usepackage[thmmarks,amsmath,amsthm]{ntheorem} % already included in beamer
\usepackage{thm-restate}

\usepackage[sort&compress,numbers]{natbib}  % to be have \citet, \citeauthor, \citeyear

\MakePerPage{footnote}

\tikzstyle{o}=[r,ppBlue]
\tikzstyle{r}=[thick,rectangle,align=center]
\tikzstyle{t}=[r,ppTrans] %,font=\bfseries]
\tikzstyle{dd}=[densely dashed]
\tikzstyle{n}=[r,ppBlue]
\tikzstyle{p}=[r,ppRed]
\tikzstyle{ppRed}  =[draw=red,  fill=  red!20]
\tikzstyle{ppBlue} =[draw=blue, fill= blue!20]
\tikzstyle{ppGreen}=[draw=green,fill=green!20]
\tikzstyle{ppTrans}=[draw=none, fill=none]

\usetheme{Warsaw}

\useoutertheme[subsection=true]{smoothbars}
%\useoutertheme[subsection=false]{miniframes}

\definecolor{bblue}{HTML}{D7DF01}	% yellow-ish actually, for better black/white printing
\definecolor{rred}{HTML}{C0504D}
\definecolor{ggreen}{HTML}{9BBB59}
\definecolor{ppurple}{HTML}{9F4C7C}
\definecolor{lightgray}{rgb}{0.3,0.3,0.3}
\definecolor{lightergray}{rgb}{0.9,0.9,0.9}
\definecolor{UniBlue}{RGB}{83,121,170}

\DeclareTextFontCommand\textintro{\normalfont\bfseries\itshape} % nice!
\newcommand{\intro}[2][]
{%
	\textintro{#2}%
}
\newcommand{\empha}[2][]
{%
	\emph{#2}%
}

%\theoremstyle{plain}
\newcounter{reqcounter}
\newtheorem{requirement}[reqcounter]{Requirement}

%setbeamercolor{structure}{fg=violet}

\makeatletter
\def\th@task{%
    \normalfont % body font
    \setbeamercolor{block title example}{bg=orange,fg=white}
    \setbeamercolor{block body example}{bg=orange!20,fg=black}
    \def\inserttheoremblockenv{exampleblock}
  }
\makeatother

\theoremstyle{task}
\newtheorem{task}{Task}

\newenvironment{assignment}%
{%\setbeamercolor{background canvas}{bg=violet}%
%\setbeamercolor{structure}{fg=cyan!90!black}%
 \setbeamercolor{frametitle}{bg=orange,fg=white}
\begin{frame}}%
{\end{frame}}%

\AtBeginSection[]{
  \begin{frame}
  \vfill
  \centering
  \begin{beamercolorbox}[sep=8pt,center,shadow=true,rounded=true]{title}
    \usebeamerfont{title}\insertsectionhead\par%
  \end{beamercolorbox}
  \tableofcontents
  \vfill
  \end{frame}
}




\pgfplotsset{compat=1.14}
\author{Markus Raab}


\title{L04 Continuous Integration}

\begin{document}

%%%%%%%%%%%%%%%%%%%%%%%%%%%%%%%%%%%%%%%%%% 
\section{Use}

\begin{frame}
	\frametitle{Learning Outcomes}
	After successful completion of L04 \\
	students will be able to

	\begin{itemize}
	\item use continuous integration
	\item remember basics of reproducibility
	\item improve continuous integration
	\end{itemize}
\end{frame}

\begin{frame}
	\frametitle{Workload}

	\begin{itemize}[<+-| alert@+>]
	\item maintainers of FLOSS usually have high workload
	\item review focus on difficult aspects
	\item use CI to answer trivial questions
	\end{itemize}
\end{frame}

\begin{frame}
	\frametitle{Check If}

	\begin{itemize}[<+-| alert@+>]
	\item the code compiles without warnings (with different compilers)
	\item all test suites run successfully (on different systems)
	\item release notes are written
	\item code is correctly formatted
	\item consistency rules are fulfilled
	\item ABI/API is compatible
	\item all content has a license
	\end{itemize}
\end{frame}

\begin{frame}
	\frametitle{Artifacts}

	The build server gives us:

	\begin{itemize}[<+-| alert@+>]
	\item documentation
	\item executables
	\item reports of
	\begin{itemize}
	\item test runs
	\item memory leaks
	\item code coverage
	\end{itemize}
	\item website, \dots
	\end{itemize}
\end{frame}

\begin{frame}
	\frametitle{Trigger Build}

	In Elektra:
	\begin{itemize}[<+-| alert@+>]
	\item Build server runs on every push in every branch
	\item Nightly/monthly/\dots builds
	\item Jenkins is the main CI
	\item Manual trigger by writing a comment: \\ ``jenkins build libelektra please''
	\item Sometimes CI problems need to be reported and fixed
	\item Maintenance reports in issues.libelektra.org/160
	\end{itemize}
\end{frame}

\begin{frame}
	\frametitle{Conclusion}

	\begin{itemize}[<+-| alert@+>]
	\item Start of build server automatically
	\item Reading of build server results manually
	\item Set label ``ready to merge'' iff build server passes
	\item Make sure that new test cases are executed
	\item CI handled like other code
	\end{itemize}
\end{frame}

%%%%%%%%%%%%%%%%%%%%%%%%%%%%%%%%%%%%%%%%%% 
\section{Reproducibility}

\begin{frame}
	\frametitle{Goals}

	\begin{enumerate}[<+-| alert@+>]

	\item get identical behavior from tools, e.g.: \\ same error messages or reformatting of source code
	\item get same failures when running tests
	\item (re)create identical binaries
	\end{enumerate}
\end{frame}

\begin{frame}
	\frametitle{Goal 1: Compilation}

	For identical results installation of exactly the same tools is needed.

	\pause

	\begin{problem}
	Different distributions offer different tool chains, compiled differently.
	\end{problem}

	\vspace{1cm}
	\pause

	\begin{solution}[Possible Solution]
	Install tools with identical version manually.
	\end{solution}
\end{frame}

\begin{frame}
	\frametitle{Goal 2: Tests}

	\begin{problem}
	For identical results the whole execution environment need to be replicated.
	\end{problem}

	\vspace{1cm}
	\pause

	\begin{solution}[Possible Solution]
	Virtualization but even then with limitations.
	\end{solution}
\end{frame}


\begin{frame}
	\frametitle{Dockerfile}

	\begin{itemize}[<+-| alert@+>]

	\item allows to define an image to be used for the build server
	\item FROM defines a base image
	\item RUN are commands to be executed that modify the image
	\end{itemize}

	\vspace{1cm}

	Alternatives: NixOS, Guix
\end{frame}

\begin{frame}[fragile]
	\frametitle{Example}

	\verb+scripts/docker/debian/bullseye/Dockerfile+:

	\begin{lstlisting}[language=docker]
FROM debian:bullseye

ENV LC_ALL C.UTF-8

RUN apt-get update && apt-get -y install \
    build-essential
RUN ldconfig

RUN useradd jenkins
USER ${JENKINS_USERID}
RUN git config --global user.email 'Jenkins <autobuilder@libelektra.org>' \
    && git config --global user.name 'Jenkins' \end{lstlisting}
\end{frame}

\begin{frame}
	\frametitle{Goal 3: Binaries}

	\begin{problem}
	\begin{itemize}[<+-| alert@+>]

	\item Different distributions offer different tool chains.
	\item Tool chained can be compiled or configured differently.
	\item Any current dates or build paths influence binaries.
	\item Any input from outside the source code (e.g. Internet) is volatile.
	\end{itemize}
	\end{problem}
\end{frame}

\begin{frame}
	\frametitle{Solutions}

	\begin{itemize}[<+-| alert@+>]

	\item Virtualization of build environment
	\item Recording of build environment (e.g. buildinfo files)
	\item Let build process ignore the environment \\
		$\rightarrow$ reduce problem to Goal~1
	\end{itemize}
\end{frame}

\begin{frame}
	\frametitle{Reproducible Builds}

	\begin{itemize}[<+-| alert@+>]

	\item verifiable path from source to binary
	\item reproducible by default (without virtualization)
	\item reduces influence of build environment
	\item make build system deterministic
	\end{itemize}
\end{frame}

\begin{frame}
	\frametitle{Limitations}

	Build system and macros enable/disable code depending on:

	\begin{itemize}
	\item compiler
	\item operating system
	\item installed dependencies
	\end{itemize}

	E.g. in Elektra plugins get disabled depending on missing dependencies.

	\pause
	\vspace{1cm}

	\begin{solution}
	In practice a combination of recording and ignoring the environment is used.
	\end{solution}
\end{frame}

%%%%%%%%%%%%%%%%%%%%%%%%%%%%%%%%%%%%%%%%%% 
\section{Improve}

\begin{frame}
	\frametitle{Jenkins}

	\begin{itemize}[<+-| alert@+>]

	\item cron on steroids
	\item many plugins
	\item implemented in Java
	\item Jenkinsfiles directly in source code
	\end{itemize}
\end{frame}

\begin{frame}
	\frametitle{Jenkinsfile}

	for Elektra in scripts/jenkins/Jenkinsfile

	\begin{itemize}[<+-| alert@+>]

	\item either declarativ or in Groovy
	\item enables review of CI
	\item different CI for different PRs
	\end{itemize}
\end{frame}

\begin{frame}[fragile]
	\frametitle{Jenkinsfile Example}

	Run one script:
	\begin{lstlisting}[language=Java]
  def tasks = [:]
  tasks.failFast = false
  tasks << buildIcheck()
\end{lstlisting}

	\begin{lstlisting}[language=Java]
def buildIcheck() {
  def stageName = "icheck"
  return [(stageName): {
    stage(stageName) {
      withDockerEnv(DOCKER_IMAGES.bullseye) {
        sh "scripts/build/run_icheck"
        deleteDir()
      }
    }
  }]
}
\end{lstlisting}
\end{frame}

\begin{frame}
	\frametitle{Useful Tests}

	\begin{problem}
	CI must be fast enough.
	\end{problem}

	\begin{itemize}[<+-| alert@+>]

	\item Verify a fact everyone agreed on is actually the case.
	\item Unit or integration tests run with different compilers and operating systems.
	\item Verify that a bug is fixed (regression).
	\item ABI/API tests: that behavior for customers is unchanged.
	\end{itemize}
\end{frame}

\begin{frame}
	\frametitle{Order is Important}

	\begin{itemize}[<+-| alert@+>]

	\item first check locally
	\item them implement automatic check
	\item then add automatic check to CI
	\end{itemize}
\end{frame}

\begin{frame}
	\frametitle{Definition of Done}

	Continuous improvements:

	\begin{itemize}
	\item underlying issue for discussion
	\item updated documentation
	\item added tests
	\item improved code comments
	\item review done
	\item QA (valgrind, ASAN, \dots)
	\end{itemize}
\end{frame}


%%%%%%%%%%%%%%%%%%%%%%%%%%%%%%%%%%%%%%%%%% 
\section{Meeting}

\subsection{Recapitulation}

\begin{frame}
	\frametitle{Recapitulation.}
	\begin{itemize}
		\item 
	\end{itemize}
\end{frame}

\subsection{Assignments}

\begin{frame}
	\frametitle{Feedback}
%	Exercises finished for this term.

	\hfill \includegraphics[width=2cm]{pics/feedback.png}
	\vspace{-1cm}
	\begin{itemize}
		\item Feedback Talk
%		\item ECTS breakdown realistic?
%		\item TISS Feedback from 16.06.2021 00:00 to 14.07.2021 23:59
	\end{itemize}
\end{frame}

\subsection{Preview}

\begin{frame}
	\frametitle{L05 Documentation}
\end{frame}


\end{document}
