% Make nice A4 pages for print:
%\usepackage{pgfpages}
%\pgfpagesuselayout{resize to}[a4paper,border shrink=5mm,landscape]

\beamertemplatenavigationsymbolsempty

\setbeamertemplate{bibliography item}[text]

\usepackage[type={CC},modifier={by-sa},version={4.0}]{doclicense}

\usepackage[utf8]{inputenc}
\usepackage{hyperref}
\usepackage{breakurl}
\usepackage{graphicx}
\usepackage{pgfplots}
\usepackage{pgf}
\usepackage{tikz}
\usetikzlibrary{positioning}
\usetikzlibrary{arrows}
\usetikzlibrary{decorations.markings}
\usetikzlibrary{calc}
\usetikzlibrary{matrix}
\usetikzlibrary{shapes}
\usetikzlibrary{decorations.pathmorphing}
\usetikzlibrary{fit}
\usetikzlibrary{backgrounds}
\usetikzlibrary{plotmarks}
\usepackage{stmaryrd}
\usepackage{listings}
\usepackage{pdflscape}
\usepackage{perpage}
\usepackage{appendixnumberbeamer}

%\usepackage[thmmarks,amsmath,amsthm]{ntheorem} % already included in beamer
\usepackage{thm-restate}

\usepackage[sort&compress,numbers]{natbib}  % to be have \citet, \citeauthor, \citeyear

\MakePerPage{footnote}

\tikzstyle{o}=[r,ppBlue]
\tikzstyle{r}=[thick,rectangle,align=center]
\tikzstyle{t}=[r,ppTrans] %,font=\bfseries]
\tikzstyle{dd}=[densely dashed]
\tikzstyle{n}=[r,ppBlue]
\tikzstyle{p}=[r,ppRed]
\tikzstyle{ppRed}  =[draw=red,  fill=  red!20]
\tikzstyle{ppBlue} =[draw=blue, fill= blue!20]
\tikzstyle{ppGreen}=[draw=green,fill=green!20]
\tikzstyle{ppTrans}=[draw=none, fill=none]

\usetheme{Warsaw}

\useoutertheme[subsection=true]{smoothbars}
%\useoutertheme[subsection=false]{miniframes}

\definecolor{bblue}{HTML}{D7DF01}	% yellow-ish actually, for better black/white printing
\definecolor{rred}{HTML}{C0504D}
\definecolor{ggreen}{HTML}{9BBB59}
\definecolor{ppurple}{HTML}{9F4C7C}
\definecolor{lightgray}{rgb}{0.3,0.3,0.3}
\definecolor{lightergray}{rgb}{0.9,0.9,0.9}
\definecolor{UniBlue}{RGB}{83,121,170}

\DeclareTextFontCommand\textintro{\normalfont\bfseries\itshape} % nice!
\newcommand{\intro}[2][]
{%
	\textintro{#2}%
}
\newcommand{\empha}[2][]
{%
	\emph{#2}%
}

%\theoremstyle{plain}
\newcounter{reqcounter}
\newtheorem{requirement}[reqcounter]{Requirement}

%setbeamercolor{structure}{fg=violet}

\makeatletter
\def\th@task{%
    \normalfont % body font
    \setbeamercolor{block title example}{bg=orange,fg=white}
    \setbeamercolor{block body example}{bg=orange!20,fg=black}
    \def\inserttheoremblockenv{exampleblock}
  }
\makeatother

\theoremstyle{task}
\newtheorem{task}{Task}

\newenvironment{assignment}%
{%\setbeamercolor{background canvas}{bg=violet}%
%\setbeamercolor{structure}{fg=cyan!90!black}%
 \setbeamercolor{frametitle}{bg=orange,fg=white}
\begin{frame}}%
{\end{frame}}%

\AtBeginSection[]{
  \begin{frame}
  \vfill
  \centering
  \begin{beamercolorbox}[sep=8pt,center,shadow=true,rounded=true]{title}
    \usebeamerfont{title}\insertsectionhead\par%
  \end{beamercolorbox}
  \tableofcontents
  \vfill
  \end{frame}
}




\pgfplotsset{compat=1.14}
\author{Markus Raab}


\title{L03 Build Tools}

\begin{document}

%%%%%%%%%%%%%%%%%%%%%%%%%%%%%%%%%%%%%%%%%% 
\section{Scripts}

\begin{frame}
	\frametitle{Learning Outcomes}
	After successful completion of L03 \\
	students will be able to

	\begin{itemize}
	\item make small modifications in (build) scripts
	\item remember basics of testing in FLOSS
	\end{itemize}
\end{frame}

\begin{frame}[fragile]
	\frametitle{Portable Shell Scripts}

	\begin{itemize}[<+-| alert@+>]
	\item can use \verb+#!/bin/sh+ as shebang
	\item use another shell as shebang if they are not portable
	\item use a minimal subset of bash, zsh, ...
	\item can be checked with \verb+make run_shellcheck+
	\end{itemize}
\end{frame}

\begin{frame}[fragile]
	\frametitle{Reformatting (1)}

	Serial variant of scripts/dev/reformat-all

	\begin{lstlisting}[language=sh]
#!/bin/sh
#
# @author Klemens
# @brief Calls all other reformat scripts
# @date 29.03.2019
# @tags reformat

DEV_SCRIPTS_DIR=$(dirname "$0")
. "${DEV_SCRIPTS_DIR}/include-common"

cd "$SOURCE"

reformat() {
	reformat=$1
	shift
	\end{lstlisting}
\end{frame}

\begin{frame}[fragile]
	\frametitle{Reformatting (2)}

	\begin{lstlisting}[language=sh]
	echo "starting $reformat ..."
	"$reformat" "$@"
	echo "finished $reformat"
}

IFS='
'
for reformat in $(ls "$DEV_SCRIPTS_DIR"/reformat-*); do
	[ "$(basename "$reformat")" = "reformat-all" ] && continue
	reformat "$reformat" "$@"
done\end{lstlisting}
\end{frame}

%%%%%%%%%%%%%%%%%%%%%%%%%%%%%%%%%%%%%%%%%% 
\section{Build Scripts}

\begin{frame}
	\frametitle{Goals}

	Build tools:

	\begin{itemize}[<+-| alert@+>]
	\item (cross-)compile the software
	\item generate documentation or other files
	\item run tests or build server scripts
	\item create packages
	\end{itemize}
\end{frame}

\begin{frame}
	\frametitle{Generation}

	How build tools typically work:

	\begin{itemize}[<+-| alert@+>]
	\item e.g.\ automake generates \texttt{./configure} shell scripts, which generates Makefiles
	\item e.g.\ CMake generates Makefiles, Ninja or project files of various IDEs
	\item other build tools directly invoke the compiler
	\end{itemize}
\end{frame}

\begin{frame}
	\frametitle{CMake}

	Elektra used automake and later switched to CMake.

	\vspace{1cm}

	CMake

	\begin{itemize}[<+-| alert@+>]
	\item cross-platform
	\item supports various programming languages
	\item has extensive modules for finding many tools and libraries
	\item see \texttt{scripts/cmake} of Elektra's repo
	\item \texttt{ctest} as test runner
	\item \texttt{cpack} for creating packages
	\end{itemize}
\end{frame}

\begin{frame}[fragile]
	\frametitle{Different Configurations}

	Common variants of how to run cmake are in \texttt{scripts/dev/configure-*}

	\begin{lstlisting}[language=sh]
cmake \
	-DPLUGINS="ALL" \
	-DTOOLS="ALL" \
	-DENABLE_DEBUG="OFF" \
	-DENABLE_LOGGER="OFF" \
	.\end{lstlisting}
\end{frame}

\begin{frame}
	\frametitle{Example: Elektra Plugins}

	\begin{itemize}[<+-| alert@+>]
	\item only \texttt{add_plugin}\footnote{implemented and documented in scripts/cmake/Modules/LibAddPlugin.cmake} needed in CMakeLists.txt
	\item README.md contains instructions when to add a plugin by looking at infos/provides and also infos/status
	\item \texttt{cmake -DPLUGINS="ALL;-EXPERIMENTAL"} excludes plugins which have EXPERIMENTAL in infos/status
	\item README.md also gets included in the plugin's code
	\end{itemize}
\end{frame}

%%%%%%%%%%%%%%%%%%%%%%%%%%%%%%%%%%%%%%%%%% 
\section{Test Tools}

\begin{frame}
	\frametitle{Test Runner}

	using ctest:

	\begin{itemize}
	\item valgrind
	\item ASAN
	\item AFL
	\end{itemize}
\end{frame}

\begin{frame}[fragile]
	\frametitle{Environment}

	Run Elektra from the build folder via \texttt{scripts/dev/run_env}.
	Basically does:

	\begin{lstlisting}[language=sh]
# common configure script
export SCRIPTS_DIR="${ELEKTRA_DIR}/scripts/dev"
. "${SCRIPTS_DIR}/include-common"

export PATH="$BUILD/bin:$SCRIPTS_DIR:$PATH"
export KDB_EXEC_PATH="$COMMON_PATH:$KDB_EXEC_PATH"
export LD_LIBRARY_PATH="$BUILD/lib:$LD_LIBRARY_PATH"
export MANPATH="$SOURCE/doc/man:$MANPATH"
export CLASSPATH="$CLASSPATH:$BUILD/lib/libelektra.jar"

export PS1="[DEV] $PS1"\end{lstlisting}
\end{frame}

%%%%%%%%%%%%%%%%%%%%%%%%%%%%%%%%%%%%%%%%%% 
\section{Meeting}

\subsection{Recapitulation}

\begin{frame}
	\frametitle{Recapitulation.}
	\begin{itemize}
		\item 
	\end{itemize}
\end{frame}

\subsection{Assignments}

\begin{frame}
	\frametitle{Feedback}
%	Exercises finished for this term.

	\hfill \includegraphics[width=2cm]{pics/feedback.png}
	\vspace{-1cm}
	\begin{itemize}
		\item Feedback Talk
%		\item ECTS breakdown realistic?
%		\item TISS Feedback from 16.06.2021 00:00 to 14.07.2021 23:59
	\end{itemize}
\end{frame}

\subsection{Preview}

\begin{frame}
	\frametitle{L04 Continuous Integration}
\end{frame}


\end{document}
