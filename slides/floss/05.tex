% Make nice A4 pages for print:
%\usepackage{pgfpages}
%\pgfpagesuselayout{resize to}[a4paper,border shrink=5mm,landscape]

\beamertemplatenavigationsymbolsempty

\setbeamertemplate{bibliography item}[text]

\usepackage[type={CC},modifier={by-sa},version={4.0}]{doclicense}

\usepackage[utf8]{inputenc}
\usepackage{hyperref}
\usepackage{breakurl}
\usepackage{graphicx}
\usepackage{pgfplots}
\usepackage{pgf}
\usepackage{tikz}
\usetikzlibrary{positioning}
\usetikzlibrary{arrows}
\usetikzlibrary{decorations.markings}
\usetikzlibrary{calc}
\usetikzlibrary{matrix}
\usetikzlibrary{shapes}
\usetikzlibrary{decorations.pathmorphing}
\usetikzlibrary{fit}
\usetikzlibrary{backgrounds}
\usetikzlibrary{plotmarks}
\usepackage{stmaryrd}
\usepackage{listings}
\usepackage{pdflscape}
\usepackage{perpage}
\usepackage{appendixnumberbeamer}

%\usepackage[thmmarks,amsmath,amsthm]{ntheorem} % already included in beamer
\usepackage{thm-restate}

\usepackage[sort&compress,numbers]{natbib}  % to be have \citet, \citeauthor, \citeyear

\MakePerPage{footnote}

\tikzstyle{o}=[r,ppBlue]
\tikzstyle{r}=[thick,rectangle,align=center]
\tikzstyle{t}=[r,ppTrans] %,font=\bfseries]
\tikzstyle{dd}=[densely dashed]
\tikzstyle{n}=[r,ppBlue]
\tikzstyle{p}=[r,ppRed]
\tikzstyle{ppRed}  =[draw=red,  fill=  red!20]
\tikzstyle{ppBlue} =[draw=blue, fill= blue!20]
\tikzstyle{ppGreen}=[draw=green,fill=green!20]
\tikzstyle{ppTrans}=[draw=none, fill=none]

\usetheme{Warsaw}

\useoutertheme[subsection=true]{smoothbars}
%\useoutertheme[subsection=false]{miniframes}

\definecolor{bblue}{HTML}{D7DF01}	% yellow-ish actually, for better black/white printing
\definecolor{rred}{HTML}{C0504D}
\definecolor{ggreen}{HTML}{9BBB59}
\definecolor{ppurple}{HTML}{9F4C7C}
\definecolor{lightgray}{rgb}{0.3,0.3,0.3}
\definecolor{lightergray}{rgb}{0.9,0.9,0.9}
\definecolor{UniBlue}{RGB}{83,121,170}

\DeclareTextFontCommand\textintro{\normalfont\bfseries\itshape} % nice!
\newcommand{\intro}[2][]
{%
	\textintro{#2}%
}
\newcommand{\empha}[2][]
{%
	\emph{#2}%
}

%\theoremstyle{plain}
\newcounter{reqcounter}
\newtheorem{requirement}[reqcounter]{Requirement}

%setbeamercolor{structure}{fg=violet}

\makeatletter
\def\th@task{%
    \normalfont % body font
    \setbeamercolor{block title example}{bg=orange,fg=white}
    \setbeamercolor{block body example}{bg=orange!20,fg=black}
    \def\inserttheoremblockenv{exampleblock}
  }
\makeatother

\theoremstyle{task}
\newtheorem{task}{Task}

\newenvironment{assignment}%
{%\setbeamercolor{background canvas}{bg=violet}%
%\setbeamercolor{structure}{fg=cyan!90!black}%
 \setbeamercolor{frametitle}{bg=orange,fg=white}
\begin{frame}}%
{\end{frame}}%

\AtBeginSection[]{
  \begin{frame}
  \vfill
  \centering
  \begin{beamercolorbox}[sep=8pt,center,shadow=true,rounded=true]{title}
    \usebeamerfont{title}\insertsectionhead\par%
  \end{beamercolorbox}
  \tableofcontents
  \vfill
  \end{frame}
}




\pgfplotsset{compat=1.14}
\author{Markus Raab}


\title{L05 Documentation}

\begin{document}

%%%%%%%%%%%%%%%%%%%%%%%%%%%%%%%%%%%%%%%%%% 
\section{Introduction}

\begin{frame}<1>[label=learning outcomes]
	\frametitle{Learning Outcomes}
	After successful completion of L05 \\
	students will be able to

	\begin{itemize}
	\item remember basics of documentation
	\item generate documentation
	\item verify documentation
	\end{itemize}
\end{frame}

\begin{frame}
	\frametitle{Correctness}

	Documentation tends to be:

	\begin{itemize}[<+-| alert@+>]
	\item outdated
	\item incorrect
	\item not helpful
	\end{itemize}
\end{frame}

\begin{frame}<1->[label=make sure that]
	\frametitle{Make Sure That}

	\begin{itemize}[<+-| alert@+>]
	\item documentation gets reviewed
	\item documentation is in source code management
	\item small distance to code
	\item avoid redundant information
	\end{itemize}
\end{frame}

\begin{frame}<1->[label=types of documentation]
	\frametitle{Different Types of Documentation}

	\begin{description}[<+-| alert@+>]
	\item[tutorials] in \texttt{doc/tutorials} for learning
	\item[how-to] solving a problem
	\item[goals] and/or use-cases ``What?''
	\item[decisions] for background information ``Why?''
	\item[explanations] in \texttt{doc/dev} and \texttt{doc/contrib} ``How?''
	\item[reference] searching for details \\ e.g. man pages, API docu
	\item[examples] for copy\&paste
	\end{description}
\end{frame}

\begin{frame}
	\frametitle{Different People}

	\begin{description}[<+-| alert@+>]
	\item[beginners] never forget everybody starts as beginner
	\item[advanced] understanding how to improve
	\item[expert] learn how to teach others, \\
		improve upon what the software is doing
	\end{description}
\end{frame}

%%%%%%%%%%%%%%%%%%%%%%%%%%%%%%%%%%%%%%%%%% 
\section{Generate}

\begin{frame}[fragile]
	\frametitle{Markdown}

	Minimal formatting abilities but implemented by many tools

	\begin{itemize}[<+-| alert@+>]
	\item \texttt{[Link](/linktarget)} or \texttt{[Link](relative/link)}
	\item \textit{*italics*} and \textbf{**bold**}
	\item \verb+`inline code`+ or \verb+```+ code fences
	\item \begin{verbatim}- item
- item
- item\end{verbatim}
	\end{itemize}
\end{frame}

\begin{frame}<1>[label=views]
	\frametitle{Views}

	The same (markdown) file can be viewed via:
	
	\begin{itemize}[<+-| alert@+>]
	\item directly viewing the source \texttt{doc/help/kdb.md}
	\item website \url{https://www.libelektra.org/manpages/kdb} (rendered by marked)
	\item API docu \url{https://doc.libelektra.org/api/master/html/doc_help_kdb_md.html} (rendered by doxygen)
	\item \texttt{man kdb} (rendered by ronn)
	\item \texttt{kdb --help} or \texttt{kdb help <command>}
	\item GitHub \texttt{https://master.libelektra.org/doc/help/kdb.md}
	\item From qt-gui (rendered by discount)
	\end{itemize}

	\pause[\thebeamerpauses]  %  show after \begin{itemize}[<+->]

	$\rightarrow$ 5 different markdown renderer involved
\end{frame}

\begin{frame}[fragile]
	\frametitle{Plugins}

	For plugins, documentation even changes the build process:
	\begin{lstlisting}[keywords={infos,author,license,provides,needs,recommends,placements,status,metadata}]
- infos = Information about the toml plugin is in keys below
- infos/author = Jakob Fischer <jakobfischer93@gmail.com>
- infos/licence = BSD
- infos/provides = storage/toml
- infos/needs = base64
- infos/recommends = type
- infos/placements = getstorage setstorage
- infos/status = experimental unfinished
- infos/metadata = order comment/# comment/#/start comment/#/space type tomltype origvalue
- infos/description = This storage plugin reads and writes TOML files using Flex and Bison.\end{lstlisting}
\end{frame}

\begin{frame}
	\frametitle{Conclusion}

	\begin{itemize}[<+-| alert@+>]
	\item reuse of documentation by generation
	\item avoids duplication and errors
	\item avoids CI checks for inconsistencies
	\item generation by CI
	\end{itemize}
\end{frame}

%%%%%%%%%%%%%%%%%%%%%%%%%%%%%%%%%%%%%%%%%% 
\section{Verify}

\begin{frame}
	\frametitle{Goals}

	Not every documentation-related task can be generated:

	\begin{itemize}[<+-| alert@+>]
	\item in the text we want to refer to the behavior
	\item we want to verify if given examples are correct
	\end{itemize}
\end{frame}

\begin{frame}
	\frametitle{Problems with Unit Tests}

	\begin{enumerate}[<+-| alert@+>]
	\item difficult to read
	\item code cannot be directly copied (asserts)
	\item cannot be easily integrated in tutorials
	\end{enumerate}

	\pause[\thebeamerpauses]  %  show after \begin{itemize}[<+->]

	$\rightarrow$ specialized verification language for documentation
\end{frame}

\begin{frame}[fragile]
	\frametitle{Verification of Tutorials}

	\begin{lstlisting}
	```sh
	kdb set user:/tests/something
	# RET: 0

	kdb get user:/tests/something
	```
\end{lstlisting}
\end{frame}

\begin{frame}[fragile]
	\frametitle{Syntax}

	\begin{enumerate}[<+-| alert@+>]
	\item starts with \verb+```sh+
	\item comments introduce checks
	\item otherwise is shell code to be executed
	\item \verb+#>+ verifies stdout output
	\item \verb+# RET:+ verifies return code (if not 0)
	\end{enumerate}
\end{frame}

\begin{frame}[fragile]
	\frametitle{Conventions}

	\begin{enumerate}[<+-| alert@+>]
	\item Test data in \verb+/tests+.
	\item Generate temporary files if needed or use HERE.
	\end{enumerate}
\end{frame}

\begin{frame}<1>[label=conclusions]
	\frametitle{Conclusions}

	\begin{enumerate}[<+-| alert@+>]
	\item If possible, generate.
	\item Otherwise, verify.
	\item Keep user and purpose in mind.
	\end{enumerate}
\end{frame}

%%%%%%%%%%%%%%%%%%%%%%%%%%%%%%%%%%%%%%%%% %
\section{Meeting}

\subsection{Recapitulation}

\againframe<10>{learning outcomes}

\begin{frame}
	\frametitle{Make Sure That}

	\begin{task}
	What should you make sure as FLOSS maintainer in respect to documentation?
	\end{task}
\end{frame}

\againframe<10>{make sure that}

\begin{frame}
	\frametitle{Make Sure That}

	\begin{task}
	Which types of documentation do you need to treat differently?
	\end{task}
\end{frame}

\againframe<10>{types of documentation}

\breakframe

\begin{frame}
	\frametitle{Views}

	\begin{task}
	Which different views can be provided by generation of documentation?
	\end{task}
\end{frame}

\againframe<10>{views}

\begin{frame}
	\frametitle{Conclusions}

	\begin{task}
	When do we generate, when do we verify, and for which users?
	\end{task}
\end{frame}

\againframe<10>{conclusions}

\breakframe


\subsection{Assignments}

\begin{assignment}
	\frametitle{Organisation}

	\begin{task}
	\begin{itemize}[<+-| alert@+>]
	\item Visit Consultation Hour (10:00–11:00) also for working together with tutor.
	\item Please mark activities in TUWEL.
	\item TUWEL no gradings visible while grading, check grades on Wednesdays.
	\item Get a Presentation Date.
	\item Mention\&request review from @markus2330 and @flo91.
	\end{itemize}
	\end{task}
\end{assignment}

\begin{assignment}
	\frametitle{H3: Pull Requests: Fix Issues}

	\begin{task}
	Any questions?
	\end{task}
\end{assignment}

\begin{assignment}
	\frametitle{T1: Reviews}

	Semicolons macros removed? (e.g. \#4710)

	\begin{task}
	Fix CI pipeline.
	\end{task}
\end{assignment}

%\begin{frame}
%	\frametitle{Feedback}
%%	Exercises finished for this term.
%
%	\hfill \includegraphics[width=2cm]{pics/feedback.png}
%	\vspace{-1cm}
%	\begin{itemize}
%		\item Feedback Talk
%		\item ECTS breakdown realistic?
%%		\item TISS Feedback from 16.06.2021 00:00 to 14.07.2021 23:59
%	\end{itemize}
%\end{frame}

\subsection{Preview}

\begin{frame}
	\frametitle{L06 Entry Barriers}
\end{frame}


\end{document}
