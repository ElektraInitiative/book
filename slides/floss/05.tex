%\ifdefined\handout
%\documentclass[handout,aspectratio=1610,xcolor={usenames,dvipsnames,table}]{beamer}
%\else
\documentclass[aspectratio=1610,xcolor={usenames,dvipsnames,table}]{beamer}
%\fi

\newcommand{\mylecture}{Configuration Management}

%\ifdefined\handout
%\documentclass[handout,aspectratio=1610,xcolor={usenames,dvipsnames,table}]{beamer}
%\else
\documentclass[aspectratio=1610,xcolor={usenames,dvipsnames,table}]{beamer}
%\fi

\newcommand{\mylecture}{Configuration Management}

%\ifdefined\handout
%\documentclass[handout,aspectratio=1610,xcolor={usenames,dvipsnames,table}]{beamer}
%\else
\documentclass[aspectratio=1610,xcolor={usenames,dvipsnames,table}]{beamer}
%\fi

\newcommand{\mylecture}{Configuration Management}

\input{../setup}
\input{../shared/setup}

\lstdefinelanguage{dump}
{
	morekeywords={kdbOpen,ksNew,keyNew,keyMeta,keyCopyMeta,keyEnd,ksEnd,kdbClose},
	sensitive=false,
	morecomment=[l]{//},
	morecomment=[s]{/*}{*/},
	morestring=[b]",
}


\lstdefinelanguage{SpecElektra}{
	%
	comment=[l]{;},
	commentstyle=\color{purple}\ttfamily,
	%
	morestring=[b]',
	morestring=[b]`,
	morestring=[b]",
	stringstyle=\color{purple}\ttfamily,
	%
	sensitive=f,% keywords are not case sensitive
	%
	% Colors see https://en.wikibooks.org/wiki/LaTeX/Colors
	%
	keywordstyle=\color{BlueViolet}\bfseries,
	keywordstyle=[2]\color{Green},
	keywordstyle=[3]\color{Aquamarine}\bfseries\textit,
	keywordstyle=[4]\color{NavyBlue}\bfseries,
	keywordstyle=[5]\color{Mahogany},
	%
	keywords={layer, require, validation, check, range, description, rationale, requirement, accessibility, enable, condition, message, default, opt, readonly, type, context, property1, property2, description, file, content, mountpoint, metadata, infos, plugins},
	keywords=[2]{},
	keywords=[3]{order, interface, network, emphasized},
	%keywords=[4]{[, ]},  %Not needed
	keywords=[4]{},
	keywords=[5]{},
	%
	literate={:=}{{{\color{red}\textbf:=}}}2
		 {\%}{{{\color{NavyBlue}\textbf\%}}}1
		 {[}{{{\color{Sepia}\textbf[}}}1
		 {]}{{{\color{Sepia}\textbf]}}}1,
}

\lstdefinelanguage{Cpp}{%
	language     = C++,
	literate=
}


\lstdefinelanguage{CfgElektra}{
	comment=[l]{;},
	commentstyle=\color{purple}\ttfamily,
	%
	morestring=[b]',
	morestring=[b]`,
	morestring=[b]",
	stringstyle=\color{purple}\ttfamily,
	%
	%
	sensitive=f,% keywords are not case sensitive
	%
	% Colors see https://en.wikibooks.org/wiki/LaTeX/Colors
	%
	keywordstyle=\color{Bittersweet}\bfseries,
	keywordstyle=[2]\color{DarkOrchid}\bfseries,
	keywordstyle=[3]\color{ForestGreen}\bfseries\textit,
	keywordstyle=[4]\color{Goldenrod}\bfseries,
	keywordstyle=[5]\color{CarnationPink},
	%
	keywords={},
	keywords=[2]{},
	keywords=[3]{},
	keywords=[4]{},
	keywords=[5]{},
	%
	literate={=}{{{\color{ForestGreen}\textbf=}}}1
		 %{<-}{{{\color{ForestGreen}\textbf<-}}}2
		 %{*}{{{\color{Bittersweet}\textbf*}}}1
		 {\%}{{{\color{NavyBlue}\textbf\%}}}1,
}




\lstset{language=SpecElektra, % Use SpecElektra as default programming language
	%boxpos=t, % make boxes a bit more unbreakable
	%frame=lines, % top+bottom line
	basicstyle=\ttfamily, % Use normal-size true type font
	showspaces,%
	showstringspaces=false, % Don't put marks in string spaces
	showlines=true, % make sure empty lines at end are shown (needed for concurrency
	tabsize=4, % spaces per tab
	xleftmargin=\parindent, % should be 18pt or 1.5em as defined by memoir
	%Does not really work well (needs to be deactivated for shortlistings):
	breaklines=false,
	%postbreak=\mbox{\textcolor{red}{$\hookrightarrow$}\space},
	%breakautoindent=true,
	%prebreak={\mbox{\ensuremath{\curvearrowright}}} % Zeichen am Zeilenende (Umbruch)
	%breaklines=true,
	%breakautoindent=true,
	%prebreak=\small\symbol{'134}, % backslash
	%prebreak={\mbox{\ensuremath{\curvearrowright}}} % lange kure
	%prebreak={\mbox{\ensuremath{\hookleftarrow}}} % lange kure
	%xleftmargin=3.0ex, %for some formats
	%xrightmargin=1.0ex, %for some formats
	%
	% Files do not work in utf8 see also:
	% http://stackoverflow.com/questions/1116266/listings-in-latex-with-utf-8-or-at-least-german-umlauts
	% http://tex.stackexchange.com/questions/24528/having-problems-with-listings-and-utf-8-can-it-be-fixed
	% Should work but doesn't? (Maybe add to literate broken?)
	%add to literate={ö}{{\"o}}1
	%	{ä}{{\"a}}1
	%	{ü}{{\"u}}1
	%	{Ö}{{\"O}}1
	%	{Ä}{{\"A}}1
	%	{Ü}{{\"U}}1
	%	{ß}{{\ss}}1,
	%
	% listingsutf8 did not work, made umlauts in comments very strange
	%extendedchars=true,
	%inputencoding=utf8,
	%
	%morecomment=[l][\color{blue}]{...}, % Line continuation (...) e.g. blue comment
	morekeywords={for_each},
	numbers=left, % Line numbers on left
	firstnumber=1, % Line numbers start with line 1
	numberstyle=\small\color{blue}, % Line numbers are blue and small
	numbersep=5pt,
	%stepnumber=5 % Line numbers go in steps of 5
}



\lstMakeShortInline[postbreak=,keywordstyle={}]^

\graphicspath{{../pic/}{../figures/}{../graphics/}{../ipe/}{../ggplot/}}



\lstdefinelanguage{dump}
{
	morekeywords={kdbOpen,ksNew,keyNew,keyMeta,keyCopyMeta,keyEnd,ksEnd,kdbClose},
	sensitive=false,
	morecomment=[l]{//},
	morecomment=[s]{/*}{*/},
	morestring=[b]",
}


\lstdefinelanguage{SpecElektra}{
	%
	comment=[l]{;},
	commentstyle=\color{purple}\ttfamily,
	%
	morestring=[b]',
	morestring=[b]`,
	morestring=[b]",
	stringstyle=\color{purple}\ttfamily,
	%
	sensitive=f,% keywords are not case sensitive
	%
	% Colors see https://en.wikibooks.org/wiki/LaTeX/Colors
	%
	keywordstyle=\color{BlueViolet}\bfseries,
	keywordstyle=[2]\color{Green},
	keywordstyle=[3]\color{Aquamarine}\bfseries\textit,
	keywordstyle=[4]\color{NavyBlue}\bfseries,
	keywordstyle=[5]\color{Mahogany},
	%
	keywords={layer, require, validation, check, range, description, rationale, requirement, accessibility, enable, condition, message, default, opt, readonly, type, context, property1, property2, description, file, content, mountpoint, metadata, infos, plugins},
	keywords=[2]{},
	keywords=[3]{order, interface, network, emphasized},
	%keywords=[4]{[, ]},  %Not needed
	keywords=[4]{},
	keywords=[5]{},
	%
	literate={:=}{{{\color{red}\textbf:=}}}2
		 {\%}{{{\color{NavyBlue}\textbf\%}}}1
		 {[}{{{\color{Sepia}\textbf[}}}1
		 {]}{{{\color{Sepia}\textbf]}}}1,
}

\lstdefinelanguage{Cpp}{%
	language     = C++,
	literate=
}


\lstdefinelanguage{CfgElektra}{
	comment=[l]{;},
	commentstyle=\color{purple}\ttfamily,
	%
	morestring=[b]',
	morestring=[b]`,
	morestring=[b]",
	stringstyle=\color{purple}\ttfamily,
	%
	%
	sensitive=f,% keywords are not case sensitive
	%
	% Colors see https://en.wikibooks.org/wiki/LaTeX/Colors
	%
	keywordstyle=\color{Bittersweet}\bfseries,
	keywordstyle=[2]\color{DarkOrchid}\bfseries,
	keywordstyle=[3]\color{ForestGreen}\bfseries\textit,
	keywordstyle=[4]\color{Goldenrod}\bfseries,
	keywordstyle=[5]\color{CarnationPink},
	%
	keywords={},
	keywords=[2]{},
	keywords=[3]{},
	keywords=[4]{},
	keywords=[5]{},
	%
	literate={=}{{{\color{ForestGreen}\textbf=}}}1
		 %{<-}{{{\color{ForestGreen}\textbf<-}}}2
		 %{*}{{{\color{Bittersweet}\textbf*}}}1
		 {\%}{{{\color{NavyBlue}\textbf\%}}}1,
}




\lstset{language=SpecElektra, % Use SpecElektra as default programming language
	%boxpos=t, % make boxes a bit more unbreakable
	%frame=lines, % top+bottom line
	basicstyle=\ttfamily, % Use normal-size true type font
	showspaces,%
	showstringspaces=false, % Don't put marks in string spaces
	showlines=true, % make sure empty lines at end are shown (needed for concurrency
	tabsize=4, % spaces per tab
	xleftmargin=\parindent, % should be 18pt or 1.5em as defined by memoir
	%Does not really work well (needs to be deactivated for shortlistings):
	breaklines=false,
	%postbreak=\mbox{\textcolor{red}{$\hookrightarrow$}\space},
	%breakautoindent=true,
	%prebreak={\mbox{\ensuremath{\curvearrowright}}} % Zeichen am Zeilenende (Umbruch)
	%breaklines=true,
	%breakautoindent=true,
	%prebreak=\small\symbol{'134}, % backslash
	%prebreak={\mbox{\ensuremath{\curvearrowright}}} % lange kure
	%prebreak={\mbox{\ensuremath{\hookleftarrow}}} % lange kure
	%xleftmargin=3.0ex, %for some formats
	%xrightmargin=1.0ex, %for some formats
	%
	% Files do not work in utf8 see also:
	% http://stackoverflow.com/questions/1116266/listings-in-latex-with-utf-8-or-at-least-german-umlauts
	% http://tex.stackexchange.com/questions/24528/having-problems-with-listings-and-utf-8-can-it-be-fixed
	% Should work but doesn't? (Maybe add to literate broken?)
	%add to literate={ö}{{\"o}}1
	%	{ä}{{\"a}}1
	%	{ü}{{\"u}}1
	%	{Ö}{{\"O}}1
	%	{Ä}{{\"A}}1
	%	{Ü}{{\"U}}1
	%	{ß}{{\ss}}1,
	%
	% listingsutf8 did not work, made umlauts in comments very strange
	%extendedchars=true,
	%inputencoding=utf8,
	%
	%morecomment=[l][\color{blue}]{...}, % Line continuation (...) e.g. blue comment
	morekeywords={for_each},
	numbers=left, % Line numbers on left
	firstnumber=1, % Line numbers start with line 1
	numberstyle=\small\color{blue}, % Line numbers are blue and small
	numbersep=5pt,
	%stepnumber=5 % Line numbers go in steps of 5
}



\lstMakeShortInline[postbreak=,keywordstyle={}]^

\graphicspath{{../pic/}{../figures/}{../graphics/}{../ipe/}{../ggplot/}}




\title{L05 Documentation}

\begin{document}

%%%%%%%%%%%%%%%%%%%%%%%%%%%%%%%%%%%%%%%%%% 
\section{Introduction}

\begin{frame}<1>[label=learning outcomes]
	\frametitle{Learning Outcomes}
	After successful completion of L05 \\
	students will be able to
	use techniques to

	\begin{itemize}
	\item remember basics
	\item generate documentation
	\item verify documentation
	\end{itemize}
\end{frame}

\begin{frame}
	\frametitle{Correctness}

	Documentation tends to be:

	\begin{itemize}[<+-| alert@+>]
	\item outdated
	\item incorrect
	\item not helpful
	\end{itemize}
\end{frame}

\begin{frame}<1>[label=make sure that]
	\frametitle{Make Sure That}

	\begin{itemize}[<+-| alert@+>]
	\item documentation gets reviewed
	\item documentation is in source code management
	\item small distance to code
	\item avoid redundant information
	\end{itemize}
\end{frame}

\begin{frame}<1>[label=types of documentation]
	\frametitle{Different Types of Documentation}

	\begin{description}[<+-| alert@+>]
	\item[tutorials] for learning
	\item[how-to] solving a problem
	\item[reference] searching for details \\ e.g. man pages, API docu
	\item[explanations] in \texttt{doc/dev} ``How?''
	\item[decisions] for background information ``Why?''
	\item[examples] for copy\&paste
	\end{description}
\end{frame}

\begin{frame}
	\frametitle{Different People}

	\begin{description}[<+-| alert@+>]
	\item[beginners] never forget everybody starts as beginner
	\item[advanced] understanding how to improve
	\item[expert] learn how to teach others, \\
		improve upon what the software is doing
	\end{description}
\end{frame}

%%%%%%%%%%%%%%%%%%%%%%%%%%%%%%%%%%%%%%%%%% 
\section{Generate}

\begin{frame}[fragile]
	\frametitle{Markdown}

	Minimal formatting abilities but implemented by many tools

	\begin{itemize}[<+-| alert@+>]
	\item \texttt{[Link](/linktarget)} or \texttt{[Link](relative/link)}
	\item \textit{*italics*} and \textbf{**bold**}
	\item \verb+`inline code`+ or \verb+```+ code fences
	\item \begin{verbatim}- item
- item
- item\end{verbatim}
	\end{itemize}
\end{frame}

\begin{frame}<1>[label=views]
	\frametitle{Views}

	The same (markdown) file can be viewed via:
	
	\begin{itemize}[<+-| alert@+>]
	\item directly viewing the source \texttt{doc/help/kdb.md}
	\item website \url{https://www.libelektra.org/manpages/kdb} (rendered by marked)
	\item API docu \url{https://doc.libelektra.org/api/master/html/doc_help_kdb_md.html} (rendered by doxygen)
	\item \texttt{man kdb} (rendered by ronn)
	\item \texttt{kdb --help} or \texttt{kdb help <command>}
	\item GitHub \texttt{https://master.libelektra.org/doc/help/kdb.md}
	\item From qt-gui (rendered by discount)
	\end{itemize}

	\pause[\thebeamerpauses]  %  show after \begin{itemize}[<+->]

	$\rightarrow$ 5 different markdown renderer involved
\end{frame}

\begin{frame}[fragile]
	\frametitle{Plugins}

	For plugins, documentation even changes the build process:
	\begin{lstlisting}[keywords={infos,author,license,provides,needs,recommends,placements,status,metadata}]
- infos = Information about the toml plugin is in keys below
- infos/author = Jakob Fischer <jakobfischer93@gmail.com>
- infos/licence = BSD
- infos/provides = storage/toml
- infos/needs = base64
- infos/recommends = type
- infos/placements = getstorage setstorage
- infos/status = experimental unfinished
- infos/metadata = order comment/# comment/#/start comment/#/space type tomltype origvalue
- infos/description = This storage plugin reads and writes TOML files using Flex and Bison.\end{lstlisting}
\end{frame}

\begin{frame}
	\frametitle{Conclusion}

	\begin{itemize}[<+-| alert@+>]
	\item reuse of documentation by generation
	\item avoids duplication and errors
	\item avoids CI checks for inconsistencies
	\item generation by CI
	\end{itemize}
\end{frame}

%%%%%%%%%%%%%%%%%%%%%%%%%%%%%%%%%%%%%%%%%% 
\section{Verify}

\begin{frame}
	\frametitle{Goals}

	Not every documentation-related task can be generated:

	\begin{itemize}[<+-| alert@+>]
	\item in the text we want to refer to the behavior
	\item we want to verify if given examples are correct
	\end{itemize}
\end{frame}

\begin{frame}
	\frametitle{Problems with Unit Tests}

	\begin{enumerate}[<+-| alert@+>]
	\item difficult to read
	\item code cannot be directly copied (asserts)
	\item cannot be easily integrated in tutorials
	\end{enumerate}

	\pause[\thebeamerpauses]  %  show after \begin{itemize}[<+->]

	$\rightarrow$ specialized verification language for documentation
\end{frame}

\begin{frame}[fragile]
	\frametitle{Verification of Tutorials}

	\begin{lstlisting}
	```sh
	kdb set user:/tests/something
	# RET: 0

	kdb get user:/tests/something
	```
\end{lstlisting}
\end{frame}

\begin{frame}[fragile]
	\frametitle{Syntax}

	\begin{enumerate}[<+-| alert@+>]
	\item starts with \verb+```sh+
	\item comments introduce checks
	\item otherwise is shell code to be executed
	\item \verb+#>+ verifies stdout output
	\item \verb+# RET:+ verifies return code (if not 0)
	\end{enumerate}
\end{frame}

\begin{frame}[fragile]
	\frametitle{Conventions}

	\begin{enumerate}[<+-| alert@+>]
	\item Test data in \verb+/tests+.
	\item Generate temporary files if needed or use HERE.
	\end{enumerate}
\end{frame}

\begin{frame}<1>[label=conclusions]
	\frametitle{Conclusions}

	\begin{enumerate}[<+-| alert@+>]
	\item If possible, generate.
	\item Otherwise, verify.
	\item Keep user and purpose in mind.
	\end{enumerate}
\end{frame}

%%%%%%%%%%%%%%%%%%%%%%%%%%%%%%%%%%%%%%%%% %
\section{Meeting}

\subsection{Recapitulation}

\againframe<10>{learning outcomes}

\begin{frame}
	\frametitle{Make Sure That}

	\begin{task}
	What should you make sure as FLOSS maintainer in respect to documentation?
	\end{task}
\end{frame}

\againframe<10>{make sure that}

\begin{frame}
	\frametitle{Make Sure That}

	\begin{task}
	Which types of documentation do you need to treat differently?
	\end{task}
\end{frame}

\againframe<10>{types of documentation}

\breakframe

\begin{frame}
	\frametitle{Views}

	\begin{task}
	Which different views can be provided by generation of documentation?
	\end{task}
\end{frame}

\againframe<10>{views}

\begin{frame}
	\frametitle{Conclusions}

	\begin{task}
	When do we generate, when do we verify, and for which users?
	\end{task}
\end{frame}

\againframe<10>{conclusions}


\subsection{Assignments}

\begin{assignment}
	\frametitle{H2: Corrections}

	\begin{task}
	How did you correct the review notes?
	\end{task}
\end{assignment}

\begin{assignment}
	\frametitle{T1: Continuous Integration}

	\begin{task}
	Do you have enough tasks?
	\end{task}

	\begin{task}
	Fix CI pipeline.
	\end{task}
\end{assignment}

\begin{frame}
	\frametitle{Feedback}
%	Exercises finished for this term.

	\hfill \includegraphics[width=2cm]{pics/feedback.png}
	\vspace{-1cm}
	\begin{itemize}
		\item Feedback Talk
		\item ECTS breakdown realistic?
%		\item TISS Feedback from 16.06.2021 00:00 to 14.07.2021 23:59
	\end{itemize}
\end{frame}

\subsection{Preview}

\begin{frame}
	\frametitle{L06 Entry Barriers}
\end{frame}


\end{document}
