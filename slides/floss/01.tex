%\ifdefined\handout
%\documentclass[handout,aspectratio=1610,xcolor={usenames,dvipsnames,table}]{beamer}
%\else
\documentclass[aspectratio=1610,xcolor={usenames,dvipsnames,table}]{beamer}
%\fi

\newcommand{\mylecture}{Configuration Management}

%\ifdefined\handout
%\documentclass[handout,aspectratio=1610,xcolor={usenames,dvipsnames,table}]{beamer}
%\else
\documentclass[aspectratio=1610,xcolor={usenames,dvipsnames,table}]{beamer}
%\fi

\newcommand{\mylecture}{Configuration Management}

%\ifdefined\handout
%\documentclass[handout,aspectratio=1610,xcolor={usenames,dvipsnames,table}]{beamer}
%\else
\documentclass[aspectratio=1610,xcolor={usenames,dvipsnames,table}]{beamer}
%\fi

\newcommand{\mylecture}{Configuration Management}

\input{../setup}
\input{../shared/setup}

\lstdefinelanguage{dump}
{
	morekeywords={kdbOpen,ksNew,keyNew,keyMeta,keyCopyMeta,keyEnd,ksEnd,kdbClose},
	sensitive=false,
	morecomment=[l]{//},
	morecomment=[s]{/*}{*/},
	morestring=[b]",
}


\lstdefinelanguage{SpecElektra}{
	%
	comment=[l]{;},
	commentstyle=\color{purple}\ttfamily,
	%
	morestring=[b]',
	morestring=[b]`,
	morestring=[b]",
	stringstyle=\color{purple}\ttfamily,
	%
	sensitive=f,% keywords are not case sensitive
	%
	% Colors see https://en.wikibooks.org/wiki/LaTeX/Colors
	%
	keywordstyle=\color{BlueViolet}\bfseries,
	keywordstyle=[2]\color{Green},
	keywordstyle=[3]\color{Aquamarine}\bfseries\textit,
	keywordstyle=[4]\color{NavyBlue}\bfseries,
	keywordstyle=[5]\color{Mahogany},
	%
	keywords={layer, require, validation, check, range, description, rationale, requirement, accessibility, enable, condition, message, default, opt, readonly, type, context, property1, property2, description, file, content, mountpoint, metadata, infos, plugins},
	keywords=[2]{},
	keywords=[3]{order, interface, network, emphasized},
	%keywords=[4]{[, ]},  %Not needed
	keywords=[4]{},
	keywords=[5]{},
	%
	literate={:=}{{{\color{red}\textbf:=}}}2
		 {\%}{{{\color{NavyBlue}\textbf\%}}}1
		 {[}{{{\color{Sepia}\textbf[}}}1
		 {]}{{{\color{Sepia}\textbf]}}}1,
}

\lstdefinelanguage{Cpp}{%
	language     = C++,
	literate=
}


\lstdefinelanguage{CfgElektra}{
	comment=[l]{;},
	commentstyle=\color{purple}\ttfamily,
	%
	morestring=[b]',
	morestring=[b]`,
	morestring=[b]",
	stringstyle=\color{purple}\ttfamily,
	%
	%
	sensitive=f,% keywords are not case sensitive
	%
	% Colors see https://en.wikibooks.org/wiki/LaTeX/Colors
	%
	keywordstyle=\color{Bittersweet}\bfseries,
	keywordstyle=[2]\color{DarkOrchid}\bfseries,
	keywordstyle=[3]\color{ForestGreen}\bfseries\textit,
	keywordstyle=[4]\color{Goldenrod}\bfseries,
	keywordstyle=[5]\color{CarnationPink},
	%
	keywords={},
	keywords=[2]{},
	keywords=[3]{},
	keywords=[4]{},
	keywords=[5]{},
	%
	literate={=}{{{\color{ForestGreen}\textbf=}}}1
		 %{<-}{{{\color{ForestGreen}\textbf<-}}}2
		 %{*}{{{\color{Bittersweet}\textbf*}}}1
		 {\%}{{{\color{NavyBlue}\textbf\%}}}1,
}




\lstset{language=SpecElektra, % Use SpecElektra as default programming language
	%boxpos=t, % make boxes a bit more unbreakable
	%frame=lines, % top+bottom line
	basicstyle=\ttfamily, % Use normal-size true type font
	showspaces,%
	showstringspaces=false, % Don't put marks in string spaces
	showlines=true, % make sure empty lines at end are shown (needed for concurrency
	tabsize=4, % spaces per tab
	xleftmargin=\parindent, % should be 18pt or 1.5em as defined by memoir
	%Does not really work well (needs to be deactivated for shortlistings):
	breaklines=false,
	%postbreak=\mbox{\textcolor{red}{$\hookrightarrow$}\space},
	%breakautoindent=true,
	%prebreak={\mbox{\ensuremath{\curvearrowright}}} % Zeichen am Zeilenende (Umbruch)
	%breaklines=true,
	%breakautoindent=true,
	%prebreak=\small\symbol{'134}, % backslash
	%prebreak={\mbox{\ensuremath{\curvearrowright}}} % lange kure
	%prebreak={\mbox{\ensuremath{\hookleftarrow}}} % lange kure
	%xleftmargin=3.0ex, %for some formats
	%xrightmargin=1.0ex, %for some formats
	%
	% Files do not work in utf8 see also:
	% http://stackoverflow.com/questions/1116266/listings-in-latex-with-utf-8-or-at-least-german-umlauts
	% http://tex.stackexchange.com/questions/24528/having-problems-with-listings-and-utf-8-can-it-be-fixed
	% Should work but doesn't? (Maybe add to literate broken?)
	%add to literate={ö}{{\"o}}1
	%	{ä}{{\"a}}1
	%	{ü}{{\"u}}1
	%	{Ö}{{\"O}}1
	%	{Ä}{{\"A}}1
	%	{Ü}{{\"U}}1
	%	{ß}{{\ss}}1,
	%
	% listingsutf8 did not work, made umlauts in comments very strange
	%extendedchars=true,
	%inputencoding=utf8,
	%
	%morecomment=[l][\color{blue}]{...}, % Line continuation (...) e.g. blue comment
	morekeywords={for_each},
	numbers=left, % Line numbers on left
	firstnumber=1, % Line numbers start with line 1
	numberstyle=\small\color{blue}, % Line numbers are blue and small
	numbersep=5pt,
	%stepnumber=5 % Line numbers go in steps of 5
}



\lstMakeShortInline[postbreak=,keywordstyle={}]^

\graphicspath{{../pic/}{../figures/}{../graphics/}{../ipe/}{../ggplot/}}



\lstdefinelanguage{dump}
{
	morekeywords={kdbOpen,ksNew,keyNew,keyMeta,keyCopyMeta,keyEnd,ksEnd,kdbClose},
	sensitive=false,
	morecomment=[l]{//},
	morecomment=[s]{/*}{*/},
	morestring=[b]",
}


\lstdefinelanguage{SpecElektra}{
	%
	comment=[l]{;},
	commentstyle=\color{purple}\ttfamily,
	%
	morestring=[b]',
	morestring=[b]`,
	morestring=[b]",
	stringstyle=\color{purple}\ttfamily,
	%
	sensitive=f,% keywords are not case sensitive
	%
	% Colors see https://en.wikibooks.org/wiki/LaTeX/Colors
	%
	keywordstyle=\color{BlueViolet}\bfseries,
	keywordstyle=[2]\color{Green},
	keywordstyle=[3]\color{Aquamarine}\bfseries\textit,
	keywordstyle=[4]\color{NavyBlue}\bfseries,
	keywordstyle=[5]\color{Mahogany},
	%
	keywords={layer, require, validation, check, range, description, rationale, requirement, accessibility, enable, condition, message, default, opt, readonly, type, context, property1, property2, description, file, content, mountpoint, metadata, infos, plugins},
	keywords=[2]{},
	keywords=[3]{order, interface, network, emphasized},
	%keywords=[4]{[, ]},  %Not needed
	keywords=[4]{},
	keywords=[5]{},
	%
	literate={:=}{{{\color{red}\textbf:=}}}2
		 {\%}{{{\color{NavyBlue}\textbf\%}}}1
		 {[}{{{\color{Sepia}\textbf[}}}1
		 {]}{{{\color{Sepia}\textbf]}}}1,
}

\lstdefinelanguage{Cpp}{%
	language     = C++,
	literate=
}


\lstdefinelanguage{CfgElektra}{
	comment=[l]{;},
	commentstyle=\color{purple}\ttfamily,
	%
	morestring=[b]',
	morestring=[b]`,
	morestring=[b]",
	stringstyle=\color{purple}\ttfamily,
	%
	%
	sensitive=f,% keywords are not case sensitive
	%
	% Colors see https://en.wikibooks.org/wiki/LaTeX/Colors
	%
	keywordstyle=\color{Bittersweet}\bfseries,
	keywordstyle=[2]\color{DarkOrchid}\bfseries,
	keywordstyle=[3]\color{ForestGreen}\bfseries\textit,
	keywordstyle=[4]\color{Goldenrod}\bfseries,
	keywordstyle=[5]\color{CarnationPink},
	%
	keywords={},
	keywords=[2]{},
	keywords=[3]{},
	keywords=[4]{},
	keywords=[5]{},
	%
	literate={=}{{{\color{ForestGreen}\textbf=}}}1
		 %{<-}{{{\color{ForestGreen}\textbf<-}}}2
		 %{*}{{{\color{Bittersweet}\textbf*}}}1
		 {\%}{{{\color{NavyBlue}\textbf\%}}}1,
}




\lstset{language=SpecElektra, % Use SpecElektra as default programming language
	%boxpos=t, % make boxes a bit more unbreakable
	%frame=lines, % top+bottom line
	basicstyle=\ttfamily, % Use normal-size true type font
	showspaces,%
	showstringspaces=false, % Don't put marks in string spaces
	showlines=true, % make sure empty lines at end are shown (needed for concurrency
	tabsize=4, % spaces per tab
	xleftmargin=\parindent, % should be 18pt or 1.5em as defined by memoir
	%Does not really work well (needs to be deactivated for shortlistings):
	breaklines=false,
	%postbreak=\mbox{\textcolor{red}{$\hookrightarrow$}\space},
	%breakautoindent=true,
	%prebreak={\mbox{\ensuremath{\curvearrowright}}} % Zeichen am Zeilenende (Umbruch)
	%breaklines=true,
	%breakautoindent=true,
	%prebreak=\small\symbol{'134}, % backslash
	%prebreak={\mbox{\ensuremath{\curvearrowright}}} % lange kure
	%prebreak={\mbox{\ensuremath{\hookleftarrow}}} % lange kure
	%xleftmargin=3.0ex, %for some formats
	%xrightmargin=1.0ex, %for some formats
	%
	% Files do not work in utf8 see also:
	% http://stackoverflow.com/questions/1116266/listings-in-latex-with-utf-8-or-at-least-german-umlauts
	% http://tex.stackexchange.com/questions/24528/having-problems-with-listings-and-utf-8-can-it-be-fixed
	% Should work but doesn't? (Maybe add to literate broken?)
	%add to literate={ö}{{\"o}}1
	%	{ä}{{\"a}}1
	%	{ü}{{\"u}}1
	%	{Ö}{{\"O}}1
	%	{Ä}{{\"A}}1
	%	{Ü}{{\"U}}1
	%	{ß}{{\ss}}1,
	%
	% listingsutf8 did not work, made umlauts in comments very strange
	%extendedchars=true,
	%inputencoding=utf8,
	%
	%morecomment=[l][\color{blue}]{...}, % Line continuation (...) e.g. blue comment
	morekeywords={for_each},
	numbers=left, % Line numbers on left
	firstnumber=1, % Line numbers start with line 1
	numberstyle=\small\color{blue}, % Line numbers are blue and small
	numbersep=5pt,
	%stepnumber=5 % Line numbers go in steps of 5
}



\lstMakeShortInline[postbreak=,keywordstyle={}]^

\graphicspath{{../pic/}{../figures/}{../graphics/}{../ipe/}{../ggplot/}}




\title{L01 Issue Tracking}

\begin{document}

%%%%%%%%%%%%%%%%%%%%%%%%%%%%%%%%%%%%%%%%%% 
\section{Issue Tracking}

\begin{frame}[label=learning outcome]
	\frametitle{Learning Outcomes}
	After successful completion of L01 and H1 \\
	students will be able to

	\begin{itemize}
	\item remember terms and characteristics of issue tracking systems,
	\item report bugs,
	\item triage bugs.
	\end{itemize}
\end{frame}

\begin{frame}<2>[label=issue]
	\frametitle{Term: Issue}

	\only<2->{
	``Issue'' is a very general term, it means nearly anything:

	\begin{itemize}
		\item a problem or bug
		\item a proposal
		\item a question
		\item a feature request
		\item a task
		\item a TODO list entry
		\item \dots
	\end{itemize}
	}
\end{frame}

\begin{frame}<2>[label=issue tracking]
	\frametitle{Issue Tracking}

	\only<2->{
	Allows to:

	\begin{itemize}
		\item track the latest status of issues: open, resolved, \dots
		\item discuss the issue (\texttt{@mention})
		\item add semantics (metadata): tags, project, milestone, relationships, priorities, \dots
	\end{itemize}
	}
\end{frame}

\begin{frame}[fragile]
	\frametitle{Queries}
	Based on full text and/or semantic queries, issues on \url{https://issues.libelektra.org}:

	\begin{itemize}
		\item I created: \\
			\verb+is:open is:issue author:@me+
		\item I am assigned to: \\
			\verb+is:open is:issue assignee:@me+
		\item without assignee: \\
			\verb+is:open is:issue no:assignee+
		\item not updated this year: \\
			\verb+is:open is:issue updated:<2022+
		\item that have a label: \\
			\verb+is:open is:issue label:floss2022W+
		\item to be fixed before 1.0: \\
			\verb+is:open is:issue milestone:0.9.*+
	\end{itemize}
\end{frame}

\begin{frame}
	\frametitle{Issue Tracking Systems}

	\begin{itemize}
		\item text files with metadata in git
		\item conversationally-rich: Debbugs, GitLab or GitHub issues
		\item semantically-rich: e.g. Bugzilla
		\item broader scope: Redmine, Trac
		\item specialization: misconfiguration tracker
	\end{itemize}
\end{frame}

\begin{frame}
	\frametitle{Unsuitable ``Issue Tracking Systems''}

	\begin{itemize}
		\item EMail
		\item Forums
		\item chats like IRC
		\item text files
		\item TODO markers
		\item \dots
	\end{itemize}

	if without metadata or not in version control.
\end{frame}

\begin{frame}
	\frametitle{Interfaces}

	\begin{itemize}
		\item Web
		\item EMail
		\item REST
		\item CLI tools, e.g. reportbug
		\item \dots
	\end{itemize}
\end{frame}

\begin{frame}
	\frametitle{Elektra has}
	\begin{itemize}
		\item TODO files in doc/todo (12 files)
		\item TODO markers in source code (263 markers)
		\item \url{https://issues.libelektra.org} (217 open issues)
	\end{itemize}
\end{frame}

\begin{frame}[fragile]
	\frametitle{Automatic Closing of Issues}

	Ideally only fixed bugs would be closed but:
	\begin{itemize}
		\item issues become irrelevant
		\item maintainers disappear
		\item systems depreciate
		\item focus shifts
		\item \dots
	\end{itemize}
	\vspace{1cm}

	In Elektra issues+PR close after 365+14 days automatically, see
	\verb+.github/stale.yml+.

	\vspace{1cm}
	The 249 issues still can be found via \verb+is:closed is:issue label:stale+
\end{frame}

\begin{frame}<1>[label=netiquette]
	\frametitle{Netiquette}

	\begin{itemize}
		\item Never forget that you are talking to human beings.
		\item Be as careful, respectful and gentle as possible.
		\item Expect as little as possible.
		\item Only judge on technical issues, never on persons.
		\item There are no golden rules, cultures can disagree on everything.
	\end{itemize}

	\only<2>{
	\begin{task}
	Do you agree with that list?
	Discuss your experiences.
	\end{task}
	}
\end{frame}

\begin{frame}[fragile]
	\frametitle{Quoting}

	\begin{itemize}
	\item If you want to reply to several points:
		\begin{code}[gobble=8]
		@ghost wrote:
		> I wrote something
		The answer
		\end{code}
	\item If you reply to a statement given somewhere else:
		\begin{code}[gobble=8]
		@ghost wrote in [link to comment]:
		> I wrote something
		The answer
		\end{code}
	\item Often no reply necessary if you only want to (dis)agree.
	\end{itemize}
\end{frame}

\begin{frame}
	\frametitle{Best Practices}
	\begin{itemize}
		\item First read attentively, then write.
		\item If in doubt: start a new issue.
		\item Split up issues that discuss unrelated problems.
		\item Prefer methods of automatic closing of issues.
		\item Fix issues you are assigned, ask for help or unassign if you give up.
	\end{itemize}
\end{frame}

%%%%%%%%%%%%%%%%%%%%%%%%%%%%%%%%%%%%%%%%%% 
\section{Bug Reporting}


\begin{frame}<2>[label=first steps]
	\frametitle{First Steps in Bug Reports}
	Make sure that:

	\only<2->{
	\begin{itemize}
		\item You use the correct issue tracker.
		\item You read about how to use that issue tracker.
		\item Use specialized helper programs, if available, like reportbug.
	\end{itemize}
	}
\end{frame}

\begin{frame}
	\frametitle{Steps to Reproduce the Problem}
	\begin{itemize}
		\item be precise
		\item be clear
		\item be complete
		\item ideally syntax of tests
	\end{itemize}
	\vspace{1cm}
	$\rightarrow$ verify yourself
\end{frame}

\begin{frame}
	\frametitle{Actual Result}
	\begin{itemize}
		\item describe the symptoms
		\item avoid opinions or conclusions here
		\item describe what you see
	\end{itemize}
\end{frame}

\begin{frame}
	\frametitle{Expected Result}
	\begin{itemize}
		\item how you would like the software to behave
		\item suggestions how to solve the problem
	\end{itemize}
\end{frame}

\begin{frame}
	\frametitle{System Information}
	\begin{itemize}
		\item version or sha of commit
		\item include errors, logs, etc.
		\item operating system or docker container
		\item versions of other relevant software
	\end{itemize}
\end{frame}

\begin{frame}
	\frametitle{Best Practices}
	\begin{itemize}
		\item learn about the community in guidelines
		\item always include symptoms, separate diagnosis
		\item reproduce using your own report
		\item sometimes an incomplete report can be better than no report
		\item reply to further questions
	\end{itemize}
\end{frame}


%%%%%%%%%%%%%%%%%%%%%%%%%%%%%%%%%%%%%%%%%% 
\section{Bug Triage}

\begin{frame}
	\frametitle{Reproduce the Problem}
	Try to do what is described in the issue, possible problems:
	\begin{itemize}
		\item There is an error in the description, e.g. a wrong command.
		\item The description is missing essential steps to do.
		\item The issue is already fixed or otherwise outdated.
	\end{itemize}
	\vspace{1cm}
	$\rightarrow$ Fix such problems in the issue description!
\end{frame}

\begin{frame}[fragile]
	\frametitle{Locate Problem in Source Code}
	\begin{itemize}
		\item Via error messages: \verb+kdb -vd+

		\item Via debugger or backtrace:
		\begin{itemize}
			\item Additionally install \verb+-dbgsym+ packages.
			\item Even better compile with: \verb+ENABLE_DEBUG+.
		\end{itemize}

		\item Via logger:
		\begin{itemize}
			\item Compile with \verb+ENABLE_LOGGER+.
			\item Modify \verb+src/libs/elektra/log.c+ as needed.
		\end{itemize}
	\end{itemize}
\end{frame}

\begin{frame}
	\frametitle{System Information}
	Hints about further affected systems, e.g.,
	\begin{itemize}
		\item information about your system
		\item version information
		\item programming language
	\end{itemize}
	might further triage the bug, i.e., help the person working on it.
\end{frame}

\begin{frame}
	\frametitle{Best Practices}

	Bug triage
	\begin{itemize}
		\item makes fixing bugs easier.
		\item can help to find the right person to fix a task.
		\item is needed for imprecise unclear or incomplete issues \\
			$\rightarrow$ avoid this in the first place
	\end{itemize}
\end{frame}

%%%%%%%%%%%%%%%%%%%%%%%%%%%%%%%%%%%%%%%%%% 
\section{Meeting}

\subsection{Organisation}

% TODO @kodebach renaming case study
% TODO presentations per video? when?
% TODO group work: implications of FLOSS to you

\begin{frame}
	\begin{task}
	Rename to ``@GitHubName Real Name'' \\
	Real Name in the order as you want to be called.
	\end{task}

	\frametitle{BigBlueButton}
	\begin{itemize}
		\item raise the hand immediately on any issues
		\item on technical problems write in chat what happened and then try another browser, e.g., recent Firefox or Chromium
	\end{itemize}
\end{frame}

\begin{assignment}
	\frametitle{Language during the meetings?}
	\begin{task}
	\begin{description}
	\item[A] English
	\item[B] Slightly Prefer English
	\item[C] Both are fine
	\item[D] Slightly Prefer German
	\item[E] German
	\end{description}
	\end{task}
\end{assignment}

\begin{assignment}
	\frametitle{Video}
	I am trying to keep meetings short with many breaks. \\
	You are allowed to:
	\begin{itemize}
		\item stretch
		\item move
		\item eat
		\item look somewhere else
		\item leave your place
	\end{itemize}

	\begin{task}
	But please turn video on.
	\end{task}
\end{assignment}

\begin{frame}
	\frametitle{Inverted Classroom}

	\begin{task}
	Did you watch the videos of L01?

	\begin{description}
	\item[A] Fully
	\item[B] Most
	\item[C] Partly
	\item[D] No
	\end{description}
	\end{task}
\end{frame}

\begin{frame}
	\frametitle{Information Flow}

	\begin{task}
	\begin{itemize}[<+-| alert@+>]
		\item please mark materials in TUWEL that you read/watched as ``Erledigt''
		\item configure TUWEL that you receive replies in discussions
		\item configure GitHub that you receive information on @mention
	\end{itemize}

	\end{task}
\end{frame}

\begin{frame}
	\frametitle{course vs.\ real-world FLOSS}

	\begin{task}
	\begin{itemize}[<+-| alert@+>]
		\item deadlines
		\item not all information open (TUWEL)
		\item teams
	\end{itemize}

	\end{task}
\end{frame}

\begin{assignment}
	\begin{task}
	Break.
	\end{task}
\end{assignment}

\subsection{Recapitulation}

\begin{frame}
	\frametitle{FLOSS}

	\floss{} allows you to:
	\pause
	\vspace{1em}
	\begin{enumerate}
		\setcounter{enumi}{-1}
		\item Use
		\item Share
		\item Study
		\item Improve
	\end{enumerate}
	\vspace{1em}
	the software (binary and source) for any purpose without restrictions.
\end{frame}

\begin{frame}
	\vspace{-0.1cm}
	\hspace{-1.5cm}
	\begin{tabular}{cl}
	\begin{tabular}{c}
	\includegraphics[width=11cm]{Overview}
	\end{tabular}
	& \begin{tabular}{l}
	\hspace{-0.5cm}
	\parbox{2.8cm}{%

	In pairs, discuss:
	\begin{itemize}[<+-| alert@+>]
		\item important topics
		\item missing topics
		\item implications
	\end{itemize}
	}
	\end{tabular}  \\
	\end{tabular}
\end{frame}

\againframe<2>{learning outcome}

\begin{assignment}
\end{assignment}

\againframe<1,3>{issue}

\againframe<1,3>{issue tracking}

\begin{assignment}
	\begin{task}
	Break.
	\end{task}
\end{assignment}

\againframe<2>{netiquette}

\againframe<1,3>{first steps}

\begin{frame}
	\frametitle{Conclusions vs. Observations}

	What are observations?

	\begin{itemize}[<+-| alert@+>]
		\item stack strace
		\item assignment to wrong variable
		\item program output
		\item missing assertion
		\item wrongly taken branch
		\item no log output
	\end{itemize}
\end{frame}

\begin{assignment}
	\begin{task}
	Break.
	\end{task}
\end{assignment}

\subsection{Assignments}

\begin{assignment}
	\frametitle{Team}
	Teamsize: 1-3

	\begin{task}
	Discuss potential collaboration.
	\end{task}
\end{assignment}

\begin{assignment}
	\frametitle{H1}

	\begin{task}
	Write in the forum which kind of issues you would like to have.
	E.g. using programming language, documentation, testing, ...
	\end{task}

	\begin{task}
	Close your issues if questions were answered.
	\end{task}

	\url{https://issues.libelektra.org/4092}
\end{assignment}

\begin{assignment}
	\frametitle{P1}

	\begin{task}
	What are your current thoughts on the project?
	\end{task}
\end{assignment}

\begin{frame}
	\frametitle{Feedback}
%	Exercises finished for this term.

	\hfill \includegraphics[width=2cm]{pics/feedback.png}
	\vspace{-1cm}
	\begin{itemize}
		\item Feedback Talk
		\item ECTS breakdown realistic?
		\item Git Demo?
		\item Videos?
	\end{itemize}
\end{frame}

\subsection{Preview}

\begin{frame}
	\frametitle{L02 Source Code Management}
\end{frame}



%%%%%%%%%%%%%%%%%%%%%%%%%%%%%%%%%%%%%%%%%% 
\nocite{raab2017introducing}

\appendix

\begin{frame}[allowframebreaks]
	\bibliographystyle{plainnat}
	\bibliography{../shared/elektra.bib}
\end{frame}

\end{document}


