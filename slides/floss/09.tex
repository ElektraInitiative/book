% Make nice A4 pages for print:
%\usepackage{pgfpages}
%\pgfpagesuselayout{resize to}[a4paper,border shrink=5mm,landscape]

\beamertemplatenavigationsymbolsempty

\setbeamertemplate{bibliography item}[text]

\usepackage[type={CC},modifier={by-sa},version={4.0}]{doclicense}

\usepackage[utf8]{inputenc}
\usepackage{hyperref}
\usepackage{breakurl}
\usepackage{graphicx}
\usepackage{pgfplots}
\usepackage{pgf}
\usepackage{tikz}
\usetikzlibrary{positioning}
\usetikzlibrary{arrows}
\usetikzlibrary{decorations.markings}
\usetikzlibrary{calc}
\usetikzlibrary{matrix}
\usetikzlibrary{shapes}
\usetikzlibrary{decorations.pathmorphing}
\usetikzlibrary{fit}
\usetikzlibrary{backgrounds}
\usetikzlibrary{plotmarks}
\usepackage{stmaryrd}
\usepackage{listings}
\usepackage{pdflscape}
\usepackage{perpage}
\usepackage{appendixnumberbeamer}

%\usepackage[thmmarks,amsmath,amsthm]{ntheorem} % already included in beamer
\usepackage{thm-restate}

\usepackage[sort&compress,numbers]{natbib}  % to be have \citet, \citeauthor, \citeyear

\MakePerPage{footnote}

\tikzstyle{o}=[r,ppBlue]
\tikzstyle{r}=[thick,rectangle,align=center]
\tikzstyle{t}=[r,ppTrans] %,font=\bfseries]
\tikzstyle{dd}=[densely dashed]
\tikzstyle{n}=[r,ppBlue]
\tikzstyle{p}=[r,ppRed]
\tikzstyle{ppRed}  =[draw=red,  fill=  red!20]
\tikzstyle{ppBlue} =[draw=blue, fill= blue!20]
\tikzstyle{ppGreen}=[draw=green,fill=green!20]
\tikzstyle{ppTrans}=[draw=none, fill=none]

\usetheme{Warsaw}

\useoutertheme[subsection=true]{smoothbars}
%\useoutertheme[subsection=false]{miniframes}

\definecolor{bblue}{HTML}{D7DF01}	% yellow-ish actually, for better black/white printing
\definecolor{rred}{HTML}{C0504D}
\definecolor{ggreen}{HTML}{9BBB59}
\definecolor{ppurple}{HTML}{9F4C7C}
\definecolor{lightgray}{rgb}{0.3,0.3,0.3}
\definecolor{lightergray}{rgb}{0.9,0.9,0.9}
\definecolor{UniBlue}{RGB}{83,121,170}

\DeclareTextFontCommand\textintro{\normalfont\bfseries\itshape} % nice!
\newcommand{\intro}[2][]
{%
	\textintro{#2}%
}
\newcommand{\empha}[2][]
{%
	\emph{#2}%
}

%\theoremstyle{plain}
\newcounter{reqcounter}
\newtheorem{requirement}[reqcounter]{Requirement}

%setbeamercolor{structure}{fg=violet}

\makeatletter
\def\th@task{%
    \normalfont % body font
    \setbeamercolor{block title example}{bg=orange,fg=white}
    \setbeamercolor{block body example}{bg=orange!20,fg=black}
    \def\inserttheoremblockenv{exampleblock}
  }
\makeatother

\theoremstyle{task}
\newtheorem{task}{Task}

\newenvironment{assignment}%
{%\setbeamercolor{background canvas}{bg=violet}%
%\setbeamercolor{structure}{fg=cyan!90!black}%
 \setbeamercolor{frametitle}{bg=orange,fg=white}
\begin{frame}}%
{\end{frame}}%

\AtBeginSection[]{
  \begin{frame}
  \vfill
  \centering
  \begin{beamercolorbox}[sep=8pt,center,shadow=true,rounded=true]{title}
    \usebeamerfont{title}\insertsectionhead\par%
  \end{beamercolorbox}
  \tableofcontents
  \vfill
  \end{frame}
}




\pgfplotsset{compat=1.14}
\author{Markus Raab}


\title{L09 Architecture}


\begin{document}

%%%%%%%%%%%%%%%%%%%%%%%%%%%%%%%%%%%%%%%%%% 
\section{Directory Structure}

\begin{frame}<1>[label=learning outcomes]
	\frametitle{Learning Outcomes}
	After successful completion of \inserttitle \\
	students will be able to

	\begin{itemize}
	\item reproduce chosen FLOSS software architectures.
	\end{itemize}
\end{frame}

\begin{frame}
	\includegraphics[width=12cm]{Overview}
\end{frame}

\begin{frame}<1-4>[label=overviewability]
	\frametitle{Overviewability}

	\begin{itemize}[<+-| alert@+>]
	\item complementary to traceability \\ (i.e.\ trace from requirements/documentation/issues to affected code and back)
	\item measurement how long a newcomer needs to find her/his way
	\item answers: ``Where to add new functionality?''
	\item in FLOSS directory structure in repository essential (= physical view)
	\end{itemize}
\end{frame}

\begin{frame}
	\frametitle{Grouping of Files}

	\begin{itemize}[<+-| alert@+>]
	\item by programming language
	\item by topics (tests, src, doc)
	\item by modules/plugins
	\end{itemize}
\end{frame}

\begin{frame}
	\frametitle{Metadata of Files and Directories}

	\begin{itemize}[<+-| alert@+>]
	\item README.md
	\item file name endings
	\item LICENSES and \texttt{.license}
	\item \texttt{scripts} vs.\ \texttt{src} (executable bits)
	\end{itemize}
\end{frame}

\begin{frame}
	\frametitle{Cleanup}

	\begin{itemize}[<+-| alert@+>]
	\item top-level
	\item main folders
	\item file names
	\item automatic formatting (encodings and line endings)
	\item reduce dependencies between folders
	\end{itemize}

	\vspace{1cm}

	But also stop worrying if needed by convention.
\end{frame}

\begin{frame}
	\frametitle{Logical Views}

	\begin{itemize}[<+-| alert@+>]
	\item Documentation Generators (``Files'' in Doxygen is physical view)
	\item Building Block View (may be identical to physical)
	\item Runtime View
	\item Deployment View
	\end{itemize}
\end{frame}

%%%%%%%%%%%%%%%%%%%%%%%%%%%%%%%%%%%%%%%%%% 
\section{Software Architecture}

\begin{frame}
	\frametitle{Software Architecture}
	\begin{itemize}[<+-| alert@+>]
	\item architecture is a high-level description of the overall system
	\item use ready-made patterns and templates for architecture
	\item e.g., \url{http://arc42.org/}
	\end{itemize}
\end{frame}

\begin{frame}<1-12>[label=arc42]
	\frametitle{Arc42}
	\begin{enumerate}[<+-| alert@+>]
	\item Introduction and Goals
	\item Constraints
	\item Context and Scope
	\item Solution Strategy
	\item Building Block View
	\item Runtime View
	\item Deployment View
	\item Crosscutting Concepts
	\item Architectural Decisions~\cite{zdun2007patterns}
	\item Quality Requirements
	\item Risks and Technical Debt
	\item Glossary
	\end{enumerate}
\end{frame}

\begin{frame}
	\frametitle{Example}

	Crosscutting concept ``configuration settings'':

	\begin{itemize}
	\item are stored in configuration files
	\item in data structure KeySet
	\item modified by configuration management tool using KeySet
	\end{itemize}

	more about it in course ``configuration management''
\end{frame}

\begin{frame}
	\frametitle{Roles}

	In FLOSS usually nobody is project manager

	\pause

	$\rightarrow$ but everyone is software architect
\end{frame}

\begin{frame}<1-5>[label=goals]
	\frametitle{Goals}

	The most important tasks of software architects are

	\begin{itemize}[<+-| alert@+>]
	\item to pursue the right goals
	\item to have good documentation (e.g.\ with arc42)
	\item to keep everything as simple as possible
	\item to communicate the architecture
	\item maintain community and quality
	\end{itemize}
\end{frame}

\begin{frame}
	\frametitle{Refactoring}

	\begin{itemize}
	\item build what community needs at the moment
	\item change according to current needs
	\item avoid over-engineering, refactor to KISS
	\end{itemize}
\end{frame}


%%%%%%%%%%%%%%%%%%%%%%%%%%%%%%%%%%%%%%%%%% 
\section{Architectural Decisions}

\begin{frame}<1-3>[label=architectural decisions]
	\frametitle{Architectural Decisions}
	\begin{itemize}[<+-| alert@+>]
	\item describe decisions that lead to the architecture
	\item decisions are high-level configuration
	\item patterns/templates are useful~\cite{zdun2007patterns}, e.g.:
	\end{itemize}
\end{frame}

\begin{frame}
	\frametitle{Template}
	\begin{enumerate}[<+-| alert@+>]
	\item problem
	\item constraints
	\item assumptions
	\item considered alternatives
	\item decision
	\item rationale
	\item implications
	\item related
	\item notes
	\end{enumerate}
\end{frame}

\begin{frame}
	\frametitle{Example: API Design}

	\begin{itemize}[<+-| alert@+>]
	\item future-proof
	\item hard to use it wrong vs.\ easy to use
	\item consistent concepts, e.g.\ for resources
	\item minimal vs.\ comfort
	\end{itemize}
\end{frame}

\begin{frame}
	\frametitle{Example: Libraries vs.\ Daemons}

	\begin{itemize}[<+-| alert@+>]
	\item both foster reuse of code
	\item daemon better if there is dynamic state
	\item but: daemon creates a single point of failure (KISS)
	\end{itemize}
\end{frame}

\begin{frame}
	\frametitle{Dangers}

	\begin{quote}
	Insanity in individuals is something rare -- but in groups, parties, nations, and epochs, it is the rule.
	-- Friedrich Nietzsche
	\end{quote}

	In groups you get confronted with the whole spectrum of psychology:

	\pause

	\begin{itemize}[<+-| alert@+>]
	\item Groupthink (conformity)
	\item Group polarization
	\end{itemize}
\end{frame}



%%%%%%%%%%%%%%%%%%%%%%%%%%%%%%%%%%%%%%%%%%
\section{Meeting}

\subsection{Recapitulation}

\begin{frame}
	\includegraphics[width=12cm]{Overview}
\end{frame}

\againframe<10>{learning outcomes}

\begin{assignment}
	\frametitle{Overviewability}

	What is traceability and what is overviewability?
\end{assignment}

\againframe<10>{overviewability}

\begin{assignment}
	\frametitle{Discussion: How to Group Files?}

	\begin{itemize}
	\item by programming language/projects (monorepo)
	\item by topics (tests, src, doc)
	\item by modules/plugins
	\end{itemize}

	\begin{task}
	How do you like files to be structured? \\
	How do you keep order over time?
	\end{task}
\end{assignment}

\breakframe

\againframe<99>{arc42}

\begin{assignment}
	\frametitle{Goals}

	\begin{task}
	What are the most important tasks of software architects?
	\end{task}
\end{assignment}

\againframe<99>{goals}

\begin{assignment}
	\frametitle{Architectural Decisions}

	\begin{task}
	What are architectural decisions? \\
	Please given an example.
	\end{task}
\end{assignment}

\againframe<99>{architectural decisions}

\begin{assignment}
	\frametitle{Reading Text}

	\begin{task}
	Summarize what you found interesting.
	\end{task}
\end{assignment}

\breakframe


\subsection{Assignments}

\begin{frame}
	\frametitle{P1 Reviews, T3 Corrections}

	\begin{task}
	Get all your PRs ready to merge.
	\end{task}
\end{frame}

\begin{frame}
	\frametitle{Skip last meeting?}

	\begin{itemize}
	\item two days earlier deadline (16.1.2023 23:59)
	\item maybe overtime in last meeting (18.1.2023 14:15)
	\end{itemize}

	\begin{task}
	Vote: Skip last meeting?
	\end{task}
\end{frame}

\begin{frame}
	\frametitle{Feedback}
	Today home and teamwork will finish for this term.

	\hfill \includegraphics[width=2cm]{pics/feedback.png}
	\vspace{-1cm}
	\begin{itemize}
		\item Feedback Talk
		\item ECTS breakdown realistic?
%		\item TISS Feedback from 16.06.2021 00:00 to 14.07.2021 23:59
	\end{itemize}
\end{frame}

\subsection{Preview}

\begin{frame}
	\frametitle{L10 Open Standards}
\end{frame}

%%%%%%%%%%%%%%%%%%%%%%%%%%%%%%%%%%%%%%%%%% 
\appendix

\begin{frame}[allowframebreaks]
	\bibliographystyle{plainnat}
	\bibliography{../shared/elektra.bib}
\end{frame}


\end{document}
