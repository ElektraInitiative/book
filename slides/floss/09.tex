%\ifdefined\handout
%\documentclass[handout,aspectratio=1610,xcolor={usenames,dvipsnames,table}]{beamer}
%\else
\documentclass[aspectratio=1610,xcolor={usenames,dvipsnames,table}]{beamer}
%\fi

\newcommand{\mylecture}{Configuration Management}

%\ifdefined\handout
%\documentclass[handout,aspectratio=1610,xcolor={usenames,dvipsnames,table}]{beamer}
%\else
\documentclass[aspectratio=1610,xcolor={usenames,dvipsnames,table}]{beamer}
%\fi

\newcommand{\mylecture}{Configuration Management}

%\ifdefined\handout
%\documentclass[handout,aspectratio=1610,xcolor={usenames,dvipsnames,table}]{beamer}
%\else
\documentclass[aspectratio=1610,xcolor={usenames,dvipsnames,table}]{beamer}
%\fi

\newcommand{\mylecture}{Configuration Management}

\input{../setup}
\input{../shared/setup}

\lstdefinelanguage{dump}
{
	morekeywords={kdbOpen,ksNew,keyNew,keyMeta,keyCopyMeta,keyEnd,ksEnd,kdbClose},
	sensitive=false,
	morecomment=[l]{//},
	morecomment=[s]{/*}{*/},
	morestring=[b]",
}


\lstdefinelanguage{SpecElektra}{
	%
	comment=[l]{;},
	commentstyle=\color{purple}\ttfamily,
	%
	morestring=[b]',
	morestring=[b]`,
	morestring=[b]",
	stringstyle=\color{purple}\ttfamily,
	%
	sensitive=f,% keywords are not case sensitive
	%
	% Colors see https://en.wikibooks.org/wiki/LaTeX/Colors
	%
	keywordstyle=\color{BlueViolet}\bfseries,
	keywordstyle=[2]\color{Green},
	keywordstyle=[3]\color{Aquamarine}\bfseries\textit,
	keywordstyle=[4]\color{NavyBlue}\bfseries,
	keywordstyle=[5]\color{Mahogany},
	%
	keywords={layer, require, validation, check, range, description, rationale, requirement, accessibility, enable, condition, message, default, opt, readonly, type, context, property1, property2, description, file, content, mountpoint, metadata, infos, plugins},
	keywords=[2]{},
	keywords=[3]{order, interface, network, emphasized},
	%keywords=[4]{[, ]},  %Not needed
	keywords=[4]{},
	keywords=[5]{},
	%
	literate={:=}{{{\color{red}\textbf:=}}}2
		 {\%}{{{\color{NavyBlue}\textbf\%}}}1
		 {[}{{{\color{Sepia}\textbf[}}}1
		 {]}{{{\color{Sepia}\textbf]}}}1,
}

\lstdefinelanguage{Cpp}{%
	language     = C++,
	literate=
}


\lstdefinelanguage{CfgElektra}{
	comment=[l]{;},
	commentstyle=\color{purple}\ttfamily,
	%
	morestring=[b]',
	morestring=[b]`,
	morestring=[b]",
	stringstyle=\color{purple}\ttfamily,
	%
	%
	sensitive=f,% keywords are not case sensitive
	%
	% Colors see https://en.wikibooks.org/wiki/LaTeX/Colors
	%
	keywordstyle=\color{Bittersweet}\bfseries,
	keywordstyle=[2]\color{DarkOrchid}\bfseries,
	keywordstyle=[3]\color{ForestGreen}\bfseries\textit,
	keywordstyle=[4]\color{Goldenrod}\bfseries,
	keywordstyle=[5]\color{CarnationPink},
	%
	keywords={},
	keywords=[2]{},
	keywords=[3]{},
	keywords=[4]{},
	keywords=[5]{},
	%
	literate={=}{{{\color{ForestGreen}\textbf=}}}1
		 %{<-}{{{\color{ForestGreen}\textbf<-}}}2
		 %{*}{{{\color{Bittersweet}\textbf*}}}1
		 {\%}{{{\color{NavyBlue}\textbf\%}}}1,
}




\lstset{language=SpecElektra, % Use SpecElektra as default programming language
	%boxpos=t, % make boxes a bit more unbreakable
	%frame=lines, % top+bottom line
	basicstyle=\ttfamily, % Use normal-size true type font
	showspaces,%
	showstringspaces=false, % Don't put marks in string spaces
	showlines=true, % make sure empty lines at end are shown (needed for concurrency
	tabsize=4, % spaces per tab
	xleftmargin=\parindent, % should be 18pt or 1.5em as defined by memoir
	%Does not really work well (needs to be deactivated for shortlistings):
	breaklines=false,
	%postbreak=\mbox{\textcolor{red}{$\hookrightarrow$}\space},
	%breakautoindent=true,
	%prebreak={\mbox{\ensuremath{\curvearrowright}}} % Zeichen am Zeilenende (Umbruch)
	%breaklines=true,
	%breakautoindent=true,
	%prebreak=\small\symbol{'134}, % backslash
	%prebreak={\mbox{\ensuremath{\curvearrowright}}} % lange kure
	%prebreak={\mbox{\ensuremath{\hookleftarrow}}} % lange kure
	%xleftmargin=3.0ex, %for some formats
	%xrightmargin=1.0ex, %for some formats
	%
	% Files do not work in utf8 see also:
	% http://stackoverflow.com/questions/1116266/listings-in-latex-with-utf-8-or-at-least-german-umlauts
	% http://tex.stackexchange.com/questions/24528/having-problems-with-listings-and-utf-8-can-it-be-fixed
	% Should work but doesn't? (Maybe add to literate broken?)
	%add to literate={ö}{{\"o}}1
	%	{ä}{{\"a}}1
	%	{ü}{{\"u}}1
	%	{Ö}{{\"O}}1
	%	{Ä}{{\"A}}1
	%	{Ü}{{\"U}}1
	%	{ß}{{\ss}}1,
	%
	% listingsutf8 did not work, made umlauts in comments very strange
	%extendedchars=true,
	%inputencoding=utf8,
	%
	%morecomment=[l][\color{blue}]{...}, % Line continuation (...) e.g. blue comment
	morekeywords={for_each},
	numbers=left, % Line numbers on left
	firstnumber=1, % Line numbers start with line 1
	numberstyle=\small\color{blue}, % Line numbers are blue and small
	numbersep=5pt,
	%stepnumber=5 % Line numbers go in steps of 5
}



\lstMakeShortInline[postbreak=,keywordstyle={}]^

\graphicspath{{../pic/}{../figures/}{../graphics/}{../ipe/}{../ggplot/}}



\lstdefinelanguage{dump}
{
	morekeywords={kdbOpen,ksNew,keyNew,keyMeta,keyCopyMeta,keyEnd,ksEnd,kdbClose},
	sensitive=false,
	morecomment=[l]{//},
	morecomment=[s]{/*}{*/},
	morestring=[b]",
}


\lstdefinelanguage{SpecElektra}{
	%
	comment=[l]{;},
	commentstyle=\color{purple}\ttfamily,
	%
	morestring=[b]',
	morestring=[b]`,
	morestring=[b]",
	stringstyle=\color{purple}\ttfamily,
	%
	sensitive=f,% keywords are not case sensitive
	%
	% Colors see https://en.wikibooks.org/wiki/LaTeX/Colors
	%
	keywordstyle=\color{BlueViolet}\bfseries,
	keywordstyle=[2]\color{Green},
	keywordstyle=[3]\color{Aquamarine}\bfseries\textit,
	keywordstyle=[4]\color{NavyBlue}\bfseries,
	keywordstyle=[5]\color{Mahogany},
	%
	keywords={layer, require, validation, check, range, description, rationale, requirement, accessibility, enable, condition, message, default, opt, readonly, type, context, property1, property2, description, file, content, mountpoint, metadata, infos, plugins},
	keywords=[2]{},
	keywords=[3]{order, interface, network, emphasized},
	%keywords=[4]{[, ]},  %Not needed
	keywords=[4]{},
	keywords=[5]{},
	%
	literate={:=}{{{\color{red}\textbf:=}}}2
		 {\%}{{{\color{NavyBlue}\textbf\%}}}1
		 {[}{{{\color{Sepia}\textbf[}}}1
		 {]}{{{\color{Sepia}\textbf]}}}1,
}

\lstdefinelanguage{Cpp}{%
	language     = C++,
	literate=
}


\lstdefinelanguage{CfgElektra}{
	comment=[l]{;},
	commentstyle=\color{purple}\ttfamily,
	%
	morestring=[b]',
	morestring=[b]`,
	morestring=[b]",
	stringstyle=\color{purple}\ttfamily,
	%
	%
	sensitive=f,% keywords are not case sensitive
	%
	% Colors see https://en.wikibooks.org/wiki/LaTeX/Colors
	%
	keywordstyle=\color{Bittersweet}\bfseries,
	keywordstyle=[2]\color{DarkOrchid}\bfseries,
	keywordstyle=[3]\color{ForestGreen}\bfseries\textit,
	keywordstyle=[4]\color{Goldenrod}\bfseries,
	keywordstyle=[5]\color{CarnationPink},
	%
	keywords={},
	keywords=[2]{},
	keywords=[3]{},
	keywords=[4]{},
	keywords=[5]{},
	%
	literate={=}{{{\color{ForestGreen}\textbf=}}}1
		 %{<-}{{{\color{ForestGreen}\textbf<-}}}2
		 %{*}{{{\color{Bittersweet}\textbf*}}}1
		 {\%}{{{\color{NavyBlue}\textbf\%}}}1,
}




\lstset{language=SpecElektra, % Use SpecElektra as default programming language
	%boxpos=t, % make boxes a bit more unbreakable
	%frame=lines, % top+bottom line
	basicstyle=\ttfamily, % Use normal-size true type font
	showspaces,%
	showstringspaces=false, % Don't put marks in string spaces
	showlines=true, % make sure empty lines at end are shown (needed for concurrency
	tabsize=4, % spaces per tab
	xleftmargin=\parindent, % should be 18pt or 1.5em as defined by memoir
	%Does not really work well (needs to be deactivated for shortlistings):
	breaklines=false,
	%postbreak=\mbox{\textcolor{red}{$\hookrightarrow$}\space},
	%breakautoindent=true,
	%prebreak={\mbox{\ensuremath{\curvearrowright}}} % Zeichen am Zeilenende (Umbruch)
	%breaklines=true,
	%breakautoindent=true,
	%prebreak=\small\symbol{'134}, % backslash
	%prebreak={\mbox{\ensuremath{\curvearrowright}}} % lange kure
	%prebreak={\mbox{\ensuremath{\hookleftarrow}}} % lange kure
	%xleftmargin=3.0ex, %for some formats
	%xrightmargin=1.0ex, %for some formats
	%
	% Files do not work in utf8 see also:
	% http://stackoverflow.com/questions/1116266/listings-in-latex-with-utf-8-or-at-least-german-umlauts
	% http://tex.stackexchange.com/questions/24528/having-problems-with-listings-and-utf-8-can-it-be-fixed
	% Should work but doesn't? (Maybe add to literate broken?)
	%add to literate={ö}{{\"o}}1
	%	{ä}{{\"a}}1
	%	{ü}{{\"u}}1
	%	{Ö}{{\"O}}1
	%	{Ä}{{\"A}}1
	%	{Ü}{{\"U}}1
	%	{ß}{{\ss}}1,
	%
	% listingsutf8 did not work, made umlauts in comments very strange
	%extendedchars=true,
	%inputencoding=utf8,
	%
	%morecomment=[l][\color{blue}]{...}, % Line continuation (...) e.g. blue comment
	morekeywords={for_each},
	numbers=left, % Line numbers on left
	firstnumber=1, % Line numbers start with line 1
	numberstyle=\small\color{blue}, % Line numbers are blue and small
	numbersep=5pt,
	%stepnumber=5 % Line numbers go in steps of 5
}



\lstMakeShortInline[postbreak=,keywordstyle={}]^

\graphicspath{{../pic/}{../figures/}{../graphics/}{../ipe/}{../ggplot/}}




\title{L09 Architecture}


\begin{document}

%%%%%%%%%%%%%%%%%%%%%%%%%%%%%%%%%%%%%%%%%% 
\section{Directory Structure}

\begin{frame}<1>[label=learning outcomes]
	\frametitle{Learning Outcomes}
	After successful completion of \inserttitle \\
	students will be able to

	\begin{itemize}
	\item reproduce chosen FLOSS software architectures.
	\end{itemize}
\end{frame}

\begin{frame}
	\includegraphics[width=12cm]{Overview}
\end{frame}

\begin{frame}<1-4>[label=overviewability]
	\frametitle{Overviewability}

	\begin{itemize}[<+-| alert@+>]
	\item complementary to traceability \\ (i.e.\ trace from requirements/documentation/issues to affected code and back)
	\item measurement how long a newcomer needs to find her/his way
	\item answers: ``Where to add new functionality?''
	\item in FLOSS directory structure in repository essential (= physical view)
	\end{itemize}
\end{frame}

\begin{frame}
	\frametitle{Grouping of Files}

	\begin{itemize}[<+-| alert@+>]
	\item by programming language
	\item by topics (tests, src, doc)
	\item by modules/plugins
	\end{itemize}
\end{frame}

\begin{frame}
	\frametitle{Metadata of Files and Directories}

	\begin{itemize}[<+-| alert@+>]
	\item README.md
	\item file name endings
	\item LICENSES and \texttt{.license}
	\item \texttt{scripts} vs.\ \texttt{src} (executable bits)
	\end{itemize}
\end{frame}

\begin{frame}
	\frametitle{Cleanup}

	\begin{itemize}[<+-| alert@+>]
	\item top-level
	\item main folders
	\item file names
	\item automatic formatting (encodings and line endings)
	\item reduce dependencies between folders
	\end{itemize}

	\vspace{1cm}

	But also stop worrying if needed by convention.
\end{frame}

\begin{frame}
	\frametitle{Logical Views}

	\begin{itemize}[<+-| alert@+>]
	\item Documentation Generators (``Files'' in Doxygen is physical view)
	\item Building Block View (may be identical to physical)
	\item Runtime View
	\item Deployment View
	\end{itemize}
\end{frame}

%%%%%%%%%%%%%%%%%%%%%%%%%%%%%%%%%%%%%%%%%% 
\section{Software Architecture}

\begin{frame}
	\frametitle{Software Architecture}
	\begin{itemize}[<+-| alert@+>]
	\item architecture is a high-level description of the overall system
	\item use ready-made patterns and templates for architecture
	\item e.g., \url{http://arc42.org/}
	\end{itemize}
\end{frame}

\begin{frame}<1-12>[label=arc42]
	\frametitle{Arc42}
	\begin{enumerate}[<+-| alert@+>]
	\item Introduction and Goals
	\item Constraints
	\item Context and Scope
	\item Solution Strategy
	\item Building Block View
	\item Runtime View
	\item Deployment View
	\item Crosscutting Concepts
	\item Architectural Decisions~\cite{zdun2007patterns}
	\item Quality Requirements
	\item Risks and Technical Debt
	\item Glossary
	\end{enumerate}
\end{frame}

\begin{frame}
	\frametitle{Example}

	Crosscutting concept ``configuration settings'':

	\begin{itemize}
	\item are stored in configuration files
	\item in data structure KeySet
	\item modified by configuration management tool using KeySet
	\end{itemize}

	more about it in course ``configuration management''
\end{frame}

\begin{frame}
	\frametitle{Roles}

	In FLOSS usually nobody is project manager

	\pause

	$\rightarrow$ but everyone is software architect
\end{frame}

\begin{frame}<1-5>[label=goals]
	\frametitle{Goals}

	The most important tasks of software architects are

	\begin{itemize}[<+-| alert@+>]
	\item to pursue the right goals
	\item to have good documentation (e.g.\ with arc42)
	\item to keep everything as simple as possible
	\item to communicate the architecture
	\item maintain community and quality
	\end{itemize}
\end{frame}

\begin{frame}
	\frametitle{Refactoring}

	\begin{itemize}
	\item build what community needs at the moment
	\item change according to current needs
	\item avoid over-engineering, refactor to KISS
	\end{itemize}
\end{frame}


%%%%%%%%%%%%%%%%%%%%%%%%%%%%%%%%%%%%%%%%%% 
\section{Architectural Decisions}

\begin{frame}<1-3>[label=architectural decisions]
	\frametitle{Architectural Decisions}
	\begin{itemize}[<+-| alert@+>]
	\item describe decisions that lead to the architecture
	\item decisions are high-level configuration
	\item patterns/templates are useful~\cite{zdun2007patterns}, e.g.:
	\end{itemize}
\end{frame}

\begin{frame}
	\frametitle{Template}
	\begin{enumerate}[<+-| alert@+>]
	\item problem
	\item constraints
	\item assumptions
	\item considered alternatives
	\item decision
	\item rationale
	\item implications
	\item related
	\item notes
	\end{enumerate}
\end{frame}

\begin{frame}
	\frametitle{Example: API Design}

	\begin{itemize}[<+-| alert@+>]
	\item future-proof
	\item hard to use it wrong vs.\ easy to use
	\item consistent concepts, e.g.\ for resources
	\item minimal vs.\ comfort
	\end{itemize}
\end{frame}

\begin{frame}
	\frametitle{Example: Libraries vs.\ Daemons}

	\begin{itemize}[<+-| alert@+>]
	\item both foster reuse of code
	\item daemon better if there is dynamic state
	\item but: daemon creates a single point of failure (KISS)
	\end{itemize}
\end{frame}

\begin{frame}
	\frametitle{Dangers}

	\begin{quote}
	Insanity in individuals is something rare -- but in groups, parties, nations, and epochs, it is the rule.
	-- Friedrich Nietzsche
	\end{quote}

	In groups you get confronted with the whole spectrum of psychology:

	\pause

	\begin{itemize}[<+-| alert@+>]
	\item Groupthink (conformity)
	\item Group polarization
	\end{itemize}
\end{frame}



%%%%%%%%%%%%%%%%%%%%%%%%%%%%%%%%%%%%%%%%%%
\section{Meeting}

\subsection{Recapitulation}

\begin{frame}
	\includegraphics[width=12cm]{Overview}
\end{frame}

\againframe<10>{learning outcomes}

\begin{assignment}
	\frametitle{Overviewability}

	What is traceability and what is overviewability?
\end{assignment}

\againframe<10>{overviewability}

\begin{assignment}
	\frametitle{Discussion: How to Group Files?}

	\begin{itemize}
	\item by programming language/projects (monorepo)
	\item by topics (tests, src, doc)
	\item by modules/plugins
	\end{itemize}

	\begin{task}
	How do you like files to be structured? \\
	How do you keep order over time?
	\end{task}
\end{assignment}

\breakframe

\begin{frame}
	\frametitle{P09 /e/OS}

	Dennis Toth
\end{frame}

\againframe<99>{arc42}

\begin{assignment}
	\frametitle{Goals}

	\begin{task}
	What are the most important tasks of software architects?
	\end{task}
\end{assignment}

\againframe<99>{goals}

\begin{assignment}
	\frametitle{Architectural Decisions}

	\begin{task}
	What are architectural decisions? \\
	Please given an example.
	\end{task}
\end{assignment}

\againframe<99>{architectural decisions}

\begin{assignment}
	\frametitle{Decision Process}
\end{assignment}

\breakframe

\begin{frame}
	\frametitle{P09 ESPHome}

	Jan de Boer
\end{frame}

\begin{assignment}
	\frametitle{Reading Text}

	\begin{task}
	Summarize what you found interesting.
	\end{task}
\end{assignment}

\breakframe

\subsection{Assignments}

\begin{frame}
	\frametitle{P1 Reviews, T3 Corrections}

	\begin{task}
	Get all your PRs ready to merge and \\
	ask for them to be merged (Label)
	\end{task}
\end{frame}

\begin{frame}
	\frametitle{Feedback}
	Today home and teamwork will finish for this term.

	\hfill \includegraphics[width=2cm]{pics/feedback.png}
	\vspace{-1cm}
	\begin{itemize}[<+-| alert@+>]
		\item \href{https://tuwel.tuwien.ac.at/mod/feedback/view.php?id=1661074}{TISS}/
			\href{https://tiss.tuwien.ac.at/survey/surveyForm.xhtml?courseNumber=194114&semesterCode=2022W}{TUWEL}
			Feedback from 12.01.2023, 00:00 to 9.02.2023, 23:59
		\item filling out before 18.01 supports continuation of this lecture
		\item TUWEL: feedback for future improvements, TISS: for TU Wien
	\end{itemize}
\end{frame}

\begin{frame}
	\frametitle{Skip last meeting?}

	\begin{itemize}
	\item two days earlier deadline (16.1.2023 23:59)
	\item full use of last meeting (18.1.2023 14:15 -- 16:00)
	\end{itemize}

	\begin{task}
	Vote: Skip last meeting at 25.01.2023?
	\end{task}
\end{frame}

\subsection{Preview}

\begin{frame}
	\frametitle{L10 Open Standards}
\end{frame}

%%%%%%%%%%%%%%%%%%%%%%%%%%%%%%%%%%%%%%%%%% 
\appendix

\begin{frame}[allowframebreaks]
	\bibliographystyle{plainnat}
	\bibliography{../shared/elektra.bib}
\end{frame}


\end{document}
