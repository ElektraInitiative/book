% Make nice A4 pages for print:
%\usepackage{pgfpages}
%\pgfpagesuselayout{resize to}[a4paper,border shrink=5mm,landscape]

\beamertemplatenavigationsymbolsempty

\setbeamertemplate{bibliography item}[text]

\usepackage[type={CC},modifier={by-sa},version={4.0}]{doclicense}

\usepackage[utf8]{inputenc}
\usepackage{hyperref}
\usepackage{breakurl}
\usepackage{graphicx}
\usepackage{pgfplots}
\usepackage{pgf}
\usepackage{tikz}
\usetikzlibrary{positioning}
\usetikzlibrary{arrows}
\usetikzlibrary{decorations.markings}
\usetikzlibrary{calc}
\usetikzlibrary{matrix}
\usetikzlibrary{shapes}
\usetikzlibrary{decorations.pathmorphing}
\usetikzlibrary{fit}
\usetikzlibrary{backgrounds}
\usetikzlibrary{plotmarks}
\usepackage{stmaryrd}
\usepackage{listings}
\usepackage{pdflscape}
\usepackage{perpage}
\usepackage{appendixnumberbeamer}

%\usepackage[thmmarks,amsmath,amsthm]{ntheorem} % already included in beamer
\usepackage{thm-restate}

\usepackage[sort&compress,numbers]{natbib}  % to be have \citet, \citeauthor, \citeyear

\MakePerPage{footnote}

\tikzstyle{o}=[r,ppBlue]
\tikzstyle{r}=[thick,rectangle,align=center]
\tikzstyle{t}=[r,ppTrans] %,font=\bfseries]
\tikzstyle{dd}=[densely dashed]
\tikzstyle{n}=[r,ppBlue]
\tikzstyle{p}=[r,ppRed]
\tikzstyle{ppRed}  =[draw=red,  fill=  red!20]
\tikzstyle{ppBlue} =[draw=blue, fill= blue!20]
\tikzstyle{ppGreen}=[draw=green,fill=green!20]
\tikzstyle{ppTrans}=[draw=none, fill=none]

\usetheme{Warsaw}

\useoutertheme[subsection=true]{smoothbars}
%\useoutertheme[subsection=false]{miniframes}

\definecolor{bblue}{HTML}{D7DF01}	% yellow-ish actually, for better black/white printing
\definecolor{rred}{HTML}{C0504D}
\definecolor{ggreen}{HTML}{9BBB59}
\definecolor{ppurple}{HTML}{9F4C7C}
\definecolor{lightgray}{rgb}{0.3,0.3,0.3}
\definecolor{lightergray}{rgb}{0.9,0.9,0.9}
\definecolor{UniBlue}{RGB}{83,121,170}

\DeclareTextFontCommand\textintro{\normalfont\bfseries\itshape} % nice!
\newcommand{\intro}[2][]
{%
	\textintro{#2}%
}
\newcommand{\empha}[2][]
{%
	\emph{#2}%
}

%\theoremstyle{plain}
\newcounter{reqcounter}
\newtheorem{requirement}[reqcounter]{Requirement}

%setbeamercolor{structure}{fg=violet}

\makeatletter
\def\th@task{%
    \normalfont % body font
    \setbeamercolor{block title example}{bg=orange,fg=white}
    \setbeamercolor{block body example}{bg=orange!20,fg=black}
    \def\inserttheoremblockenv{exampleblock}
  }
\makeatother

\theoremstyle{task}
\newtheorem{task}{Task}

\newenvironment{assignment}%
{%\setbeamercolor{background canvas}{bg=violet}%
%\setbeamercolor{structure}{fg=cyan!90!black}%
 \setbeamercolor{frametitle}{bg=orange,fg=white}
\begin{frame}}%
{\end{frame}}%

\AtBeginSection[]{
  \begin{frame}
  \vfill
  \centering
  \begin{beamercolorbox}[sep=8pt,center,shadow=true,rounded=true]{title}
    \usebeamerfont{title}\insertsectionhead\par%
  \end{beamercolorbox}
  \tableofcontents
  \vfill
  \end{frame}
}




\pgfplotsset{compat=1.14}
\author{Markus Raab}


\title{L07 Maintenance}

\begin{document}

%%%%%%%%%%%%%%%%%%%%%%%%%%%%%%%%%%%%%%%%%% 
\section{Maintenance}

\begin{frame}<1>[label=learning outcomes]
	\frametitle{Learning Outcomes}
	After successful completion of L07 \\
	students will be able to

	\begin{itemize}
	\item maintain FLOSS initiatives
	\end{itemize}
\end{frame}

\begin{frame}<1-3>[label=releases]
	\frametitle{Releases}

	\begin{itemize}[<+-| alert@+>]
	\item main communication to the ``outside world''
	\item tag which version to use (e.g. when some feature is ready)
	\item points in time when to test, package and announce
	\end{itemize}
\end{frame}

\begin{frame}<1-5>[label=release notes]
	\frametitle{Release Notes}

	\begin{itemize}[<+-| alert@+>]
	\item communicate what changed
	\item say thanks to contributors
	\item give incentive to try out the latest version
	\item give maintainers hints how to package
	\item communicate API changes to developers
	\end{itemize}
\end{frame}

\begin{frame}
	\frametitle{Release Schema}

	If possible, avoid them, e.g. use date as release number. \\
	If you want to convey some semantic meaning, like:

	\begin{itemize}[<+-| alert@+>]
	\item compatibility in APIs (major.minor.micro)
	\item stable/unstable branches (even/odd, alpha/beta) or LTS
	\end{itemize}

	\pause[\thebeamerpauses]  %  show after \begin{itemize}[<+->]

	then a different release schema might make sense.
\end{frame}

\begin{frame}<1-4>[label=licensing]
	\frametitle{Licensing}

	\begin{itemize}[<+-| alert@+>]
	\item choose one license for your whole initiative
	\item if you copy code from others:
	\begin{itemize}
	\item it either has the same license
	\item it has a compatible license AND you document all licenses you use (reuse.software)
	\end{itemize}
	\end{itemize}
\end{frame}

\begin{frame}
	\frametitle{Source Releases}

	\begin{itemize}[<+-| alert@+>]
	\item tar the source code
	\item sign the tar ball
	\end{itemize}

	\pause[\thebeamerpauses]  %  show after \begin{itemize}[<+->]
	\vspace{1cm}

	$\rightarrow$ gets less important \\
	distributions now also pull from version control
\end{frame}

\begin{frame}
	\frametitle{Binary Releases}

	important if you need to directly publish to end users:

	\begin{itemize}[<+-| alert@+>]
	\item compile on some old distributions and tar the binaries
	\item flatpack, homebrew, 0install
%	\item GNU Guix, Nix % binary?, not important here
	\end{itemize}
\end{frame}

\begin{frame}<1-5>[label=linux distributions]
	\frametitle{Linux Distributions}

	\begin{itemize}[<+-| alert@+>]
	\item packages maintained by the distributions
	\item can be considerable effort, so it is good if you can help by reporting:
	\begin{itemize}
	\item which licenses are used (reuse)
	\item which dependencies are needed
	\item which files are installed
	\end{itemize}
	\end{itemize}
\end{frame}

\begin{frame}
	\frametitle{Continuous Releases}

	Automate:

	\begin{itemize}[<+-| alert@+>]
	\item which installed files were added/removed
	\item publishing release notes on web-site
	\item creating and publishing packages
	\item tests
	\item generate version-specific information (like tested compilers)
	\end{itemize}
\end{frame}

%%%%%%%%%%%%%%%%%%%%%%%%%%%%%%%%%%%%%%%%% %
\section{Meeting}

\subsection{Recapitulation}

\againframe<10>{learning outcomes}

\begin{frame}
	\frametitle{Releases}

	\begin{task}
	What purpose do releases have?
	\end{task}
\end{frame}

\againframe<10>{releases}

\begin{frame}
	\frametitle{Release Notes}

	\begin{task}
	What purpose do release notes have?
	\end{task}
\end{frame}

\againframe<10>{release notes}

\begin{frame}
	\frametitle{Release Notes}

	\begin{task}
	Brainstorm how good release notes should be written.
	\end{task}
\end{frame}

\breakframe

\begin{frame}
	\frametitle{Licensing}

	\begin{task}
	What needs to be considered in regards to licensing?
	\end{task}
\end{frame}

\againframe<10>{licensing}

\begin{frame}
	\frametitle{Licensing}

	\begin{task}
	Which license do you (personally) prefer?
	\end{task}
\end{frame}

\breakframe

\begin{frame}
	\frametitle{Linux Distributions}

	\begin{task}
	Who maintains packages in Linux distributions? \\
	How can you help as FLOSS maintainer?
	\end{task}
\end{frame}

\againframe<10>{linux distributions}

\begin{frame}
	\frametitle{Linux Distributions}

	\begin{task}
	Which FLOSS operating system did you try? \\
	What was your experience?
	\end{task}
\end{frame}

\subsection{Assignments}

\begin{frame}
	\frametitle{ready to merge}

	set label ``ready to merge'' for all your PRs iff

	\begin{itemize}
	\item build server is happy
	\item you ticked all relevant points in PR
	\end{itemize}

	ping me (@markus2330) on any questions
\end{frame}

\begin{frame}
	\frametitle{H3+T2 reviews}

	\begin{itemize}
	\item actually do testing (e.g. run shell recorder of PR)
	\item write what exactly you did (e.g. post output of your run)
	\end{itemize}
\end{frame}

\begin{frame}
	\frametitle{P2}

	\begin{itemize}
	\item documentation
	\item entry barriers
	\item literature research
	\end{itemize}
\end{frame}

%\begin{frame}
%	\frametitle{Feedback}
%
%	\hfill \includegraphics[width=2cm]{pics/feedback.png}
%	\vspace{-1cm}
%	\begin{itemize}
%		\item Feedback Talk
%		\item ECTS breakdown realistic?
%		\item TISS Feedback from 16.06.2021 00:00 to 14.07.2021 23:59
%	\end{itemize}
%\end{frame}

\subsection{Preview}

\begin{frame}
	\frametitle{L08 Collaboration}

	before holidays

	with Rhonda
\end{frame}


%%%%%%%%%%%%%%%%%%%%%%%%%%%%%%%%%%%%%%%%%% 
\appendix

\begin{frame}[allowframebreaks]
	\bibliographystyle{plainnat}
	\bibliography{../shared/elektra.bib}
\end{frame}


\end{document}

