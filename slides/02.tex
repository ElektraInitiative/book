% Make nice A4 pages for print:
%\usepackage{pgfpages}
%\pgfpagesuselayout{resize to}[a4paper,border shrink=5mm,landscape]

\beamertemplatenavigationsymbolsempty

\setbeamertemplate{bibliography item}[text]

\usepackage[type={CC},modifier={by-sa},version={4.0}]{doclicense}

\usepackage[utf8]{inputenc}
\usepackage{hyperref}
\usepackage{breakurl}
\usepackage{graphicx}
\usepackage{pgfplots}
\usepackage{pgf}
\usepackage{tikz}
\usetikzlibrary{positioning}
\usetikzlibrary{arrows}
\usetikzlibrary{decorations.markings}
\usetikzlibrary{calc}
\usetikzlibrary{matrix}
\usetikzlibrary{shapes}
\usetikzlibrary{decorations.pathmorphing}
\usetikzlibrary{fit}
\usetikzlibrary{backgrounds}
\usetikzlibrary{plotmarks}
\usepackage{stmaryrd}
\usepackage{listings}
\usepackage{pdflscape}
\usepackage{perpage}
\usepackage{appendixnumberbeamer}

%\usepackage[thmmarks,amsmath,amsthm]{ntheorem} % already included in beamer
\usepackage{thm-restate}

\usepackage[sort&compress,numbers]{natbib}  % to be have \citet, \citeauthor, \citeyear

\MakePerPage{footnote}

\tikzstyle{o}=[r,ppBlue]
\tikzstyle{r}=[thick,rectangle,align=center]
\tikzstyle{t}=[r,ppTrans] %,font=\bfseries]
\tikzstyle{dd}=[densely dashed]
\tikzstyle{n}=[r,ppBlue]
\tikzstyle{p}=[r,ppRed]
\tikzstyle{ppRed}  =[draw=red,  fill=  red!20]
\tikzstyle{ppBlue} =[draw=blue, fill= blue!20]
\tikzstyle{ppGreen}=[draw=green,fill=green!20]
\tikzstyle{ppTrans}=[draw=none, fill=none]

\usetheme{Warsaw}

\useoutertheme[subsection=true]{smoothbars}
%\useoutertheme[subsection=false]{miniframes}

\definecolor{bblue}{HTML}{D7DF01}	% yellow-ish actually, for better black/white printing
\definecolor{rred}{HTML}{C0504D}
\definecolor{ggreen}{HTML}{9BBB59}
\definecolor{ppurple}{HTML}{9F4C7C}
\definecolor{lightgray}{rgb}{0.3,0.3,0.3}
\definecolor{lightergray}{rgb}{0.9,0.9,0.9}
\definecolor{UniBlue}{RGB}{83,121,170}

\DeclareTextFontCommand\textintro{\normalfont\bfseries\itshape} % nice!
\newcommand{\intro}[2][]
{%
	\textintro{#2}%
}
\newcommand{\empha}[2][]
{%
	\emph{#2}%
}

%\theoremstyle{plain}
\newcounter{reqcounter}
\newtheorem{requirement}[reqcounter]{Requirement}

%setbeamercolor{structure}{fg=violet}

\makeatletter
\def\th@task{%
    \normalfont % body font
    \setbeamercolor{block title example}{bg=orange,fg=white}
    \setbeamercolor{block body example}{bg=orange!20,fg=black}
    \def\inserttheoremblockenv{exampleblock}
  }
\makeatother

\theoremstyle{task}
\newtheorem{task}{Task}

\newenvironment{assignment}%
{%\setbeamercolor{background canvas}{bg=violet}%
%\setbeamercolor{structure}{fg=cyan!90!black}%
 \setbeamercolor{frametitle}{bg=orange,fg=white}
\begin{frame}}%
{\end{frame}}%

\AtBeginSection[]{
  \begin{frame}
  \vfill
  \centering
  \begin{beamercolorbox}[sep=8pt,center,shadow=true,rounded=true]{title}
    \usebeamerfont{title}\insertsectionhead\par%
  \end{beamercolorbox}
  \tableofcontents
  \vfill
  \end{frame}
}




\pgfplotsset{compat=1.14}
\author{Markus Raab}


\date{20.3.2018}

\begin{document}

\renewcommand{\enquote}[1]{\emph{``#1''}} % Cannot be done earlier

%%%%%%%%%%%%%%%%%%%%%%%%%%%%%%%
\begin{frame}
	\titlepage
	\doclicenseThis
\end{frame}

\begin{frame}
	Lecture is every week Wednesday 09:00 - 11:00.

	\begin{description}
		\item[06.03.2019:] {\color{gray}topic, teams}
		\item[13.03.2019:] {\color{gray}TISS registration, initial PR}
		\item[20.03.2019:] {\color{red}other registrations, guest lecture}
		\item[27.03.2019:] {\color{orange}PR for first issue done, second started, \\ HS: kleiner Schiffbau}
		\item[03.04.2019:]
		\item[10.04.2019:] mid-term submission of exercises
		\item[08.05.2019:] (HS?)
		\item[15.05.2019:]
		\item[22.05.2019:]
		\item[29.05.2019:]
		\item[05.06.2019:] final submission of exercises
		\item[12.06.2019:]
		\item[19.06.2019:] last corrections of exercises
		\item[26.06.2019:] exam
	\end{description}
\end{frame}

\begin{frame}
	\frametitle{Popular Topics}
	\vspace{-0.55cm}
	\setlength{\columnsep}{-1.3cm}
	\raggedright
	\begin{multicols}{2}
	\begin{description}
	\item[14] tools
	\item[9] testability
	\item[9] code-generation
	\item[7] context-awareness
	\item[6] specification
	\item[6] misconfiguration
	\item[6] {\color{red} complexity reduction}
	\item[5] validation
	\item[5] points in time % (early detection)
	\item[5] error messages
	\item[5] auto-detection
	\item[4] user interface
	\item[4] introspection
	\item[4] design
	\item[4] cascading
	\item[4] architecture of access
	\item[3] {\color{red} configuration sources}
	\item[3] config-less systems
	\item[2] secure conf
	\item[2] architectural decisions
	\item[1] push vs.\ pull
	\item[1] infrastructure as code
	\item[1] full vs.\ partial
	\item[1] convention over conf %iguration
	\item[1] CI/CD
	\item[0] documentation
	\end{description}
	\end{multicols}
\end{frame}

\begin{assignment}
	\frametitle{Tasks for today}
	(until 20.03.2019 23:59)

	\begin{task}
	Make sure to say which programming languages you know in STUDENTS.ini.
	\end{task}

	\begin{task}
	Registration for talk, homework and teamwork \\ (including team and working title as required in private repo).
	\end{task}

	\begin{task}
	Write something in at least one issue (e.g.\ ask if you can have it).
	\end{task}
\end{assignment}

\begin{assignment}
	\frametitle{Tasks for next week}
	(until 27.03.2019 23:59) \\ 
	(hint: to get help submit at least one day earlier)

	\begin{task}
	Description of homework.
	\end{task}

	\begin{task}
	Description of teamwork (which application, which CM tool).
	\end{task}

	Either:
	\begin{itemize}
	\item Inside a folder (use GitHub name for the folder name) of the private repo as pull request.
	\item As new issue in the public libelektra repo.
	\end{itemize}

	\begin{task}
	Fix at least one issue and write some text in at least one other issue.
	\end{task}
\end{assignment}

\begin{frame}
	\hspace*{-1cm}\includegraphics[width=\paperwidth]{dot/topics}
\end{frame}

\begin{frame}
	\frametitle{Talk}
	about anything related to configuration management
	\begin{itemize}
		\item The duration must be not longer than 20 minutes (shorter is ok, content matters).
		\item It must be about your experience.
		\begin{itemize}
			\item E.g., about the homework you did.
			\item I.e., not only about study of literature.
			\item If you extensively use some tool before, please share your experience.
		\end{itemize}
		\item Two persons per date.
		\item Same topic allowed if persons coordinate their talk.
		\item \textbf{Slides must be in repo. \\ (please CC licensed and with source)}
	\end{itemize}
\end{frame}


%%%%%%%%%%%%%%%%%%%%%%%%%%%%%%%%%%%%%%%%%% 
\section{Command-line Arguments}

\subsection{Usage and Popularity}

\begin{frame}
	\frametitle{Is there something else?}
	\begin{itemize}
	\item configuration files are the most researched of all configuration sources~\cite{jin2014configurations}
	\item but it is neither the most used nor most popular~\cite{raab2017challenges}
	\end{itemize}
\end{frame}

\begin{frame}
	\methodQuestion{} \question{Which configuration systems/libraries/APIs have you already used or would like to use in one of your FLOSS project(s)?}
	\begin{itemize}
	\item command-line arguments (\p{92}, $n=222$)
	\item environment variables (\p{79}, $n=218$)
	\item \methodSource{} API \texttt{getenv} is used omnipresently with 2,683 occurrences
	\item configuration files (\p{74}, $n=218$))
	\end{itemize}
\end{frame}


\begin{frame}
	\methodQuestion{} \question{What is your experience with the following configuration systems/libraries/APIs?}
	\begin{itemize}
	\item \texttt{getenv} (\p{10}, $n=198$)
	\item configuration files (\p{6}, $n=190$)
	\item command-line options (\p{4}, $n=210$)
	\item X/Q/GSettings (\p{41}, \p{14}, \p{35})
	\item KConfig (\p{21})
	\item dconf (\p{42})
	\item plist (\p{32})
	\item Windows Registry (\p{69})
	\end{itemize}
\end{frame}

\begin{assignment}
	\begin{task}
	Which configuration source do you use most?
	\end{task}

	\begin{task}
	Possible talk: About one of these sources.
	\end{task}
\end{assignment}



%%%%%%%%%%%%%%%%%%%%%%%%%%%%%%%%%%%%%%%%%% 
\nocite{raab2017introducing}

\appendix

\begin{frame}[allowframebreaks]
	\bibliographystyle{plainnat}
	\bibliography{../shared/elektra.bib}
\end{frame}

\end{document}


