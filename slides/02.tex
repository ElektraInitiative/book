\documentclass{beamer}

%TODO: motivation why specification language (complexity+misconfiguration)
%TODO: visibility
%TODO: requirements for CM: uniform configuration access with introspection

%\ifdefined\handout
%\documentclass[handout,aspectratio=1610,xcolor={usenames,dvipsnames,table}]{beamer}
%\else
\documentclass[aspectratio=1610,xcolor={usenames,dvipsnames,table}]{beamer}
%\fi

\newcommand{\mylecture}{Configuration Management}

%\ifdefined\handout
%\documentclass[handout,aspectratio=1610,xcolor={usenames,dvipsnames,table}]{beamer}
%\else
\documentclass[aspectratio=1610,xcolor={usenames,dvipsnames,table}]{beamer}
%\fi

\newcommand{\mylecture}{Configuration Management}

%\ifdefined\handout
%\documentclass[handout,aspectratio=1610,xcolor={usenames,dvipsnames,table}]{beamer}
%\else
\documentclass[aspectratio=1610,xcolor={usenames,dvipsnames,table}]{beamer}
%\fi

\newcommand{\mylecture}{Configuration Management}

\input{../setup}
\input{../shared/setup}

\lstdefinelanguage{dump}
{
	morekeywords={kdbOpen,ksNew,keyNew,keyMeta,keyCopyMeta,keyEnd,ksEnd,kdbClose},
	sensitive=false,
	morecomment=[l]{//},
	morecomment=[s]{/*}{*/},
	morestring=[b]",
}


\lstdefinelanguage{SpecElektra}{
	%
	comment=[l]{;},
	commentstyle=\color{purple}\ttfamily,
	%
	morestring=[b]',
	morestring=[b]`,
	morestring=[b]",
	stringstyle=\color{purple}\ttfamily,
	%
	sensitive=f,% keywords are not case sensitive
	%
	% Colors see https://en.wikibooks.org/wiki/LaTeX/Colors
	%
	keywordstyle=\color{BlueViolet}\bfseries,
	keywordstyle=[2]\color{Green},
	keywordstyle=[3]\color{Aquamarine}\bfseries\textit,
	keywordstyle=[4]\color{NavyBlue}\bfseries,
	keywordstyle=[5]\color{Mahogany},
	%
	keywords={layer, require, validation, check, range, description, rationale, requirement, accessibility, enable, condition, message, default, opt, readonly, type, context, property1, property2, description, file, content, mountpoint, metadata, infos, plugins},
	keywords=[2]{},
	keywords=[3]{order, interface, network, emphasized},
	%keywords=[4]{[, ]},  %Not needed
	keywords=[4]{},
	keywords=[5]{},
	%
	literate={:=}{{{\color{red}\textbf:=}}}2
		 {\%}{{{\color{NavyBlue}\textbf\%}}}1
		 {[}{{{\color{Sepia}\textbf[}}}1
		 {]}{{{\color{Sepia}\textbf]}}}1,
}

\lstdefinelanguage{Cpp}{%
	language     = C++,
	literate=
}


\lstdefinelanguage{CfgElektra}{
	comment=[l]{;},
	commentstyle=\color{purple}\ttfamily,
	%
	morestring=[b]',
	morestring=[b]`,
	morestring=[b]",
	stringstyle=\color{purple}\ttfamily,
	%
	%
	sensitive=f,% keywords are not case sensitive
	%
	% Colors see https://en.wikibooks.org/wiki/LaTeX/Colors
	%
	keywordstyle=\color{Bittersweet}\bfseries,
	keywordstyle=[2]\color{DarkOrchid}\bfseries,
	keywordstyle=[3]\color{ForestGreen}\bfseries\textit,
	keywordstyle=[4]\color{Goldenrod}\bfseries,
	keywordstyle=[5]\color{CarnationPink},
	%
	keywords={},
	keywords=[2]{},
	keywords=[3]{},
	keywords=[4]{},
	keywords=[5]{},
	%
	literate={=}{{{\color{ForestGreen}\textbf=}}}1
		 %{<-}{{{\color{ForestGreen}\textbf<-}}}2
		 %{*}{{{\color{Bittersweet}\textbf*}}}1
		 {\%}{{{\color{NavyBlue}\textbf\%}}}1,
}




\lstset{language=SpecElektra, % Use SpecElektra as default programming language
	%boxpos=t, % make boxes a bit more unbreakable
	%frame=lines, % top+bottom line
	basicstyle=\ttfamily, % Use normal-size true type font
	showspaces,%
	showstringspaces=false, % Don't put marks in string spaces
	showlines=true, % make sure empty lines at end are shown (needed for concurrency
	tabsize=4, % spaces per tab
	xleftmargin=\parindent, % should be 18pt or 1.5em as defined by memoir
	%Does not really work well (needs to be deactivated for shortlistings):
	breaklines=false,
	%postbreak=\mbox{\textcolor{red}{$\hookrightarrow$}\space},
	%breakautoindent=true,
	%prebreak={\mbox{\ensuremath{\curvearrowright}}} % Zeichen am Zeilenende (Umbruch)
	%breaklines=true,
	%breakautoindent=true,
	%prebreak=\small\symbol{'134}, % backslash
	%prebreak={\mbox{\ensuremath{\curvearrowright}}} % lange kure
	%prebreak={\mbox{\ensuremath{\hookleftarrow}}} % lange kure
	%xleftmargin=3.0ex, %for some formats
	%xrightmargin=1.0ex, %for some formats
	%
	% Files do not work in utf8 see also:
	% http://stackoverflow.com/questions/1116266/listings-in-latex-with-utf-8-or-at-least-german-umlauts
	% http://tex.stackexchange.com/questions/24528/having-problems-with-listings-and-utf-8-can-it-be-fixed
	% Should work but doesn't? (Maybe add to literate broken?)
	%add to literate={ö}{{\"o}}1
	%	{ä}{{\"a}}1
	%	{ü}{{\"u}}1
	%	{Ö}{{\"O}}1
	%	{Ä}{{\"A}}1
	%	{Ü}{{\"U}}1
	%	{ß}{{\ss}}1,
	%
	% listingsutf8 did not work, made umlauts in comments very strange
	%extendedchars=true,
	%inputencoding=utf8,
	%
	%morecomment=[l][\color{blue}]{...}, % Line continuation (...) e.g. blue comment
	morekeywords={for_each},
	numbers=left, % Line numbers on left
	firstnumber=1, % Line numbers start with line 1
	numberstyle=\small\color{blue}, % Line numbers are blue and small
	numbersep=5pt,
	%stepnumber=5 % Line numbers go in steps of 5
}



\lstMakeShortInline[postbreak=,keywordstyle={}]^

\graphicspath{{../pic/}{../figures/}{../graphics/}{../ipe/}{../ggplot/}}



\lstdefinelanguage{dump}
{
	morekeywords={kdbOpen,ksNew,keyNew,keyMeta,keyCopyMeta,keyEnd,ksEnd,kdbClose},
	sensitive=false,
	morecomment=[l]{//},
	morecomment=[s]{/*}{*/},
	morestring=[b]",
}


\lstdefinelanguage{SpecElektra}{
	%
	comment=[l]{;},
	commentstyle=\color{purple}\ttfamily,
	%
	morestring=[b]',
	morestring=[b]`,
	morestring=[b]",
	stringstyle=\color{purple}\ttfamily,
	%
	sensitive=f,% keywords are not case sensitive
	%
	% Colors see https://en.wikibooks.org/wiki/LaTeX/Colors
	%
	keywordstyle=\color{BlueViolet}\bfseries,
	keywordstyle=[2]\color{Green},
	keywordstyle=[3]\color{Aquamarine}\bfseries\textit,
	keywordstyle=[4]\color{NavyBlue}\bfseries,
	keywordstyle=[5]\color{Mahogany},
	%
	keywords={layer, require, validation, check, range, description, rationale, requirement, accessibility, enable, condition, message, default, opt, readonly, type, context, property1, property2, description, file, content, mountpoint, metadata, infos, plugins},
	keywords=[2]{},
	keywords=[3]{order, interface, network, emphasized},
	%keywords=[4]{[, ]},  %Not needed
	keywords=[4]{},
	keywords=[5]{},
	%
	literate={:=}{{{\color{red}\textbf:=}}}2
		 {\%}{{{\color{NavyBlue}\textbf\%}}}1
		 {[}{{{\color{Sepia}\textbf[}}}1
		 {]}{{{\color{Sepia}\textbf]}}}1,
}

\lstdefinelanguage{Cpp}{%
	language     = C++,
	literate=
}


\lstdefinelanguage{CfgElektra}{
	comment=[l]{;},
	commentstyle=\color{purple}\ttfamily,
	%
	morestring=[b]',
	morestring=[b]`,
	morestring=[b]",
	stringstyle=\color{purple}\ttfamily,
	%
	%
	sensitive=f,% keywords are not case sensitive
	%
	% Colors see https://en.wikibooks.org/wiki/LaTeX/Colors
	%
	keywordstyle=\color{Bittersweet}\bfseries,
	keywordstyle=[2]\color{DarkOrchid}\bfseries,
	keywordstyle=[3]\color{ForestGreen}\bfseries\textit,
	keywordstyle=[4]\color{Goldenrod}\bfseries,
	keywordstyle=[5]\color{CarnationPink},
	%
	keywords={},
	keywords=[2]{},
	keywords=[3]{},
	keywords=[4]{},
	keywords=[5]{},
	%
	literate={=}{{{\color{ForestGreen}\textbf=}}}1
		 %{<-}{{{\color{ForestGreen}\textbf<-}}}2
		 %{*}{{{\color{Bittersweet}\textbf*}}}1
		 {\%}{{{\color{NavyBlue}\textbf\%}}}1,
}




\lstset{language=SpecElektra, % Use SpecElektra as default programming language
	%boxpos=t, % make boxes a bit more unbreakable
	%frame=lines, % top+bottom line
	basicstyle=\ttfamily, % Use normal-size true type font
	showspaces,%
	showstringspaces=false, % Don't put marks in string spaces
	showlines=true, % make sure empty lines at end are shown (needed for concurrency
	tabsize=4, % spaces per tab
	xleftmargin=\parindent, % should be 18pt or 1.5em as defined by memoir
	%Does not really work well (needs to be deactivated for shortlistings):
	breaklines=false,
	%postbreak=\mbox{\textcolor{red}{$\hookrightarrow$}\space},
	%breakautoindent=true,
	%prebreak={\mbox{\ensuremath{\curvearrowright}}} % Zeichen am Zeilenende (Umbruch)
	%breaklines=true,
	%breakautoindent=true,
	%prebreak=\small\symbol{'134}, % backslash
	%prebreak={\mbox{\ensuremath{\curvearrowright}}} % lange kure
	%prebreak={\mbox{\ensuremath{\hookleftarrow}}} % lange kure
	%xleftmargin=3.0ex, %for some formats
	%xrightmargin=1.0ex, %for some formats
	%
	% Files do not work in utf8 see also:
	% http://stackoverflow.com/questions/1116266/listings-in-latex-with-utf-8-or-at-least-german-umlauts
	% http://tex.stackexchange.com/questions/24528/having-problems-with-listings-and-utf-8-can-it-be-fixed
	% Should work but doesn't? (Maybe add to literate broken?)
	%add to literate={ö}{{\"o}}1
	%	{ä}{{\"a}}1
	%	{ü}{{\"u}}1
	%	{Ö}{{\"O}}1
	%	{Ä}{{\"A}}1
	%	{Ü}{{\"U}}1
	%	{ß}{{\ss}}1,
	%
	% listingsutf8 did not work, made umlauts in comments very strange
	%extendedchars=true,
	%inputencoding=utf8,
	%
	%morecomment=[l][\color{blue}]{...}, % Line continuation (...) e.g. blue comment
	morekeywords={for_each},
	numbers=left, % Line numbers on left
	firstnumber=1, % Line numbers start with line 1
	numberstyle=\small\color{blue}, % Line numbers are blue and small
	numbersep=5pt,
	%stepnumber=5 % Line numbers go in steps of 5
}



\lstMakeShortInline[postbreak=,keywordstyle={}]^

\graphicspath{{../pic/}{../figures/}{../graphics/}{../ipe/}{../ggplot/}}




\date{16.3.2018}

\begin{document}

\renewcommand{\enquote}[1]{\emph{``#1''}} % Cannot be done earlier

%%%%%%%%%%%%%%%%%%%%%%%%%%%%%%%
\begin{frame}
	\titlepage
	\doclicenseThis
\end{frame}

\begin{frame}
	\frametitle{Organization}
	Next lectures:
	\begin{description}
		\item[16.3.2018:] \textbf{topic homework and talk} (GitHub account!)
		\item[23.3.2018:] teams found together
		\item[13.4.2018:] homework submitted, topics of team exercise
		\item[20.4.2018:] no lecture

		\item[18.5.2018:] guest lecture
		\item[25.5.2018:] team exercise submitted
		\item[22.6.2018:] last corrections of team exercise
		%\item[29.6.2018:] test
	\end{description}
\end{frame}


\begin{frame}
	\frametitle{Popular Topics}
	\vspace{-0.5cm}
	\begin{multicols}{2}
	\begin{description}
	\item[4] validation
	\item[4] user interface
	\item[3] tools (benefits?)
	\item[3] testability
	\color{red}
	\item[3] complexity reduction (when conf. needed?)
	\item[3] architectural decisions
	\color{black}
	\item[2] Puppet
	\item[2] modularity
	\color{gray}
	\item[2] environment variables
	\color{black}
	\item[2] documentation
	\color{red}
	\item[2] configuration specification
	\color{gray}
	\item[2] command-line args\color{black}
	\item[2] code generation
	\item[1] variability
	\item[1] self-description
	\item[1] round-tripping
	\item[1] introspection
	\item[1] early
	\item[1] dependences
	\item[1] context-awareness
	\item[1] auto-detection
	\item[1] administrators
	\end{description}
	\end{multicols}
\end{frame}

\begin{frame}
	\hspace*{-1cm}\includegraphics[width=\paperwidth]{dot/topics}
\end{frame}



\section{Complexity}

\subsection{Trend}

\begin{frame}
	\frametitle{Trend Firefox}
	\includegraphics[scale=0.7]{firefox}
\end{frame}

\begin{frame}
	\frametitle{Trend Chromium}
	\includegraphics[scale=0.7]{chromium}
\end{frame}

\begin{frame}
	\frametitle{Trend Configuration Files}
	\includegraphics[scale=0.5]{pics/trend.png}
	\citet{xu2015hey}
\end{frame}

\subsection{Calculation}

\begin{frame}
	\frametitle{Types of Complexity}
	\begin{itemize}
	\item complexity in access:
		\begin{itemize}
		\item many different formats
		\item non-uniformity
		\item transformations
		\end{itemize}
	\item configuration settings
		\begin{itemize}
		\item number of settings $s$
		\item number of values $n$
		\item dependences between settings
		\end{itemize}
	\end{itemize}
\end{frame}

\lstDeleteShortInline^
\begin{frame}
	\frametitle{Calculation of Complexity}

	Using enumerative combinatorics:
	\begin{itemize}
	\item number of configurations: $n^s$
	\item for $N$ groups of different $n$ and $s$ (i.e., $n_1 \dots n_N$ with $s_1 \dots s_N$ occurrences):  $$\prod_{i=1}^{N} n_i^{s_i}$$
	\item more difficult to calculate (or unbounded) for dependences, module instantiations, arrays, \dots
	\end{itemize}
\end{frame}

\begin{frame}
	\frametitle{Calculation of Complexity}

	Examples:
	\begin{itemize}
	\item 600 boolean settings in Apache httpd:
	\pause
	$2^{600} \approx 10^{180}$

	\item 19 integer settings:
	\pause
	${2^{32}}^{19} = 2^{32 \cdot 19} = 2^{609} \approx 10^{183}$

	\item 2000 boolean settings in Firefox:
	\pause
	$2^{2000} \approx 10^{602}$

	\item for 20 boolean and 20 enums with 5 possibilities:
	\pause
	$$2^{20}*5^{20} = 10^{20}$$

	\item an array with $1-20$ boolean settings:
	\pause
	$2^{20}$
	\end{itemize}
\end{frame}
\lstMakeShortInline[postbreak=,keywordstyle={}]^
%XXX

\begin{assignment}
	\begin{task}
	Calculate complexity for some tool you know.
	\end{task}

	\begin{task}
	Possible Homework: Write tool to calculate complexity with a given configuration specification.
	\end{task}
\end{assignment}

\begin{frame}
	\frametitle{Decision Tree}
	\begin{itemize}
	\item configuration settings may depend on each other
	\item form a decision tree~\cite{reiser2009cvm,czarnecki2012cool}
	\item the decision tree is an instantiation of chosen configuration settings
	\end{itemize}
\end{frame}

\subsection{Usage}

\begin{frame}
	\frametitle{Unnecessary Settings~\cite{xu2015hey}}
	\begin{itemize}
	\item \p{6} to \p{17} of settings set by majority
	\item up to \p{54} are seldom set
	\item up to \p{47} of numeric parameters have no more than five distinct values
	\item \p{17} to \p{48} of configuration issues are about difficulties in finding settings
	\item up to \p{54} of settings are rarely used
	\end{itemize}
\end{frame}

\begin{frame}
	\frametitle{Reduction}
	\methodQuestion{}
	\question{Why do you think configuration should be reduced?}
	\begin{itemize}
	\item to simplify code maintenance (\p{50}),
	\item to prevent errors and misconfiguration (\p{43}),
	\item to provide better user experience (\p{40}),
	\item \textbf{\question{I do not think it should be reduced} (\p{30})},
	\item because they prefer auto-detection (\p{29}) \\ (with a possibility to override configuration settings: \p{32}),
	\item \question{because use-cases which are rarely used should not be supported} (\p{13}),
	\item \question{never find time for this task} (\p{9}), and
	\item \question{because only standard use-cases should be supported} (\p{1})
	\end{itemize}
\end{frame}

\begin{frame}
	\begin{alertblock}{Question}
	How to specify reduction strategies of configuration settings?
	\end{alertblock}
	\pause
	\begin{exampleblock}{Answer}
	Configuration Specification
	\end{exampleblock}
\end{frame}

%%%%%%%%%%%%%%%%%%%%%%%%%%%%%%%%%%%%%%%%%% 
\section{Configuration Specification}

\subsection{Why}

\begin{frame}
	\methodQuestion{}
	\question{Configuration specification (e.g. XSD/JSON schemas) allows you to describe possible values and their meaning.  Why do/would you specify configuration?}
	\ExecuteMetaData[../book/motivation.tex]{question-introduce-spec}
\end{frame}

\begin{frame}
	\frametitle{Limitation of Schemata designed for Data}
	\begin{itemize}
	\item not key-value based
	\item not possible to write own plugins
	\item tied to specific formats (e.g. XML/JSON)
	\end{itemize}
\end{frame}

\begin{frame}
	\frametitle{Limitation of Zero-Configuration-Free}
	\begin{itemize}
	\item e.g. gpsd\footnote{\url{www.aosabook.org/en/gpsd.html}}
	\item broken hardware or protocols
	\item auto-detection may go wrong
	\item the configuration actually lives elsewhere (e.g., in the GPS devices)
	\end{itemize}
\end{frame}

\subsection{Hiding}

\begin{frame}
	\frametitle{Limitation of Schemata designed for Data}
	\methodQuestion{}
	\begin{itemize}
	\item not key-value based
	\item not possible to write own plugins
	\item tied to specific formats (e.g. XML/JSON)
	\end{itemize}
\end{frame}

\subsection{Calculate Default Values}


%%%%%%%%%%%%%%%%%%%%%%%%%%%%%%%%%%%%%%%%%% 
\section{Architectural Decisions}


\begin{frame}
	\frametitle{Conclusion}
	\begin{itemize}
	\item alarming trend in number and complexity of configuration settings
	\item visibility and default value calculation helps
	\item but also more courageous decisions needed
	\item abstractions: specifications
	\end{itemize}
\end{frame}



%%%%%%%%%%%%%%%%%%%%%%%%%%%%%%%%%%%%%%%%%% 
\nocite{raab2017introducing}

\appendix

\begin{frame}[allowframebreaks]
	\bibliographystyle{plainnat}
	\bibliography{../shared/elektra.bib}
\end{frame}

\end{document}


