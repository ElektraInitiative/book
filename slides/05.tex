% Make nice A4 pages for print:
%\usepackage{pgfpages}
%\pgfpagesuselayout{resize to}[a4paper,border shrink=5mm,landscape]

\beamertemplatenavigationsymbolsempty

\setbeamertemplate{bibliography item}[text]

\usepackage[type={CC},modifier={by-sa},version={4.0}]{doclicense}

\usepackage[utf8]{inputenc}
\usepackage{hyperref}
\usepackage{breakurl}
\usepackage{graphicx}
\usepackage{pgfplots}
\usepackage{pgf}
\usepackage{tikz}
\usetikzlibrary{positioning}
\usetikzlibrary{arrows}
\usetikzlibrary{decorations.markings}
\usetikzlibrary{calc}
\usetikzlibrary{matrix}
\usetikzlibrary{shapes}
\usetikzlibrary{decorations.pathmorphing}
\usetikzlibrary{fit}
\usetikzlibrary{backgrounds}
\usetikzlibrary{plotmarks}
\usepackage{stmaryrd}
\usepackage{listings}
\usepackage{pdflscape}
\usepackage{perpage}
\usepackage{appendixnumberbeamer}

%\usepackage[thmmarks,amsmath,amsthm]{ntheorem} % already included in beamer
\usepackage{thm-restate}

\usepackage[sort&compress,numbers]{natbib}  % to be have \citet, \citeauthor, \citeyear

\MakePerPage{footnote}

\tikzstyle{o}=[r,ppBlue]
\tikzstyle{r}=[thick,rectangle,align=center]
\tikzstyle{t}=[r,ppTrans] %,font=\bfseries]
\tikzstyle{dd}=[densely dashed]
\tikzstyle{n}=[r,ppBlue]
\tikzstyle{p}=[r,ppRed]
\tikzstyle{ppRed}  =[draw=red,  fill=  red!20]
\tikzstyle{ppBlue} =[draw=blue, fill= blue!20]
\tikzstyle{ppGreen}=[draw=green,fill=green!20]
\tikzstyle{ppTrans}=[draw=none, fill=none]

\usetheme{Warsaw}

\useoutertheme[subsection=true]{smoothbars}
%\useoutertheme[subsection=false]{miniframes}

\definecolor{bblue}{HTML}{D7DF01}	% yellow-ish actually, for better black/white printing
\definecolor{rred}{HTML}{C0504D}
\definecolor{ggreen}{HTML}{9BBB59}
\definecolor{ppurple}{HTML}{9F4C7C}
\definecolor{lightgray}{rgb}{0.3,0.3,0.3}
\definecolor{lightergray}{rgb}{0.9,0.9,0.9}
\definecolor{UniBlue}{RGB}{83,121,170}

\DeclareTextFontCommand\textintro{\normalfont\bfseries\itshape} % nice!
\newcommand{\intro}[2][]
{%
	\textintro{#2}%
}
\newcommand{\empha}[2][]
{%
	\emph{#2}%
}

%\theoremstyle{plain}
\newcounter{reqcounter}
\newtheorem{requirement}[reqcounter]{Requirement}

%setbeamercolor{structure}{fg=violet}

\makeatletter
\def\th@task{%
    \normalfont % body font
    \setbeamercolor{block title example}{bg=orange,fg=white}
    \setbeamercolor{block body example}{bg=orange!20,fg=black}
    \def\inserttheoremblockenv{exampleblock}
  }
\makeatother

\theoremstyle{task}
\newtheorem{task}{Task}

\newenvironment{assignment}%
{%\setbeamercolor{background canvas}{bg=violet}%
%\setbeamercolor{structure}{fg=cyan!90!black}%
 \setbeamercolor{frametitle}{bg=orange,fg=white}
\begin{frame}}%
{\end{frame}}%

\AtBeginSection[]{
  \begin{frame}
  \vfill
  \centering
  \begin{beamercolorbox}[sep=8pt,center,shadow=true,rounded=true]{title}
    \usebeamerfont{title}\insertsectionhead\par%
  \end{beamercolorbox}
  \tableofcontents
  \vfill
  \end{frame}
}




\pgfplotsset{compat=1.14}
\author{Markus Raab}


\title{L05 Configuration Sources}
\date{21.04.2021}

\begin{document}


%%%%%%%%%%%%%%%%%%%%%%%%%%%%%%%%%%%%%%%%%% 
\section{History of Configuration Management}

\begin{frame}
	\frametitle{Learning Outcomes}
	Students will be able to
	\begin{itemize}
	%\item remember the history of configuration management.
	\item remember differences between CM languages and historical approaches.
	\item write simple configuration management scripts.
	%\item connect needs of CM tools with configuration specifications.
	%\item contextualize CM languages with the view point of administrators.
	\end{itemize}
\end{frame}

\begin{frame}
	\frametitle{Definition}

	\intro[configuration management]{Configuration Management}:

	\begin{itemize}
	\item is a discipline in which configuration (in the broader sense) is administered.
	\item makes sure computers are assembled from desired parts and the correct applications are installed.
	\item ensures that the execution environment of installed applications is as required.
	\end{itemize}
\end{frame}


\begin{frame}
	\frametitle{Definition}

	\intro[configuration management tool]{Configuration Management Tools}:

	\pause

	\begin{itemize}
	\item help people involved in configuration management.
	\item have means to describe the desired configuration of the whole managed system.
	\item try to converge the actual configuration to the desired one~\cite{burgess1995cfengine}.
	\end{itemize}
\end{frame}


\begin{frame}
	\frametitle{}

	Challenging tasks in configuration management:

	\pause

	\begin{itemize}
	\item inventory list
	\item installing packages
	\item monitoring
	\item add/replace machines
	\item maintaining files/databases
	\item \intro{configuration file manipulation}
	\end{itemize}
\end{frame}

%TODO: add more examples!


\begin{frame}
	\frametitle{Cloning}

	It all started with:

	\begin{itemize}
	\item clone all files with dd, rdist, rsync or unison (``golden image'')
	\item then do necessary modifications with scripts or profiles
	\end{itemize}

	\pause

	\setbeamersize{description width=1cm}
	\begin{description}
	\item[$+$] works good for many identical stateless machines
	\item[$-$] fails if differences between machines are too big
	\end{description}
\end{frame}

\begin{frame}
	\frametitle{Scripts}

	First improvement: have a script to create the ``golden image''.
	Possible benefits:

	\begin{itemize}[<+-| alert@+>]
	\item Documentation
	\item Customization (using configuration settings)
	\item \textbf{Reproducability}: Reproduce creation using different operating system versions
	\end{itemize}
\end{frame}

\begin{frame}[fragile]
	\frametitle{Profiles}

	\intro[profile]{Profiles} are groups of configuration settings between which the user can easily switch.

	\begin{itemize}
	\item by hostname, information EEPROM, manual selection, \dots
	\item can be activated via the ^profile^ plugin:
	\end{itemize}

	\begin{code}[morekeywords={long},gobble=4]]
	[application/profile]
	  type:=string
	  opt:=p
	  opt/long:=profile
	  default:=current
	\end{code}

	with a config like:
	\begin{code}[gobble=4,language=CfgElektra]
	application/current/key = "current"
	application/myprofile/key = "myprofile"
	application/%/key = "default"
	\end{code}
\end{frame}


\begin{frame}
	\frametitle{First four configuration management tools}
	Cloning, and then NIS/NFS, was state of the art for a long time, until in 1994 when \enquote{the community nearly exploded with four new configuration systems}~\cite{evard1997analysis}:

	\ExecuteMetaData[../book/background.tex]{configuration-management-first-four}
\end{frame}

\begin{frame}
	\frametitle{Possible Benefits}

	\begin{itemize}[<+-| alert@+>]
	\item All advantages scripts have: \\
		Documentation, Customization, Reproducability
	\item Declarative description of the system \\
		(Infrastructure as Code~\cite{waldemar2013testing})
	\item Less configuration drift
	\item Error handling
	\item Pull/Push
	\item Reusability
	\item (Resource) Abstractions
	\end{itemize}
\end{frame}


%%%%%%%%%%%%%%%%%%%%%%%%%%%%%%%%%%%%%%%%%% 
\section{CM Languages}

\subsection{}

\begin{frame}[fragile]
	\frametitle{Proteus (PCL)}
	\textsc{Proteus}~\cite{tryggeseth1995modelling} shows the tight relation between software configuration management, like Git or Svn, and configuration specification languages.
	\textsc{Proteus} (PCL) combines both worlds in a powerful build system.

	\begin{code}[basicstyle=\tiny,morekeywords={family,attributes,end,physical,default,classifications},gobble=4,language=]
	family CalcProg
		attributes
			HOME : string default "/home/ask/proteus/test";
			workspace := HOME ++ "/calc/src/"; // string concatenation
			repository := "calc/";
			end
		physical
			main => "main.C";
			defs => "defs.h";
			exe => "calc.x" attributes workspace := HOME ++ "/calc/bin"; end
			classifications status := standard.derived; end;
		end
	end
	\end{code}
\end{frame}

\begin{frame}
	\frametitle{NIX}

	The NIX language~\cite{dolstra2007purely} is purely functional as a novel feature.
	The main concept is the referential transparency both for the configuration specification language and for the system itself.

	\textbf{Expressiveness:}
	NIX expressions, for example functions, describe how to build software packages.

	\textbf{Reasoning:}
	Because of the referential transparency of the system itself, every solution derived from the NIX expressions should be valid, so no reasoning or conflict handling is necessary.

	\textbf{Modularity:}
	The NIX expressions are modular because they ensure absence of side effects and thus can be easily composed.

	\textbf{Reusability:}
	Derivations that describe atomic build actions are reused in other derivations.
\end{frame}

\begin{frame}
	\frametitle{UML}
	\citet{felfernig1999knowledge,felfernig2000uml,felfernig2002joint} describe an approach where the unified modeling language (UML) is used as notation.

	\textbf{Expressiveness:}
	All UML features, including cardinality, domain-specific stereotypes and OCL-constraints are available.
	The basic structure of the system is specified using classes, generalization and aggregation.

	\textbf{Reasoning:}
	Customers provide additional input data and requirements for the actual variant of the product.

	\textbf{Modularity:}
	Generalization is present without multiple inheritance with disjunctive semantics, i.\,e., only one of the given subtypes will be instantiated.

	\textbf{Reusability:}
	For shared aggregation additional ports are defined for a part.
\end{frame}


\begin{frame}
	\frametitle{CFEngine}

	CFEngine~\cite{burgess2003theory,burgess1995cfengine,pandey2012investigating} is a language-based system administration tool that pioneered idempotent behavior.

	\textbf{Expressiveness:}
	CFEngine allows us to declare dependences and facilitates some high-level configuration specification constructs.
	In its initial variants it neither had validation specifications, cardinalities, nor higher-level relationships.

	\textbf{Reasoning:}
	Not supported.

	\textbf{Modularity:}
	Not supported.

	\textbf{Reusability:}
	Existing system administrator scripts can be profitably run from CFEngine.
\end{frame}



\begin{frame}
	\frametitle{Quattor (Pan)}

	\citet{cons2002pan} invented and used PAN for many machines within CERN.

	\textbf{Expressiveness:}
	The Pan language allows users to specify data types, validation with code snippets and constraints.
	The compiler uses a 5 step process: compilation, execution, insertions-of-defaults, validation, and serialization.

	\textbf{Reasoning:}
	Pan focuses on validating configurations, it is not able to generate new configurations.
	Pan provides type enforcement with embedded validation code.

	\textbf{Modularity:}
	The language has user-defined data types (called templates) but otherwise has only minimal support for modularity.

	\textbf{Reusability:}
	Reusability and collaboration is only possible via simple include statements and a simple inheritance mechanism of templates.
\end{frame}


\begin{frame}
	\frametitle{ConfValley (CPL)}

	\citet{huang2015confvalley} introduce systematic validation for cloud services.
	ConfValley uses a unified configuration settings representation for tens of different configuration file formats.

	\textbf{Expressiveness:}
	CPL is not able to specify dynamic and complex requirements.

	\textbf{Reasoning:}
	Constraints can be inferred by running an inference engine on configuration settings that are considered good (black-box approach).
	Within the validation engine, however, no constraint solver is available.

	\textbf{Modularity:}
	CPL aims at easy grouping of constraints.
	Adding language primitives need modifications in the compiler.

	\textbf{Reusability:}
	Using transformations and compositions, predicates can be reused in different contexts.
	Also with language constructs like \texttt{let}, specifications can be reused.
\end{frame}




%%%%%%%%%%%%%%%%%%%%%%%%%%%%%%%%%%%%%%%%%% 
\section{Configuration Management Tools}

\begin{frame}
	\frametitle{List of CM tools}

	\begin{itemize}[<+-| alert@+>]
	\item CFengine (1993)
	\item LCFG (1994)
	\item Quattor (2005)
	\item Puppet (2005)
	\item Chef (2009)
	\item Salt (2011)
	\item Ansible (2012)
	\item Itamae (2014)
	\item OpsMops (2019)
	% TODO: something new?
	\end{itemize}
\end{frame}

\begin{frame}[fragile]
	Key/value access in puppet-libelektra~\cite{raab2020unified}:
	\vspace{0.5cm}

	\begin{code}[morekeywords={kdbkey,kdbmount,ensure,value},gobble=4]
	kdbkey {'/slapd/threads/listener':
		ensure => 'present',
		value => '4'
		check => {
			'type' => 'short',
			'range' => '1,2,4,8,16',
			'default' => '1'
		}
	}
	\end{code}
\end{frame}

\begin{frame}[fragile]
	Key/value access in puppet-libelektra:

	\begin{code}[morekeywords={kdbkey,kdbmount,ensure,value},gobble=4]
	kdbmount {'system/sw/samba':
		ensure => 'present',
		file => '/etc/samba/smb.conf',
		plugins => 'ini'
	}
	kdbkey {'system/sw/samba/global/workgroup':
		ensure => 'present',
		value => 'MY_WORKGROUP'
	}
	kdbkey {'system/sw/samba/global/log level':
		ensure => 'absent'
	}
	\end{code}

	Uniqueness of keys is essential.
	Ideally, applications already mount their configuration at installation.
\end{frame}


\begin{frame}[fragile]
	Key/value specifications in puppet-libelektra:

	\begin{code}[morekeywords={kdbkey,ensure,value},gobble=4]
	kdbkey {'system/sw/samba/global/log level':
		ensure => 'present',
		value => 'MY_WORKGROUP',
		check => {
			'type' => 'short',
			'range' => '0-10',
			'default' => '1',
			'description' => 'Sets the amount of log/
				debug messages that are sent to the
				log file. 0 is none, 3 is consider-
				able.'
	}
	\end{code}
\end{frame}

\begin{frame}[fragile]
	Key/value specifications in puppet-libelektra:

	\begin{code}[morekeywords={kdbkey,ensure,value},gobble=4]
	kdbkey {'spec/xfce/pointers/Mouse/RightHanded':
		ensure => 'present',
		check => {
			'namespaces/#0' => 'user',
			'namespaces/#1' => 'system',
			'visibility' => 'important',
			'default' => 'false',
			'check/type' => 'boolean'
	}
	\end{code}

	Ideally, applications already specify their settings.
\end{frame}

\begin{frame}[fragile]
	Key/value access in Chef:

	\begin{code}[morekeywords={kdbset,do,action,value,end},gobble=4]
	kdbset 'system/sw/samba/global/workgroup' do
		value 'MY_WORKGROUP'
		action :create
	end
	\end{code}
\end{frame}

\begin{frame}[fragile]
	Key/value access in Chef:
	\vspace{0.5cm}

	\begin{code}[morekeywords={kdbset,do,action,value,end},gobble=4]
	kdbset '/slapd/threads/listener' do
		value '4'
		action :create
	end
	\end{code}

	\pause
	\begin{alertblock}{Finding}
	We have CM code representing the settings.
	\end{alertblock}
\end{frame}

\begin{frame}[fragile]
	Key/value access in Ansible:
	\vspace{0.5cm}

	\begin{code}[morekeywords={name,connection,key,value,elektra,mountpoint,file,plugins,hosts,tasks},gobble=4]
	- name: setup LDAP
	  connection: local
	  hosts: localhost
	  tasks:
	  - name: set listening threads
	    elektra:
	      key: '/slapd/threads/listener'
	      value: '4'
	\end{code}
\end{frame}


\begin{frame}[fragile]
	Key/value access in Ansible:

	\begin{code}[morekeywords={name,connection,key,value,elektra,mountpoint,file,plugins,hosts,tasks},gobble=4]
	- name: setup samba
	  connection: local
	  hosts: localhost
	  tasks:
	  - name: set workgroup
	    elektra:
	      mountpoint: system/sw/samba
	      file: /etc/samba/smb.conf
	      plugins: ini
	    elektra:
	      key: 'system/sw/samba/global/workgroup'
	      value: 'MY_WORKGROUP'
	\end{code}
\end{frame}

\begin{frame}
	\frametitle{Key/Values Revisited}

	Decide about \textbf{changeability} per key:

	\begin{itemize}[<+-| alert@+>]
	\item Who is responsible (end user, packages, admin manual or CM).
	\item In which namespaces apps search the key (cascading lookup).
	\item Who can see it (visibility).
	\item Who can edit it (admin, end user, both).
	\item Which configuration values are allowed (validation).
	\end{itemize}

	\pause[\thebeamerpauses]  %  show after \begin{itemize}[<+->]

	\begin{alertblock}{Changeability}
	Ownership of every key must be very clear and documented.
	\end{alertblock}
\end{frame}

\begin{frame}
	\frametitle{Layers of Abstractions}

	Recursively define useful abstractions (meta-levels):

	\begin{itemize}[<+-| alert@+>]
	\item Bits in (configuration) files and memory
	\item Key/value view of configuration settings
	\item Goals/specifications of settings per node and instantiations of modules
	\vspace{1em}
	\item CM code to instantiate settings in the whole network
	\item Global optimization: allocation of nodes and decision regarding topology in the whole network
	\item Global goals/specifications of the whole network
	\end{itemize}
\end{frame}

\begin{frame}
	\frametitle{Design Rules~\cite{burgess2006modeling}}

	\begin{itemize}[<+-| alert@+>]
	\item Factor processes into containers to avoid overlaps in settings.
	\item Maintain clear separation of ownership (for every key).
	\item Specify replicated settings in a single source (use links and derivations).
	\item Document all remaining overlaps (in the specification).
	\item The manageability of settings is reduced by the number of possible configuration values.
	\end{itemize}
\end{frame}


\begin{frame}
	\frametitle{Open Topics}

	\begin{itemize}[<+-| alert@+>]
	\item global optimizations/self-healing
	\item configuration integration
	\item safe migrations of settings and data
	\item collaboration
	\item management (including knowledge)
	\item centralized vs.\ distributed
	\end{itemize}
\end{frame}


\begin{frame}
	\frametitle{Conclusion}

	\begin{itemize}[<+-| alert@+>]
	\item have unique identifier for your configurations settings \\ $\rightarrow$ allows to get/set configurations and specifications
	\item solving CM is solving constraints \\ $\rightarrow$ be aware of the specifications
	\item do not design around tools but design tools around you
	\item be brave and remove all configuration settings you can
	\item use all help you can get: e.g.\ build tools, preseeding, installer automation, virtualization, package managers, distributions
	\item complexity in CM vs.\ complexity in applications' specification
%	\item modularity is essential for validation and legacy support
%	\item artifact generation improves consistency and type safety
	\end{itemize}
\end{frame}



%%%%%%%%%%%%%%%%%%%%%%%%%%%%%%%%%%%%%%%%%%
\section{Meeting}

%\begin{frame}
%	\frametitle{CM Languages (Partly Recapitulation)}
%
%	\begin{itemize}[<+-| alert@+>]
%	\item What is the relationship to software configuration management (Proteus/PCL)?
%	\item[] Build systems may provide configuration management features.
%	\item How is it possible to provide referential transparency both for the configuration specification language and for the system itself (NIX, GNU Guix)?
%	\item[] By functional languages and file system (layouts).
%	\item Which notations for CM exist?
%	\item[] Text,  Graphical (UML), Semi-structured, Key-value, Structured
%	\end{itemize}
%\end{frame}
%
%\begin{assignment}
%	\begin{task}
%	Break.
%	\end{task}
%\end{assignment}



\appendix

\begin{frame}[allowframebreaks]
	\bibliographystyle{plainnat}
	\bibliography{../shared/elektra.bib}
\end{frame}


\end{document}
