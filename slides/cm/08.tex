% Make nice A4 pages for print:
%\usepackage{pgfpages}
%\pgfpagesuselayout{resize to}[a4paper,border shrink=5mm,landscape]

\beamertemplatenavigationsymbolsempty

\setbeamertemplate{bibliography item}[text]

\usepackage[type={CC},modifier={by-sa},version={4.0}]{doclicense}

\usepackage[utf8]{inputenc}
\usepackage{hyperref}
\usepackage{breakurl}
\usepackage{graphicx}
\usepackage{pgfplots}
\usepackage{pgf}
\usepackage{tikz}
\usetikzlibrary{positioning}
\usetikzlibrary{arrows}
\usetikzlibrary{decorations.markings}
\usetikzlibrary{calc}
\usetikzlibrary{matrix}
\usetikzlibrary{shapes}
\usetikzlibrary{decorations.pathmorphing}
\usetikzlibrary{fit}
\usetikzlibrary{backgrounds}
\usetikzlibrary{plotmarks}
\usepackage{stmaryrd}
\usepackage{listings}
\usepackage{pdflscape}
\usepackage{perpage}
\usepackage{appendixnumberbeamer}

%\usepackage[thmmarks,amsmath,amsthm]{ntheorem} % already included in beamer
\usepackage{thm-restate}

\usepackage[sort&compress,numbers]{natbib}  % to be have \citet, \citeauthor, \citeyear

\MakePerPage{footnote}

\tikzstyle{o}=[r,ppBlue]
\tikzstyle{r}=[thick,rectangle,align=center]
\tikzstyle{t}=[r,ppTrans] %,font=\bfseries]
\tikzstyle{dd}=[densely dashed]
\tikzstyle{n}=[r,ppBlue]
\tikzstyle{p}=[r,ppRed]
\tikzstyle{ppRed}  =[draw=red,  fill=  red!20]
\tikzstyle{ppBlue} =[draw=blue, fill= blue!20]
\tikzstyle{ppGreen}=[draw=green,fill=green!20]
\tikzstyle{ppTrans}=[draw=none, fill=none]

\usetheme{Warsaw}

\useoutertheme[subsection=true]{smoothbars}
%\useoutertheme[subsection=false]{miniframes}

\definecolor{bblue}{HTML}{D7DF01}	% yellow-ish actually, for better black/white printing
\definecolor{rred}{HTML}{C0504D}
\definecolor{ggreen}{HTML}{9BBB59}
\definecolor{ppurple}{HTML}{9F4C7C}
\definecolor{lightgray}{rgb}{0.3,0.3,0.3}
\definecolor{lightergray}{rgb}{0.9,0.9,0.9}
\definecolor{UniBlue}{RGB}{83,121,170}

\DeclareTextFontCommand\textintro{\normalfont\bfseries\itshape} % nice!
\newcommand{\intro}[2][]
{%
	\textintro{#2}%
}
\newcommand{\empha}[2][]
{%
	\emph{#2}%
}

%\theoremstyle{plain}
\newcounter{reqcounter}
\newtheorem{requirement}[reqcounter]{Requirement}

%setbeamercolor{structure}{fg=violet}

\makeatletter
\def\th@task{%
    \normalfont % body font
    \setbeamercolor{block title example}{bg=orange,fg=white}
    \setbeamercolor{block body example}{bg=orange!20,fg=black}
    \def\inserttheoremblockenv{exampleblock}
  }
\makeatother

\theoremstyle{task}
\newtheorem{task}{Task}

\newenvironment{assignment}%
{%\setbeamercolor{background canvas}{bg=violet}%
%\setbeamercolor{structure}{fg=cyan!90!black}%
 \setbeamercolor{frametitle}{bg=orange,fg=white}
\begin{frame}}%
{\end{frame}}%

\AtBeginSection[]{
  \begin{frame}
  \vfill
  \centering
  \begin{beamercolorbox}[sep=8pt,center,shadow=true,rounded=true]{title}
    \usebeamerfont{title}\insertsectionhead\par%
  \end{beamercolorbox}
  \tableofcontents
  \vfill
  \end{frame}
}




\pgfplotsset{compat=1.14}
\author{Markus Raab}


\title{L08 Early Detection of Misconfiguration}
\date{19.05.2021}

\begin{document}

%%%%%%%%%%%%%%%%%%%%%%%%%%%%%%%%%%%%%%%%%% 
\section{Points in Time}

\begin{frame}
	\frametitle{Learning Outcomes}
	Students will be able to
	\begin{itemize}
	\item recall points of time relevant in configuration management.
	\item remind some arguments about pull vs.\ push.
	\item remember various strategies for earlier reduction of misconfiguration.
	\end{itemize}
\end{frame}

\begin{frame}
\frametitle{When are settings used?}
	From the application's perspective:
	\vspace{1em}

	\setbeamersize{description width=3.5cm}
	\begin{description}[<+-| alert@+>]
	\item[Implementation-time:] Configuration accesses \index{implementation-time}
	are hard-coded in the sou\-rce code.
	For example, architectural decisions~\cite{zdun2007patterns} lead to impl\-ementation-time settings.

	\item[Compile-time:] Configuration accesses \index{compile-time}
	are resolved by the build system while compiling.

	\item[Deployment-time:] Configuration accesses \index{deployment-time}
	are while the software is installed.

	\item[Load-time:] Configuration accesses \index{load-time}
	are during the start of applications.

	\item[Run-time:] Configuration accesses \index{run-time}
	are during execution after the startup procedure.
	\end{description}
\end{frame}

\begin{frame}
	\frametitle{Detection of Misconfiguration}

	\begin{alertblock}{Viewpoint}
	Different viewpoint: now from configuration management perspective.
	\end{alertblock}

	Phases when we can detect misconfigurations:
	\begin{itemize}[<+-| alert@+>]
	\item Compilation stage in configuration management tool
	\item Writing configuration settings on nodes
	\item Starting applications (load-time)
	\item When configuration setting is actually used (run-time) \\
		$\rightarrow$ Latent Misconfiguration
	\end{itemize}

	\pause[\thebeamerpauses]

	\begin{alertblock}{Problem}
	Earlier versus more context.
	\end{alertblock}
\end{frame}

%%%%%%%%%%%%%%%%%%%%%%%%%%%%%%%%%%%%%%%%%% 
\section{Push vs.\ Pull}

\begin{frame}
	\frametitle{Push vs.\ Pull}

	\begin{itemize}[<+-| alert@+>]
	\item Push is more interactive.
	\item Push cannot do its job if nodes are not reachable.
	\item Push needs additional techniques to scale with many nodes.
	\item Push demands access to servers from a single server.
	\item Pull needs additional monitoring to know when a patch has been applied.
	\item Pull needs resources even if nothing is to do.
	\end{itemize}

	\pause[\thebeamerpauses]

	\begin{task}
	Do you prefer push or pull?
	What does your CM tool of choice use?
	\end{task}
\end{frame}

%%%%%%%%%%%%%%%%%%%%%%%%%%%%%%%%%%%%%%%%%% 
\section{Early Detection}

\begin{frame}
	As shown by \citet{xu2016early}:
	\begin{itemize}[<+-| alert@+>]
	\item \p{12} -- \p{39} configuration settings are not used at all during the application's startup procedure.
	\item Applications often have latent misconfigurations (\p{14} -- \p{93}).
	\item Latent misconfigurations are particular severe (\p{75} of high-severity misconfigurations).
	\item Latent misconfiguration needs longer to diagnose.
	\end{itemize}
\end{frame}

\begin{frame}
	\frametitle{Checkers as plugins}

	Using checkers as plugins exclude whole classes of errors such as:
	\begin{itemize}[<+-| alert@+>]
	\item Invalid file paths using the plugin \plugin{path}.
	\item Invalid IP addresses or host names using the plugins \plugin{network} or \plugin{ipaddr}.
	\end{itemize}

	\pause[\thebeamerpauses]
	Because the checks occur before the resources are actually used, the checks are subject to race conditions.\footnotemark
	\only<3->{\footnotetext[1]{For example, a path that was present during the check, can have been removed when the application tries to access it.}}

	\pause
	In some situations facilities of the operating system help\footnotemark, in others we have fundamental problems.\footnotemark
	\only<4>{\footnotetext[2]{For example, we open the file during the check and pass \texttt{/proc/<pid>/fd/<fd>} to the application. This file cannot be unlinked, but unfortunately the file descriptor requires resources.}}
	\only<4>{\footnotetext[3]{For example, if the host we want to reach has gone offline after validation.}}
\end{frame}

\begin{frame}[fragile]
	\frametitle{Example~\cite{xu2016early}}

	Squid uses ^diskd_program^ but not before requests are served.
	Latent misconfiguration caused 7h downtime and 48h diagnosis effort.

	\pause

	\begin{alertblock}{Finding}
	Configuration from all externals programs need to be checked, too.
	\end{alertblock}
\end{frame}

\begin{frame}
	\frametitle{Conclusion}

	\begin{itemize}[<+-| alert@+>]
	\item provide external specifications for other tooling and configuration management
	\item use code generation to keep internal specifications consistent with external specifications (e.g. for refactoring)
	\item implement checkers as plugins
	\item execute checkers as early as possible,
	also for external programs executed later
	\item keep important resources allocated after checking
	\end{itemize}
\end{frame}

%%%%%%%%%%%%%%%%%%%%%%%%%%%%%%%%%%%%%%%%%% 
\section{Meeting}

\subsection{Recapitulation}

\begin{frame}
\frametitle{When are settings used?}
	From the application's perspective:
	\vspace{1em}

	\pause

	\setbeamersize{description width=3.5cm}
	\begin{description} %[<+-| alert@+>]
	\item[Implementation-time:] Configuration accesses \index{implementation-time}
	are hard-coded in the sou\-rce code.
	For example, architectural decisions~\cite{zdun2007patterns} lead to impl\-ementation-time settings.

	\item[Compile-time:] Configuration accesses \index{compile-time}
	are resolved by the build system while compiling.

	\item[Deployment-time:] Configuration accesses \index{deployment-time}
	are while the software is installed.

	\item[Load-time:] Configuration accesses \index{load-time}
	are during the start of applications.

	\item[Run-time:] Configuration accesses \index{run-time}
	are during execution after the startup procedure.
	\end{description}
\end{frame}

\begin{frame}
	\frametitle{Application vs. CM tool perspective?}

	\pause

	Phases when we can detect misconfigurations:
	\begin{itemize} %[<+-| alert@+>]
	\item Compilation stage in configuration management tool
	\item Writing configuration settings on nodes
	\item Starting applications (load-time)
	\item When configuration setting is actually used (run-time) \\
		$\rightarrow$ Latent Misconfiguration
	\end{itemize}
\end{frame}

\begin{assignment}
	\begin{task}
	Break.
	\end{task}
\end{assignment}

\begin{frame}
	\frametitle{Push vs.\ Pull}

	\pause

	\begin{itemize} %[<+-| alert@+>]
	\item Push is more interactive.
	\item Push cannot do its job if nodes are not reachable.
	\item Push needs additional techniques to scale with many nodes.
	\item Push demands access to servers from a single server.
	\item Pull needs additional monitoring to know when a patch has been applied.
	\item Pull needs resources even if nothing is to do.
	\end{itemize}

	\pause[\thebeamerpauses]

	\begin{task}
	Do you prefer push or pull?
	What does your CM tool of choice use?
	\end{task}
\end{frame}

\begin{assignment}
	\begin{task}
	Break.
	\end{task}
\end{assignment}

\subsection{Talks}

\begin{frame}[fragile]
	\frametitle{Talks}

	\begin{code}[language=xml]
  <talk>
    <date>19.05.2021</date>
    <name>@aaronabebe</name>
    <topic>externalized configuration in distributed systems</topic>
  </talk>

  <talk>
    <date>19.05.2021</date>
    <name>@philippoppel</name>
    <topic>Configuration Integration based on T2</topic>
  </talk>
	\end{code}
\end{frame}

\subsection{Assignments}

\begin{assignment}
	Please add slides for talk in TUWEL \textbf{and} private git repo, dates:

	\begin{itemize}[<+-| alert@+>]
	\item 26 May: peer review 
	\end{itemize}
\end{assignment}

\begin{assignment}
	T3 deadline today:

	What means ``only partial changes''?
\end{assignment}

\begin{frame}
	\frametitle{Feedback}
	\hfill \includegraphics[width=2cm]{pics/feedback.png}
	\vspace{-1cm}
	\begin{itemize}
		\item ECTS breakdown realistic?
		\item Feedback Talk
	\end{itemize}
\end{frame}

\subsection{Preview}

\begin{frame}[fragile]
	\frametitle{Talks}

	\small
	\begin{code}[language=xml]
  <talk>
    <date>26.05.2021</date>
    <name>@a-kraschitzer</name>
    <topic>configuration migration</topic>
  </talk>

  <talk>
    <date>26.05.2021</date>
    <name>@robaerd</name>
    <topic>infrastructure as code</topic>
  </talk>

  <talk>
    <date>26.05.2021</date>
    <name>@tucek</name>
    <topic>A short introduction on how we configure our services at work</topic>
  </talk>
	\end{code}
\end{frame}




\appendix

\begin{frame}[allowframebreaks]
	\bibliographystyle{plainnat}
	\bibliography{../shared/elektra.bib}
\end{frame}



\end{document}
