%\ifdefined\handout
%\documentclass[handout,aspectratio=1610,xcolor={usenames,dvipsnames,table}]{beamer}
%\else
\documentclass[aspectratio=1610,xcolor={usenames,dvipsnames,table}]{beamer}
%\fi

\newcommand{\mylecture}{Configuration Management}

%\ifdefined\handout
%\documentclass[handout,aspectratio=1610,xcolor={usenames,dvipsnames,table}]{beamer}
%\else
\documentclass[aspectratio=1610,xcolor={usenames,dvipsnames,table}]{beamer}
%\fi

\newcommand{\mylecture}{Configuration Management}

%\ifdefined\handout
%\documentclass[handout,aspectratio=1610,xcolor={usenames,dvipsnames,table}]{beamer}
%\else
\documentclass[aspectratio=1610,xcolor={usenames,dvipsnames,table}]{beamer}
%\fi

\newcommand{\mylecture}{Configuration Management}

\input{../setup}
\input{../shared/setup}

\lstdefinelanguage{dump}
{
	morekeywords={kdbOpen,ksNew,keyNew,keyMeta,keyCopyMeta,keyEnd,ksEnd,kdbClose},
	sensitive=false,
	morecomment=[l]{//},
	morecomment=[s]{/*}{*/},
	morestring=[b]",
}


\lstdefinelanguage{SpecElektra}{
	%
	comment=[l]{;},
	commentstyle=\color{purple}\ttfamily,
	%
	morestring=[b]',
	morestring=[b]`,
	morestring=[b]",
	stringstyle=\color{purple}\ttfamily,
	%
	sensitive=f,% keywords are not case sensitive
	%
	% Colors see https://en.wikibooks.org/wiki/LaTeX/Colors
	%
	keywordstyle=\color{BlueViolet}\bfseries,
	keywordstyle=[2]\color{Green},
	keywordstyle=[3]\color{Aquamarine}\bfseries\textit,
	keywordstyle=[4]\color{NavyBlue}\bfseries,
	keywordstyle=[5]\color{Mahogany},
	%
	keywords={layer, require, validation, check, range, description, rationale, requirement, accessibility, enable, condition, message, default, opt, readonly, type, context, property1, property2, description, file, content, mountpoint, metadata, infos, plugins},
	keywords=[2]{},
	keywords=[3]{order, interface, network, emphasized},
	%keywords=[4]{[, ]},  %Not needed
	keywords=[4]{},
	keywords=[5]{},
	%
	literate={:=}{{{\color{red}\textbf:=}}}2
		 {\%}{{{\color{NavyBlue}\textbf\%}}}1
		 {[}{{{\color{Sepia}\textbf[}}}1
		 {]}{{{\color{Sepia}\textbf]}}}1,
}

\lstdefinelanguage{Cpp}{%
	language     = C++,
	literate=
}


\lstdefinelanguage{CfgElektra}{
	comment=[l]{;},
	commentstyle=\color{purple}\ttfamily,
	%
	morestring=[b]',
	morestring=[b]`,
	morestring=[b]",
	stringstyle=\color{purple}\ttfamily,
	%
	%
	sensitive=f,% keywords are not case sensitive
	%
	% Colors see https://en.wikibooks.org/wiki/LaTeX/Colors
	%
	keywordstyle=\color{Bittersweet}\bfseries,
	keywordstyle=[2]\color{DarkOrchid}\bfseries,
	keywordstyle=[3]\color{ForestGreen}\bfseries\textit,
	keywordstyle=[4]\color{Goldenrod}\bfseries,
	keywordstyle=[5]\color{CarnationPink},
	%
	keywords={},
	keywords=[2]{},
	keywords=[3]{},
	keywords=[4]{},
	keywords=[5]{},
	%
	literate={=}{{{\color{ForestGreen}\textbf=}}}1
		 %{<-}{{{\color{ForestGreen}\textbf<-}}}2
		 %{*}{{{\color{Bittersweet}\textbf*}}}1
		 {\%}{{{\color{NavyBlue}\textbf\%}}}1,
}




\lstset{language=SpecElektra, % Use SpecElektra as default programming language
	%boxpos=t, % make boxes a bit more unbreakable
	%frame=lines, % top+bottom line
	basicstyle=\ttfamily, % Use normal-size true type font
	showspaces,%
	showstringspaces=false, % Don't put marks in string spaces
	showlines=true, % make sure empty lines at end are shown (needed for concurrency
	tabsize=4, % spaces per tab
	xleftmargin=\parindent, % should be 18pt or 1.5em as defined by memoir
	%Does not really work well (needs to be deactivated for shortlistings):
	breaklines=false,
	%postbreak=\mbox{\textcolor{red}{$\hookrightarrow$}\space},
	%breakautoindent=true,
	%prebreak={\mbox{\ensuremath{\curvearrowright}}} % Zeichen am Zeilenende (Umbruch)
	%breaklines=true,
	%breakautoindent=true,
	%prebreak=\small\symbol{'134}, % backslash
	%prebreak={\mbox{\ensuremath{\curvearrowright}}} % lange kure
	%prebreak={\mbox{\ensuremath{\hookleftarrow}}} % lange kure
	%xleftmargin=3.0ex, %for some formats
	%xrightmargin=1.0ex, %for some formats
	%
	% Files do not work in utf8 see also:
	% http://stackoverflow.com/questions/1116266/listings-in-latex-with-utf-8-or-at-least-german-umlauts
	% http://tex.stackexchange.com/questions/24528/having-problems-with-listings-and-utf-8-can-it-be-fixed
	% Should work but doesn't? (Maybe add to literate broken?)
	%add to literate={ö}{{\"o}}1
	%	{ä}{{\"a}}1
	%	{ü}{{\"u}}1
	%	{Ö}{{\"O}}1
	%	{Ä}{{\"A}}1
	%	{Ü}{{\"U}}1
	%	{ß}{{\ss}}1,
	%
	% listingsutf8 did not work, made umlauts in comments very strange
	%extendedchars=true,
	%inputencoding=utf8,
	%
	%morecomment=[l][\color{blue}]{...}, % Line continuation (...) e.g. blue comment
	morekeywords={for_each},
	numbers=left, % Line numbers on left
	firstnumber=1, % Line numbers start with line 1
	numberstyle=\small\color{blue}, % Line numbers are blue and small
	numbersep=5pt,
	%stepnumber=5 % Line numbers go in steps of 5
}



\lstMakeShortInline[postbreak=,keywordstyle={}]^

\graphicspath{{../pic/}{../figures/}{../graphics/}{../ipe/}{../ggplot/}}



\lstdefinelanguage{dump}
{
	morekeywords={kdbOpen,ksNew,keyNew,keyMeta,keyCopyMeta,keyEnd,ksEnd,kdbClose},
	sensitive=false,
	morecomment=[l]{//},
	morecomment=[s]{/*}{*/},
	morestring=[b]",
}


\lstdefinelanguage{SpecElektra}{
	%
	comment=[l]{;},
	commentstyle=\color{purple}\ttfamily,
	%
	morestring=[b]',
	morestring=[b]`,
	morestring=[b]",
	stringstyle=\color{purple}\ttfamily,
	%
	sensitive=f,% keywords are not case sensitive
	%
	% Colors see https://en.wikibooks.org/wiki/LaTeX/Colors
	%
	keywordstyle=\color{BlueViolet}\bfseries,
	keywordstyle=[2]\color{Green},
	keywordstyle=[3]\color{Aquamarine}\bfseries\textit,
	keywordstyle=[4]\color{NavyBlue}\bfseries,
	keywordstyle=[5]\color{Mahogany},
	%
	keywords={layer, require, validation, check, range, description, rationale, requirement, accessibility, enable, condition, message, default, opt, readonly, type, context, property1, property2, description, file, content, mountpoint, metadata, infos, plugins},
	keywords=[2]{},
	keywords=[3]{order, interface, network, emphasized},
	%keywords=[4]{[, ]},  %Not needed
	keywords=[4]{},
	keywords=[5]{},
	%
	literate={:=}{{{\color{red}\textbf:=}}}2
		 {\%}{{{\color{NavyBlue}\textbf\%}}}1
		 {[}{{{\color{Sepia}\textbf[}}}1
		 {]}{{{\color{Sepia}\textbf]}}}1,
}

\lstdefinelanguage{Cpp}{%
	language     = C++,
	literate=
}


\lstdefinelanguage{CfgElektra}{
	comment=[l]{;},
	commentstyle=\color{purple}\ttfamily,
	%
	morestring=[b]',
	morestring=[b]`,
	morestring=[b]",
	stringstyle=\color{purple}\ttfamily,
	%
	%
	sensitive=f,% keywords are not case sensitive
	%
	% Colors see https://en.wikibooks.org/wiki/LaTeX/Colors
	%
	keywordstyle=\color{Bittersweet}\bfseries,
	keywordstyle=[2]\color{DarkOrchid}\bfseries,
	keywordstyle=[3]\color{ForestGreen}\bfseries\textit,
	keywordstyle=[4]\color{Goldenrod}\bfseries,
	keywordstyle=[5]\color{CarnationPink},
	%
	keywords={},
	keywords=[2]{},
	keywords=[3]{},
	keywords=[4]{},
	keywords=[5]{},
	%
	literate={=}{{{\color{ForestGreen}\textbf=}}}1
		 %{<-}{{{\color{ForestGreen}\textbf<-}}}2
		 %{*}{{{\color{Bittersweet}\textbf*}}}1
		 {\%}{{{\color{NavyBlue}\textbf\%}}}1,
}




\lstset{language=SpecElektra, % Use SpecElektra as default programming language
	%boxpos=t, % make boxes a bit more unbreakable
	%frame=lines, % top+bottom line
	basicstyle=\ttfamily, % Use normal-size true type font
	showspaces,%
	showstringspaces=false, % Don't put marks in string spaces
	showlines=true, % make sure empty lines at end are shown (needed for concurrency
	tabsize=4, % spaces per tab
	xleftmargin=\parindent, % should be 18pt or 1.5em as defined by memoir
	%Does not really work well (needs to be deactivated for shortlistings):
	breaklines=false,
	%postbreak=\mbox{\textcolor{red}{$\hookrightarrow$}\space},
	%breakautoindent=true,
	%prebreak={\mbox{\ensuremath{\curvearrowright}}} % Zeichen am Zeilenende (Umbruch)
	%breaklines=true,
	%breakautoindent=true,
	%prebreak=\small\symbol{'134}, % backslash
	%prebreak={\mbox{\ensuremath{\curvearrowright}}} % lange kure
	%prebreak={\mbox{\ensuremath{\hookleftarrow}}} % lange kure
	%xleftmargin=3.0ex, %for some formats
	%xrightmargin=1.0ex, %for some formats
	%
	% Files do not work in utf8 see also:
	% http://stackoverflow.com/questions/1116266/listings-in-latex-with-utf-8-or-at-least-german-umlauts
	% http://tex.stackexchange.com/questions/24528/having-problems-with-listings-and-utf-8-can-it-be-fixed
	% Should work but doesn't? (Maybe add to literate broken?)
	%add to literate={ö}{{\"o}}1
	%	{ä}{{\"a}}1
	%	{ü}{{\"u}}1
	%	{Ö}{{\"O}}1
	%	{Ä}{{\"A}}1
	%	{Ü}{{\"U}}1
	%	{ß}{{\ss}}1,
	%
	% listingsutf8 did not work, made umlauts in comments very strange
	%extendedchars=true,
	%inputencoding=utf8,
	%
	%morecomment=[l][\color{blue}]{...}, % Line continuation (...) e.g. blue comment
	morekeywords={for_each},
	numbers=left, % Line numbers on left
	firstnumber=1, % Line numbers start with line 1
	numberstyle=\small\color{blue}, % Line numbers are blue and small
	numbersep=5pt,
	%stepnumber=5 % Line numbers go in steps of 5
}



\lstMakeShortInline[postbreak=,keywordstyle={}]^

\graphicspath{{../pic/}{../figures/}{../graphics/}{../ipe/}{../ggplot/}}




\title{L08 Early Detection of Misconfiguration}
%\date{19.05.2021}

\begin{document}

%%%%%%%%%%%%%%%%%%%%%%%%%%%%%%%%%%%%%%%%%% 
\section{Points in Time}

\begin{frame}<1>[label=learning outcomes]
	\frametitle{Learning Outcomes}
	Students will be able to
	\begin{itemize}
	\item recall points of time relevant in configuration management.
	\item remind some arguments about pull vs.\ push.
	\item remember various strategies for earlier reduction of misconfiguration.
	\end{itemize}
\end{frame}

\begin{frame}<2-6>[label=application]
	\frametitle{When are settings used?}
	From the application's perspective:
	\vspace{1em}

	\pause

	\setbeamersize{description width=3.5cm}
	\begin{description}[<+-| alert@+>]
	\item[Implementation-time:] Configuration accesses \index{implementation-time}
	are hard-coded in the sou\-rce code.
	For example, architectural decisions~\cite{zdun2007patterns} lead to impl\-ementation-time settings.

	\item[Compile-time:] Configuration accesses \index{compile-time}
	are resolved by the build system while compiling.

	\item[Deployment-time:] Configuration accesses \index{deployment-time}
	are while the software is installed.

	\item[Load-time:] Configuration accesses \index{load-time}
	are during the start of applications.

	\item[Run-time:] Configuration accesses \index{run-time}
	are during execution after the startup procedure.
	\end{description}
\end{frame}

\begin{frame}<2-7>[label=CM tool]
	\frametitle{Detection of Misconfiguration}

	\begin{alertblock}{Viewpoint}
	Different viewpoint: now from configuration management perspective.
	\end{alertblock}

	\pause

	Phases when we can detect misconfigurations:

	\pause
	\begin{itemize}[<+-| alert@+>]
	\item Compilation stage in configuration management tool
	\item Writing configuration settings on nodes
	\item Starting applications (load-time)
	\item When configuration setting is actually used (run-time) \\
		$\rightarrow$ Latent Misconfiguration
	\end{itemize}

	\pause[\thebeamerpauses]

	\begin{alertblock}{Problem}
	Earlier versus more context.
	\end{alertblock}
\end{frame}

%%%%%%%%%%%%%%%%%%%%%%%%%%%%%%%%%%%%%%%%%% 
\section{Push vs.\ Pull}

\begin{frame}<2-7>[label=push vs. pull]
	\frametitle{Push vs.\ Pull}

	\pause

	\begin{itemize}[<+-| alert@+>]
	\item Push is more interactive.
	\item Push cannot do its job if nodes are not reachable.
	\item Push needs additional techniques to scale with many nodes.
	\item Push demands access to servers from a single server.
	\item Pull needs additional monitoring to know when a patch has been applied.
	\item Pull needs resources even if nothing is to do.
	\end{itemize}

	\pause[\thebeamerpauses]

	\begin{task}
	Do you prefer push or pull?
	What does your CM tool of choice use?
	\end{task}
\end{frame}

%%%%%%%%%%%%%%%%%%%%%%%%%%%%%%%%%%%%%%%%%% 
\section{Early Detection}

\begin{frame}
	As shown by \citet{xu2016early}:
	\begin{itemize}[<+-| alert@+>]
	\item \p{12} -- \p{39} configuration settings are not used at all during the application's startup procedure.
	\item Applications often have latent misconfigurations (\p{14} -- \p{93}).
	\item Latent misconfigurations are particular severe (\p{75} of high-severity misconfigurations).
	\item Latent misconfiguration needs longer to diagnose.
	\end{itemize}
\end{frame}

\begin{frame}
	\frametitle{Checkers as plugins}

	Using checkers as plugins exclude whole classes of errors such as:
	\begin{itemize}[<+-| alert@+>]
	\item Invalid file paths using the plugin \plugin{path}.
	\item Invalid IP addresses or host names using the plugins \plugin{network} or \plugin{ipaddr}.
	\end{itemize}

	\pause[\thebeamerpauses]
	Because the checks occur before the resources are actually used, the checks are subject to race conditions.\footnotemark
	\only<3->{\footnotetext[1]{For example, a path that was present during the check, can have been removed when the application tries to access it.}}

	\pause
	In some situations facilities of the operating system help\footnotemark, in others we have fundamental problems.\footnotemark
	\only<4>{\footnotetext[2]{For example, we open the file during the check and pass \texttt{/proc/<pid>/fd/<fd>} to the application. This file cannot be unlinked, but unfortunately the file descriptor requires resources.}}
	\only<4>{\footnotetext[3]{For example, if the host we want to reach has gone offline after validation.}}
\end{frame}

\begin{frame}[fragile]
	\frametitle{Example~\cite{xu2016early}}

	Squid uses ^diskd_program^ but not before requests are served.
	Latent misconfiguration caused 7h downtime and 48h diagnosis effort.

	\pause

	\begin{alertblock}{Finding}
	Configuration from all externals programs need to be checked, too.
	\end{alertblock}
\end{frame}

\begin{frame}
	\frametitle{Conclusion}

	\begin{itemize}[<+-| alert@+>]
	\item provide external specifications for other tooling and configuration management
	\item use code generation to keep internal specifications consistent with external specifications (e.g. for refactoring)
	\item implement checkers as plugins
	\item execute checkers as early as possible,
	also for external programs executed later
	\item keep important resources allocated after checking
	\end{itemize}
\end{frame}

%%%%%%%%%%%%%%%%%%%%%%%%%%%%%%%%%%%%%%%%%% 
\section{Meeting}

\subsection{Recapitulation}

\againframe<1,99>{application}

\againframe<1,99>{CM tool}

\breakframe

\againframe<1,99>{push vs. pull}

\begin{assignment}
	\frametitle{Brainstorming}

	\begin{alertblock}{Question}
	How to improve your apps to avoid latent misconfiguration?
	\end{alertblock}
\end{assignment}

\againframe<99>{learning outcomes}

\breakframe

\subsection{Assignments}

\begin{assignment}
	\frametitle{Release Notes}

	\begin{description}
	\item[For your best work,] you can write a ``Highlight'' section.
	\item[For other work,] changes should be completely listed.
	\end{description}
\end{assignment}

\begin{assignment}
	\frametitle{H3 Corrections}

	\begin{itemize}
	\item Install with given installation scripts
	\item Configure with your own CM code and your own configuration settings
	\end{itemize}
\end{assignment}

\begin{assignment}
	\frametitle{T3 Fix Issues}

	\begin{itemize}
	\item Always show that bug is fixed with test case.
	\item Enough issues?
	\end{itemize}
\end{assignment}

\breakframe

\begin{assignment}
	\frametitle{Feedback}
	\hfill \includegraphics[width=2cm]{pics/feedback.png}
	\vspace{-1cm}
	\begin{itemize}
		\item TISS feedback starting from 16.6.2022
		\item ECTS breakdown realistic?
		\item Topics for T3?
		\item Videos?
		\item Any other suggestions for improvements?
	\end{itemize}
\end{assignment}

\subsection{Preview}

\begin{frame}
	\frametitle{Outlook}
	Office hour today 13:00, next week 16:00.

	Next week last time without presentation.

	L09 Configuration as a User Interface:
	\begin{itemize}
		\item Three-way merge
		\item Error messages
		\item System Administrator Research
	\end{itemize}
\end{frame}




\appendix

\begin{frame}[allowframebreaks]
	\bibliographystyle{plainnat}
	\bibliography{../shared/elektra.bib}
\end{frame}



\end{document}
