%\ifdefined\handout
%\documentclass[handout,aspectratio=1610,xcolor={usenames,dvipsnames,table}]{beamer}
%\else
\documentclass[aspectratio=1610,xcolor={usenames,dvipsnames,table}]{beamer}
%\fi

\newcommand{\mylecture}{Configuration Management}

%\ifdefined\handout
%\documentclass[handout,aspectratio=1610,xcolor={usenames,dvipsnames,table}]{beamer}
%\else
\documentclass[aspectratio=1610,xcolor={usenames,dvipsnames,table}]{beamer}
%\fi

\newcommand{\mylecture}{Configuration Management}

%\ifdefined\handout
%\documentclass[handout,aspectratio=1610,xcolor={usenames,dvipsnames,table}]{beamer}
%\else
\documentclass[aspectratio=1610,xcolor={usenames,dvipsnames,table}]{beamer}
%\fi

\newcommand{\mylecture}{Configuration Management}

\input{../setup}
\input{../shared/setup}

\lstdefinelanguage{dump}
{
	morekeywords={kdbOpen,ksNew,keyNew,keyMeta,keyCopyMeta,keyEnd,ksEnd,kdbClose},
	sensitive=false,
	morecomment=[l]{//},
	morecomment=[s]{/*}{*/},
	morestring=[b]",
}


\lstdefinelanguage{SpecElektra}{
	%
	comment=[l]{;},
	commentstyle=\color{purple}\ttfamily,
	%
	morestring=[b]',
	morestring=[b]`,
	morestring=[b]",
	stringstyle=\color{purple}\ttfamily,
	%
	sensitive=f,% keywords are not case sensitive
	%
	% Colors see https://en.wikibooks.org/wiki/LaTeX/Colors
	%
	keywordstyle=\color{BlueViolet}\bfseries,
	keywordstyle=[2]\color{Green},
	keywordstyle=[3]\color{Aquamarine}\bfseries\textit,
	keywordstyle=[4]\color{NavyBlue}\bfseries,
	keywordstyle=[5]\color{Mahogany},
	%
	keywords={layer, require, validation, check, range, description, rationale, requirement, accessibility, enable, condition, message, default, opt, readonly, type, context, property1, property2, description, file, content, mountpoint, metadata, infos, plugins},
	keywords=[2]{},
	keywords=[3]{order, interface, network, emphasized},
	%keywords=[4]{[, ]},  %Not needed
	keywords=[4]{},
	keywords=[5]{},
	%
	literate={:=}{{{\color{red}\textbf:=}}}2
		 {\%}{{{\color{NavyBlue}\textbf\%}}}1
		 {[}{{{\color{Sepia}\textbf[}}}1
		 {]}{{{\color{Sepia}\textbf]}}}1,
}

\lstdefinelanguage{Cpp}{%
	language     = C++,
	literate=
}


\lstdefinelanguage{CfgElektra}{
	comment=[l]{;},
	commentstyle=\color{purple}\ttfamily,
	%
	morestring=[b]',
	morestring=[b]`,
	morestring=[b]",
	stringstyle=\color{purple}\ttfamily,
	%
	%
	sensitive=f,% keywords are not case sensitive
	%
	% Colors see https://en.wikibooks.org/wiki/LaTeX/Colors
	%
	keywordstyle=\color{Bittersweet}\bfseries,
	keywordstyle=[2]\color{DarkOrchid}\bfseries,
	keywordstyle=[3]\color{ForestGreen}\bfseries\textit,
	keywordstyle=[4]\color{Goldenrod}\bfseries,
	keywordstyle=[5]\color{CarnationPink},
	%
	keywords={},
	keywords=[2]{},
	keywords=[3]{},
	keywords=[4]{},
	keywords=[5]{},
	%
	literate={=}{{{\color{ForestGreen}\textbf=}}}1
		 %{<-}{{{\color{ForestGreen}\textbf<-}}}2
		 %{*}{{{\color{Bittersweet}\textbf*}}}1
		 {\%}{{{\color{NavyBlue}\textbf\%}}}1,
}




\lstset{language=SpecElektra, % Use SpecElektra as default programming language
	%boxpos=t, % make boxes a bit more unbreakable
	%frame=lines, % top+bottom line
	basicstyle=\ttfamily, % Use normal-size true type font
	showspaces,%
	showstringspaces=false, % Don't put marks in string spaces
	showlines=true, % make sure empty lines at end are shown (needed for concurrency
	tabsize=4, % spaces per tab
	xleftmargin=\parindent, % should be 18pt or 1.5em as defined by memoir
	%Does not really work well (needs to be deactivated for shortlistings):
	breaklines=false,
	%postbreak=\mbox{\textcolor{red}{$\hookrightarrow$}\space},
	%breakautoindent=true,
	%prebreak={\mbox{\ensuremath{\curvearrowright}}} % Zeichen am Zeilenende (Umbruch)
	%breaklines=true,
	%breakautoindent=true,
	%prebreak=\small\symbol{'134}, % backslash
	%prebreak={\mbox{\ensuremath{\curvearrowright}}} % lange kure
	%prebreak={\mbox{\ensuremath{\hookleftarrow}}} % lange kure
	%xleftmargin=3.0ex, %for some formats
	%xrightmargin=1.0ex, %for some formats
	%
	% Files do not work in utf8 see also:
	% http://stackoverflow.com/questions/1116266/listings-in-latex-with-utf-8-or-at-least-german-umlauts
	% http://tex.stackexchange.com/questions/24528/having-problems-with-listings-and-utf-8-can-it-be-fixed
	% Should work but doesn't? (Maybe add to literate broken?)
	%add to literate={ö}{{\"o}}1
	%	{ä}{{\"a}}1
	%	{ü}{{\"u}}1
	%	{Ö}{{\"O}}1
	%	{Ä}{{\"A}}1
	%	{Ü}{{\"U}}1
	%	{ß}{{\ss}}1,
	%
	% listingsutf8 did not work, made umlauts in comments very strange
	%extendedchars=true,
	%inputencoding=utf8,
	%
	%morecomment=[l][\color{blue}]{...}, % Line continuation (...) e.g. blue comment
	morekeywords={for_each},
	numbers=left, % Line numbers on left
	firstnumber=1, % Line numbers start with line 1
	numberstyle=\small\color{blue}, % Line numbers are blue and small
	numbersep=5pt,
	%stepnumber=5 % Line numbers go in steps of 5
}



\lstMakeShortInline[postbreak=,keywordstyle={}]^

\graphicspath{{../pic/}{../figures/}{../graphics/}{../ipe/}{../ggplot/}}



\lstdefinelanguage{dump}
{
	morekeywords={kdbOpen,ksNew,keyNew,keyMeta,keyCopyMeta,keyEnd,ksEnd,kdbClose},
	sensitive=false,
	morecomment=[l]{//},
	morecomment=[s]{/*}{*/},
	morestring=[b]",
}


\lstdefinelanguage{SpecElektra}{
	%
	comment=[l]{;},
	commentstyle=\color{purple}\ttfamily,
	%
	morestring=[b]',
	morestring=[b]`,
	morestring=[b]",
	stringstyle=\color{purple}\ttfamily,
	%
	sensitive=f,% keywords are not case sensitive
	%
	% Colors see https://en.wikibooks.org/wiki/LaTeX/Colors
	%
	keywordstyle=\color{BlueViolet}\bfseries,
	keywordstyle=[2]\color{Green},
	keywordstyle=[3]\color{Aquamarine}\bfseries\textit,
	keywordstyle=[4]\color{NavyBlue}\bfseries,
	keywordstyle=[5]\color{Mahogany},
	%
	keywords={layer, require, validation, check, range, description, rationale, requirement, accessibility, enable, condition, message, default, opt, readonly, type, context, property1, property2, description, file, content, mountpoint, metadata, infos, plugins},
	keywords=[2]{},
	keywords=[3]{order, interface, network, emphasized},
	%keywords=[4]{[, ]},  %Not needed
	keywords=[4]{},
	keywords=[5]{},
	%
	literate={:=}{{{\color{red}\textbf:=}}}2
		 {\%}{{{\color{NavyBlue}\textbf\%}}}1
		 {[}{{{\color{Sepia}\textbf[}}}1
		 {]}{{{\color{Sepia}\textbf]}}}1,
}

\lstdefinelanguage{Cpp}{%
	language     = C++,
	literate=
}


\lstdefinelanguage{CfgElektra}{
	comment=[l]{;},
	commentstyle=\color{purple}\ttfamily,
	%
	morestring=[b]',
	morestring=[b]`,
	morestring=[b]",
	stringstyle=\color{purple}\ttfamily,
	%
	%
	sensitive=f,% keywords are not case sensitive
	%
	% Colors see https://en.wikibooks.org/wiki/LaTeX/Colors
	%
	keywordstyle=\color{Bittersweet}\bfseries,
	keywordstyle=[2]\color{DarkOrchid}\bfseries,
	keywordstyle=[3]\color{ForestGreen}\bfseries\textit,
	keywordstyle=[4]\color{Goldenrod}\bfseries,
	keywordstyle=[5]\color{CarnationPink},
	%
	keywords={},
	keywords=[2]{},
	keywords=[3]{},
	keywords=[4]{},
	keywords=[5]{},
	%
	literate={=}{{{\color{ForestGreen}\textbf=}}}1
		 %{<-}{{{\color{ForestGreen}\textbf<-}}}2
		 %{*}{{{\color{Bittersweet}\textbf*}}}1
		 {\%}{{{\color{NavyBlue}\textbf\%}}}1,
}




\lstset{language=SpecElektra, % Use SpecElektra as default programming language
	%boxpos=t, % make boxes a bit more unbreakable
	%frame=lines, % top+bottom line
	basicstyle=\ttfamily, % Use normal-size true type font
	showspaces,%
	showstringspaces=false, % Don't put marks in string spaces
	showlines=true, % make sure empty lines at end are shown (needed for concurrency
	tabsize=4, % spaces per tab
	xleftmargin=\parindent, % should be 18pt or 1.5em as defined by memoir
	%Does not really work well (needs to be deactivated for shortlistings):
	breaklines=false,
	%postbreak=\mbox{\textcolor{red}{$\hookrightarrow$}\space},
	%breakautoindent=true,
	%prebreak={\mbox{\ensuremath{\curvearrowright}}} % Zeichen am Zeilenende (Umbruch)
	%breaklines=true,
	%breakautoindent=true,
	%prebreak=\small\symbol{'134}, % backslash
	%prebreak={\mbox{\ensuremath{\curvearrowright}}} % lange kure
	%prebreak={\mbox{\ensuremath{\hookleftarrow}}} % lange kure
	%xleftmargin=3.0ex, %for some formats
	%xrightmargin=1.0ex, %for some formats
	%
	% Files do not work in utf8 see also:
	% http://stackoverflow.com/questions/1116266/listings-in-latex-with-utf-8-or-at-least-german-umlauts
	% http://tex.stackexchange.com/questions/24528/having-problems-with-listings-and-utf-8-can-it-be-fixed
	% Should work but doesn't? (Maybe add to literate broken?)
	%add to literate={ö}{{\"o}}1
	%	{ä}{{\"a}}1
	%	{ü}{{\"u}}1
	%	{Ö}{{\"O}}1
	%	{Ä}{{\"A}}1
	%	{Ü}{{\"U}}1
	%	{ß}{{\ss}}1,
	%
	% listingsutf8 did not work, made umlauts in comments very strange
	%extendedchars=true,
	%inputencoding=utf8,
	%
	%morecomment=[l][\color{blue}]{...}, % Line continuation (...) e.g. blue comment
	morekeywords={for_each},
	numbers=left, % Line numbers on left
	firstnumber=1, % Line numbers start with line 1
	numberstyle=\small\color{blue}, % Line numbers are blue and small
	numbersep=5pt,
	%stepnumber=5 % Line numbers go in steps of 5
}



\lstMakeShortInline[postbreak=,keywordstyle={}]^

\graphicspath{{../pic/}{../figures/}{../graphics/}{../ipe/}{../ggplot/}}




\title{LRE Recapitulation}
\date{23.06.2021}

\begin{document}


%%%%%%%%%%%%%%%%%%%%%%%%%%%%%%%%%%%%%%%%%% 
\section{Recapitulation}

\begin{frame}
	\frametitle{Metalevels}
	\begin{alertblock}{Question}
	Describe the three Metalevels in Elektra and CM.
	\end{alertblock}

	\pause
	\includegraphics{metalevels}
\end{frame}

\begin{frame}
	\frametitle{Integration: Current Situation}

	\pause
	\includegraphics[scale=0.7]{cursituation}
\end{frame}

\begin{frame}
	\frametitle{Integration: Wanted Situation}

	\pause
	\includegraphics[scale=0.7]{wantsituation}
\end{frame}

\begin{frame}
	\frametitle{Possible Benefits of CM}

	\begin{task}
	What are the goals of CM Tools?
	\end{task}

	\pause

	\begin{itemize} %[<+-| alert@+>]
	\item The same goals scripts have: \\
		Documentation, Customization, Reproducability
	\item Declarative description of the system \\
		Single Source of Truth 
		(Infrastructure as Code~\cite{waldemar2013testing})
	\item Less configuration drift
	\item Error handling
	\item Pull vs.\ Push
	\item Reusability
	%\item (Resource) Abstractions
	\end{itemize}
\end{frame}

\begin{frame}
	\frametitle{Develop CM-aware applications}

	\begin{task}
	What needs to be considered by developers to make applications CM-aware?
	\end{task}

	\pause[\thebeamerpauses]  %  show after \begin{itemize}[<+->]

	\begin{itemize} %[<+-| alert@+>]
	\item reduce the configuration complexity
	\item intensively review and improve the specifications
	\item test (and debug) different configuration settings
	\item allow introspection
	\item consider context
	\end{itemize}
\end{frame}

%\breakframe

\section{Conclusion}

\begin{frame}
	\frametitle{Popular Topics 2022S}
	\vspace{-0.55cm}
	\setlength{\columnsep}{-1.3cm}
	\raggedright
	\definecolor{amethyst}{rgb}{0.6, 0.4, 0.8}
	\begin{multicols}{2}
	\begin{description}
	\item[6] {\color{amethyst} CM-Tools: Ansible}
	\item[4] {\color{amethyst} Context Awareness} % Collected in L10
	\item[3] {\color{amethyst} Design Configuration} % Collected in L10
	\item[3] {\color{amethyst} Validation}
	\item[3] {\color{amethyst} configuration versioning}
	\item[2] {\color{gray} Spring Initializr (https://start.spring.io/)}
	\item[2] {\color{amethyst} Specification and Integration}
	\item[2] {\color{red} Kubernetes CM}
	\item[2] {\color{amethyst} Integration (Common view on different configuration sources)}
	\item[2] {\color{amethyst} Infrastructure CM}
	\item[1] {\color{gray} Docker Compose YAML}
	\end{description}
	\end{multicols}
\end{frame}

\begin{frame}
	\frametitle{Learning Outcomes (TISS)}
	Students will be able to

	\begin{enumerate}
	% Technical and Methodological Knowledge
	\item support configuration management during software engineering,
	\item describe systematic approaches for configuration management and exemplary configuration management tools.
	%Cognitive and Practical Skills
	\item use configuration specification languages,
	\item implement such specified variability during program construction,
	\item apply techniques of quality assurance in configurable applications.
	% Social and Personal Skills
	\item communicate variability with system administrators.
	\end{enumerate}
\end{frame}

\begin{frame}
	\tiny
	\frametitle{Learning Outcomes}
	Students will be able to

	\begin{itemize}
	% L01
	\item remember definitions of configuration settings.
	% TODO: meta-levels
	($\rightarrow$ TISS 2)

	% L02
	\item use configuration specification languages.
	(= TISS 3)

	% L03
	\item remember strategies for configuration integration.
	($\rightarrow$ TISS 1)

	% L04
	\item differentiate between configuration sources.
	($\rightarrow$ TISS 1)
	\item unify configuration sources via specifications.
	($\rightarrow$ TISS 1)

	% L05
	%\item {\color{gray} write simple configuration management scripts.}
	\item describe systematic approaches for configuration management and exemplary configuration management tools.
	($=$ TISS 2)

	% L06
	\item write simple checker plugins.
	($\rightarrow$ TISS 4)

	% L07
	\item remember terms of properties of CM.
	($\rightarrow$ TISS 2)
	\item remember various strategies for reduction of misconfiguration.
	%\item find unused settings.
	($\rightarrow$ TISS 2)
	($\rightarrow$ TISS 5)

	% L08
	\item recall points of time relevant in configuration management.
	%\item remind some arguments about pull vs.\ push.
	%\item remember various strategies for earlier reduction of misconfiguration.
	($\rightarrow$ TISS 5)

	% L09
	%\item recall a method of avoiding errors.
	\item apply some principles of good error messages.
	($\rightarrow$ TISS 5)
	\item remind some basics of system administrator research.
	($\rightarrow$ TISS 6)

	% L10
	\item design and document configuration settings and specifications.
	($\rightarrow$ TISS 1)
	\item evaluate a configuration system and decide about use of
	($\rightarrow$ TISS 5)
	\begin{itemize}
	\tiny
	\item code generation.
	\item introspection.
	\item context-awareness.
	\end{itemize}

	% L11
	\item remember connections between the many different topics within CM.
	\end{itemize}
\end{frame}

\begin{frame}
	\frametitle{Map}

	\vspace{-0.5cm}
	\includegraphics[width=8cm]{pics/map.pdf}
\end{frame}

\section{Preview}


\begin{frame}
	\frametitle{Feedback}
	\hfill \includegraphics[width=2cm]{pics/feedback.png}
	\vspace{-1cm}
	\begin{itemize}
		\item TUWEL Feedback %\linebreak
		%{\scriptsize \url{https://tuwel.tuwien.ac.at/mod/feedback/view.php?id=1259052}}
		\vspace{0.2cm}
		\item TISS Feedback \linebreak
		{\small from 16.06.2022 00:00 until 14.07.2022 23:59
		\scriptsize \url{https://tiss.tuwien.ac.at/survey/surveyForm.xhtml?courseNumber=194030&semesterCode=2022S}}
	\end{itemize}
\end{frame}

\begin{frame}
	\frametitle{Conclusion}

	Doing CM is easier if the applications support it.
	\vspace{1cm}

	I hope the lecture helped you to know how to write such applications.
\end{frame}




%%%%%%%%%%%%%%%%%%%%%%%%%%%%%%%%%%%%%%%%%% 
\nocite{raab2017introducing}

\appendix

\begin{frame}[allowframebreaks]
	\bibliographystyle{plainnat}
	\bibliography{../shared/elektra.bib}
\end{frame}

\end{document}

