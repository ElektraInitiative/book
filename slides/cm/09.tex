%\ifdefined\handout
%\documentclass[handout,aspectratio=1610,xcolor={usenames,dvipsnames,table}]{beamer}
%\else
\documentclass[aspectratio=1610,xcolor={usenames,dvipsnames,table}]{beamer}
%\fi

\newcommand{\mylecture}{Configuration Management}

%\ifdefined\handout
%\documentclass[handout,aspectratio=1610,xcolor={usenames,dvipsnames,table}]{beamer}
%\else
\documentclass[aspectratio=1610,xcolor={usenames,dvipsnames,table}]{beamer}
%\fi

\newcommand{\mylecture}{Configuration Management}

%\ifdefined\handout
%\documentclass[handout,aspectratio=1610,xcolor={usenames,dvipsnames,table}]{beamer}
%\else
\documentclass[aspectratio=1610,xcolor={usenames,dvipsnames,table}]{beamer}
%\fi

\newcommand{\mylecture}{Configuration Management}

\input{../setup}
\input{../shared/setup}

\lstdefinelanguage{dump}
{
	morekeywords={kdbOpen,ksNew,keyNew,keyMeta,keyCopyMeta,keyEnd,ksEnd,kdbClose},
	sensitive=false,
	morecomment=[l]{//},
	morecomment=[s]{/*}{*/},
	morestring=[b]",
}


\lstdefinelanguage{SpecElektra}{
	%
	comment=[l]{;},
	commentstyle=\color{purple}\ttfamily,
	%
	morestring=[b]',
	morestring=[b]`,
	morestring=[b]",
	stringstyle=\color{purple}\ttfamily,
	%
	sensitive=f,% keywords are not case sensitive
	%
	% Colors see https://en.wikibooks.org/wiki/LaTeX/Colors
	%
	keywordstyle=\color{BlueViolet}\bfseries,
	keywordstyle=[2]\color{Green},
	keywordstyle=[3]\color{Aquamarine}\bfseries\textit,
	keywordstyle=[4]\color{NavyBlue}\bfseries,
	keywordstyle=[5]\color{Mahogany},
	%
	keywords={layer, require, validation, check, range, description, rationale, requirement, accessibility, enable, condition, message, default, opt, readonly, type, context, property1, property2, description, file, content, mountpoint, metadata, infos, plugins},
	keywords=[2]{},
	keywords=[3]{order, interface, network, emphasized},
	%keywords=[4]{[, ]},  %Not needed
	keywords=[4]{},
	keywords=[5]{},
	%
	literate={:=}{{{\color{red}\textbf:=}}}2
		 {\%}{{{\color{NavyBlue}\textbf\%}}}1
		 {[}{{{\color{Sepia}\textbf[}}}1
		 {]}{{{\color{Sepia}\textbf]}}}1,
}

\lstdefinelanguage{Cpp}{%
	language     = C++,
	literate=
}


\lstdefinelanguage{CfgElektra}{
	comment=[l]{;},
	commentstyle=\color{purple}\ttfamily,
	%
	morestring=[b]',
	morestring=[b]`,
	morestring=[b]",
	stringstyle=\color{purple}\ttfamily,
	%
	%
	sensitive=f,% keywords are not case sensitive
	%
	% Colors see https://en.wikibooks.org/wiki/LaTeX/Colors
	%
	keywordstyle=\color{Bittersweet}\bfseries,
	keywordstyle=[2]\color{DarkOrchid}\bfseries,
	keywordstyle=[3]\color{ForestGreen}\bfseries\textit,
	keywordstyle=[4]\color{Goldenrod}\bfseries,
	keywordstyle=[5]\color{CarnationPink},
	%
	keywords={},
	keywords=[2]{},
	keywords=[3]{},
	keywords=[4]{},
	keywords=[5]{},
	%
	literate={=}{{{\color{ForestGreen}\textbf=}}}1
		 %{<-}{{{\color{ForestGreen}\textbf<-}}}2
		 %{*}{{{\color{Bittersweet}\textbf*}}}1
		 {\%}{{{\color{NavyBlue}\textbf\%}}}1,
}




\lstset{language=SpecElektra, % Use SpecElektra as default programming language
	%boxpos=t, % make boxes a bit more unbreakable
	%frame=lines, % top+bottom line
	basicstyle=\ttfamily, % Use normal-size true type font
	showspaces,%
	showstringspaces=false, % Don't put marks in string spaces
	showlines=true, % make sure empty lines at end are shown (needed for concurrency
	tabsize=4, % spaces per tab
	xleftmargin=\parindent, % should be 18pt or 1.5em as defined by memoir
	%Does not really work well (needs to be deactivated for shortlistings):
	breaklines=false,
	%postbreak=\mbox{\textcolor{red}{$\hookrightarrow$}\space},
	%breakautoindent=true,
	%prebreak={\mbox{\ensuremath{\curvearrowright}}} % Zeichen am Zeilenende (Umbruch)
	%breaklines=true,
	%breakautoindent=true,
	%prebreak=\small\symbol{'134}, % backslash
	%prebreak={\mbox{\ensuremath{\curvearrowright}}} % lange kure
	%prebreak={\mbox{\ensuremath{\hookleftarrow}}} % lange kure
	%xleftmargin=3.0ex, %for some formats
	%xrightmargin=1.0ex, %for some formats
	%
	% Files do not work in utf8 see also:
	% http://stackoverflow.com/questions/1116266/listings-in-latex-with-utf-8-or-at-least-german-umlauts
	% http://tex.stackexchange.com/questions/24528/having-problems-with-listings-and-utf-8-can-it-be-fixed
	% Should work but doesn't? (Maybe add to literate broken?)
	%add to literate={ö}{{\"o}}1
	%	{ä}{{\"a}}1
	%	{ü}{{\"u}}1
	%	{Ö}{{\"O}}1
	%	{Ä}{{\"A}}1
	%	{Ü}{{\"U}}1
	%	{ß}{{\ss}}1,
	%
	% listingsutf8 did not work, made umlauts in comments very strange
	%extendedchars=true,
	%inputencoding=utf8,
	%
	%morecomment=[l][\color{blue}]{...}, % Line continuation (...) e.g. blue comment
	morekeywords={for_each},
	numbers=left, % Line numbers on left
	firstnumber=1, % Line numbers start with line 1
	numberstyle=\small\color{blue}, % Line numbers are blue and small
	numbersep=5pt,
	%stepnumber=5 % Line numbers go in steps of 5
}



\lstMakeShortInline[postbreak=,keywordstyle={}]^

\graphicspath{{../pic/}{../figures/}{../graphics/}{../ipe/}{../ggplot/}}



\lstdefinelanguage{dump}
{
	morekeywords={kdbOpen,ksNew,keyNew,keyMeta,keyCopyMeta,keyEnd,ksEnd,kdbClose},
	sensitive=false,
	morecomment=[l]{//},
	morecomment=[s]{/*}{*/},
	morestring=[b]",
}


\lstdefinelanguage{SpecElektra}{
	%
	comment=[l]{;},
	commentstyle=\color{purple}\ttfamily,
	%
	morestring=[b]',
	morestring=[b]`,
	morestring=[b]",
	stringstyle=\color{purple}\ttfamily,
	%
	sensitive=f,% keywords are not case sensitive
	%
	% Colors see https://en.wikibooks.org/wiki/LaTeX/Colors
	%
	keywordstyle=\color{BlueViolet}\bfseries,
	keywordstyle=[2]\color{Green},
	keywordstyle=[3]\color{Aquamarine}\bfseries\textit,
	keywordstyle=[4]\color{NavyBlue}\bfseries,
	keywordstyle=[5]\color{Mahogany},
	%
	keywords={layer, require, validation, check, range, description, rationale, requirement, accessibility, enable, condition, message, default, opt, readonly, type, context, property1, property2, description, file, content, mountpoint, metadata, infos, plugins},
	keywords=[2]{},
	keywords=[3]{order, interface, network, emphasized},
	%keywords=[4]{[, ]},  %Not needed
	keywords=[4]{},
	keywords=[5]{},
	%
	literate={:=}{{{\color{red}\textbf:=}}}2
		 {\%}{{{\color{NavyBlue}\textbf\%}}}1
		 {[}{{{\color{Sepia}\textbf[}}}1
		 {]}{{{\color{Sepia}\textbf]}}}1,
}

\lstdefinelanguage{Cpp}{%
	language     = C++,
	literate=
}


\lstdefinelanguage{CfgElektra}{
	comment=[l]{;},
	commentstyle=\color{purple}\ttfamily,
	%
	morestring=[b]',
	morestring=[b]`,
	morestring=[b]",
	stringstyle=\color{purple}\ttfamily,
	%
	%
	sensitive=f,% keywords are not case sensitive
	%
	% Colors see https://en.wikibooks.org/wiki/LaTeX/Colors
	%
	keywordstyle=\color{Bittersweet}\bfseries,
	keywordstyle=[2]\color{DarkOrchid}\bfseries,
	keywordstyle=[3]\color{ForestGreen}\bfseries\textit,
	keywordstyle=[4]\color{Goldenrod}\bfseries,
	keywordstyle=[5]\color{CarnationPink},
	%
	keywords={},
	keywords=[2]{},
	keywords=[3]{},
	keywords=[4]{},
	keywords=[5]{},
	%
	literate={=}{{{\color{ForestGreen}\textbf=}}}1
		 %{<-}{{{\color{ForestGreen}\textbf<-}}}2
		 %{*}{{{\color{Bittersweet}\textbf*}}}1
		 {\%}{{{\color{NavyBlue}\textbf\%}}}1,
}




\lstset{language=SpecElektra, % Use SpecElektra as default programming language
	%boxpos=t, % make boxes a bit more unbreakable
	%frame=lines, % top+bottom line
	basicstyle=\ttfamily, % Use normal-size true type font
	showspaces,%
	showstringspaces=false, % Don't put marks in string spaces
	showlines=true, % make sure empty lines at end are shown (needed for concurrency
	tabsize=4, % spaces per tab
	xleftmargin=\parindent, % should be 18pt or 1.5em as defined by memoir
	%Does not really work well (needs to be deactivated for shortlistings):
	breaklines=false,
	%postbreak=\mbox{\textcolor{red}{$\hookrightarrow$}\space},
	%breakautoindent=true,
	%prebreak={\mbox{\ensuremath{\curvearrowright}}} % Zeichen am Zeilenende (Umbruch)
	%breaklines=true,
	%breakautoindent=true,
	%prebreak=\small\symbol{'134}, % backslash
	%prebreak={\mbox{\ensuremath{\curvearrowright}}} % lange kure
	%prebreak={\mbox{\ensuremath{\hookleftarrow}}} % lange kure
	%xleftmargin=3.0ex, %for some formats
	%xrightmargin=1.0ex, %for some formats
	%
	% Files do not work in utf8 see also:
	% http://stackoverflow.com/questions/1116266/listings-in-latex-with-utf-8-or-at-least-german-umlauts
	% http://tex.stackexchange.com/questions/24528/having-problems-with-listings-and-utf-8-can-it-be-fixed
	% Should work but doesn't? (Maybe add to literate broken?)
	%add to literate={ö}{{\"o}}1
	%	{ä}{{\"a}}1
	%	{ü}{{\"u}}1
	%	{Ö}{{\"O}}1
	%	{Ä}{{\"A}}1
	%	{Ü}{{\"U}}1
	%	{ß}{{\ss}}1,
	%
	% listingsutf8 did not work, made umlauts in comments very strange
	%extendedchars=true,
	%inputencoding=utf8,
	%
	%morecomment=[l][\color{blue}]{...}, % Line continuation (...) e.g. blue comment
	morekeywords={for_each},
	numbers=left, % Line numbers on left
	firstnumber=1, % Line numbers start with line 1
	numberstyle=\small\color{blue}, % Line numbers are blue and small
	numbersep=5pt,
	%stepnumber=5 % Line numbers go in steps of 5
}



\lstMakeShortInline[postbreak=,keywordstyle={}]^

\graphicspath{{../pic/}{../figures/}{../graphics/}{../ipe/}{../ggplot/}}




\title{L09 Configuration as a User Interface}

\begin{document}


%%%%%%%%%%%%%%%%%%%%%%%%%%%%%%%%%%%%%%%%%% 
\section{Three-way merge}


\begin{frame}
	\frametitle{Learning Outcomes}
	Students will be able to
	\begin{itemize}
	\item recall a method of avoiding errors.
	\item apply some principles of good error messages.
	\item remind some basics of system administrator research.
	\end{itemize}
\end{frame}


\begin{frame}
	\frametitle{Synchronization}

	Problem: transient and persistent configuration settings might be out-of-sync~\cite{jin2014configurations}

	\pause

	\ExecuteMetaData[../../book/motivation.tex]{req-consistency}

	\pause

	Solutions:
	\begin{itemize}
	\item Often write out configuration settings.
	\end{itemize}
\end{frame}

\begin{frame}
	\frametitle{Semantic Three-way merge}

	Problem: When trying to writing out configuration settings, the configuration settings might not be as they were before. (Conflict)

	\pause

	Solution: Many conflicts can be resolved automatically with a semantic three-way merge.

	\pause

	We can resolve many conflicts automatically if we consider:
	\begin{itemize}[<+-| alert@+>]
	\item the key/value structure (vs. line-based)
	\item the origin of the configuration settings
	\item the type of settings
	\end{itemize}

	\pause[\thebeamerpauses]

	For example, when upgrading slapd:
	\begin{itemize}
	\item System administrator changed the file (Ours).
	\item Package maintainer changed the file (Theirs).
	\end{itemize}
\end{frame}

\begin{frame}[fragile]
	\frametitle{Conflicts Example}

	\textbf{Ours:}
	\begin{code}[gobble=4,language=CfgElektra]
	slapd/threads/listener=4

	slapd/threads/enable= \
		yes # must be enabled for listener

	\end{code}

	\textbf{Theirs:}
	\begin{code}[gobble=4,language=CfgElektra]
	slapd/threads/enable = on
	slapd/threads/listener = 8
	\end{code}

	\pause
	\textbf{Origin:}
	\begin{code}[gobble=4,language=CfgElektra]
	slapd/threads/listener=8
	slapd/threads/enable = true
	\end{code}
\end{frame}






%%%%%%%%%%%%%%%%%%%%%%%%%%%%%%%%%%%%%%%%%% 
\section{Error Messages}

\subsection{}

\begin{frame}[fragile]
	\frametitle{Motivation (Recapitulation)}

	Error messages are extremely important as they are the main communication channel to system administrators.\\
	\vspace{0.5em}

	\begin{code}[morekeywords={assign,math},gobble=4]
	[a]
	  check/type:=long
	[b]
	  check/type:=long
	[c]
	  check/range:=0-10
	  assign/math:=../a+../b
	\end{code}

	\begin{task}
	Where should the error message point to if we change b to 10 \\ (a is unchanged 1)?
	\end{task}
\end{frame}

\begin{frame}
	\frametitle{Considerations (Recapitulation)}

	\begin{task}
	What needs to be considered when designing error messages?
	\end{task}

	\begin{itemize} %[<+-| alert@+>]
	\item Generic vs.\ specific plugins
	\item Precisely locate the cause (and do not report aftereffects)
	\item Give context
	\item Personification~\cite{lee2011personifying}
	\end{itemize}
\end{frame}

\begin{frame}
	\frametitle{Further Considerations}

	\begin{itemize}[<+-| alert@+>]
	\item configuration design first: avoid errors if possible
	\item ``edit here mentality'': do not point to correct statements~\cite{marceau2011mind}
	\item precision and recall\footnote{terms from classification, it is the numerical counterpart of soundness and completeness}~\cite{wrenn2017error}
	\item error messages should not leak internals~\cite{brown1983error}
	\item do not propose solutions~\cite{marceau2011mind} if you are not sure
	\item reduce vocabulary~\cite{marceau2011mind}
	\item tension between providing enough information and not overwhelming the user~\cite{wrenn2017error}
	\item colors might help~\cite{wrenn2017error}
	\end{itemize}
\end{frame}

\begin{frame}
	\frametitle{Error Messages for Misconfiguration~\cite{zhang2015proactive}}

	\begin{itemize}[<+-| alert@+>]
	\item error messages are sometimes the only interface
	\item tool uses misconfiguration injection and checks if error message point to the correct setting
	\item tool requires system tests
	\item they considered error message as okay if key \emph{or} value is present
	\end{itemize}

	\pause[\thebeamerpauses]  %  show after \begin{itemize}[<+->]

	\begin{alertblock}{Implication}
	Missing error message means the configuration specification is not complete.
	\end{alertblock}
\end{frame}

\begin{frame}
	\frametitle{Context for error messages}

	Error messages should contain:

	\begin{itemize}[<+-| alert@+>]
	\item pin-point key (which also pin-points to the specification)
	\item repeat relevant parts of values and the specification
	\item show mountpoint (to make relative keys unique)
	\item show file name and line number
	\item for reporting bugs: show source code lines
	\end{itemize}
\end{frame}

\begin{frame}[fragile]
	\frametitle{Precise Location (Recapitulation)}

	\begin{code}[language=CfgElektra,gobble=4]
	a=5  ; unmodified
	b=10 ; modification bit in metadata
	     ; is only set here
	c=15 ; unmodified by user but changed
	     ; later by assign/math
	\end{code}
\end{frame}

\begin{frame}[fragile]
	\frametitle{Example Error Messages (Recapitulation)}
\begin{verbatim}
Sorry, I was unable to change the configuration settings!
Description: I tried to set a value outside the range!
Reason: I tried to modify b to be 10 but this caused c to
        be outside of the allowed range (0-10).
Module: range
At: sourcefile.c:1234
Mountpoint: /test
Configfile: /etc/testfile.conf
\end{verbatim}
\end{frame}

\begin{frame}[fragile]
	\frametitle{Example Error Messages (Improvement)}
\begin{verbatim}
Sorry, module range issued error C03100:
I tried to modify b to be 10 but this caused c to
be outside of the allowed range (0-10).
\end{verbatim}
\end{frame}




%%%%%%%%%%%%%%%%%%%%%%%%%%%%%%%%%%%%%%%%%% 
\section{System Administrator Research}

\subsection{}

\begin{frame}
	\frametitle{User View}

	Who is the user of CM?

	\begin{itemize}[<+-| alert@+>]
	\item End Users?
	\item Developers (devs)?
	\item System Administrators (admins)?
	\end{itemize}
\end{frame}

\begin{frame}
	\frametitle{System Administrator Research}

	\begin{itemize}[<+-| alert@+>]
	\item Interest of understanding administrators emerged around 2002~\cite{anderson2002researching}.
	\item Typical methods are surveys, diary studies, interviews and observations (ethnographic field studies).
	\item Field studies also done in industry \cite{barrett2004field}.
	\item Barrett \cite{barrett2003system} tried to initiate a workshop at CHI 2003 to draw the attention of the HCI community towards system administration.
	\item The workshop was already dropped in the next year.
	\item The tenor is that ``tools ... are not well aligned''~\cite{haber2007design}.
	\item Research mainly looks at pre-CM. Manual administration is still standard \\ (Source: e.g., Luke Kanies).
	\end{itemize}
\end{frame}


\begin{frame}
	\frametitle{CM research}

	In the meanwhile at Large Installation System Administrator Conference (LISA):

	\begin{itemize}[<+-| alert@+>]
	\item began as CFengine Workshop at LISA 2001
	\item CM workshop by Paul Anderson~\cite{anderson2002researching}
	\item in LISA 2003 an informal poll asked about CM tools: \\
	 	the only user of each tool in the room at the time was its author~\cite{burgess2006modeling}
	\item it is easy to invent CM tools (and configuration file formats)
	\item it is difficult to make it useful beyond your own goals
	\end{itemize}
\end{frame}

\begin{frame}
	\frametitle{Tasks}

	What do system administrators do?

	\begin{itemize}[<+-| alert@+>]
	\item keep our infrastructure running
	\item coordinate
	\item do backups
	\item manage hardware
	\item do inventory
	\item install applications
	\item manage security
	\item configure applications
	\item troubleshoot
	\item $\implies$ the unsung heroes!
	\end{itemize}
\end{frame}


\begin{frame}
	\frametitle{7 people, 1 command-line~\cite{barrett2004field}}

	\begin{itemize}[<+-| alert@+>]
	\item system administrator misunderstood problem \\ (had a wrong assumption)
	\item 7 people sought attention and trust, competing to tell the admin what to do
	\item due to wrong assumption the admin communicated to everyone, people could not help
	\item there were several instances in which the admin ignored or misinterpreted evidence of the real problem
	\item eventually someone else solved the problem: admin confused ``from''/``to'' port in the settings and firewall blocked requests
	\end{itemize}
\end{frame}

\begin{frame}
	\frametitle{other cases~\cite{barrett2004field}}

	\begin{itemize}[<+-| alert@+>]
	\item lost semicolon: execution of script failed due to missing semicolon, then they tried to delete a non-existent table.
	\item crontab: onltape/ofltape confused because of discussion about offline backup (although an online backup should be performed).
	\item crit sit: many system administrators competed against each other trying to write a simple script. The crit sit continued for two weeks.
	\end{itemize}
\end{frame}


\begin{frame}
	\frametitle{\citet{haber2007design}}

	Later \citet{haber2007design} repeated an ethnographic field study.
	The stories are similar to \citet{barrett2004field}.
	Their study was also conducted in the same company.
	They created personas:

	\begin{itemize}
	\item database administrator
	\item web administrator
	\item security administrator
	\end{itemize}
\end{frame}



\begin{frame}
	\frametitle{Database Administrator~\cite{haber2007design}}

	\begin{itemize}[<+-| alert@+>]
	\item frequent contact via phone, e-mail and IM
	\item needs to work on weekends
	\item pair-programming for new tasks
	\item typical errors: stopping wrong database process
	\end{itemize}
\end{frame}

\begin{frame}
	\frametitle{Web Administrator~\cite{haber2007design}}

	\begin{itemize}[<+-| alert@+>]
	\item crit sit
	\item deploying new Web applications
	\item about 20-400 steps to deploy an application
	\item moving from test to production done by hand
	\end{itemize}
\end{frame}

\begin{frame}
	\frametitle{Security Administrator~\cite{haber2007design}}

	\begin{itemize}[<+-| alert@+>]
	\item gets emails on suspicious activities
	\item multi-user chat
	\item ad-hoc scripts
	\end{itemize}
\end{frame}


\begin{frame}
	\frametitle{\citet{haber2007design}}

	\begin{itemize}[<+-| alert@+>]
	\item ``if data is lost...that is when you write your résumé.''
	\item \p{90} is spent with communicating with other admins
	\item only \p{6} is gathering information and running commands
	\item quality control: monitoring found that non-functional service was down two days
	\end{itemize}
\end{frame}


\begin{frame}
	\frametitle{\citet{barrett2004field}}

	\begin{itemize}[<+-| alert@+>]
	\item \p{20} of the time is spent in diversions
	\item \p{20} of the time people communicated about \emph{how to communicate}
	\item CLIs were generally preferred
	\item configuration and log files are scattered, poorly organized and often used inconsistent terminology
	\end{itemize}
\end{frame}


\begin{frame}
	\frametitle{Findings \cite{barrett2004field}}

	\begin{itemize}[<+-| alert@+>]
	\item syntax checking is essential
	\item replicating actions (e.g., to production) is error-prone
	\item undo not available
	\item do not assume a complete mental model (``if understand the system is a prerequisite [...], we are lost'')
	\item do not assume programming skills (only \p{35} reported having a bachelor's degree)
	\item trust in CLI tools but little trust in GUIs (is the information up-to-date?)
	\item errors while executing scripts lead to inconsistent state, rerunning often does not work
	\item[] (not idempotent)
	\end{itemize}
\end{frame}

\begin{frame}
	\frametitle{Design Principles \cite{haber2007design}}

	Many design principles for tools were given~\cite{haber2007design}:

	\begin{itemize}[<+-| alert@+>]
	\item configuration and logs should be displayed in a uniform way
	\item APIs/plugins for tools should be provided
	\item errors in configuration need to be discovered quickly
	\item confusion of similar settings should be avoided
	\item provide means of comparing configuration settings
	\item provide consistent profiles of information
	\item both transient and persistent settings should be visible
	\item when errors occur: always display which changes have been made (modern approach is idempotence)
	\end{itemize}
\end{frame}

\begin{frame}
	\frametitle{Apply to CM}

	What can we learn from manual system administration?

	\setbeamersize{description width=1cm}
	\begin{description}[<+-| alert@+>]
	\item[$+$] intensive review process catches errors
	\item[$-$] collaboration ineffective
	\item[$-$] context/situational awareness is essential
	\item[$+$] precise editing of configuration files works well
	\item[$+$] self-written tools are very efficient
	\end{description}

	\pause[\thebeamerpauses]  %  show after \begin{itemize}[<+->]

	\begin{alertblock}{Idea}
	Replicate parts that work well, automate error-prone parts.
	\end{alertblock}
\end{frame}

%TODO: duplicate from 05
\begin{frame}
	\frametitle{Precise Editing}

	Partial modifications (precise editing) is natural for humans. \\
	It ensures preservations of (potentially security-relevant!) defaults. \\
	In CM following methods are used:

	\begin{itemize}[<+-| alert@+>]
	\item embed shell commands to do the work
	\item replace full content of configuration files
	\item replace full content of configuration files with templates
	\item line based manipulation (e.g., file\_line): match line and replace it
	\item Augeas/XML: match a key with XPath and replace it
	\item Elektra: set the value of a key
	\end{itemize}
\end{frame}

\begin{frame}
	\frametitle{Apply to CM}

	Elektra's goals are:

	\begin{itemize}[<+-| alert@+>]
	\item it should be easy to develop new high-level tools
	\item precise editing: change the configuration value as specified
	\end{itemize}

	\pause[\thebeamerpauses]  %  show after \begin{itemize}[<+->]

	Administrators/Devs still need to:

	\begin{itemize}[<+-| alert@+>]
	\item reduce the configuration complexity
	\item intensively review and improve the specifications
	\item test (and debug) configuration settings
	\end{itemize}

	\pause[\thebeamerpauses]  %  show after \begin{itemize}[<+->]

	Open topics (incomplete):

	\begin{itemize}[<+-| alert@+>]
	\item safe migrations of settings and data
	\item collaboration
	\item management (including knowledge)
	\end{itemize}
\end{frame}

\begin{frame}
	\frametitle{Conclusion}

	\begin{itemize}[<+-| alert@+>]
	\item Configuration management languages differ widely.
	\item Configuration specifications are helpful in different ways.
	\item Do not design around tools but design tools around you.
	\end{itemize}
\end{frame}

%%%%%%%%%%%%%%%%%%%%%%%%%%%%%%%%%%%%%%%%%% 
\section{Meeting}

%\subsection{Recapitulation}
%
%\begin{frame}
%	\frametitle{Learning Outcomes}
%	Students will be able to
%	\begin{itemize}
%	\item recall a method of avoiding errors
%	\item apply some principles of good error messages
%	\item remind some basics of system administrator research
%	\end{itemize}
%\end{frame}
%
%\begin{frame}[fragile]
%	\frametitle{Conflicts Example}
%	How can you avoid errors when merging these configuration settings?
%
%	\textbf{Ours:}
%	\begin{code}[gobble=4,language=CfgElektra]
%	slapd/threads/listener=4
%
%	slapd/threads/enable= \
%		yes # must be enabled for listener
%
%	\end{code}
%
%	\textbf{Theirs:}
%	\begin{code}[gobble=4,language=CfgElektra]
%	slapd/threads/enable = on
%	slapd/threads/listener = 8
%	\end{code}
%
%	\textbf{Origin:}
%	\begin{code}[gobble=4,language=CfgElektra]
%	slapd/threads/listener=8
%	slapd/threads/enable = true
%	\end{code}
%\end{frame}
%
%\begin{frame}
%	\frametitle{Context for error messages}
%
%	Error messages should contain:
%
%	\pause
%
%	\begin{itemize} %[<+-| alert@+>]
%	\item pin-point key (which also pin-points to the specification)
%	\item repeat relevant parts of values and the specification
%	\item show mountpoint (to make relative keys unique)
%	\item show file name and line number
%	\item for reporting bugs: show source code lines
%	\end{itemize}
%\end{frame}
%
%\begin{assignment}
%	\begin{task}
%	Break.
%	\end{task}
%\end{assignment}
%
%\begin{frame}
%	\frametitle{Tasks}
%
%	What do system administrators do?
%
%	\pause
%
%	\begin{itemize} %[<+-| alert@+>]
%	\item keep our infrastructure running
%	\item coordinate
%	\item do backups
%	\item manage hardware
%	\item do inventory
%	\item install applications
%	\item manage security
%	\item configure applications
%	\item troubleshoot
%	\item $\implies$ the unsung heroes!
%	\end{itemize}
%\end{frame}
%
%\begin{assignment}
%	\begin{task}
%	Tell a story about system administrators.
%	\end{task}
%\end{assignment}
%
%\subsection{Assignments}
%
%\begin{assignment}
%	Finalize all PRs:
%
%	\begin{itemize} %[<+-| alert@+>]
%	\item Rebase PRs (+1P).
%	\item Make CI happy.
%	\item Correction for H3 (for JNI: only with tests, no mounting).
%	\item Correction for T3.
%	\end{itemize}
%\end{assignment}
%
%\begin{frame}
%	\frametitle{Feedback}
%	\hfill \includegraphics[width=2cm]{pics/feedback.png}
%	\vspace{-1cm}
%	\begin{itemize}
%		\item ECTS breakdown realistic?
%		\item Feedback Talk
%		\item TISS Feedback from 16.06.2021 00:00 to 14.07.2021 23:59
%	\end{itemize}
%\end{frame}
%
%\subsection{Preview}
%
%\begin{frame}
%	\frametitle{Outlook}
%
%	How ``everything'' in CM connects:
%	\begin{itemize} %[<+-| alert@+>]
%	\item documentation
%	\item introspection
%	\item (code) generation
%	\item context awareness
%	\end{itemize}
%	Best topics at last.
%\end{frame}



%%%%%%%%%%%%%%%%%%%%%%%%%%%%%%%%%%%%%%%%%% 
\nocite{raab2017introducing}

\appendix

\begin{frame}[allowframebreaks]
	\bibliographystyle{plainnat}
	\bibliography{../shared/elektra.bib}
\end{frame}

\end{document}

