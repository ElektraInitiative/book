%\ifdefined\handout
%\documentclass[handout,aspectratio=1610,xcolor={usenames,dvipsnames,table}]{beamer}
%\else
\documentclass[aspectratio=1610,xcolor={usenames,dvipsnames,table}]{beamer}
%\fi

\newcommand{\mylecture}{Configuration Management}

%\ifdefined\handout
%\documentclass[handout,aspectratio=1610,xcolor={usenames,dvipsnames,table}]{beamer}
%\else
\documentclass[aspectratio=1610,xcolor={usenames,dvipsnames,table}]{beamer}
%\fi

\newcommand{\mylecture}{Configuration Management}

%\ifdefined\handout
%\documentclass[handout,aspectratio=1610,xcolor={usenames,dvipsnames,table}]{beamer}
%\else
\documentclass[aspectratio=1610,xcolor={usenames,dvipsnames,table}]{beamer}
%\fi

\newcommand{\mylecture}{Configuration Management}

\input{../setup}
\input{../shared/setup}

\lstdefinelanguage{dump}
{
	morekeywords={kdbOpen,ksNew,keyNew,keyMeta,keyCopyMeta,keyEnd,ksEnd,kdbClose},
	sensitive=false,
	morecomment=[l]{//},
	morecomment=[s]{/*}{*/},
	morestring=[b]",
}


\lstdefinelanguage{SpecElektra}{
	%
	comment=[l]{;},
	commentstyle=\color{purple}\ttfamily,
	%
	morestring=[b]',
	morestring=[b]`,
	morestring=[b]",
	stringstyle=\color{purple}\ttfamily,
	%
	sensitive=f,% keywords are not case sensitive
	%
	% Colors see https://en.wikibooks.org/wiki/LaTeX/Colors
	%
	keywordstyle=\color{BlueViolet}\bfseries,
	keywordstyle=[2]\color{Green},
	keywordstyle=[3]\color{Aquamarine}\bfseries\textit,
	keywordstyle=[4]\color{NavyBlue}\bfseries,
	keywordstyle=[5]\color{Mahogany},
	%
	keywords={layer, require, validation, check, range, description, rationale, requirement, accessibility, enable, condition, message, default, opt, readonly, type, context, property1, property2, description, file, content, mountpoint, metadata, infos, plugins},
	keywords=[2]{},
	keywords=[3]{order, interface, network, emphasized},
	%keywords=[4]{[, ]},  %Not needed
	keywords=[4]{},
	keywords=[5]{},
	%
	literate={:=}{{{\color{red}\textbf:=}}}2
		 {\%}{{{\color{NavyBlue}\textbf\%}}}1
		 {[}{{{\color{Sepia}\textbf[}}}1
		 {]}{{{\color{Sepia}\textbf]}}}1,
}

\lstdefinelanguage{Cpp}{%
	language     = C++,
	literate=
}


\lstdefinelanguage{CfgElektra}{
	comment=[l]{;},
	commentstyle=\color{purple}\ttfamily,
	%
	morestring=[b]',
	morestring=[b]`,
	morestring=[b]",
	stringstyle=\color{purple}\ttfamily,
	%
	%
	sensitive=f,% keywords are not case sensitive
	%
	% Colors see https://en.wikibooks.org/wiki/LaTeX/Colors
	%
	keywordstyle=\color{Bittersweet}\bfseries,
	keywordstyle=[2]\color{DarkOrchid}\bfseries,
	keywordstyle=[3]\color{ForestGreen}\bfseries\textit,
	keywordstyle=[4]\color{Goldenrod}\bfseries,
	keywordstyle=[5]\color{CarnationPink},
	%
	keywords={},
	keywords=[2]{},
	keywords=[3]{},
	keywords=[4]{},
	keywords=[5]{},
	%
	literate={=}{{{\color{ForestGreen}\textbf=}}}1
		 %{<-}{{{\color{ForestGreen}\textbf<-}}}2
		 %{*}{{{\color{Bittersweet}\textbf*}}}1
		 {\%}{{{\color{NavyBlue}\textbf\%}}}1,
}




\lstset{language=SpecElektra, % Use SpecElektra as default programming language
	%boxpos=t, % make boxes a bit more unbreakable
	%frame=lines, % top+bottom line
	basicstyle=\ttfamily, % Use normal-size true type font
	showspaces,%
	showstringspaces=false, % Don't put marks in string spaces
	showlines=true, % make sure empty lines at end are shown (needed for concurrency
	tabsize=4, % spaces per tab
	xleftmargin=\parindent, % should be 18pt or 1.5em as defined by memoir
	%Does not really work well (needs to be deactivated for shortlistings):
	breaklines=false,
	%postbreak=\mbox{\textcolor{red}{$\hookrightarrow$}\space},
	%breakautoindent=true,
	%prebreak={\mbox{\ensuremath{\curvearrowright}}} % Zeichen am Zeilenende (Umbruch)
	%breaklines=true,
	%breakautoindent=true,
	%prebreak=\small\symbol{'134}, % backslash
	%prebreak={\mbox{\ensuremath{\curvearrowright}}} % lange kure
	%prebreak={\mbox{\ensuremath{\hookleftarrow}}} % lange kure
	%xleftmargin=3.0ex, %for some formats
	%xrightmargin=1.0ex, %for some formats
	%
	% Files do not work in utf8 see also:
	% http://stackoverflow.com/questions/1116266/listings-in-latex-with-utf-8-or-at-least-german-umlauts
	% http://tex.stackexchange.com/questions/24528/having-problems-with-listings-and-utf-8-can-it-be-fixed
	% Should work but doesn't? (Maybe add to literate broken?)
	%add to literate={ö}{{\"o}}1
	%	{ä}{{\"a}}1
	%	{ü}{{\"u}}1
	%	{Ö}{{\"O}}1
	%	{Ä}{{\"A}}1
	%	{Ü}{{\"U}}1
	%	{ß}{{\ss}}1,
	%
	% listingsutf8 did not work, made umlauts in comments very strange
	%extendedchars=true,
	%inputencoding=utf8,
	%
	%morecomment=[l][\color{blue}]{...}, % Line continuation (...) e.g. blue comment
	morekeywords={for_each},
	numbers=left, % Line numbers on left
	firstnumber=1, % Line numbers start with line 1
	numberstyle=\small\color{blue}, % Line numbers are blue and small
	numbersep=5pt,
	%stepnumber=5 % Line numbers go in steps of 5
}



\lstMakeShortInline[postbreak=,keywordstyle={}]^

\graphicspath{{../pic/}{../figures/}{../graphics/}{../ipe/}{../ggplot/}}



\lstdefinelanguage{dump}
{
	morekeywords={kdbOpen,ksNew,keyNew,keyMeta,keyCopyMeta,keyEnd,ksEnd,kdbClose},
	sensitive=false,
	morecomment=[l]{//},
	morecomment=[s]{/*}{*/},
	morestring=[b]",
}


\lstdefinelanguage{SpecElektra}{
	%
	comment=[l]{;},
	commentstyle=\color{purple}\ttfamily,
	%
	morestring=[b]',
	morestring=[b]`,
	morestring=[b]",
	stringstyle=\color{purple}\ttfamily,
	%
	sensitive=f,% keywords are not case sensitive
	%
	% Colors see https://en.wikibooks.org/wiki/LaTeX/Colors
	%
	keywordstyle=\color{BlueViolet}\bfseries,
	keywordstyle=[2]\color{Green},
	keywordstyle=[3]\color{Aquamarine}\bfseries\textit,
	keywordstyle=[4]\color{NavyBlue}\bfseries,
	keywordstyle=[5]\color{Mahogany},
	%
	keywords={layer, require, validation, check, range, description, rationale, requirement, accessibility, enable, condition, message, default, opt, readonly, type, context, property1, property2, description, file, content, mountpoint, metadata, infos, plugins},
	keywords=[2]{},
	keywords=[3]{order, interface, network, emphasized},
	%keywords=[4]{[, ]},  %Not needed
	keywords=[4]{},
	keywords=[5]{},
	%
	literate={:=}{{{\color{red}\textbf:=}}}2
		 {\%}{{{\color{NavyBlue}\textbf\%}}}1
		 {[}{{{\color{Sepia}\textbf[}}}1
		 {]}{{{\color{Sepia}\textbf]}}}1,
}

\lstdefinelanguage{Cpp}{%
	language     = C++,
	literate=
}


\lstdefinelanguage{CfgElektra}{
	comment=[l]{;},
	commentstyle=\color{purple}\ttfamily,
	%
	morestring=[b]',
	morestring=[b]`,
	morestring=[b]",
	stringstyle=\color{purple}\ttfamily,
	%
	%
	sensitive=f,% keywords are not case sensitive
	%
	% Colors see https://en.wikibooks.org/wiki/LaTeX/Colors
	%
	keywordstyle=\color{Bittersweet}\bfseries,
	keywordstyle=[2]\color{DarkOrchid}\bfseries,
	keywordstyle=[3]\color{ForestGreen}\bfseries\textit,
	keywordstyle=[4]\color{Goldenrod}\bfseries,
	keywordstyle=[5]\color{CarnationPink},
	%
	keywords={},
	keywords=[2]{},
	keywords=[3]{},
	keywords=[4]{},
	keywords=[5]{},
	%
	literate={=}{{{\color{ForestGreen}\textbf=}}}1
		 %{<-}{{{\color{ForestGreen}\textbf<-}}}2
		 %{*}{{{\color{Bittersweet}\textbf*}}}1
		 {\%}{{{\color{NavyBlue}\textbf\%}}}1,
}




\lstset{language=SpecElektra, % Use SpecElektra as default programming language
	%boxpos=t, % make boxes a bit more unbreakable
	%frame=lines, % top+bottom line
	basicstyle=\ttfamily, % Use normal-size true type font
	showspaces,%
	showstringspaces=false, % Don't put marks in string spaces
	showlines=true, % make sure empty lines at end are shown (needed for concurrency
	tabsize=4, % spaces per tab
	xleftmargin=\parindent, % should be 18pt or 1.5em as defined by memoir
	%Does not really work well (needs to be deactivated for shortlistings):
	breaklines=false,
	%postbreak=\mbox{\textcolor{red}{$\hookrightarrow$}\space},
	%breakautoindent=true,
	%prebreak={\mbox{\ensuremath{\curvearrowright}}} % Zeichen am Zeilenende (Umbruch)
	%breaklines=true,
	%breakautoindent=true,
	%prebreak=\small\symbol{'134}, % backslash
	%prebreak={\mbox{\ensuremath{\curvearrowright}}} % lange kure
	%prebreak={\mbox{\ensuremath{\hookleftarrow}}} % lange kure
	%xleftmargin=3.0ex, %for some formats
	%xrightmargin=1.0ex, %for some formats
	%
	% Files do not work in utf8 see also:
	% http://stackoverflow.com/questions/1116266/listings-in-latex-with-utf-8-or-at-least-german-umlauts
	% http://tex.stackexchange.com/questions/24528/having-problems-with-listings-and-utf-8-can-it-be-fixed
	% Should work but doesn't? (Maybe add to literate broken?)
	%add to literate={ö}{{\"o}}1
	%	{ä}{{\"a}}1
	%	{ü}{{\"u}}1
	%	{Ö}{{\"O}}1
	%	{Ä}{{\"A}}1
	%	{Ü}{{\"U}}1
	%	{ß}{{\ss}}1,
	%
	% listingsutf8 did not work, made umlauts in comments very strange
	%extendedchars=true,
	%inputencoding=utf8,
	%
	%morecomment=[l][\color{blue}]{...}, % Line continuation (...) e.g. blue comment
	morekeywords={for_each},
	numbers=left, % Line numbers on left
	firstnumber=1, % Line numbers start with line 1
	numberstyle=\small\color{blue}, % Line numbers are blue and small
	numbersep=5pt,
	%stepnumber=5 % Line numbers go in steps of 5
}



\lstMakeShortInline[postbreak=,keywordstyle={}]^

\graphicspath{{../pic/}{../figures/}{../graphics/}{../ipe/}{../ggplot/}}




%TODO: clarify links

\title{L03 Configuration Integration}
\date{24.03.2021}

\begin{document}

%%%%%%%%%%%%%%%%%%%%%%%%%%%%%%%%%%%%%%%%%% 
\section{Configuration Libraries}


\begin{frame}<1>[label=learning outcomes]
	\frametitle{Learning Outcomes}
	Students will be able to
	\begin{itemize}
	\item remember strategies for configuration integration.
	\end{itemize}
\end{frame}

\begin{frame}
	\frametitle{Configuration Access APIs}

	\Large
	\ExecuteMetaData[../../book/background.tex]{definition-api}

	\ExecuteMetaData[../../book/background.tex]{definition-configuration-access}
\end{frame}

\begin{frame}[fragile]
	\frametitle{Configuration Access APIs}

	\begin{itemize}[<+-| alert@+>]
	\item ^char * getenv (const char * key)^
	\item ^ConfigStatus xf86HandleConfigFile(Bool autoconf)^
	\item ^long pathconf (const char *path, int^ ^name)^
	\item ^long sysconf (int name)^
	\item ^size_t confstr (int name, char *buf, size_t len)^
	\end{itemize}
\end{frame}

\begin{frame}[fragile]
	\frametitle{Configuration Access Points}
	\ExecuteMetaData[../../book/background.tex]{definition-configuration-access-points}

	\begin{code}[language=Cpp,gobble=4,showspaces=no]
	int main()
	{
		getenv ("PATH");
	}
	\end{code}
\end{frame}

\begin{frame}[fragile]
	\frametitle{Configuration Libraries}
	\ExecuteMetaData[../../book/background.tex]{definition-configuration-library}

	Trends:
	\begin{itemize}[<+-| alert@+>]
	\item flexibility to configure configuration access (e.g., \url{https://commons.apache.org/proper/commons-configuration/})
	\item more type safety (e.g., \url{http://owner.aeonbits.org/}, code generation)
	\item specifications and introspection (gsettings, XML/JSON, Elektra)
	\item configuration integration (UCI, Augeas, Elektra)
	\end{itemize}
\end{frame}

\begin{frame}
	\frametitle{Current Situation}
	\includegraphics[scale=0.7]{cursituation}
\end{frame}

\begin{frame}
	\frametitle{Wanted Situation}
	\includegraphics[scale=0.7]{wantsituation}
\end{frame}


%%%%%%%%%%%%%%%%%%%%%%%%%%%%%%%%%%%%%%%%%% 
\section{Lightweight to Strong Integration}

\begin{frame}[fragile]
	\frametitle{Lightweight Integration}

	Specify already-existing configuration files:
	\begin{code}[language=Cpp,gobble=4,showspaces=no]
	[ntp]
	  mountpoint:=ntp.conf
	  infos/plugins:=ntp
	\end{code}

	\vspace{1cm}
	Works well for configuration management tools.
\end{frame}

\begin{frame}[fragile]
	\frametitle{Medium Integration}

	Having frontends that implement existing \textbf{APIs} decouple applications from each other.
	These applications continue to use their specific configuration accesses, but \elektra{} redirects their configuration accesses to the shared key database.

	Possible APIs:
	\begin{itemize}[<+-| alert@+>]
	\item ^getenv^
	\item open/close of configuration files
	\end{itemize}

	\pause[\thebeamerpauses]
	\vspace{1cm}

	Also needs application-specific specifications.
\end{frame}

\begin{frame}[fragile]
	\frametitle{Strong Integration}

	Change the application so that it directly uses Elektra.

	Advantages:
	\begin{itemize}[<+-| alert@+>]
	\item Elektra's features always available
	\item more type safety
	\item administrators can choose configuration file formats
	\item notification and logging
	\item only one parser involved
	\item no specification for binding needed
	\item no built-in defaults: everything is introspectable
	\end{itemize}
\end{frame}

\begin{frame}[fragile]
	\frametitle{Strong Integration}

	Different implementations strategies:

	\begin{itemize}[<+-| alert@+>]
	\item have some application-specific API which uses ^KeySet^
	\item use one of KeySet's language bindings
	\item use Elektra's high-level API (currently only C)
	\item use code generation
	\end{itemize}
\end{frame}

\begin{frame}[fragile]
	\frametitle{Strong Integration}
	Examples:
	\begin{itemize}
	\item LCDproc
	\item Oyranos
	\item for GNOME: ^gsettings^
	\item for KDE: ^kconfig^
	\end{itemize}
\end{frame}



\section{Sharing Configuration}

\begin{frame}
	\begin{itemize}
	\item idea: make default values better
	\item generalization of sharing configuration values
	\item examples: language settings, default printer, \dots
	\end{itemize}
	\pause
	\vspace{1cm}
	Can be derived from:
	\begin{itemize}
	\item other configuration settings (override/fallback)
	\item context~\cite{raab2017introducing}
	\item hardware/system (problem with dependences)
	\pause
	\vspace{1cm}
	\item XServer vs.\ gpsd
	\end{itemize}
\end{frame}

\begin{frame}[fragile]
	\frametitle{Examples}
	Context:
	\begin{code}[gobble=4]
	[slapd/threads/listener]
	context:=/slapd/threads/%cpu%/listener
	\end{code}

	\pause
	\vspace{1cm}
	Calculation with conditionals plugin
	\\ (e.g., switch off GPS if battery low):
	\begin{code}[gobble=4]
	[gps/status]
	assign:=(battery > 'low') ? ('on') : ('off')
	\end{code}
\end{frame}

%%%%%%%%%%%%%%%%%%%%%%%%%%%%%%%%%%%%%%%%%% 
\section{Meeting}

\subsection{Recapitulation}

\againframe<99>{learning outcomes}

\begin{frame}
	\begin{task}
	Which configuration access APIs do you know? \\
	What are the differences between these APIs?
	\end{task}
\end{frame}

\begin{frame}
	\begin{alertblock}{Question}
	Which forms of configuration integration exist?
	\end{alertblock}

	\begin{exampleblock}{Answer}
	\begin{itemize}
	\item Lightweight Integration: Mount \only<2->{existing configuration files for CM tools}
	\item Medium Integration: Specify \only<3->{how to access via existing APIs}
	\item Strong Integration: Modify \only<4->{the application to use a configuration library that has support for configuration specifications.}
	\end{itemize}
	\end{exampleblock}
\end{frame}

\begin{frame}
	\begin{alertblock}{Question}
	Given a strong integration,
	how can we reuse the same configuration setting for different applications?
	\end{alertblock}

	\pause
	\begin{exampleblock}{Answer}
	\begin{itemize}
	\item Implement support directly in application to fetch setting from central location.
	\item Override/fallback links in specification.
	\item Calculate/transform values in specification.
	\item Use CM code to copy the settings to all places as needed.
	\end{itemize}
	\end{exampleblock}
\end{frame}

\subsection{Assignments}

\begin{frame}
	\frametitle{H1}

	\begin{task}
	What do you want to do as homework?
	\end{task}
\end{frame}

\begin{frame}
	\frametitle{P0}

	\begin{task}
	Do you already have a suitable project?
	\end{task}
\end{frame}

\begin{frame}
	\frametitle{T0 Issues}

	\begin{task}
	Did you already find enough suitable issues?
	\end{task}
\end{frame}

\begin{frame}
	\frametitle{Develop with Elektra}

	\begin{task}
	Can you already compile Elektra and software using Elektra?
	\end{task}
\end{frame}

\subsection{Preview}

\begin{frame}
	\frametitle{L04: Sources of Configuration}

	\begin{itemize}
	\item Configuration file formats
	\item Environment variables
	\item Command-line arguments
	\item Abstractions
	\item Complexity
	\end{itemize}
\end{frame}

\appendix

\begin{frame}[allowframebreaks]
	\bibliographystyle{plainnat}
	\bibliography{../shared/elektra.bib}
\end{frame}


\end{document}
