%\ifdefined\handout
%\documentclass[handout,aspectratio=1610,xcolor={usenames,dvipsnames,table}]{beamer}
%\else
\documentclass[aspectratio=1610,xcolor={usenames,dvipsnames,table}]{beamer}
%\fi

\newcommand{\mylecture}{Configuration Management}

%\ifdefined\handout
%\documentclass[handout,aspectratio=1610,xcolor={usenames,dvipsnames,table}]{beamer}
%\else
\documentclass[aspectratio=1610,xcolor={usenames,dvipsnames,table}]{beamer}
%\fi

\newcommand{\mylecture}{Configuration Management}

%\ifdefined\handout
%\documentclass[handout,aspectratio=1610,xcolor={usenames,dvipsnames,table}]{beamer}
%\else
\documentclass[aspectratio=1610,xcolor={usenames,dvipsnames,table}]{beamer}
%\fi

\newcommand{\mylecture}{Configuration Management}

\input{../setup}
\input{../shared/setup}

\lstdefinelanguage{dump}
{
	morekeywords={kdbOpen,ksNew,keyNew,keyMeta,keyCopyMeta,keyEnd,ksEnd,kdbClose},
	sensitive=false,
	morecomment=[l]{//},
	morecomment=[s]{/*}{*/},
	morestring=[b]",
}


\lstdefinelanguage{SpecElektra}{
	%
	comment=[l]{;},
	commentstyle=\color{purple}\ttfamily,
	%
	morestring=[b]',
	morestring=[b]`,
	morestring=[b]",
	stringstyle=\color{purple}\ttfamily,
	%
	sensitive=f,% keywords are not case sensitive
	%
	% Colors see https://en.wikibooks.org/wiki/LaTeX/Colors
	%
	keywordstyle=\color{BlueViolet}\bfseries,
	keywordstyle=[2]\color{Green},
	keywordstyle=[3]\color{Aquamarine}\bfseries\textit,
	keywordstyle=[4]\color{NavyBlue}\bfseries,
	keywordstyle=[5]\color{Mahogany},
	%
	keywords={layer, require, validation, check, range, description, rationale, requirement, accessibility, enable, condition, message, default, opt, readonly, type, context, property1, property2, description, file, content, mountpoint, metadata, infos, plugins},
	keywords=[2]{},
	keywords=[3]{order, interface, network, emphasized},
	%keywords=[4]{[, ]},  %Not needed
	keywords=[4]{},
	keywords=[5]{},
	%
	literate={:=}{{{\color{red}\textbf:=}}}2
		 {\%}{{{\color{NavyBlue}\textbf\%}}}1
		 {[}{{{\color{Sepia}\textbf[}}}1
		 {]}{{{\color{Sepia}\textbf]}}}1,
}

\lstdefinelanguage{Cpp}{%
	language     = C++,
	literate=
}


\lstdefinelanguage{CfgElektra}{
	comment=[l]{;},
	commentstyle=\color{purple}\ttfamily,
	%
	morestring=[b]',
	morestring=[b]`,
	morestring=[b]",
	stringstyle=\color{purple}\ttfamily,
	%
	%
	sensitive=f,% keywords are not case sensitive
	%
	% Colors see https://en.wikibooks.org/wiki/LaTeX/Colors
	%
	keywordstyle=\color{Bittersweet}\bfseries,
	keywordstyle=[2]\color{DarkOrchid}\bfseries,
	keywordstyle=[3]\color{ForestGreen}\bfseries\textit,
	keywordstyle=[4]\color{Goldenrod}\bfseries,
	keywordstyle=[5]\color{CarnationPink},
	%
	keywords={},
	keywords=[2]{},
	keywords=[3]{},
	keywords=[4]{},
	keywords=[5]{},
	%
	literate={=}{{{\color{ForestGreen}\textbf=}}}1
		 %{<-}{{{\color{ForestGreen}\textbf<-}}}2
		 %{*}{{{\color{Bittersweet}\textbf*}}}1
		 {\%}{{{\color{NavyBlue}\textbf\%}}}1,
}




\lstset{language=SpecElektra, % Use SpecElektra as default programming language
	%boxpos=t, % make boxes a bit more unbreakable
	%frame=lines, % top+bottom line
	basicstyle=\ttfamily, % Use normal-size true type font
	showspaces,%
	showstringspaces=false, % Don't put marks in string spaces
	showlines=true, % make sure empty lines at end are shown (needed for concurrency
	tabsize=4, % spaces per tab
	xleftmargin=\parindent, % should be 18pt or 1.5em as defined by memoir
	%Does not really work well (needs to be deactivated for shortlistings):
	breaklines=false,
	%postbreak=\mbox{\textcolor{red}{$\hookrightarrow$}\space},
	%breakautoindent=true,
	%prebreak={\mbox{\ensuremath{\curvearrowright}}} % Zeichen am Zeilenende (Umbruch)
	%breaklines=true,
	%breakautoindent=true,
	%prebreak=\small\symbol{'134}, % backslash
	%prebreak={\mbox{\ensuremath{\curvearrowright}}} % lange kure
	%prebreak={\mbox{\ensuremath{\hookleftarrow}}} % lange kure
	%xleftmargin=3.0ex, %for some formats
	%xrightmargin=1.0ex, %for some formats
	%
	% Files do not work in utf8 see also:
	% http://stackoverflow.com/questions/1116266/listings-in-latex-with-utf-8-or-at-least-german-umlauts
	% http://tex.stackexchange.com/questions/24528/having-problems-with-listings-and-utf-8-can-it-be-fixed
	% Should work but doesn't? (Maybe add to literate broken?)
	%add to literate={ö}{{\"o}}1
	%	{ä}{{\"a}}1
	%	{ü}{{\"u}}1
	%	{Ö}{{\"O}}1
	%	{Ä}{{\"A}}1
	%	{Ü}{{\"U}}1
	%	{ß}{{\ss}}1,
	%
	% listingsutf8 did not work, made umlauts in comments very strange
	%extendedchars=true,
	%inputencoding=utf8,
	%
	%morecomment=[l][\color{blue}]{...}, % Line continuation (...) e.g. blue comment
	morekeywords={for_each},
	numbers=left, % Line numbers on left
	firstnumber=1, % Line numbers start with line 1
	numberstyle=\small\color{blue}, % Line numbers are blue and small
	numbersep=5pt,
	%stepnumber=5 % Line numbers go in steps of 5
}



\lstMakeShortInline[postbreak=,keywordstyle={}]^

\graphicspath{{../pic/}{../figures/}{../graphics/}{../ipe/}{../ggplot/}}



\lstdefinelanguage{dump}
{
	morekeywords={kdbOpen,ksNew,keyNew,keyMeta,keyCopyMeta,keyEnd,ksEnd,kdbClose},
	sensitive=false,
	morecomment=[l]{//},
	morecomment=[s]{/*}{*/},
	morestring=[b]",
}


\lstdefinelanguage{SpecElektra}{
	%
	comment=[l]{;},
	commentstyle=\color{purple}\ttfamily,
	%
	morestring=[b]',
	morestring=[b]`,
	morestring=[b]",
	stringstyle=\color{purple}\ttfamily,
	%
	sensitive=f,% keywords are not case sensitive
	%
	% Colors see https://en.wikibooks.org/wiki/LaTeX/Colors
	%
	keywordstyle=\color{BlueViolet}\bfseries,
	keywordstyle=[2]\color{Green},
	keywordstyle=[3]\color{Aquamarine}\bfseries\textit,
	keywordstyle=[4]\color{NavyBlue}\bfseries,
	keywordstyle=[5]\color{Mahogany},
	%
	keywords={layer, require, validation, check, range, description, rationale, requirement, accessibility, enable, condition, message, default, opt, readonly, type, context, property1, property2, description, file, content, mountpoint, metadata, infos, plugins},
	keywords=[2]{},
	keywords=[3]{order, interface, network, emphasized},
	%keywords=[4]{[, ]},  %Not needed
	keywords=[4]{},
	keywords=[5]{},
	%
	literate={:=}{{{\color{red}\textbf:=}}}2
		 {\%}{{{\color{NavyBlue}\textbf\%}}}1
		 {[}{{{\color{Sepia}\textbf[}}}1
		 {]}{{{\color{Sepia}\textbf]}}}1,
}

\lstdefinelanguage{Cpp}{%
	language     = C++,
	literate=
}


\lstdefinelanguage{CfgElektra}{
	comment=[l]{;},
	commentstyle=\color{purple}\ttfamily,
	%
	morestring=[b]',
	morestring=[b]`,
	morestring=[b]",
	stringstyle=\color{purple}\ttfamily,
	%
	%
	sensitive=f,% keywords are not case sensitive
	%
	% Colors see https://en.wikibooks.org/wiki/LaTeX/Colors
	%
	keywordstyle=\color{Bittersweet}\bfseries,
	keywordstyle=[2]\color{DarkOrchid}\bfseries,
	keywordstyle=[3]\color{ForestGreen}\bfseries\textit,
	keywordstyle=[4]\color{Goldenrod}\bfseries,
	keywordstyle=[5]\color{CarnationPink},
	%
	keywords={},
	keywords=[2]{},
	keywords=[3]{},
	keywords=[4]{},
	keywords=[5]{},
	%
	literate={=}{{{\color{ForestGreen}\textbf=}}}1
		 %{<-}{{{\color{ForestGreen}\textbf<-}}}2
		 %{*}{{{\color{Bittersweet}\textbf*}}}1
		 {\%}{{{\color{NavyBlue}\textbf\%}}}1,
}




\lstset{language=SpecElektra, % Use SpecElektra as default programming language
	%boxpos=t, % make boxes a bit more unbreakable
	%frame=lines, % top+bottom line
	basicstyle=\ttfamily, % Use normal-size true type font
	showspaces,%
	showstringspaces=false, % Don't put marks in string spaces
	showlines=true, % make sure empty lines at end are shown (needed for concurrency
	tabsize=4, % spaces per tab
	xleftmargin=\parindent, % should be 18pt or 1.5em as defined by memoir
	%Does not really work well (needs to be deactivated for shortlistings):
	breaklines=false,
	%postbreak=\mbox{\textcolor{red}{$\hookrightarrow$}\space},
	%breakautoindent=true,
	%prebreak={\mbox{\ensuremath{\curvearrowright}}} % Zeichen am Zeilenende (Umbruch)
	%breaklines=true,
	%breakautoindent=true,
	%prebreak=\small\symbol{'134}, % backslash
	%prebreak={\mbox{\ensuremath{\curvearrowright}}} % lange kure
	%prebreak={\mbox{\ensuremath{\hookleftarrow}}} % lange kure
	%xleftmargin=3.0ex, %for some formats
	%xrightmargin=1.0ex, %for some formats
	%
	% Files do not work in utf8 see also:
	% http://stackoverflow.com/questions/1116266/listings-in-latex-with-utf-8-or-at-least-german-umlauts
	% http://tex.stackexchange.com/questions/24528/having-problems-with-listings-and-utf-8-can-it-be-fixed
	% Should work but doesn't? (Maybe add to literate broken?)
	%add to literate={ö}{{\"o}}1
	%	{ä}{{\"a}}1
	%	{ü}{{\"u}}1
	%	{Ö}{{\"O}}1
	%	{Ä}{{\"A}}1
	%	{Ü}{{\"U}}1
	%	{ß}{{\ss}}1,
	%
	% listingsutf8 did not work, made umlauts in comments very strange
	%extendedchars=true,
	%inputencoding=utf8,
	%
	%morecomment=[l][\color{blue}]{...}, % Line continuation (...) e.g. blue comment
	morekeywords={for_each},
	numbers=left, % Line numbers on left
	firstnumber=1, % Line numbers start with line 1
	numberstyle=\small\color{blue}, % Line numbers are blue and small
	numbersep=5pt,
	%stepnumber=5 % Line numbers go in steps of 5
}



\lstMakeShortInline[postbreak=,keywordstyle={}]^

\graphicspath{{../pic/}{../figures/}{../graphics/}{../ipe/}{../ggplot/}}




\title{L02 Configuration Specification Languages}
%\date{17.03.2021}

\begin{document}

%%%%%%%%%%%%%%%%%%%%%%%%%%%%%%%%%%%%%%%%%% 
\section{Theory}

\begin{frame}
	\frametitle{Rationale}
	\begin{itemize}
	\item without specification you and others do not even know which settings are available
	\item needed for any further techniques we will discuss
	\pause
	\item essential for \intro[no-futz computing]{no-futz computing}~\citet{holland2001nofutz}
	\item the foundation for any advanced tooling like configuration management tools
	\pause
	\item needed as communication of producers and consumers of configuration
	\end{itemize}
\end{frame}

\begin{frame}
	\methodQuestion{}
	\question{Configuration specification (e.g. XSD/JSON schemas) allows you to describe possible values and their meaning.  Why do/would you specify configuration?}
	\ExecuteMetaData[../../book/motivation.tex]{question-introduce-spec}
\end{frame}

\begin{frame}
	\frametitle{Limitations of Schemata designed for Data}
	\begin{itemize}
	\item e.g.\ XSD/JSON schemas
	\item they are already very helpful but:
	\pause
	\begin{itemize}
	\item not key-value based
	\item not easy to introspect
	\item designed to validate data without semantics: \\ file path vs.\ presence of file
	\item not always possible to extend with plugins
	\item tied to specific formats (e.g. XML/JSON)
	\end{itemize}
	\end{itemize}
\end{frame}

\begin{frame}
	\frametitle{Types of Specifications}
	\includegraphics[scale=0.8]{specifications}
\end{frame}


\begin{frame}
	\frametitle{Requirements}

	\begin{itemize}
	\item formal/informal?
	\item complete?
	\pause
	\item should be extensible
	\item should be external to application
	\item open for introspection
	\item should talk to users
	\item should allow generation of artefacts
	\end{itemize}
\end{frame}


\begin{frame}[fragile]
	\frametitle{Grammar}
	\begin{grammar}
	<configuration specifications> ::= \{ <configuration specification> \}

	<configuration specification> ::= '[' <key> ']' <properties>

	<properties> ::= \{ <property> \}

	<property> ::= <property name> ':=' [ <property value> ]
	\end{grammar}

	\vspace{1cm}
	Example:
	\begin{code}[gobble=4]
	[slapd/threads/listener]
	default:=1
	type:=long
	\end{code}
\end{frame}


\section{Practice}

\begin{frame}
	\frametitle{Learning Outcomes}
	Students will be able to
	\begin{itemize}
	\item use configuration specification languages.
	\end{itemize}
\end{frame}

\begin{frame}
	\frametitle{Metalevels (Recapitulation)}
	\includegraphics{metalevels}

	We will now walk through metalevels bottom-up.
\end{frame}

\begin{frame}[fragile]
	\frametitle{Configuration Settings (Recapitulation)}

	A configuration file may look like:

	\begin{code}[language=CfgElektra]
	a=5
	b=10
	c=15
	\end{code}

	We apply these configuration settings imperatively using:

	\begin{code}[language=bash]
	kdb set /a 5
	kdb set /b 10
	kdb set /c 15
	\end{code}

	And we list them with \lstinline[language=bash,morekeywords={ls},showspaces=no]^kdb ls /^.
\end{frame}

\begin{frame}[fragile]
	\frametitle{Specifications (Recapitulation)}
	For specifications such as:

	\begin{code}
	[slapd/threads/listener]
	  type:=short
	  default:=1
	\end{code}

	We apply the specifications imperatively using:

	\begin{code}[language=bash,morekeywords={meta,set,default}]
	kdb meta-set /slapd/threads/listener\
		type short
	kdb meta-set /slapd/threads/listener\
		default 1
	\end{code}

	(automatically uses ^spec:^ namespace)
\end{frame}

\begin{frame}[fragile]
	\frametitle{Meta-Specifications (Recapitulation)}
	For meta-specifications such as:

	\small
	\begin{code}[gobble=4]
	[type]
	type:=enum short unsigned_short long \
		float double char boolean any string ...
	description:=Defines the type of the value, \
		 as specified in CORBA
	\end{code}

	We apply the meta-specifications imperatively using:

	\begin{code}[language=bash,morekeywords={meta,set},gobble=4]
	kdb meta-set system:/info/elektra/metadata/type/#0 \
		type "enum short ..."
	kdb meta-set system:/info/elektra/metadata/type/#0 \
		description "Defines ..."
	\end{code}

	\large
	see ^doc/METADATA.ini^
\end{frame}

\begin{frame}
	\frametitle{SpecElektra}

	\begin{itemize}
	\item we use it to demonstrate configuration specification languages
	\item a modular \intro{specification language} for configuration settings
	\item we use properties to specify configuration settings and configuration access
	\item \elektra{Spec} specifies the behavior of \elektra{}
	\end{itemize}
\end{frame}


\begin{frame}[fragile]
	\frametitle{Mountpoint}

	The root of each configuration specification, e.g. in ni syntax:

	\begin{code}[morekeywords={mountpoint,infos,plugins},gobble=4]]
	[]
	mountpoint = vlc.ini
	infos/plugins = ni
	\end{code}
\end{frame}

\begin{frame}[fragile]
	\frametitle{Hierarchy}

	Always prefer hierarchy separator (^/^) as only separator:

	\begin{code}[gobble=4]
	[server/ip]
	\end{code}

	\vspace{1cm}

	Avoid other separators:

	\begin{code}[gobble=4]
	[server_ip]
	[server-ip]
	[server.ip]
	\end{code}

	Because they limit extensibility
	as they do not create sections in configuration files.
\end{frame}

\begin{frame}[fragile]
	\frametitle{Types}

	Presence alone indicates availability of a configuration setting:

	\begin{code}[gobble=4]
	[server/port]
	\end{code}

	Equivalent to ^type:=any^.

	\vspace{1cm}

	Properties give restrictions:

	\begin{code}[gobble=4]
	[server/port]
	type:=short
	\end{code}
\end{frame}

\begin{frame}[fragile]
	\frametitle{Require vs. Default}
	Prefer default values:
	\begin{code}[gobble=4]
	[server/ip]
	default:=127.0.0.1
	\end{code}

	Note that defaults must be sane and secure.

	\pause
	\vspace{1cm}

	Avoid require:

	\begin{code}[gobble=4]
	[server/ip]
	require:=
	\end{code}

	Because this forces the user to take action.
	\pause

	\begin{warn}
	\texttt{require and default} do not make sense together.
	\end{warn}
\end{frame}

\begin{frame}[fragile]
	\frametitle{IP Addresses}
	\begin{code}[morekeywords={ipaddr,example},gobble=4]
	[server/ip]
	check/ipaddr:=ipv4
	example:=0.0.0.0
	default:=127.0.0.1
	\end{code}

	Two plugins provide ^check/ipaddr^: ipaddr and network

	Will be automatically selected.
\end{frame}

\begin{frame}[fragile]
	\frametitle{Arrays}
	\begin{code}[gobble=4]
	[servers]
	array:=

	[servers/#/ip]
	check/ipaddr:=ipv4

	[servers/#/port]
	type:=short
	\end{code}
\end{frame}

\begin{frame}[fragile]
	\frametitle{Command-line Options}

	Environment and command-line options can be considered with:

	\begin{code}[morekeywords={long,env},gobble=4]]
	[recursive]
	  type:=boolean
	  opt:=r
	  opt/long:=recursive
	  env:=RECURSIVE
	  default:=0
	\end{code}
\end{frame}

\begin{frame}[fragile]
	\frametitle{Dates}
	\small
	\begin{code}[morekeywords={type,date,format,example},gobble=4]
	[mydate]
	example:=2021-03-01
	type:=string
	check/date:=ISO8601
	check/date/format:=calendardate complete extended
	\end{code}
\end{frame}

\begin{frame}[fragile]
	\frametitle{Design Considerations}
	Percentages
	\\ (e.g., configured image should be additionally cropped):
	\begin{code}[gobble=4]
	[image/width]
	type:=long

	[crop/width]
	type:=long
	check/range:=0-100
	\end{code}
\end{frame}

\begin{frame}
	\frametitle{Artefacts}
	\begin{itemize}
	\item plugins in configuration framework (e.g. validate settings)
	\item tooling (GUI, Web UI)
	\item generate examples/documentation
	\item auto-completion/syntax highlighting/IDE support
	\end{itemize}
\end{frame}




\section{Meeting}
%\subsection{Recapitulation}
%
%\begin{frame}
%	\frametitle{Metalevels (Recapitulation)}
%	\begin{alertblock}{Question}
%	Describe the three Metalevels in Elektra.
%	\end{alertblock}
%
%	\pause
%	\includegraphics{metalevels}
%\end{frame}
%
%\begin{assignment}
%	\begin{task}
%	Is meta-data separated from or included in the data structure KeySet?
%	\end{task}
%\end{assignment}
%
%\begin{assignment}
%	\begin{task}
%	Break.
%	\end{task}
%\end{assignment}
%
%\begin{assignment}
%	\begin{task}
%	What do we mean with a configuration specification?
%	\end{task}
%
%	\begin{task}
%	Which requirements do we have for a configuration specification?
%	\end{task}
%
%	\pause
%
%	\begin{itemize}
%	\item should be extensible
%	\item should be external to application
%	\item open for introspection (for tooling)
%	\item should talk to users
%	\item should allow generation of artefacts
%	\end{itemize}
%\end{assignment}
%
%\begin{assignment}
%	\begin{task}
%	What can be part of a configuration specification?
%	What can they be used for?
%	\end{task}
%\end{assignment}
%
%\begin{assignment}
%	\begin{task}
%	Break.
%	\end{task}
%\end{assignment}
%
%\begin{assignment}
%	\begin{task}
%	Now, how do we implement such a specification?
%	Which artefacts can we generate?
%	\end{task}
%\end{assignment}
%
%\subsection{Assignments}
%
%\begin{frame}
%	\frametitle{Develop with Elektra}
%
%	\begin{task}
%	Can you already compile software using Elektra?
%	\end{task}
%\end{frame}
%
%\begin{frame}
%	\frametitle{Teams}
%
%	\begin{task}
%	All Teams formed?
%	\end{task}
%\end{frame}
%
%\begin{frame}
%	\frametitle{Reformatting}
%
%	\begin{task}
%	Can you reformat the code?
%	\end{task}
%\end{frame}
%
%\begin{frame}
%	\frametitle{Running Tests}
%
%	\begin{task}
%	Can you run all the tests?
%	\end{task}
%\end{frame}
%
%\subsection{L03: Configuration Integration}
%
%\begin{frame}
%	\frametitle{Preview Next Week}
%
%	\begin{itemize}
%	\item Configuration Libraries
%	\item Lightweight to Strong Integration
%	\item Sharing Configuration
%	\end{itemize}
%\end{frame}




%%%%%%%%%%%%%%%%%%%%%%%%%%%%%%%%%%%%%%%%%% 
\nocite{raab2017introducing}

\appendix

\begin{frame}[allowframebreaks]
	\bibliographystyle{plainnat}
	\bibliography{../../shared/elektra.bib}
\end{frame}

\end{document}


