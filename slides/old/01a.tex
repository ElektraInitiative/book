% Make nice A4 pages for print:
%\usepackage{pgfpages}
%\pgfpagesuselayout{resize to}[a4paper,border shrink=5mm,landscape]

\beamertemplatenavigationsymbolsempty

\setbeamertemplate{bibliography item}[text]

\usepackage[type={CC},modifier={by-sa},version={4.0}]{doclicense}

\usepackage[utf8]{inputenc}
\usepackage{hyperref}
\usepackage{breakurl}
\usepackage{graphicx}
\usepackage{pgfplots}
\usepackage{pgf}
\usepackage{tikz}
\usetikzlibrary{positioning}
\usetikzlibrary{arrows}
\usetikzlibrary{decorations.markings}
\usetikzlibrary{calc}
\usetikzlibrary{matrix}
\usetikzlibrary{shapes}
\usetikzlibrary{decorations.pathmorphing}
\usetikzlibrary{fit}
\usetikzlibrary{backgrounds}
\usetikzlibrary{plotmarks}
\usepackage{stmaryrd}
\usepackage{listings}
\usepackage{pdflscape}
\usepackage{perpage}
\usepackage{appendixnumberbeamer}

%\usepackage[thmmarks,amsmath,amsthm]{ntheorem} % already included in beamer
\usepackage{thm-restate}

\usepackage[sort&compress,numbers]{natbib}  % to be have \citet, \citeauthor, \citeyear

\MakePerPage{footnote}

\tikzstyle{o}=[r,ppBlue]
\tikzstyle{r}=[thick,rectangle,align=center]
\tikzstyle{t}=[r,ppTrans] %,font=\bfseries]
\tikzstyle{dd}=[densely dashed]
\tikzstyle{n}=[r,ppBlue]
\tikzstyle{p}=[r,ppRed]
\tikzstyle{ppRed}  =[draw=red,  fill=  red!20]
\tikzstyle{ppBlue} =[draw=blue, fill= blue!20]
\tikzstyle{ppGreen}=[draw=green,fill=green!20]
\tikzstyle{ppTrans}=[draw=none, fill=none]

\usetheme{Warsaw}

\useoutertheme[subsection=true]{smoothbars}
%\useoutertheme[subsection=false]{miniframes}

\definecolor{bblue}{HTML}{D7DF01}	% yellow-ish actually, for better black/white printing
\definecolor{rred}{HTML}{C0504D}
\definecolor{ggreen}{HTML}{9BBB59}
\definecolor{ppurple}{HTML}{9F4C7C}
\definecolor{lightgray}{rgb}{0.3,0.3,0.3}
\definecolor{lightergray}{rgb}{0.9,0.9,0.9}
\definecolor{UniBlue}{RGB}{83,121,170}

\DeclareTextFontCommand\textintro{\normalfont\bfseries\itshape} % nice!
\newcommand{\intro}[2][]
{%
	\textintro{#2}%
}
\newcommand{\empha}[2][]
{%
	\emph{#2}%
}

%\theoremstyle{plain}
\newcounter{reqcounter}
\newtheorem{requirement}[reqcounter]{Requirement}

%setbeamercolor{structure}{fg=violet}

\makeatletter
\def\th@task{%
    \normalfont % body font
    \setbeamercolor{block title example}{bg=orange,fg=white}
    \setbeamercolor{block body example}{bg=orange!20,fg=black}
    \def\inserttheoremblockenv{exampleblock}
  }
\makeatother

\theoremstyle{task}
\newtheorem{task}{Task}

\newenvironment{assignment}%
{%\setbeamercolor{background canvas}{bg=violet}%
%\setbeamercolor{structure}{fg=cyan!90!black}%
 \setbeamercolor{frametitle}{bg=orange,fg=white}
\begin{frame}}%
{\end{frame}}%

\AtBeginSection[]{
  \begin{frame}
  \vfill
  \centering
  \begin{beamercolorbox}[sep=8pt,center,shadow=true,rounded=true]{title}
    \usebeamerfont{title}\insertsectionhead\par%
  \end{beamercolorbox}
  \tableofcontents
  \vfill
  \end{frame}
}




\pgfplotsset{compat=1.14}
\author{Markus Raab}


\date{29.01.2020}

\begin{document}

\renewcommand{\enquote}[1]{\emph{``#1''}} % Cannot be done earlier

%%%%%%%%%%%%%%%%%%%%%%%%%%%%%%%
\begin{frame}
	\titlepage
	\doclicenseThis
\end{frame}

\section{Configuration File Formats}

\subsection{Definitions}

%copied to 01:
\begin{frame}
	\frametitle{Basic Definitions}
	The \intro[execution environment]{execution environment} is information outside the boundaries of each currently running process~\cite{corbato1971multics}.

	Controlling the execution environment is essential for configuration management~\cite{cons2002pan,huang2015confvalley}, testing~\cite{van2010automating,wang2009context}, and security~\cite{goldberg1996secure,schreuders2012towards,perkins2009automatically,liang2003isolated}.
\end{frame}

\begin{frame}
	\frametitle{Configuration Setting}
	\begin{definition}
\label{def:configuration-setting}
A \intro[configuration setting]{configuration setting},
or \intro[setting|see{configuration setting}]{setting} in short,
fulfills these properties:
\begin{enumerate}
\item
It is provided by the execution environment.
\item
It is \empha[consume]{consumed} by an application.
\item
It consists of a key, a configuration value, and potentially \empha{metadata}.
The \intro{configuration value}, or \intro[value|see{configuration value}]{value} in short, influences the application's behavior.
\item
It can be \empha[produce]{produced} by the maintainer, user, or system administrator of the software.
\end{enumerate}
\end{definition}

\end{frame}

\begin{frame}[fragile]
	\frametitle{Synonyms for Configuration Settings}
	\ExecuteMetaData[../book/background.tex]{synonyms}
\end{frame}
%:copied to 01

\begin{frame}[fragile]
	\frametitle{Definition}
	A \intro{configuration file} is a file containing configuration settings.

	\pause
	A Web server configuration file:

	\begin{lstlisting}[gobble=4]
	port=80 ; comment
	address=127.0.0.1\end{lstlisting}

	\only<2-2>{
	\begin{task}
	What are keys? What are configuration values? What is metadata?
	\end{task}
	}
	\pause

	The configuration values are ^80^ and ^127.0.0.1^, respectively.
	Other information in the configuration file is metadata for the configuration settings (such as the comment).
\end{frame}

\subsection{Formats}

\begin{frame}
	\frametitle{Types of Formats}
	\begin{itemize}
	\item CSV (comma-separated values)
	\item semi-structured
	\item programming language
	\item document-oriented
	\item literate
	\end{itemize}
\end{frame}

\begin{frame}
	\frametitle{CSV formats}
	\begin{itemize}
	\item passwd: \formatdate{3}{11}{1971}
	\item passwd and group use : as separator
	\item are difficult to extend (e.g., GECOS)
	\item today mostly used for legacy reasons
	\item are replaced one-by-one (e.g., inetd, crontab)
	\end{itemize}
\end{frame}

\begin{frame}
	\frametitle{Trends}
	\begin{itemize}
	\item away from CSV
	\item towards general-purpose serialization formats (INI, JSON)
	\item human-read/writable (YAML, HOCON, TOML)
	\item programming language as configuration file
	\end{itemize}
\end{frame}

\begin{frame}
	\frametitle{Programming Language}
	\begin{description}
	\item[$+$] very easy for developers (simply source the file)
	\item[$+$] above-overage quality of error message
	\item[$-$] makes automatic change of individual values harder
	\item[$-$] very hard to use for people who do not know the programming language
	\item[$-$] does not separate code and data
	\end{description}
\end{frame}

\begin{assignment}
	\frametitle{Introduce somebody}
	\begin{task}
	Talk with someone about your favourite configuration file format.
	\end{task}

	\begin{task}
	Did you implement a configuration file parser and/or invented a new configuration file format?
	\end{task}

	\begin{task}
	Explain to everyone about the other person and his/her favourite configuration file format.
	\end{task}
\end{assignment}

\begin{frame}
	\frametitle{Method}

	What do FLOSS developers say?

	\begin{description}
	\item[\methodQuestion{}] survey with 672 persons visiting, 162 persons completing the survey~\cite{raab2017challenges}
	\item[\methodSource{}] source code analysis of 16 applications, comprising 50 million lines of code~\cite{raab2017challenges}
	\end{description}
\end{frame}

\begin{frame}
	\frametitle{Why are so many formats present?}
	\methodQuestion{} \question{In which way have you used or contributed to the configuration system/library/API in your previously mentioned FLOSS project(s)?}~\cite{raab2017challenges}
	\begin{itemize}
	\item \p{19} persons ($n=251$) have introduced a configuration file format.
	\item \p{29} implemented a configuration file parser.
	\item \p{15} introduced a configuration system/library/API.
	\item \p{34} used external configuration access APIs.
	\end{itemize}
\end{frame}

\begin{frame}
	\frametitle{Multitude of Formats}
	\begin{itemize}
	\item on every system a multitude of (legacy) configuration file formats exist
	\item the number grows fast
	\item thus applications usually have to deal with some legacy formats
	\end{itemize}
	

	\begin{restatable}{requirement}{reqLegacy}
	A configuration library must be able to integrate (legacy) systems and must fully support (legacy) configuration files.%
	\label{req:legacy}
	\end{restatable}
\end{frame}

\appendix

\begin{frame}[allowframebreaks]
	\bibliographystyle{plainnat}
	\bibliography{../shared/elektra.bib}
\end{frame}

\end{document}

