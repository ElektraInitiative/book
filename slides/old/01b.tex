%\ifdefined\handout
%\documentclass[handout,aspectratio=1610,xcolor={usenames,dvipsnames,table}]{beamer}
%\else
\documentclass[aspectratio=1610,xcolor={usenames,dvipsnames,table}]{beamer}
%\fi

\newcommand{\mylecture}{Configuration Management}

%\ifdefined\handout
%\documentclass[handout,aspectratio=1610,xcolor={usenames,dvipsnames,table}]{beamer}
%\else
\documentclass[aspectratio=1610,xcolor={usenames,dvipsnames,table}]{beamer}
%\fi

\newcommand{\mylecture}{Configuration Management}

%\ifdefined\handout
%\documentclass[handout,aspectratio=1610,xcolor={usenames,dvipsnames,table}]{beamer}
%\else
\documentclass[aspectratio=1610,xcolor={usenames,dvipsnames,table}]{beamer}
%\fi

\newcommand{\mylecture}{Configuration Management}

\input{../setup}
\input{../shared/setup}

\lstdefinelanguage{dump}
{
	morekeywords={kdbOpen,ksNew,keyNew,keyMeta,keyCopyMeta,keyEnd,ksEnd,kdbClose},
	sensitive=false,
	morecomment=[l]{//},
	morecomment=[s]{/*}{*/},
	morestring=[b]",
}


\lstdefinelanguage{SpecElektra}{
	%
	comment=[l]{;},
	commentstyle=\color{purple}\ttfamily,
	%
	morestring=[b]',
	morestring=[b]`,
	morestring=[b]",
	stringstyle=\color{purple}\ttfamily,
	%
	sensitive=f,% keywords are not case sensitive
	%
	% Colors see https://en.wikibooks.org/wiki/LaTeX/Colors
	%
	keywordstyle=\color{BlueViolet}\bfseries,
	keywordstyle=[2]\color{Green},
	keywordstyle=[3]\color{Aquamarine}\bfseries\textit,
	keywordstyle=[4]\color{NavyBlue}\bfseries,
	keywordstyle=[5]\color{Mahogany},
	%
	keywords={layer, require, validation, check, range, description, rationale, requirement, accessibility, enable, condition, message, default, opt, readonly, type, context, property1, property2, description, file, content, mountpoint, metadata, infos, plugins},
	keywords=[2]{},
	keywords=[3]{order, interface, network, emphasized},
	%keywords=[4]{[, ]},  %Not needed
	keywords=[4]{},
	keywords=[5]{},
	%
	literate={:=}{{{\color{red}\textbf:=}}}2
		 {\%}{{{\color{NavyBlue}\textbf\%}}}1
		 {[}{{{\color{Sepia}\textbf[}}}1
		 {]}{{{\color{Sepia}\textbf]}}}1,
}

\lstdefinelanguage{Cpp}{%
	language     = C++,
	literate=
}


\lstdefinelanguage{CfgElektra}{
	comment=[l]{;},
	commentstyle=\color{purple}\ttfamily,
	%
	morestring=[b]',
	morestring=[b]`,
	morestring=[b]",
	stringstyle=\color{purple}\ttfamily,
	%
	%
	sensitive=f,% keywords are not case sensitive
	%
	% Colors see https://en.wikibooks.org/wiki/LaTeX/Colors
	%
	keywordstyle=\color{Bittersweet}\bfseries,
	keywordstyle=[2]\color{DarkOrchid}\bfseries,
	keywordstyle=[3]\color{ForestGreen}\bfseries\textit,
	keywordstyle=[4]\color{Goldenrod}\bfseries,
	keywordstyle=[5]\color{CarnationPink},
	%
	keywords={},
	keywords=[2]{},
	keywords=[3]{},
	keywords=[4]{},
	keywords=[5]{},
	%
	literate={=}{{{\color{ForestGreen}\textbf=}}}1
		 %{<-}{{{\color{ForestGreen}\textbf<-}}}2
		 %{*}{{{\color{Bittersweet}\textbf*}}}1
		 {\%}{{{\color{NavyBlue}\textbf\%}}}1,
}




\lstset{language=SpecElektra, % Use SpecElektra as default programming language
	%boxpos=t, % make boxes a bit more unbreakable
	%frame=lines, % top+bottom line
	basicstyle=\ttfamily, % Use normal-size true type font
	showspaces,%
	showstringspaces=false, % Don't put marks in string spaces
	showlines=true, % make sure empty lines at end are shown (needed for concurrency
	tabsize=4, % spaces per tab
	xleftmargin=\parindent, % should be 18pt or 1.5em as defined by memoir
	%Does not really work well (needs to be deactivated for shortlistings):
	breaklines=false,
	%postbreak=\mbox{\textcolor{red}{$\hookrightarrow$}\space},
	%breakautoindent=true,
	%prebreak={\mbox{\ensuremath{\curvearrowright}}} % Zeichen am Zeilenende (Umbruch)
	%breaklines=true,
	%breakautoindent=true,
	%prebreak=\small\symbol{'134}, % backslash
	%prebreak={\mbox{\ensuremath{\curvearrowright}}} % lange kure
	%prebreak={\mbox{\ensuremath{\hookleftarrow}}} % lange kure
	%xleftmargin=3.0ex, %for some formats
	%xrightmargin=1.0ex, %for some formats
	%
	% Files do not work in utf8 see also:
	% http://stackoverflow.com/questions/1116266/listings-in-latex-with-utf-8-or-at-least-german-umlauts
	% http://tex.stackexchange.com/questions/24528/having-problems-with-listings-and-utf-8-can-it-be-fixed
	% Should work but doesn't? (Maybe add to literate broken?)
	%add to literate={ö}{{\"o}}1
	%	{ä}{{\"a}}1
	%	{ü}{{\"u}}1
	%	{Ö}{{\"O}}1
	%	{Ä}{{\"A}}1
	%	{Ü}{{\"U}}1
	%	{ß}{{\ss}}1,
	%
	% listingsutf8 did not work, made umlauts in comments very strange
	%extendedchars=true,
	%inputencoding=utf8,
	%
	%morecomment=[l][\color{blue}]{...}, % Line continuation (...) e.g. blue comment
	morekeywords={for_each},
	numbers=left, % Line numbers on left
	firstnumber=1, % Line numbers start with line 1
	numberstyle=\small\color{blue}, % Line numbers are blue and small
	numbersep=5pt,
	%stepnumber=5 % Line numbers go in steps of 5
}



\lstMakeShortInline[postbreak=,keywordstyle={}]^

\graphicspath{{../pic/}{../figures/}{../graphics/}{../ipe/}{../ggplot/}}



\lstdefinelanguage{dump}
{
	morekeywords={kdbOpen,ksNew,keyNew,keyMeta,keyCopyMeta,keyEnd,ksEnd,kdbClose},
	sensitive=false,
	morecomment=[l]{//},
	morecomment=[s]{/*}{*/},
	morestring=[b]",
}


\lstdefinelanguage{SpecElektra}{
	%
	comment=[l]{;},
	commentstyle=\color{purple}\ttfamily,
	%
	morestring=[b]',
	morestring=[b]`,
	morestring=[b]",
	stringstyle=\color{purple}\ttfamily,
	%
	sensitive=f,% keywords are not case sensitive
	%
	% Colors see https://en.wikibooks.org/wiki/LaTeX/Colors
	%
	keywordstyle=\color{BlueViolet}\bfseries,
	keywordstyle=[2]\color{Green},
	keywordstyle=[3]\color{Aquamarine}\bfseries\textit,
	keywordstyle=[4]\color{NavyBlue}\bfseries,
	keywordstyle=[5]\color{Mahogany},
	%
	keywords={layer, require, validation, check, range, description, rationale, requirement, accessibility, enable, condition, message, default, opt, readonly, type, context, property1, property2, description, file, content, mountpoint, metadata, infos, plugins},
	keywords=[2]{},
	keywords=[3]{order, interface, network, emphasized},
	%keywords=[4]{[, ]},  %Not needed
	keywords=[4]{},
	keywords=[5]{},
	%
	literate={:=}{{{\color{red}\textbf:=}}}2
		 {\%}{{{\color{NavyBlue}\textbf\%}}}1
		 {[}{{{\color{Sepia}\textbf[}}}1
		 {]}{{{\color{Sepia}\textbf]}}}1,
}

\lstdefinelanguage{Cpp}{%
	language     = C++,
	literate=
}


\lstdefinelanguage{CfgElektra}{
	comment=[l]{;},
	commentstyle=\color{purple}\ttfamily,
	%
	morestring=[b]',
	morestring=[b]`,
	morestring=[b]",
	stringstyle=\color{purple}\ttfamily,
	%
	%
	sensitive=f,% keywords are not case sensitive
	%
	% Colors see https://en.wikibooks.org/wiki/LaTeX/Colors
	%
	keywordstyle=\color{Bittersweet}\bfseries,
	keywordstyle=[2]\color{DarkOrchid}\bfseries,
	keywordstyle=[3]\color{ForestGreen}\bfseries\textit,
	keywordstyle=[4]\color{Goldenrod}\bfseries,
	keywordstyle=[5]\color{CarnationPink},
	%
	keywords={},
	keywords=[2]{},
	keywords=[3]{},
	keywords=[4]{},
	keywords=[5]{},
	%
	literate={=}{{{\color{ForestGreen}\textbf=}}}1
		 %{<-}{{{\color{ForestGreen}\textbf<-}}}2
		 %{*}{{{\color{Bittersweet}\textbf*}}}1
		 {\%}{{{\color{NavyBlue}\textbf\%}}}1,
}




\lstset{language=SpecElektra, % Use SpecElektra as default programming language
	%boxpos=t, % make boxes a bit more unbreakable
	%frame=lines, % top+bottom line
	basicstyle=\ttfamily, % Use normal-size true type font
	showspaces,%
	showstringspaces=false, % Don't put marks in string spaces
	showlines=true, % make sure empty lines at end are shown (needed for concurrency
	tabsize=4, % spaces per tab
	xleftmargin=\parindent, % should be 18pt or 1.5em as defined by memoir
	%Does not really work well (needs to be deactivated for shortlistings):
	breaklines=false,
	%postbreak=\mbox{\textcolor{red}{$\hookrightarrow$}\space},
	%breakautoindent=true,
	%prebreak={\mbox{\ensuremath{\curvearrowright}}} % Zeichen am Zeilenende (Umbruch)
	%breaklines=true,
	%breakautoindent=true,
	%prebreak=\small\symbol{'134}, % backslash
	%prebreak={\mbox{\ensuremath{\curvearrowright}}} % lange kure
	%prebreak={\mbox{\ensuremath{\hookleftarrow}}} % lange kure
	%xleftmargin=3.0ex, %for some formats
	%xrightmargin=1.0ex, %for some formats
	%
	% Files do not work in utf8 see also:
	% http://stackoverflow.com/questions/1116266/listings-in-latex-with-utf-8-or-at-least-german-umlauts
	% http://tex.stackexchange.com/questions/24528/having-problems-with-listings-and-utf-8-can-it-be-fixed
	% Should work but doesn't? (Maybe add to literate broken?)
	%add to literate={ö}{{\"o}}1
	%	{ä}{{\"a}}1
	%	{ü}{{\"u}}1
	%	{Ö}{{\"O}}1
	%	{Ä}{{\"A}}1
	%	{Ü}{{\"U}}1
	%	{ß}{{\ss}}1,
	%
	% listingsutf8 did not work, made umlauts in comments very strange
	%extendedchars=true,
	%inputencoding=utf8,
	%
	%morecomment=[l][\color{blue}]{...}, % Line continuation (...) e.g. blue comment
	morekeywords={for_each},
	numbers=left, % Line numbers on left
	firstnumber=1, % Line numbers start with line 1
	numberstyle=\small\color{blue}, % Line numbers are blue and small
	numbersep=5pt,
	%stepnumber=5 % Line numbers go in steps of 5
}



\lstMakeShortInline[postbreak=,keywordstyle={}]^

\graphicspath{{../pic/}{../figures/}{../graphics/}{../ipe/}{../ggplot/}}




\date{29.01.2020}

\begin{document}

\renewcommand{\enquote}[1]{\emph{``#1''}} % Cannot be done earlier

%%%%%%%%%%%%%%%%%%%%%%%%%%%%%%%
\begin{frame}
	\titlepage
	\doclicenseThis
\end{frame}




%%%%%%%%%%%%%%%%%%%%%%%%%%%%%%%%%%%%%%%%%% 
\section{Elektra}

\subsection{Basics}

\begin{frame}
	\frametitle{Elektra as Virtual Filesystem}
	\begin{itemize}
	\item configuration files are seen like ``block devices''
	\item are mounted with respective filesystem drivers into the filesystem
	\item many tools and APIs evolved to work with files
	\item Idea of Elektra: establish a similar ecosystem for configuration
	\end{itemize}
\end{frame}

\begin{frame}
	\frametitle{Why is Elektra not a Filesystem then?}
	\begin{itemize}
	\item API semantics: key/value get/set
	\item namespaces: based on established semantics
	\item many features essential for misconfiguration hardening:
		\begin{itemize}
		\item validation
		\item visibility
		\item defaults
		\item \dots (extensible specification)
		\end{itemize}
	\end{itemize}
\end{frame}

% hack: needed to render graphics properly
\begin{frame}<0>[noframenumbering]
	\begin{columns}[c]
	\begin{column}{7cm}
	\includegraphics[scale=0.8]{horizontalmodularity}
	\end{column}
	\begin{column}{4cm}
	Cylinders are configuration files, P? are plugins~\cite{raab2016improving}.

	Key ideas:
	\begin{itemize}
	\item all work is done by plugins
	\item central data structure implements semantics
	\end{itemize}
	\end{column}
	\end{columns}
\end{frame}

\begin{frame}
	\begin{columns}[c]
	\begin{column}{7cm}
	\includegraphics[scale=0.8]{horizontalmodularity}
	\end{column}
	\begin{column}{4cm}
	Cylinders are configuration files, P? are plugins~\cite{raab2016improving}.

	Key ideas:
	\begin{itemize}
	\item all work is done by plugins
	\item central data structure implements semantics
	\end{itemize}
	\end{column}
	\end{columns}
\end{frame}

\begin{frame}
	\frametitle{KeySet}

	The common data structure between plugins:
	\vspace{1cm}

	\includegraphics{keyset}
\end{frame}

\begin{assignment}
	\begin{task}
	Is meta-data separated from or included in the data structure?
	\end{task}
\end{assignment}

\begin{frame}[fragile]
	\begin{description}[align=left]
	\item[kdb.open():]
	The first step is to bootstrap into a situation where the necessary plugins can be loaded.
	\item[kdb.get(\texttt{KeySet}):] \index{kdb.get}
	The application (initially) fetches and (later) updates its configuration settings as a key set of type ^KeySet^ from the execution environment by one or many calls to ^kdb.get^.
	%If all relevant configuration files are unmodified since the last invocation, ^kdb.get^ will do nothing.
	\item[kdb.set(\texttt{KeySet}):] \index{kdb.set}
	When a user finishes editing configuration settings, ^kdb.set^ is in charge of writing all changes back to the key database.
	%This function atomically persists a whole key set in involved parts of the execution environment.
	%In the case of an error no action takes place.
	\item[kdb.close():] \index{kdb.close}
	The last step is to close the connection to the key database.
	\end{description}
\end{frame}

\begin{assignment}
	\begin{task}
	Break.
	\end{task}
\end{assignment}

\subsection{Metalevels}

\begin{frame}
	\frametitle{Metalevels}
	\includegraphics{metalevels}

	We will now walk through metalevels bottom-up.
\end{frame}

\begin{frame}[fragile]
	\frametitle{Configuration Settings}

	A configuration file may look like:

	\begin{code}[language=CfgElektra]
	a=5
	b=10
	c=15
	\end{code}

	We apply these configuration settings imperatively using:

	\begin{code}[language=bash]
	kdb set /a 5
	kdb set /b 10
	kdb set /c 15
	\end{code}

	And we list them with \lstinline[language=bash,morekeywords={ls},showspaces=no]^kdb ls /^.
\end{frame}

\begin{frame}[fragile]
	\frametitle{Specifications}
	For specifications such as:

	\begin{code}
	[slapd/threads/listener]
	  check/range:=1,2,4,8,16
	  default:=1
	\end{code}

	We apply the specifications imperatively using:

	\begin{code}[language=bash,morekeywords={meta-set}]
	kdb meta-set /slapd/threads/listener\
		check/range 1,2,4,8,16
	kdb meta-set /slapd/threads/listener\
	       	default 1
	\end{code}

	(automatically uses ^spec^ namespace)
\end{frame}

\begin{frame}[fragile]
	\frametitle{Meta-Specifications}
	For meta-specifications such as:

	\small
	\begin{code}
	[visibility]
	type:=enum critical important user\
	      advanced developer debug disabled
	description:=Who should see this\
	     configuration setting?
	\end{code}

	We apply the meta-specifications imperatively using:

	\begin{code}[language=bash,morekeywords={meta-set}]
	kdb meta-set /elektra/meta/\
		visibility type enum ...
	kdb meta-set /elektra/meta/\
		visibility description "Who ...
	\end{code}

	(see ^doc/METADATA.ini^, disclaimer: 1.0 not yet released)
\end{frame}

\begin{assignment}
	\begin{task}
	Brainstorming: Ideas for (meta-)specifications.
	\end{task}
\end{assignment}

\subsection{Conclusions}

\begin{frame}
	\frametitle{Introspection}
	\begin{itemize}[<+->]
	\item unified get/set access to (meta*)-key/values
	\item access via applications, CLI, GUI, web-UI, ...
	\item GUI, web-UI can semantically interpret metadata
	\item access via any programming language
	\item access via any configuration management system
	\end{itemize}
\end{frame}

\begin{frame}
	\frametitle{Users of Elektra}
	\begin{itemize}[<+->]
	\item Embedded systems
	\begin{itemize}
	\item OpenWRT (distribution)
	\item Broadcom (blue-ray devices)
	\item Kapsch (cameras)
	\item Toshiba (TVs)
	\end{itemize}
	\item Server
	\begin{itemize}
	\item Allianz (insurance)
	\item TU Wien
	\item puppet-libelektra
	\item Other Universities
	\end{itemize}
	\item Desktop
	\begin{itemize}
	\item Oyranos
	\item LCDproc (in progress)
	\item KDE
	\end{itemize}
	\end{itemize}
\end{frame}

\begin{frame}
	\frametitle{Conclusion}
	\begin{itemize}
	\item goals:
		\begin{itemize}
		\item make simple configuration management tasks simple
		\item improve robustness
		\item improve extensibility (reusable plugins operating on key/value)
		\item improve performance
		\item good defaults
		\item system-wide introspection
		\item system-level dependency injection
		\end{itemize}
	\item \elektra{} has no dependence to other libraries but only concrete plugins introduce dependences.
	\end{itemize}
\end{frame}




\appendix

\begin{frame}[allowframebreaks]
	\bibliographystyle{plainnat}
	\bibliography{../shared/elektra.bib}
\end{frame}

\end{document}

