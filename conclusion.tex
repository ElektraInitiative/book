\chapter{Conclusion and Future Work}
\label{chapter:conclusion}

\chapterprecishere{We are what we think. All that we are arises with our thoughts. With our thoughts we make the world.\par\raggedleft--- \textup{Siddhartha Gautama, rendering of Thomas Byrom}}
\chapterhung

Humans are involved in an endless struggle to express themselves.
In this book we described some contributions that help users, system administrators, and developers to reveal their thoughts related to context-aware configuration.
In particular, the framework \elektra{} and the modular specification language \elektra{Spec} are a promising way to specify configuration access for FLOSS systems.
We are ready to answer our main research question for the different \empha[stakeholder]{stakeholders}:
\rqMain*


\begin{description}
\item[Developers] had obstacles that they were missing decent frontends with a high level of abstraction.
Thus they hard-coded inflexible and mostly context-unaware configuration accesses.
\elektra{} improves on the problem by the concept of contextual values as configuration settings.
These better frontends enable developers to directly work with variables that contain context-aware configuration.

\item[System administrators] were not empowered to connect configuration settings and context information.
\elektra{} improves on the configuration integration problem by globally sharing configuration settings and context information in a key database.
To not lose any modularity, we mount backends and put needed functionality in plugins.
\elektra{Spec} empowers system administrators to improve on configuration validation and context awareness with requirements and context information encoded as configuration settings.

\item[End users] rarely had context awareness in their applications.
Instead, they had to manually change settings of every application.
With \elektra{}, they benefit from increased context awareness of the configuration settings.
We showed that this even works for applications in which no source code is modified.
Furthermore, \elektra{} provides user interfaces so that end users can avoid any contact with configuration~files.
\end{description}



We conclude the answer of \rqref{main} by walking through the four initial goals.
We revisit contributions to each goal and describe future work.
As structure, we use the steps needed to create and evaluate the modular, system-level, context-oriented configuration specification language \elektra{Spec}:
\begin{itemize}
 \item Before we created \elektra{Spec}, we had to understand the requirements and challenges.
 We conclude on our challenges in \secref{conclusion-challenges}.
 \item \elektra{Spec} alone would be an incomplete solution if applications misinterpret configuration settings internally.
 In \secref{conclusion-programming}, we conclude our efforts to provide better and safer frontends that avoid missing considerations of context.
 \item In \secref{conclusion-language} we give final remarks about the modularity and efficiency of \elektra{Spec}.
 \item In \secref{conclusion-awareness} we summarize how we achieve context-aware configuration for both elektrified applications and applications without source code modifications.
\end{itemize}

















\section{Challenges}
\label{sec:conclusion-challenges}

In the beginning of our work we found that state-of-the-art literature only reported the phenomenon misconfiguration but hardly elaborated on its causes.
Thus we had to unveil these causes ourselves.
From the causes, we derived requirements aiming at \goalref{requirements}:
\goalRequirements*

To accomplish the goal, we addressed the following research question:
\rqMotivation*

The answer is that validations and context information is encoded in the applications in a way so that they are~\cite{raab2017challenges}
\begin{itemize}
\item not reusable, requiring error-prone duplication, %finding of RQ 2.1; Contribution 5
\item not introspectable by external tools, and
\item incapable of using knowledge of the system and its context.
\end{itemize}
We supported our claims by an experience report, a questionnaire, and a large-scale source code analysis~\cite{raab2017challenges}.
We framed the disclosed challenges as \empha{configuration integration problem}:
Applications are currently unable to access configuration settings and specifications present in the system.

In this book, we collected the first empirically-founded requirements for configuration access.
Some requirements fundamentally differ from all current implementations.
In particular, \elektra{Lib} is the first implementation with external configuration specifications and consistent introspection.
These properties are, however, essential to mitigate the configuration integration problem that was found important by \p{96} of $173$ survey participants.

Furthermore, we learned that current frontends are already used as if they were context aware, although they are not.
Based on that, we split our efforts into two paths for possible improvements on context-aware configuration:
\begin{itemize}
\item By providing novel frontends, we mitigate problems for newly-written software.
\item By reimplementing current frontends, we mitigate problems for legacy software.
\end{itemize}

\paragraph{Future Work:}
We plan to conduct studies on misconfiguration with systems that use \elektra{}.
In particular, similar studies as discussed in \chapref{motivation} and \secref{implication-misconfiguration} should be repeated.
This way, new ways of misconfiguration can be unveiled and the effectiveness of \elektra{} can be improved.










\section{Context-oriented Programming}
\label{sec:conclusion-programming}

We would not achieve much if developers continue to access configuration settings in error-prone, non-unifiable ways.
Thus we first looked at \goalref{frontends}:
\goalFrontends*

We demonstrated the possibility of supporting contextual values without overhead on reading them---unconcerned of the number of active layers.
A declarative specification of contextual values diminishes the burden of implementing type-safe contextual classes implementing contextual values and layers.
Additionally, we avoid superfluous cache updates on context changes by specifying dependences between values and contexts~\cite{raab2014program}.


Specifications in combination with active layers yield unique names for all contextual values in each context.
These names allow introspection and improve debugging support of contextual values~\cite{raab2014program}.


\paragraph{Ubiquitous Computing:}

In this book, we improved context awareness and customizations without compromise on efficiency in multi-threaded applications.
Developers can choose whether context changes are across all threads or target specific thread(s)~\cite{raab2015global}.

We demonstrated that our approach improves on multi-core-processor support for context-aware ubiquitous computing.
\elektra{} facilitates algorithms to read contextual values concurrently with full control with respect to performance~\cite{raab2015global}.

In benchmarks we demonstrated that even frequent context changes do not slow down the page replies of a Web server application.
Only on a single-core processor we noticed decay if using a high number of context changes.
On a multi-core processor we needed an unrealistic-long global lock to reduce the number of replies per seconds~\cite{raab2015global}.

Our contributions for ubiquitous computing are summarized as follows~\cite{raab2015global}:
\begin{itemize}
\item \elektra{} enables developers to facilitate contextual values in embedded, multi-threaded applications.
\item In a case study, we reported on our experience on developing a Web server running on embedded hardware.
\item We analyzed the performance in single-core and multi-core setups.
\end{itemize}


These contributions are significant because in other implementations, context awareness had a much larger performance impact unsuitable for embedded systems~\cite{raab2015global}.


Another important aspect of our tool is the build-in persistence of the execution environment.
In our approach, command-line options and configuration files initialize all contextual values for every known context.
\elektra{} enables serializing users' customizations to configuration files.


\paragraph{Mobile Computing:}


In this book we demonstrated how contextual values are used for layer activation.
We considered several limitations and benefits~\cite{raab2016persistent}:
\begin{itemize}
\item Layer activations automatically consider context.
\item Applications are enabled to synchronize layers activations.
\item Applications share context information across programming languages.
\end{itemize}


The approach is practical and relevant to mobile development.
It simplifies accounting for the current context.
Furthermore, it supports individual customization and sharing context between applications.
Even end users can redefine configuration settings in a specific context~\cite{raab2016persistent}.

In benchmarks, we demonstrated that activating contextual values does not add much cost to layer activations.
\elektra{} also supports synchronization with the key database to enable sharing of contextual values between applications.
A real-world Web server benchmark illustrates that only high synchronization rates with the key database, such as every few milliseconds, influence the number of served requests on a multi-core computer~\cite{raab2016persistent}.%
{\parfillskip=0pt plus .7\textwidth \emergencystretch=.5\textwidth \par}

Our contributions in mobile computing are summarized as follows~\cite{raab2016persistent}:
\begin{itemize}
\item With \elektra{}, we introduced a tool with a unique combination of performance, context awareness, and customization.
\item \elektra{} empowers developers to facilitate contextual values in multi-threaded and multi-process applications.
\item Contextual values are shareable across applications.
\item \elektra{} currently supports development, also for embedded systems, in \supportedbindings.
\item In a case study we reported our experience in embedded development.
	We analyzed the performance in microbenchmarks and using a Web server.
\end{itemize}

\paragraph{Future Work:} Many further optimization techniques are open to be explored.
For example, layer switching can be accelerated by using an array per contextual value.
It might require compression techniques to keep the size of arrays acceptable.











\section{Modular Specification Language \textsc{SpecElektra}}
\label{sec:conclusion-language}

We conclude on \goalref{abstraction}:
\goalAbstraction*

\elektra{Spec} is in a sweet spot of configuration specification languages that enables modularity:
In \elektra{Spec}, modularity supports developers to reach system-oriented and context-oriented requirements.
The modularization in the specification language supports applications to include their specific validation strategies.
This way we reach an improvement of the validation precision without introducing a complex configuration specification language.
\elektra{Spec} enables us to externalize many cross-cutting concerns related to configuration access~\cite{raab2016improving}.


We implemented most parts of \elektra{Spec} in backends so that specifications are automatically enforced for all applications.
Bugs are easily fixed for all applications at a single place.
We took care to keep the backends extensible by implementing all features as plugins.
We presented an algorithm that automatically assembles plugins using the configuration specifications and the plugins' contracts.

In benchmarks we showed that overhead introduced by modular abstraction is negligible.
All current features, except of the cache for contextual values, can be implemented within plugins in the backends.
Even better, the impact of a higher number of plugins is irrelevant compared to overall costs~\cite{raab2016improving}.

The benefits of \elektra{Spec} are thus that it~\cite{raab2016improving}
\begin{itemize}
\item opens up a unified way to access configuration settings,
\item enables documentation, validation, and even calculations of configuration settings,
\item avoids restrictions for plugins,
\item introduces plugins with run-time and compile-time variability, and
\item enables us to introduce high-level configuration settings from which \elektra{} derives suitable context-aware configurations.
\end{itemize}

\paragraph{Future Work:} There is some doubt that POSIX abstractions cover the needs of applications~\cite{atlidakis2016posix}.
We agree and think that configuration access is an important need of applications.
In further work, \elektra{}'s API, KDB, and its abstractions should be reimplemented in different programming languages.
For implementations of \elektra{}'s API, compliance levels should be defined.


Further research is needed to decide which type systems fit best for such a configuration specification language.
Several type systems, as described in \secref{related-type-systems}, can be implemented and compared.
We also leave type-safe upgrading of configuration specifications as further work.

















\section{Context awareness}
\label{sec:conclusion-awareness}

We contributed to \goalref{context}:
\goalContext*

In this book, we claimed that applications can be enriched to be more context aware without any source code modifications.
We showed that such a run-time system exists and, furthermore, is practical.
We evaluated \elektra{} on 16 large, real-world FLOSS applications and presented more detailed case studies on some of the applications.
Only by changing configuration settings and writing simple specifications, we improved context awareness in several case studies, often even flawlessly.
We applied a context-oriented software engineering process that supports systematic applicability~\cite{raab2017introducing}.

We facilitated a context-aware key database using configuration files.
Calls to frontends, such as ^getenv^ and ^open^, are forwarded to \elektra{}'s implementation of context-aware configuration access.
Furthermore, the same context information, configuration settings, and specifications are reused between applications to improve on the configuration integration problem.
A unique property of our approach is that it enables context awareness without any modifications in source code~\cite{raab2016unanticipated}.


Our results related to unanticipated context awareness are~\cite{raab2016unanticipated}:
\begin{itemize}
\item With our approach applications are more aware of their context.
This context awareness leverages application integration.
\item Our work demonstrates that deducing configuration settings from context is realistic.
\item We provide experimental validation in a case study of significant complexity.
\item The evaluation offers some clues on the potential of context awareness.
\end{itemize}

Avoiding manual considerations of context and validation in configuration access addresses a source of misconfiguration~\cite{raab2017introducing}.


Our approach demonstrates that it is not required to foresee every possible context during development.
Instead we introduce layers and configuration settings during deployment.
\elektra{} is modular due to the separation of context sensors and applications.
The source code and run-time analysis shows that dependence injection, i.\,e., `hijacking' existing ^getenv^ and ^open^ invocations, enables more context awareness~\cite{raab2017introducing}.

We showed how we systematically integrated all 1,957 configuration settings of Firefox to provide seamless adaptation to workplaces.
We never needed to modify Firefox's source code~\cite{raab2017introducing}.%
{\parfillskip=0pt plus .7\textwidth \emergencystretch=.5\textwidth \par}

\paragraph{Future Work:} We plan to integrate more context awareness and to conduct larger case studies~\cite{raab2016improving}.
Ideally, whole desktop environments are elektrified with a single implementation of an API.
The ^open^ interception can be reimplemented as file system in user space (FUSE), which is a less intruding solution.
We take advantage of the fact that \elektra{} is by no means limited to intercepting ^getenv^ and ^open^.
For example, we started implementing the API GSettings used within the desktop environment GNOME.
It has the potential to make all GNOME settings context aware.
Extensions to make even more forms of configuration context aware (configuration for plugins, modules, mobile APIs etc.) remain as future work~\cite{raab2017introducing}.
