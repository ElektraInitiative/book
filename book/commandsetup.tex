%% For Elektra names
\newcommand{\elektra}[2][]
{%
	\ifthenelse{\isempty{#2}}%
		{\textsc{Elektra}}% if #2 is empty
		{%
		\ifthenelse{\isempty{#1}}%
			{\textsc{#2\-Elektra}}% if #1 is empty, do not index it!
			{\index{#1Elektra|boldindex}\textsc{#2\-Elektra}}% if #1 is not empty, index it
		}%
}

%% Highlight an important word + index it
\newcommand{\empha}[2][]
{%
	\ifthenelse{\isempty{#1}}%
		{\index{#2}\emph{#2}}% if #1 is empty
		{\index{#1}\emph{#2}}% if #1 is not empty
}


%% https://tex.stackexchange.com/questions/391448/list-all-theorems-that-were-not-restated/391484#391484
\makeatletter
\long\def\g@addto@toks#1#2{\global#1\expandafter{\the#1#2}}
\newtoks\@toks@restatable
\newtoks\listofrestatable
\newtoks\listofrestatated
\let\tre@restatable\restatable
\let\tre@endrestatable\endrestatable
\renewenvironment{restatable}[3][]{%
  \g@addto@toks\@toks@restatable{#3,}%
  \g@addto@toks\listofrestatable{\process@comma#3 (#2)}%
  \global\@namedef{add@restated@#3}{%
    \g@addto@toks\listofrestatated{#3; }%
    \global\cslet{add@restated@#3}{\relax}%
  }
  \tre@restatable[#1]{#2}{#3}%
  \csgpreto{#3}{\@nameuse{add@restated@#3}}%
}{\tre@endrestatable}
\def\process@comma{\def\process@comma{,\space}}
\def\notrestated{}
\def\checknotrestated{%
  \@for\@rthm:=\the\@toks@restatable\do{%
    \ifx\@rthm\@empty\else
      \expandafter\ifx\csname add@restated@\@rthm\endcsname\relax\else
        \xdef\notrestated{\notrestated\@rthm; }\fi
    \fi
  }
  \ifx\notrestated\empty\else
    \par\vspace{1pc}\par\noindent
    Warning! -- There were theorems that have not been restated: \notrestated
    \@latex@warning{There were theorems that have not been restated: \notrestated}%
\fi
}
\makeatletter



% https://en.wikibooks.org/wiki/LaTeX/Fonts
\DeclareTextFontCommand\textintro{\normalfont\bfseries\itshape} % nice!
%\DeclareTextFontCommand\textintro{\normalfont\slshape\scshape} % also nice! needs slantsc
%\DeclareTextFontCommand{\question}{\em}

%amsmath stuff
\DeclareMathSymbol{\mlq}{\mathord}{operators}{``}  % math left quote
\DeclareMathSymbol{\mrq}{\mathord}{operators}{`'}  % math right quote

%% Introduces a new word in the book + index it
\newcommand{\intro}[2][]
{%
	\ifthenelse{\isempty{#1}}%
		{\index{#2|boldindex}\textintro{#2}}% if #1 is empty
		{\index{#1|boldindex}\textintro{#2}}% if #1 is not empty
}

% https://tex.stackexchange.com/questions/390144/why-does-lstlisting-removes-a-space-after-morestring/390156#390156
\makeatletter
%\def\lst@visiblespace{\lst@ttfamily{\ }{\ }}
\def\lst@visiblespace{\lst@ttfamily{\phantom{\char32}}{\phantom{\textvisiblespace}}}
\makeatother

%https://stackoverflow.com/questions/2767389/referencing-a-theorem-like-environment-by-its-name
\makeatletter
\def\namedlabel#1#2{\begingroup
   \def\@currentlabel{#2}%
   \label{#1}\endgroup
}
\makeatother


% From here:
% https://tex.stackexchange.com/questions/24101/theorem-decorations-that-stay-with-theorem-environment
\newcommand{\theoremhang}{% top theorem decoration
  \begingroup%
    \setlength{\unitlength}{.005\linewidth}% \linewidth/200
    \begin{picture}(0,0)(1.5,0)%
      \linethickness{0.45pt} \color{Black}%
      \put(-3,2){\line(1,0){206}}% Top line
      \multido{\iA=2+-1,\iB=50+-10}{5}{% Top hangs
        \color{black!\iB}%
        \put(-3,\iA){\line(0,-1){1}}% Top left hang
        \put(203,\iA){\line(0,-1){1}}% Top right hang
      }%
    \end{picture}%
  \endgroup%
}%

\newcommand{\theoremhung}{% bottom theorem decoration
  \nobreak%
  \vspace*{-1.8ex}%
  \begingroup%
    \setlength{\unitlength}{.005\linewidth}% \linewidth/200
    \begin{picture}(0,0)(1.5,0)%
      \linethickness{0.45pt} \color{Black}%
      \put(-3,0){\line(1,0){206}}% Bottom line
      \multido{\iA=0+1,\iB=50+-10}{5}{% Bottom hangs
        \color{black!\iB}%
        \put(-3,\iA){\line(0,1){1}}% Bottom left hang
        \put(203,\iA){\line(0,1){1}}% Bottom right hang
      }%
    \end{picture}%
  \endgroup%
}%

\newcommand{\theoremspace}{\needspace{2\baselineskip}}
\newcommand{\theoremspuce}{\nopagebreak\noindent}

\newcommand{\chapterhung}{\vspace{-1.4em}\theoremhung\vspace{2em}}

\newcounter{defexample}
\numberwithin{defexample}{chapter}


\theoremstyle{plain}
\newtheorem{definition}[defexample]{Definition}

\theoremstyle{definition}
\theoremsymbol{\ensuremath{\blacktriangle}}%
\newtheorem{example}[defexample]{Example}

\theoremstyle{definition}
\theoremsymbol{}%
\theoremprework{\theoremspace\theoremhang\vspace*{-1.3ex}}%
\theorempostwork{\vspace{-1em}\theoremspuce\theoremhung\vspace{1em}}%
\theoremheaderfont{\upshape\bfseries}
\theoremseparator{.}
\theorembodyfont{\upshape}
\newtheorem{lemma}{Lemma}

\theoremstyle{plain}
\newtheorem{contribution}{Contribution}

\theoremstyle{plain}
\newcounter{reqcounter}
\newtheorem{requirement}[reqcounter]{Requirement}

\theoremstyle{nonumberplain}
\newtheorem{goal}{Goal}

\theoremstyle{definition}
\newtheorem{RQ}{RQ}

\theoremstyle{definition}
\newtheorem{RQsub}{RQ}[RQ]

\theoremstyle{definition}
\newtheorem{RQsubsub}{RQ}[RQsub]


\theoremstyle{nonumberplain}
\theoremheaderfont{\sffamily\bfseries\upshape}%
\theorembodyfont{\itshape}%
\theoremsymbol{}%
\theoremseparator{\ }%
\theoremprework{\theoremspace\theoremhang\vspace*{-1.3ex}}%
\theorempostwork{\vspace{-1em}\theoremspuce\theoremhung\vspace{1em}}%
\setlength{\theorempreskipamount}{-1ex}
\setlength{\theorempostskipamount}{-1ex}
\newtheorem{hypothesis}{Hypothesis}

\theoremstyle{nonumberplain}
\theoremheaderfont{\sffamily\bfseries\upshape}%
\theorembodyfont{\itshape}%
\theoremsymbol{}%
\theoremseparator{\ }%
\theoremprework{\theoremspace\theoremhang\vspace*{-1.3ex}}%
\theorempostwork{\vspace{-1em}\theoremspuce\theoremhung\vspace{1em}}%
\setlength{\theorempreskipamount}{-1ex}
\setlength{\theorempostskipamount}{-1ex}
\newtheorem{finding}{Finding}

\theoremstyle{nonumberplain}
\newtheorem{implication}{Implication}


% https://tex.stackexchange.com/questions/229383/alignment-of-modified-veelo-chapter-style-memoir
\makeatletter
\newlength{\numberheight}
\setlength{\numberheight}{\beforechapskip}

\makechapterstyle{meelo}{
	\setlength{\afterchapskip}{40pt}
	\renewcommand*{\chapterheadstart}{\vspace*{40pt}}
	\renewcommand*{\afterchapternum}{\par\nobreak\vskip 25pt}
	\renewcommand*{\chapnamefont}{\normalfont\LARGE\flushright}
	\renewcommand*{\chapnumfont}{\normalfont\HUGE}
	\renewcommand*{\chaptitlefont}{\normalfont\HUGE\bfseries\flushright}
	\renewcommand*{\printchaptername}{%
		\chapnamefont\MakeUppercase{\@chapapp}}
	\renewcommand*{\chapternamenum}{}
	\setlength{\beforechapskip}{18mm}
	\setlength{\midchapskip}{\paperwidth}
	\addtolength{\midchapskip}{-\textwidth}
	\addtolength{\midchapskip}{-\spinemargin}
	\addtolength{\midchapskip}{-11.5em}
	\renewcommand*{\printchapternum}{%
		\enspace\resizebox{!}{\numberheight}{\chapnumfont\thechapter}%
			\rlap{\hspace{2cm}\rule{\midchapskip}{\beforechapskip}}%
	}%
	\makeoddfoot{plain}{}{}{\thepage}%
}

\chapterstyle{meelo}
\makeatother


%% Make grammar(s) pretty:
%\setlength{\grammarparsep}{20pt plus 1pt minus 1pt} % increase separation between rules
\setlength{\grammarindent}{12em} % increase separation between LHS/RHS

%% to be used to mark the main page of an index
\newcommand{\boldindex}[1]{\textbf{\hyperpage{#1}}}

%\newunicodechar{⏎}{$\hookleftarrow$} % \hookleftarrow ≅ ↩︎


%% Refer to a Goal
\newcommand{\goalref}[1]
{Goal~\ref{goal:#1}}

%% Refer to a Section
\newcommand{\secref}[1]
{Section~\ref{sec:#1}}
%{Section~\ref{sec:#1} on page~\pageref{sec:#1}}

%% Refer to a Table
\newcommand{\tabref}[1]
{Table~\ref{tab:#1}}

%% Refer to a Figure
\newcommand{\figref}[1]
{Figure~\ref{fig:#1}}

%% Refer to a Example
\newcommand{\exref}[1]
{Example~\ref{ex:#1}}

%% Refer to a Lemma
\newcommand{\lemmaref}[1]
{Lemma~\ref{lemma:#1}}

%% Refer to a Definition
\newcommand{\defref}[1]
{Definition~\ref{def:#1} on page~\pageref{def:#1}}

\let\oldeqref\eqref
%% Refer to an equation
\renewcommand{\eqref}[1]
{Equation~\oldeqref{eq:#1}}

%% Refer to RQ for sections
\newcommand{\rqsubsection}[2]{\subsection{RQ \ref*{rq:#1}: #2}}


\newcommand{\fixtheorem}[0]{\leavevmode\vspace{-1em}}

%% Refer to a chapter
\newcommand{\chapref}[1]
{Chapter~\ref{chapter:#1}}
%{Chapter~\ref{chapter:#1} on page~\pageref{chapter:#1}}

\newcommand{\question}[1]
{\emph{O: ``#1''}}

\newcommand{\methodQuestion}{\emph{Q:}}

\newcommand{\methodSource}{\emph{S:}}



%define namespace
%textsc also used in backend.tex:344
\makeatletter
\newcommand{\namespace}[1]{\@namespace{#1}}
\newcommand{\@@namespace}[1]{\textsc{\let\@namespace\@@@namespace#1}}
\newcommand{\@@@namespace}[1]{\textnormal{\let\@namespace\@@namespace#1}}
\let\@namespace\@@namespace
\makeatother


\makeatletter
\newcommand{\notsupported}[1]{\@notsupported{Not supported#1}}
\newcommand{\@@notsupported}[1]{\textsc{\let\@notsupported\@@@notsupported#1}}
\newcommand{\@@@notsupported}[1]{\textnormal{\let\@notsupported\@@notsupported#1}}
\let\@notsupported\@@notsupported
\makeatother

% code environment (avoids listings to have page breaks within them)
% \begin{code}[caption={a},language=Cpp]
% \end{code}
\lstnewenvironment{code}[1][]%
{%
\microtypesetup{activate=false}%
\noindent%
\minipage{\linewidth}%
%\vspace{0.5\baselineskip}\medskip%
\theoremspace%
\theoremhang%
\lstset{#1}}%
{\theoremhung\endminipage%
\microtypesetup{activate=true}%
}
%{}

\newcommand{\StashGrammar}{\let\syntleft\relax \let\syntright\relax}

%two lines within a cell
\newcommand{\blap}[1]{{\begin{tabular}[t]{@{}c@{}}#1\end{tabular}}}

%operators
\DeclareMathOperator*{\concat}{\mathnormal{+\!\!+}}
\DeclareMathOperator*{\concatkey}{\mathnormal{\colon\!\!\!/}}

%symbols
%make Linebreak work without math
\newcommand{\NullValue}{\epsilon}
\newcommand{\CleanString}{\overrightarrow{C'}}
\newcommand{\RelativeKeyNames}{\mathbb{R}}
\newcommand{\KeyNames}{\mathbb{N}}
\newcommand{\MetaKeyValues}{\bigcup\limits_{g \in G} X_g}
\newcommand{\EmptyNameSpace}{\epsilon}
\newcommand{\Keys}{\mathbb{K}}
\newcommand{\KeySets}{\mathcal{K}}
\newcommand{\NullKey}{\varnothing}
\newcommand{\EmptyKeySet}{\kappa}
\newcommand{\NameSpaces}{\mathnormal{N}}
\newcommand{\LineBreak}{\ensuremath{\hookleftarrow}}
\newcommand{\WhiteSpace}{\textvisiblespace}
\newcommand{\SetOfValues}{\ensuremath{\mathbb{C}_\NullValue}}
\newcommand{\Strings}{\ensuremath{\mathbb{C}}}
\newcommand{\NonEmptyStrings}{\ensuremath{\mathbb{C}_{\geq1}}}
\newcommand{\CleanStrings}{\ensuremath{\mathbb{C}'}}
\newcommand{\NonEmptyCleanStrings}{\ensuremath{\mathbb{C}_{\geq1}'}}

%% Define a key
\newcommand{\key}[1]{\lstinline[language=,literate={/}{}{0\discretionary{/}{}{/}},postbreak=,breaklines=false]{#1}}

% write a string
\newcommand{\str}[1]
{\syntax{\lq\texttt{#1}'}}

% write a plugin (hardly used)
\newcommand{\plugin}[1]
{\texttt{#1}}

%% Define a property (also used for clauses)
\newcommand{\property}[1]{\lstinline[language=,literate={/}{}{0\discretionary{/}{}{/}},postbreak=,breaklines=false]{#1}}


%: WARNING: needs to be kept in sync with background.tex:1169
\lstMakeShortInline[postbreak=,keywordstyle={}]^
%\lstMakeShortInline[breaklines=true]^



%define strong
\makeatletter
\newcommand{\strong}[1]{\@strong{#1}}
\newcommand{\@@strong}[1]{\noindent \textbf{\let\@strong\@@@strong#1}}
\newcommand{\@@@strong}[1]{\textnormal{\let\@strong\@@strong#1}}
\let\@strong\@@strong
\makeatother

%define rqref
\makeatletter
\newcommand{\rqref}[1]{\@rqref{RQ~\ref{rq:#1}}}
\newcommand{\@@rqref}[1]{\noindent \textbf{\let\@rqref\@@@rqref#1}}
\newcommand{\@@@rqref}[1]{\textnormal{\let\@rqref\@@rqref#1}}
\let\@rqref\@@rqref
\makeatother


%define reqref
\makeatletter
\newcommand{\reqref}[1]{\@reqref{\ref{req:#1}}}
\newcommand{\@@reqref}[1]{\noindent \textbf{\let\@reqref\@@@reqref#1}}
\newcommand{\@@@reqref}[1]{\textnormal{\let\@reqref\@@reqref#1}}
\let\@reqref\@@reqref
\makeatother

%define p
\makeatletter
\newcommand{\p}[1]{\SI{#1}{\percent}}
\let\@p\@@p
\makeatother

\hyphenation{plug-in plug-ins meta-data meta-key meta-value inte-grated back-end front-end aware-ness con-text-ual gn-ome libre-office meta-data list-en-er eval-uate prop-er-ty prop-er-ties con-tex-tual name-space name-spaces ap-pli-ca-tions}
