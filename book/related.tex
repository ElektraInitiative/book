\chapter{Related Work}
\label{chapter:related}

\chapterprecishere{begin virus \\
I am a book virus! Copy me into your book to help me spread! \\
end.}
\chapterhung

In this chapter we compare our work with others.
In~\secref{related-configuration} we look into other systems providing configuration access.
In this book, we introduced context awareness and related new programming techniques, related work of which we discuss in~\secref{related-awareness} and \secref{related-programming}.
In~\secref{related-methodology} we focus on methodology in other work.
We aim to answer the question:
\rqRelated*



\section{Configuration}
\label{sec:related-configuration}

Systems that provide access to configuration are naturally related to \elektra{}.
We found, however, that only in rare exceptions the abstractions are similar to the abstractions in \elektra{}.
Thus most systems providing configuration access are not as similar to \elektra{} as one might think.
In this section we discuss different approaches to various systems and describe the differences in their abstractions.

\subsection{Configuration Libraries}

In general, other configuration libraries differ from \elektra{} by having no or little external specifications.
They require applications to hard code their configuration access, which prevents introspection, external validation, adapted tooling, and other wanted properties.%
{\parfillskip=0pt plus .7\textwidth \emergencystretch=.5\textwidth \par}

Apache Commons Configuration~\cite{nosal2012supporting,online2016apache} abstracts over different configuration sources using the factory pattern~\cite{gamma1994design,denisov2013overview}.
Different from \elektra{}, developers need to hard code which sources and validations they want to use.
Thus it does not enable introspection.
Furthermore, it is tightly integrated in Java technology.
In the Version~2.1 (released on \formatdate{20}{8}{2016}), Apache Commons Configuration
\begin{itemize}
\item
requires applications to completely specify which configuration files in which syntax shall be used, causing the configuration integration problem,
\item
does not support any form of external specifications except for those that are tied to specific configuration file formats such as XML,
\item
does not provide any context awareness, and
\item
introduces a complex multi-threading model.
\end{itemize}

\citet{nosal2012supporting,nosal2014xml}~extended the ideas of Apache Commons Configuration.
Their work enables users to integrate source code annotations into a coherent system abstracting configuration sources.
Their meta-configuration is similar to bootstrapping in \elektra{}.
Different from \elektra{}, the solution is tightly coupled to Java technology and neither has support for validation nor for context awareness.


\citet{roll2003towards} introduced a way to generate CORBA initialization code from XML specifications.
Her idea was to avoid hard-coded initialization code.
She had goals similar to our high-level API but she focused on CORBA and did not support context awareness.

\citet{denisov2015functional} summarized different requirements for configuration libraries.
Unfortunately, the outlined configuration library was not fully implemented.
No empirical evidence was given for these requirements.
Nevertheless, our empirical results confirm most of these requirements.
From our data, however, we cannot find argumentation for two of his requirements:
\begin{itemize}
\item \enquote{support for complex data structures}:
We did not find any case of complex data structures in the configuration settings of the software we analyzed.
Complex data structures that refer to other elements can easily be avoided by improved lookup that resolve links in data structures.
We argue that objects shall not be directly serialized into configuration files, but instead configuration settings shall be designed from a system administrator's perspective.
We agree that collections shall be supported but we would not say that they are ``complex data structures''.
\item We did not see why \enquote{support of cloud services} requires more than \enquote{support for different configuration sources/persistent backing stores} for a configuration library.
Non-file-based storage is needed for many systems and its need is not limited to cloud services.
\elektra{} works with network file systems, and can directly request configuration files from URLs via Curl resolvers.
From the survey we cannot confirm that there are additional requirements for cloud-based systems:
Only a single person of the survey mentioned ``cloud-init based configuration'' (without giving new requirements) although several OpenStack developers participated in the questionnaire.
We agree that cloud-based setups can be different in aspects related to configuration management.
\end{itemize}

\subsection{Configuration File Parsing}

One of the main difficulties in \elektra{} is the support of the many configuration file formats.
A significant portion of the development time went into the many parsers and serializers.
We considered different ways of how to implement configuration file parsers efficiently.
Here we mention some techniques of parsing configuration files, which got attention from the research community.

Lenses promise to solve the problem of having separate implementations of parsers and serializers~\cite{foster2008boomerang,lutterkort2008augeas,ko2016bigul}:
A single specification, i.\,e., a lens, is used to parse and serialize configuration files, leading to trouble-free \empha{round-tripping} properties.
We found that most of the development time is invested in better abstracting the configuration settings from the concrete syntax; and not in having separate source code for parsers and serializers.
Augeas~\cite{lutterkort2008augeas} is an implementation of these ideas focusing on configuration file manipulation.
We already discussed its properties in \secref{augeas}.
We found Augeas useful because of its preservation of all white spaces and comments.
It is well suited to access legacy formats such as ^sshd^ or ^ntp^ that do not provide parsing libraries.
Thus we integrated Augeas in \elektra{}.

The PADS language~\cite{fisher2005pads,fisher2010next,fisher2011pads} tackles the more general problem of processing ad hoc data sources.
From a type specification of ad hoc data, parsers and serializers are generated.
Different from \elektra{}, PADS has fewer assumptions on how the data looks like.
The data might even contain errors, and PADS can still work with them.
We find the (error) model of PADS elegant, but a potential adaptation to \elektra{} is future work.


\subsection{Validation}

Validation languages are a wide topic we cannot fully cover here.
\citet{hartmann2016validation} evaluated requirements of different stakeholders for data applications and published 81 types of constraints.
We interpret the many types of constraints as confirmation that \elektra{} needs to be extensible to capture any types of constraints.

\citet{murata2005taxonomy} created a taxonomy of XML schema languages using formal language theory.
XML schemas excel most of \elektra{}'s validation strategies in terms of expressibility.
\elektra{}, however, allows users to combine many strategies, which is difficult using XML schemas.
It would require to solve the problem of fully-automatic schema matching, which is impossible in general~\cite{batini1986comparative,do2003comparison,rahm2001survey}.
For example, if we need to check integrity constraints, we use Schematron, and miss the features of XDuce~\cite{hosoya2003xduce}.
Changing from one schema to the other often means rewriting the whole schema.\footnote{Only for the most popular XML schemas conversion tools exist, for example, Trang or the Multi-Schema Validator.}
One of the rare exceptions within XML is RELAX~NG~\cite{clark2002relax}, which allows users at least to combine different data types.
In \elektra{} we easily combine validation strategies.
If a feature is missing, we extend the configuration specification language with new plugins.

XML technology is not known to be ideally suitable for describing key-value pairs nor configuration settings~\cite{bray1997xml}.
Furthermore, it is easily too verbose to be written by hand~\cite{brabrand2008dual}.
Thus often syntactic alternatives are proposed, which are less verbose and can be transformed with a single specification~\cite{brabrand2008dual}.
As another example, RELAX~NG supports an XML syntax and a compact syntax~\cite{clark2002relax}.


\subsection{Links}

XML technologies have a variety of ways to express links.
They are not specified to be used outside of XML technologies.
Configuration files, however, are often not in XML syntax.
XPointer~\cite{grosso2003xpointer} permits identifying XML fragments.
XInclude~\cite{marsh2006xml} provides an element ``fallback'' similar to the property \property{default} as described in this book.
For the other link properties we described no similar concepts are provided~\cite{raab2015kps}.


Configuration links were proposed and formally developed by~\citet{reiser2009core,reiser2009managing}.
They differ from our fallback and override links because they~\cite{raab2015kps}
\begin{itemize}
\item exclusively refer within specifications, while \elektra{}'s links refer to configuration settings and specifications,
\item are interpreted when serializing configuration settings, while \elektra{}'s links are evaluated at run-time,
\item adopt propositional logic to identify selections, and
\item seem to lack support for transformation rules.
\end{itemize}


\subsection {Misconfiguration}

There is a large body of research for fixing misconfigurations that already became manifest~\cite{yin2011empirical,zhang2013automated,zhang2014configuration,xiong2012range,attariyan2010automating}.
On contrary, \elektra{} wants to avoid that misconfigurations occur in the first place.
Taking more context into account avoids some kinds of misconfigurations.%
{\parfillskip=0pt plus .7\textwidth \emergencystretch=.5\textwidth \par}

\citet{nagaraja2004understanding} tried to avoid having a completely duplicated production environment and wanted to nevertheless catch misconfigurations degrading performance.
In an extensive user study with system administrators they observed 42 misconfigurations.
They distinguished between local and global misconfigurations.
Both kinds of misconfigurations involve context at different levels.
While some misconfigurations would be caught by trivial \elektra{Spec} validations, other operation errors are out of scope for \elektra{} (for example restarting services).
We are positive, however, that the combination of modern configuration management tools together with \elektra{} catches all errors described in the paper~\cite{nagaraja2004understanding}.\footnote{
Assuming that the workflows are adapted to a more modern style coherent with how the configuration management tools work.
For example, you would not manually migrate database servers in modern, redundant, infrastructure-as-a-code systems.}

ConfErr~\cite{keller2008conferr} is able to localize misconfiguration by trying to inject possibly invalid configuration settings.
ConfErr does not use a configuration specification nor does it analyze the source code, which puts severe limitations on its effectiveness.
ConfErr could benefit from guidance of \elektra{Spec} to explore border cases in a more targeted way.

AutoBash~\cite{su2007autobash} and ConfAid~\cite{attariyan2010automating} have similar goals as \elektra{}.
Unlike \elektra{}, predicates that test the application must be available on the productive system.
In \elektra{}, problems are ruled out by the specification, so that they cannot occur in the productive system.

Spex~\cite{xu2013blame} infers parts of the configuration specification by analyzing the source code of applications.
Spex's approach is complementing \elektra{} in the sense that it can be used for initial construction of the specification for large existing applications.
Spex is not suitable to be used for development and generation of end-user documentation.

\citet{xu2015hey} questioned which of the many configuration settings are used in practice.
They argue that users are confused by too many settings.
We fully agree and are positive that \elektra{Spec} helps by automatically deducing many settings from context.
These augmented settings can be removed from user guides.
Advanced users still have the possibility to override such configuration settings, addressing our findings in the survey~\cite{raab2017introducing}.%
{\parfillskip=0pt plus .7\textwidth \emergencystretch=.5\textwidth \par}

\citet{jin2014configurations} portrayed obstacles in configuring real-world systems.
The authors calculated that Firefox has 1,957 settings and guessed that they overlooked only a small part of settings.
In our study we show, however, that the configuration access API ^getenv^ drastically increases the total amount of available configuration settings~\cite{raab2017introducing}.

\citet{nadi2014mining,rabkin2011static} extracted program configuration specifications from source code.
They confirm that, even though many specifications are extracted, we need additional external knowledge.
We show in our work how context awareness contributes as external knowledge~\cite{raab2016unanticipated}.

PCheck~\cite{xu2016early} aims at validating configuration settings early.
The tool searches for validation checks within the source code and moves them into start-up code.
Different from \elektra{}, it requires validation code to be present in the application.
But as unveiled in our study, the present validation specifications are often incomplete because of missing context information.

Another idea is to detect inadequate error messages for misconfiguration~\cite{zhang2015proactive}.
\elektra{} systematically avoids some problems in error messages by automatically adding important information, such as the affected configuration file, the mountpoint, etc.

Other related work assumes that only a single programming language is applied.
\citet{rabkin2011static} describe how configuration settings are extracted statically.
\citet{zhang2013automated} present a tool that finds wrongly configured settings in a fully automated way.
\citet{rabkin2011precomputing} precompute possible misconfigurations diagnosis.
In contrast, \elektra{} has no assumption on the choice of programming languages.

\citet{xu2015systems} surveyed different ways of how to improve on misconfiguration.
At that time, the idea of context-aware configuration was in an early stage~\cite{raab2014program}.
Thus context-aware configuration was not part of this survey.


\subsection{Configuration Management}
\label{sec:related-management}

Configuration management is related to the topic of the book, but configuration management is situated one level higher.
While configuration management is concerned about the content of configuration settings, \elektra{} is concerned about how applications access these configuration settings.
In \elektra{} both configuration settings and specifications are well suited to be managed via configuration management tools.
This gives an important advantage because configuration management tools struggle to work with the many ways to access configuration settings~\cite{burgess1995cfengine}.

Configuration management includes managing configuration settings for all nodes in a network~\cite{delaet2010short}.
In this context, directly changing configuration files is regarded as anti-pattern.
For smart phones and other customizable devices users often desire to directly reconfigure their personal devices.
\elektra{Spec} allows personal customization, while still enforcing the configuration specification managed by system administrators~\cite{raab2016improving}.
Configuration management tools do not support to locally validate configuration.
Misconfiguration easily reaches the systems, in particular if the consistency problems are related to context.
\elektra{} mitigates these problems by local, context-aware validation.

\citet{zdun2006tailorable} argues that the concern \enquote{behavioral composition and configuration} shall be treated as a first-class entity.
This approach goes a different direction than \elektra{}.
It values composition, reusability and modularity of source code more than the resulting system administrators' interfaces~\cite{raab2015kps}.

\citet{gruber1995ontologies} used ontologies for sharing data.
He aimed at minimal ontological commitment, i.\,e., to tolerate different forms of representation.
For example, different date formats such as ``1993'' or ``March 1993'' shall be accepted.
In \elektra{} plugins make sure that different formats are canonicalized so that frontends receive expected values.
\citet{gruber1995ontologies} facilitated references similar to \elektra{}'s key names~\cite{raab2015kps}.


\citet{syrjanen2000including}~formalized the dependences of the Debian Package management system using the stable model semantics.
The goal was to get better diagnostic information for error messages.


































\section{Context Awareness}
\label{sec:related-awareness}

In this section we discuss related work on context awareness.


\subsection{Context-oriented Programming}

Context-oriented programming plays a role within software engineering~\cite{baldauf2007survey,hong2009context,salvaneschi2012context}.
It aims at comprehensible programs being more context aware.
As extension, \elektra{} adds context awareness without source code modification~\cite{raab2016unanticipated}.

\citet{plaice2010contextcartesian} applied a Cartesian approach to context.
One of the differences is the use of an $n$-dimensional table with lazy computation instead of our one-dimensional key set with eagerly computed values.
While the Cartesian approach is theoretically more powerful, it has the disadvantage that its contents cannot easily be serialized.
However, efficient serialization is a requirement for working with execution environments~\cite{raab2014program}.%
{\parfillskip=0pt \emergencystretch=.5\textwidth \par}

\citet{watanabe2014reflective} introduced an actor-based model for cross-context messages.
They improved the receiving of messages in the correct context.
Different from them \elektra{} works with threads and processes instead of actors~\cite{raab2015global}.


\citet{costanza2006efficientlayeractivation} provided ContextL as an extension to Common Lisp Object System (CLOS) and relied on its features: dynamic class generation, multiple inheritance, dynamically scoped variables, and multiple dispatch.
In most programming languages these features are not available.
For example, C++ only supports multiple inheritance from this set of features~\cite{raab2014program}.

Dynamic aspect weaving, for example, in the Steamloom virtual machine~\cite{bockisch2004steamloom}, provides language constructs for the activation of partial program definitions.
Different from \elektra{}, it requires virtual machines~\cite{raab2014program}.

\citet{kamina2015generalized} proposed a generalized activation mechanism using contexts and subscribers.
\elektra{} only provides a subset of these generalized activation mechanism.
Their implicit activation, however, can have serious impact on the performance, conflicting with our performance requirements~\cite{raab2016persistent}.

\citet{pape2016optimizing}~used a meta-tracing just-in-time compiler to better cope with non-standard lookups.
This line of research promises to get implicit activation with acceptable costs.
Their work needs just-in-time compilation to work, not available in some of \elektra{}'s supported languages.

\citet{bainomugisha2012interruptible} suggested that even currently running code shall be interrupted and restarted to better fit the viewpoint time of context awareness.
This idea nicely fits into our concept of synchronization points:
\elektra{}'s frontends can be extended to jump back to the last synchronization point on context changes.
\citet{cardozo2014validation,taing2016unanticipated} instead focused on consistency of unanticipated adaptation of context.

\citet{chiba2012we} demonstrated that modularity does not necessarily need syntactic extensions.
Instead they introduced a database and browser for cross-cutting concerns.
The database contains the information which code snippets shall be kept synchronous.
The ideas of the approach can be applied to \elektra{Spec}.

\subsection{Contextual Values}

\citet{montangero1975magma,asirelli1979flexible} pioneered contextual values already in 1975.
Similar to the key set in \elektra{}, they use context-value pairs where all values in all contexts are stored.

Löwis et al.~\cite{lowis2007contextbeyond} proposed an updated form of contextual values\footnote{Named context variables in their work.} extending on context-oriented programming.
Their layer activation\footnote{They call it binding of contextual values.} forces the developer to explicitly declare layers.
Thus their approach would benefit from having contextual values as layers~\cite{lowis2007contextbeyond}.
Their proposal for implicit layer activations avoids explicit synchronization points at a high price:
They check context on every use of every contextual value~\cite{lowis2007contextbeyond}.


\citet{tanter2008contextvalues} analyzed the semantics of contextual values in-depth.
Different from \elektra{}, these contextual values need modifications in the Scheme interpreter.


\subsection{Context-aware Applications}

\index{context-aware application|boldindex}

\citet{schilit1994context} proposed context-aware computing.
They already envisioned automatic contextual reconfiguration.
Since the work of \citet{dey2000towards}, the research topic received more attention.

Most approaches---to make applications more context aware---require modifications in the source code.
To mitigate efforts, context-oriented middleware~\cite{geihs2009comprehensive,henricksen2005middleware,gu2004middleware,anthony2009context} and frameworks~\cite{williams2014contextion,raento2005contextphone} were introduced.
\elektra{} can be seen a light-weight context-aware middleware or framework for local configuration settings~\cite{raab2016unanticipated}.

\citet{riva2006unearthing} extracted design patterns from context-aware applications.
They found that a hybrid mediator-observer is used in almost all of their surveyed context-oriented programming systems.
\elektra{} is no exception and uses the observer pattern for thread synchronization~\cite{raab2015global}.



\citet{baldauf2007survey,hong2009context,alegre2016engineering} surveyed the field of context-aware applications in depth.
\citet{mens2016taxonomy} introduced a taxonomy of approaches to context-aware applications.
As in our work, they found the execution environment as source for context information.
Based on their taxonomy, \elektra{} shall be classified as follows:%
{\parfillskip=0pt plus .7\textwidth \emergencystretch=.5\textwidth \par}
\begin{itemize}
\item \elektra{} does not support behavior missing in the source code.
It enables, however, to switch between existing behavior at run-time as needed by the context.
\item \elektra{} supports both contextual and context features:
The decision is up to the application.
\item \elektra{} uses a one-branch context tree:
Without context placeholders contextual values are non-context-aware configuration settings.
\item \elektra{} supports programmer-declared dependences~\cite{raab2016persistent}.
\end{itemize}


\citet{alexandrov1998ufo} uses intercepting of library calls to promote user experience but with a different goal than \elektra{}.

\citet{lee2014deployment} proposes a context-aware, deployment-oriented development process similar to our context-oriented software engineering process.
Different from \elektra{}, context needs to be known at design time.


\citet{parra2007introducing} argued that context-aware specifications need integration into context-aware applications using compiler technologies.
While this might be a good idea for some specifications, in general we propose the opposite:
Even if many context specifications are interpreted at run-time, the overhead is little.
Additionally, only if context specifications are evaluated at run-time flexible adaptations for new viewpoints of contexts are possible.


\citet{kamina2014context} defined layers as binary information:
They are either activated or deactivated.
For location information, for example, we could use ^inAustria^ and ^inArgentinia^ as layers.
With a high number of locations, however, the number of layers gets unmanageable.
To avoid such high numbers of layers we prefer key-value pairs for layer information~\cite{keays2003context}.
Then even contextual specifications with few layer names describe a large set of possible values.


\subsection{Context-aware Web Services}

\citet{niu2016wif4inl} reported on a framework, called WIF4InL, for indoor localization.
Similar to \elektra{}, it provides an application-neutral API.
\citet{niu2016wif4inl} used layering of Web-based APIs:
Their composite API \enquote{allows high-level queries by internally combing some fundamental API.}~\cite{niu2016wif4inl}.
Different from \elektra{}, only queries regarding locations are provided.
\elektra{}'s design is different:
We would expect the application to not care about localization details and directly query the configuration settings.
For example, suppose a screen shall automatically change its brightness according to the indoor location.
In WIF4InL, the developer would directly query the API and encode the rules about the brightness as part of the program.
In \elektra{}, the developer would ask for the current brightness.
Internally, \elektra{} could use WIF4InL as context sensor to calculate the correct brightness.
While \elektra{} provides more portability for its applications, using WIF4InL directly opens more possibilities for behavior where several real-world objects interact.
Furthermore, WIF4InL provides topological and navigational features, for which \elektra{} is not the appropriate tool.

\citet{kapitsaki2008architecture} argued that context awareness is an \enquote{essential aspect---almost a requirement---of mobile services}.
They propose a context-oriented Web service architecture using the SOAP protocol and build upon work of~\citet{keidl2004towards}.
They share with \elektra{} that their architecture is plugin based and that it supports manipulation of responses the user gets.













\section{Programming Techniques}
\label{sec:related-programming}


In this section we elaborate on programming techniques similar to programming techniques used in \elektra{}.

\subsection{Product Lines}

Software product lines~\cite{pohl2005software,schaefer2011formal,berger2015feature,midtgaard2014systematic} investigate configuration settings at design time using feature models~\cite{lee2002concepts}.
They aim at manufacturing software products by combining features.
While they share some goals with \elektra{}, other goals are completely different.
Whereas product lines focus on creating customized products, \elektra{} targets on customizing software at run-time and on-demand.
\elektra{}'s specifications can be seen as variability model, but \elektra{} has its focus on run-time configuration access for FLOSS applications.
Thus \elektra{}'s goals are more in line with goals of FLOSS software, where distributions avoid variants of the same software packages.

Linux, by using the Kconfig language, can be considered as highly-configurable product~\cite{passos2015feature}.
Compilation variants of plugins in \elektra{} work similarly.
\citet{berger2010variability} compare two variability modeling languages: Kconfig and CDL.

\citet{leite2016computing}~suggested to facilitate ideas from software product line for configuration management.
They also propose to replace imperative scripts with descriptions of desired features.


Recent advances in software product lines switched from compile-time to load-time variability~\cite{rhein2016variability}.
\citet{mauro2016context} built on these dynamic software product lines and extended them with context awareness.
Similar to \elektra{}, they proposed a single specification that includes context information.
\citet{mauro2016context} relied on constraint solving, an approach that could also be applied as validation plugin within \elektra{}.%
{\parfillskip=0pt \emergencystretch=.5\textwidth \par}


\subsection{Database Management Systems}

Stored procedures~\cite{eisenberg1996standard} are used to validate or transform data but cannot tamper with some semantics of the database.
For example, there are limitations in which way a stored procedure can reconfigure the database.
Furthermore, automatic program modifications of the stored procedures are challenging.

\citet{elmongui2009chameleon} introduced context-aware data management systems.
Such query languages can be on top of \elektra{}~\cite{raab2016persistent}.

\citet{grier09how} argued that security of plugins can be enforced if direct access to the system is restricted.
Plugin architectures were also proposed on operating system level~\cite{decasper2000router}.


\subsection{Meta-level Programming}

\citet{umuhoza2015automatic} studied different code generation techniques for mobile development.
\citet{loques2000integration} discussed the correspondence between concepts of configuration and meta-level programming.
\citet{aktemur2009comparative} compared different techniques to implement customizable libraries.
They did not consider context~\cite{raab2016persistent}.


\citet{jung2005embedded} facilitated code generation.
They target embedded systems with focus on resource utilization as discussed in~\secref{resource-utilization}.
Using partial evaluation they removed Libxml2 dependences to make the resulting binaries fit on their target platform.
Different from \elektra{}, they assume configuration settings to be static, i.\,e., neither context aware nor customizable.

\citet{osterlund2014concurrent}---using the ideas of \citet{ericsson2008composition}---presented a way of how programs, that use conservative locking, can be accelerated.
They used run-time knowledge in order to choose the optimal locking scheme~\cite{osterlund2017adaptive}.
The frontends of \elektra{} can profit from this technique.



\section{Methodology}
\label{sec:related-methodology}

Here we describe work that used similar methods or elaborated on methodology we used.

\subsection{Type systems}
\label{sec:related-type-systems}

Type systems~\cite{pierce2002types,guttag1977abstract,puntigam1995type,schwartzbach1994object,liskov1994behavioral,wegner1990concepts,cardelli1985understanding,puntigam1997coordination,puntigam2007see,puntigam2007black} allow developers to specify constraints on data.
Configuration values are data, thus configuration specifications can be supported by type systems.

\citet{gannon1977experimental} conducted the first user research on type systems.
The first experiment was quantitative, with nearly no qualitative aspect.
Their work has a subjective tone, for example, the paper states that these \enquote{results come as no surprise}.
The next paper conducting a user study was written ten years later~\cite{prechelt1998controlled}.
Again it follows a quantitative approach, but it puts as goal to not take personal anecdotes as granted.
It was still not mentioned that results might not be universally valid.
The work of \citet{daly2009work} is one of the early qualitative papers.
Their results focused on the usefulness of error messages.

\citet{hanenberg2010experiment} finally established user research for type systems at a greater scale.
The major contribution was that he doubted some previously well-established opinions about the positive impact of static type systems.
The effort of his study was huge:
He developed two computer languages that were identical with the only exception that one had static types and the other had not.
Even with 49 subjects, of whom each worked over 27 hours, he could not demonstrate a statistical significant difference.
In his experimental setup, he combined a small application with a larger one.
While in the smaller application the static-typed language yielded disadvantages, this was not the case for the larger~application.%
{\parfillskip=0pt \emergencystretch=.5\textwidth \par}


The main methodology for type systems is to prove soundness and completeness~\cite{pierce2002types}.
\elektra{Spec} currently does not have sound and complete checking of the configuration specification nor settings.
Candidate type systems for \elektra{} are:

\begin{enumerate}
\item More powerful data types, for example, supporting units of measurement~\cite{pierce2002types,dhungana2013generation}.
\item Ways to define subtyping between configuration settings and subtrees thereof~\cite{pierce2002types}.
\item Sum types are both self-describing and their tag is one of the simplest form of metadata.
Sum types are the opposite of the subtypes:
They allow us to loosen the strictness of a type.
For example, to allow a string to be empty or to contain a number, we use sum types.
\item Constraint programming~\cite{fruhwirth2003essentials}, for example with Gecode, COIN-OR LP, and Z3.
\item Data schemas, for example, Relax~NG schema \cite{clark2002relax}, and XSD \cite{wadler2003xml}.
\item Check relations and infer types from relations (type reconstruction)~\cite{pierce2002types,harkes2016icedust}.
\item Any combination of the techniques above.
\end{enumerate}


\subsection{Surveys}

\citet{berger2013survey} and \citet{villela2014survey} conducted a questionnaire asking about variability modeling.
Their survey targets a different group.

Several studies focused on FLOSS developers.
\citet{michlmayr2005quality} explored quality problems using interviews.
We affirm that documentation often is missing.
\citet{barcomb2015developers} explored how developers acquire FLOSS skills.
\citet{crowston2012free} surveyed other FLOSS development studies.


\subsection{Human Computer Interaction}

Human computer interaction has a broad spectrum of user research methodology, including ``research through design'' \cite{gaver2012expect}.
Its goal is to produce knowledge by studying the process of designing and the interaction of design with users \cite{storni2015personal}.
So design is merely the means to extend the ability to investigate and acquire new knowledge.
Nevertheless, the main interest of this research is how things can be improved and not how things are.
We agree with \citet{tichy1997should}, who argued that computer scientists should experiment more.

The similarly called ``design science'' is a rigorous method used to design and evaluate information systems in case studies~\cite{wieringa2009design}.
For software engineering ``theory of cases'' was developed~\cite{eisenhardt2007theory,easterbrook2008selecting}.%
{\parfillskip=0pt plus .7\textwidth \emergencystretch=.5\textwidth \par}




\subsection{Application Integration}

Integration via middleware is a well-established research topic.
For example, \citet{paul1997integration}~introduced the tool interconnection language.
It improves the integration of components by specifying operational and event interfaces.
The tool interconnection language compiler generates adapter code, which influences the run-time environment.
In the case study, they used CORBA.

Integration of software engineering tooling is well-researched, too.
For example, DesignSpace~\cite{demuth2015designspace} integrates development artifacts and their relations.
Monto~\cite{keidel2016ide} allows users to integrate different programming language environments into integrated development environments.

Enterprise application integration~\cite{conrad2005enterprise} has a completely different focus:
It tries to better integrate support of business processes.

Application integration via configuration settings is a novelty of this book.
We could not find literature that tried to integrate software configuration settings and specifications by context-aware configuration.
