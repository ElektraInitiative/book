\chapter{Kurzfassung}

Software ist mit Hilfe von Konfigurationseinstellungen, welche üblicherweise in Konfigurationsdateien gespeichert werden, hochgradig adaptiv.
Moderne Systeme beinhalten bereits detaillierte Informationen, in welchem Kontext sich das System gerade befindet.
Wir definieren Kontext als jede Information relevant für Konfigurationseinstellungen, zum Beispiel der aktuelle Ort, vorhandene Hardware, Netzwerkkonfigurationen, Konfigurationen anderer Programme, etc.

Heutzutage sind Konfigurationseinstellungen und Kontext nicht verbunden.
Adaptionen von Konfigurationseinstellungen, wie besseres Anpassen an den Kontext, werden manuell durchgeführt -- oftmals in komplizierten Schnittstellen und ohne hilfreiche Rückmeldungen bei Fehlern.
Mit einer in der Arbeit durchgeführten Umfrage und Quelltextanalyse erkannten wir Ursachen dafür, warum Programme derzeit selten Kontext berücksichtigen:
Entwickler haben Informationen über den Kontext nicht bequem verfügbar und vermeiden Abhängigkeiten zu Software, welche die Informationen bereitstellen könnte.

Wir zielen darauf ab, diese Probleme durch eine systemnahe Konfigurationsspezifikationssprache, welche die Beziehungen zwischen Konfigurationseinstellungen und Kontext beschreibt, zu lösen.
Der Hintergedanke ist, dass die Berücksichtigung von Kontext in den Konfigurationseinstellungen die Benutzerfreundlichkeit erhöht und fehlerhafte Konfigurationseinstellungen reduziert.
Unsere Konfigurationsspezifikationssprache orchestriert dabei Frontends und Backends, um den Zugriff auf Konfigurationseinstellungen zu vereinheitlichen.
Wir führen ein Frontend (eine Programmierschnittstelle für Entwickler) ein, welches mittels Quelltextgenerierung die Konfigurationsspezifikationssprache in kontextsensitive Variablen abbildet und dabei Kontextsensitivität in dynamischen Sichtbarkeitsbereichen ermöglicht.
Die Konfigurationsspezifikationssprache modularisiert Quelltexte in Form von Plugins, mit deren Hilfe die Backends aufgebaut werden.
Dadurch werden die zuvor genannten Probleme von fehlenden Kontextinformationen in Applikationen und unerwünschten Abhängigkeiten gemindert.
Um unseren Ansatz zu validieren, haben wir mehrere Sprachkonstrukte einer modularen Konfigurationsspezifikationssprache implementiert.%
{\parfillskip=0pt plus .8\textwidth \emergencystretch=.5\textwidth \par}

Wir haben die Implikationen unserer neuartigen modularen Abstraktionen der Konfigurationsspezifikationssprache ausführlich evaluiert.
Dabei diskutieren wir Funktionalität zur Introspektion, neu entwickelte Werkzeuge, Analysen zur Fehlerbehebung und Entwicklungszeit.
Ebenfalls messen wir den durch Backends verursachten Mehraufwand und vergleichen Lösungen, implementiert als Frontends und Backends, mit dem Ergebnis, dass Mehraufwände in modularen Backends gering sind.
Trotz Kontextsensitivität ermöglicht das Frontend lesende Zugriffe auf Konfigurationseinstellungen mit der Laufzeit-Effizienz von native Variablen.
Da es unrealistisch ist alle existierenden Applikationen auf solche typsicheren Frontends umzuschreiben, demonstrieren wir verschiedene Möglichkeiten, wie bestehende Applikationen ebenfalls an unsere Backends angebunden werden können.
Mit 16 bekannten Standardapplikationen, wie etwa Firefox, zeigen wir, dass die Kontextsensitivität der Konfigurationseinstellungen auch ohne Quelltextänderungen verbessert werden kann.
\selectlanguage{english}
